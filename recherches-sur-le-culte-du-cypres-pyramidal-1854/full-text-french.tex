%XeLaTeX
\documentclass{article}

\setlength{\emergencystretch}{100pt}
\usepackage{tocloft}
\cftsetindents{subsubsection}{3em}{7em}
\cftsetindents{subsection}{3em}{6em}
\cftsetindents{section}{3em}{6em}
\usepackage{float}
\usepackage{svg}
\usepackage{pdflscape}
\usepackage{graphicx}
\usepackage{microtype}
\usepackage{fontspec}
\usepackage{polyglossia}
%--------------------------------------------------------------
% Languages
\setdefaultlanguage{french}
\setotherlanguage{greek}
\setotherlanguage{armenian}
\setotherlanguage{hebrew}
\setotherlanguage{coptic}
\setotherlanguage{chinese}
%--------------------------------------------------------------
% Global font features (must precede the \set...font calls)
\defaultfontfeatures{Scale=MatchLowercase}
%--------------------------------------------------------------
% Latin‑script families
\setmainfont{Latin Modern Roman}
\setsansfont{Latin Modern Roman}
\setmonofont{Latin Modern Roman}
%--------------------------------------------------------------
% Auxiliary families
\newfontfamily\smallcaps{Latin Modern Roman Caps}
%--------------------------------------------------------------
% Greek families (after language declaration)
\newfontfamily\greekfont[Script=Greek]{Gentium Plus}
\newfontfamily\greekfontsf[Script=Greek]{Gentium Plus}
\newfontfamily\greekfontitalic[Script=Greek]{Gentium Plus}
\newfontfamily\greekfonttt[Script=Greek]{Gentium Plus}
%--------------------------------------------------------------
% Armenian families (example)
\newfontfamily\armenianfont[Script=Armenian]{DejaVuSerif}
\newfontfamily\armenianfontsf[Script=Armenian]{DejaVuSerif}
\newfontfamily\armenianfontitalic[Script=Armenian]{DejaVuSerif-Italic}
%--------------------------------------------------------------
% Hebrew families – **all four** are defined now
\newfontfamily\hebrewfont[Script=Hebrew]{Keter YG}          % serif
\newfontfamily\hebrewfontsf[Script=Hebrew]{Keter YG}    % sans‑serif (SR = “Sans‑Roman” variant)
\newfontfamily\hebrewfonttt[Script=Hebrew]{Keter YG}       % monospaced (fallback to same font)
%--------------------------------------------------------------
% Coptic family (example)
\newfontfamily\copticfont[Script=Coptic]{Noto Sans Coptic}
\newfontfamily\copticfontsf[Script=Coptic]{Noto Sans Coptic}
%--------------------------------------------------------------
% Chinese
\newfontfamily\chinesefont{Noto Serif CJK SC}
\newfontfamily\cJKfontsf{Noto Sans CJK SC}
%--------------------------------------------------------------
\newcommand{\textoverline}[1]{$\overline{\mbox{#1}}$}   

%SVGs
\newcommand*\arabicAAA{\raisebox{-0.5ex}{\includesvg[height=0.85em]{svgs/001.svg}}}
\newcommand*\arabicAAB{\raisebox{-0.5ex}{\includesvg[height=0.75em]{svgs/002.svg}}}
\newcommand*\arabicAAC{\raisebox{-0.5ex}{\includesvg[height=0.9em]{svgs/003.svg}}}
\newcommand*\arabicAAD{\raisebox{-0.5ex}{\includesvg[height=0.75em]{svgs/004.svg}}}
\newcommand*\arabicAAE{\raisebox{-0.5ex}{\includesvg[height=0.75em]{svgs/005.svg}}}
\newcommand*\arabicAAF{\raisebox{-0.1ex}{\includesvg[height=0.75em]{svgs/006.svg}}}
\newcommand*\arabicAAG{\includesvg[width=2em]{svgs/008.svg}}
\newcommand*\svgAAA{\includesvg[height=0.7em]{svgs/007.svg}}
\newcommand*\svgAAB{\includesvg[height=0.6em]{svgs/009.svg}}
\newcommand*\svgAAC{\includesvg[width=1em]{svgs/010.svg}}
\newcommand*\svgAAD{\includesvg[height=0.7em]{svgs/011.svg}}
\newcommand*\svgAAE{\includesvg[height=0.7em]{svgs/012.svg}}
\newcommand*\svgAAF{\includesvg[height=0.6em]{svgs/013.svg}}
\newcommand*\hieroAAAA{\includesvg[height=1em]{svgs/014.svg}}
\newcommand*\hieroAAAB{\includesvg[height=1em]{svgs/015.svg}}
\newcommand*\hieroAAAC{\includesvg[height=0.8em]{svgs/016.svg}}
\newcommand*\hieroAAAD{\includesvg[height=0.8em]{svgs/017.svg}}
\newcommand*\hieroAAAE{\includesvg[height=0.8em]{svgs/018.svg}}
\newcommand*\hieroAAAF{\raisebox{0.6ex}{\includesvg[height=0.5em]{svgs/019.svg}}}
\newcommand*\hieroAAAG{\includesvg[height=1em]{svgs/020.svg}}
\newcommand*\hieroAAAH{\includesvg[height=1em]{svgs/021.svg}}
\newcommand*\hieroAAAI{\includesvg[height=1em]{svgs/022.svg}}
\newcommand*\hieroAAAJ{\includesvg[height=1em]{svgs/023.svg}}
\newcommand*\hieroAAAK{\includesvg[height=1em]{svgs/024.svg}}
\newcommand*\hieroAAAL{\includesvg[height=1em]{svgs/025.svg}}
\newcommand*\hieroAAAM{\includesvg[height=1em]{svgs/026.svg}}
\newcommand*\hieroAAAN{\includesvg[height=1em]{svgs/027.svg}}
\newcommand*\hieroAAAO{\includesvg[height=1em]{svgs/028.svg}}
\newcommand*\hieroAAAP{\includesvg[height=1em]{svgs/029.svg}}
\newcommand*\hieroAAAQ{\includesvg[height=1em]{svgs/030.svg}}
\newcommand*\hieroAAAR{\includesvg[height=1em]{svgs/031.svg}}
\newcommand*\hieroAAAS{\includesvg[height=1em]{svgs/032.svg}}
\newcommand*\hieroAAAT{\raisebox{0.6ex}{\includesvg[width=1em]{svgs/033.svg}}}
\newcommand*\hieroAAAU{\includesvg[height=1em]{svgs/034.svg}}
\newcommand*\hieroAAAV{\includesvg[height=1em]{svgs/035.svg}}
\newcommand*\hieroAAAW{\includesvg[height=1em]{svgs/036.svg}}
\newcommand*\hieroAAAX{\includesvg[height=1em]{svgs/037.svg}}
\newcommand*\hieroAAAY{\includesvg[height=1em]{svgs/038.svg}}
\newcommand*\hieroAAAZ{\includesvg[height=1em]{svgs/039.svg}}
\newcommand*\hieroAABA{\raisebox{0.6ex}{\includesvg[width=1em]{svgs/040.svg}}}
\newcommand*\hieroAABB{\includesvg[height=1em]{svgs/041.svg}}
\newcommand*\hieroAABC{\includesvg[height=1em]{svgs/042.svg}}
\newcommand*\hieroAABD{\raisebox{0.6ex}{\includesvg[width=1em]{svgs/043.svg}}}
\newcommand*\hieroAABE{\includesvg[height=1em]{svgs/044.svg}}
\newcommand*\hieroAABF{\includesvg[height=1em]{svgs/045.svg}}
\newcommand*\hieroAABG{\includesvg[height=1em]{svgs/046.svg}}
\newcommand*\hieroAABH{\includesvg[height=1em]{svgs/047.svg}}
\newcommand*\hieroAABI{\includesvg[height=1em]{svgs/048.svg}}
\newcommand*\hieroAABJ{\includesvg[height=1em]{svgs/049.svg}}
\newcommand*\hieroAABK{\raisebox{0.6ex}{\includesvg[width=1em]{svgs/050.svg}}}
\newcommand*\hieroAABL{\includesvg[height=1em]{svgs/051.svg}}
\newcommand*\hieroAABM{\raisebox{0.6ex}{\includesvg[height=0.5em]{svgs/052.svg}}}
\newcommand*\hieroAABN{\includesvg[height=1em]{svgs/053.svg}}
\newcommand*\hieroAABO{\includesvg[height=1em]{svgs/054.svg}}
\newcommand*\hieroAABP{\includesvg[height=1em]{svgs/055.svg}}
\newcommand*\hieroAABQ{\includesvg[height=0.7em]{svgs/056.svg}}
\newcommand*\hieroAABR{\includesvg[height=1em]{svgs/057.svg}}
\newcommand*\hieroAABS{\includesvg[height=1em]{svgs/058.svg}}
\newcommand*\hieroAABT{\includesvg[height=1em]{svgs/059.svg}}
\newcommand*\hieroAABU{\includesvg[height=1em]{svgs/060.svg}}
\newcommand*\hieroAABV{\includesvg[height=1em]{svgs/061.svg}}
\newcommand*\hieroAABW{\includesvg[height=1em]{svgs/062.svg}}
\newcommand*\hieroAABX{\includesvg[height=1em]{svgs/063.svg}}
\newcommand*\hieroAABY{\includesvg[height=1em]{svgs/064.svg}}
\newcommand*\hieroAABZ{\includesvg[height=1em]{svgs/065.svg}}
\newcommand*\hieroAACA{\raisebox{0.3ex}{\includesvg[height=0.8em]{svgs/066.svg}}}
\newcommand*\hieroAACB{\includesvg[height=1em]{svgs/067.svg}}
\newcommand*\hieroAACC{\includesvg[height=1em]{svgs/068.svg}}
\newcommand*\hieroAACD{\includesvg[height=1em]{svgs/069.svg}}
\newcommand*\hieroAACE{\includesvg[height=1em]{svgs/070.svg}}
\newcommand*\hieroAACF{\includesvg[height=1em]{svgs/071.svg}}
\newcommand*\hieroAACG{\includesvg[height=1em]{svgs/072.svg}}
\newcommand*\hieroAACH{\includesvg[height=1em]{svgs/073.svg}}
\newcommand*\hieroAACI{\includesvg[height=1em]{svgs/074.svg}}
\newcommand*\hieroAACJ{\includesvg[height=1em]{svgs/075.svg}}
\newcommand*\hieroAACK{\includesvg[height=1em]{svgs/076.svg}}
\newcommand*\hieroAACL{\raisebox{0.3ex}{\includesvg[height=0.8em]{svgs/077.svg}}}
\newcommand*\hieroAACM{\includesvg[height=1em]{svgs/078.svg}}
\newcommand*\hieroAACN{\includesvg[height=1em]{svgs/079.svg}}
\newcommand*\hieroAACO{\raisebox{0.3ex}{\includesvg[height=0.8em]{svgs/080.svg}}}
\newcommand*\hieroAACP{\includesvg[height=1em]{svgs/081.svg}}
\newcommand*\hieroAACQ{\includesvg[height=1em]{svgs/082.svg}}
\newcommand*\hieroAACR{\raisebox{0.9ex}{\includesvg[width=0.5em]{svgs/083.svg}}}
\newcommand*\hieroAACS{\includesvg[height=1em]{svgs/084.svg}}
\newcommand*\hieroAACT{\raisebox{0.4ex}{\includesvg[width=1em]{svgs/085.svg}}}
\newcommand*\hieroAACU{\includesvg[height=1em]{svgs/086.svg}}
\newcommand*\hieroAACV{\raisebox{0.2ex}{\includesvg[height=0.6em]{svgs/087.svg}}}
\newcommand*\hieroAACW{\includesvg[height=1em]{svgs/088.svg}}
\newcommand*\hieroAACX{\includesvg[height=1em]{svgs/089.svg}}
\newcommand*\hieroAACY{\raisebox{0.5ex}{\includesvg[width=1em]{svgs/090.svg}}}
\newcommand*\hieroAACZ{\includesvg[height=1em]{svgs/091.svg}}
\newcommand*\hieroAADA{\includesvg[height=1em]{svgs/092.svg}}
\newcommand*\hieroAADB{\includesvg[height=1em]{svgs/093.svg}}
\newcommand*\hieroAADC{\includesvg[height=1em]{svgs/094.svg}}
\newcommand*\hieroAADD{\includesvg[height=1em]{svgs/095.svg}}
\newcommand*\hieroAADE{\includesvg[height=1em]{svgs/096.svg}}
\newcommand*\hieroAADF{\includesvg[height=1em]{svgs/097.svg}}
\newcommand*\hieroAADG{\includesvg[height=1em]{svgs/098.svg}}
\newcommand*\hieroAADH{\includesvg[height=1em]{svgs/099.svg}}
\newcommand*\hieroAADI{\includesvg[height=1em]{svgs/100.svg}}
\newcommand*\hieroAADJ{\includesvg[height=1em]{svgs/101.svg}}
\newcommand*\hieroAADK{\includesvg[height=1em]{svgs/102.svg}}
\newcommand*\hieroAADL{\includesvg[height=1em]{svgs/103.svg}}
\newcommand*\hieroAADM{\includesvg[height=1em]{svgs/104.svg}}
\newcommand*\hieroAADN{\includesvg[height=1em]{svgs/105.svg}}
\newcommand*\hieroAADO{\raisebox{0.2ex}{\includesvg[height=0.8em]{svgs/106.svg}}}
\newcommand*\hieroAADP{\includesvg[height=1em]{svgs/107.svg}}
\newcommand*\hieroAADQ{\raisebox{0.6ex}{\includesvg[width=1em]{svgs/108.svg}}}
\newcommand*\hieroAADR{\includesvg[height=1em]{svgs/109.svg}}
\newcommand*\hieroAADS{\includesvg[height=1em]{svgs/110.svg}}
\newcommand*\hieroAADT{\includesvg[height=1em]{svgs/111.svg}}
\newcommand*\hieroAADU{\includesvg[height=1em]{svgs/112.svg}}
\newcommand*\hieroAADV{\includesvg[height=1em]{svgs/113.svg}}
\newcommand*\hieroAADW{\includesvg[height=1em]{svgs/114.svg}}
\newcommand*\hieroAADX{\includesvg[height=1em]{svgs/115.svg}}
\newcommand*\hieroAADY{\raisebox{0.2ex}{\includesvg[height=0.6em]{svgs/116.svg}}}
\newcommand*\hieroAADZ{\includesvg[height=1em]{svgs/117.svg}}
\newcommand*\hieroAAEA{\includesvg[height=1.5em]{svgs/118.svg}}

\begin{document}

\renewcommand\thefootnote{\tiny\arabic{footnote}}
\let\oldfootnote\footnote
    \renewcommand{\footnote}[1]{\oldfootnote{#1}}
\begin{titlepage} % Suppresses headers and footers on the title page
	\centering % Centre everything on the title page
	%\scshape % Use small caps for all text on the title page

	%————————————————
	%	Title
	%————————————————

	\rule{\textwidth}{1.6pt}\vspace*{-\baselineskip}\vspace*{2pt} % Thick horizontal rule
	\rule{\textwidth}{0.4pt} % Thin horizontal rule
	
	\vspace{1\baselineskip} % Whitespace above the title
	
	{\Huge\smallcaps{Recherches sur le Culte \\ du Cyprès pyramidal \\ chez les \\ Peuples civilisés \\ de \\ l'Antiquité}}
	
	\vspace{1\baselineskip} % Whitespace above the title

	\rule{\textwidth}{0.4pt}\vspace*{-\baselineskip}\vspace{3.2pt} % Thin horizontal rule
	\rule{\textwidth}{1.6pt} % Thick horizontal rule
	
	\vspace{1\baselineskip} % Whitespace after the title block
	
	%————————————————
	%	Subtitle
	%————————————————
	

        {\smallcaps{Par \\\large M. Félix Lajard}}
 ‌
	%————————————————
	%	Editor(s)
	%————————————————
        \vspace*{\fill}    

        \vspace{1.0\baselineskip}

        {\small\smallcaps{Mémoires de l'Institut de France}}
	    
	    {\footnotesize\smallcaps{Académie des Inscriptions et Belles-Lettres \\ Tome Vingtième}}
        
	    \vspace{1\baselineskip}

        {\small\smallcaps{Paris 1854}}
		
	    \vspace{0.25\baselineskip} % Whitespace after the title block

        {\small\smallcaps{Solar Anamnesis Edition}}% Publication year}
    
	{\footnotesize\smallcaps{CC0 1.0 Universal}} % Publisher
\end{titlepage}
\clearpage
\tableofcontents
\clearpage
\section*{Introduction}
\begin{center}
(\emph{1\textsuperscript{re} lecture, le 3 mars 1843. 2\textsuperscript{e} lecture, le 2 juin 1843.})
\end{center}
\paragraph{}
L'étude des symboles qui entrèrent dans le langage hiératique des nations civilisées de l'antiquité n'est pas seulement utile pour parvenir à la connaissance du sens intime de tel ou tel mot employé dans un texte religieux, et de tel ou tel objet placé sur un monument figuré ; elle peut parfois servir aussi à constater un fait historique qui était resté inaperçu ; à confirmer un événement sur lequel planait quelque doute ; ou bien encore à suivre les traces des conquêtes ou des migrations d'un peuple. Si je ne me fais illusion, les observations que j'ai recueillies pendant le cours de mes recherches sur le symbole du cyprès pyramidal, justifieront pleinement ces diverses assertions.

Dès l'année 1642, le fils du fameux Meursius avait fait de cet arbre le sujet d'un chapitre du petit traité qu'il publia sous le titre d'\emph{Arboretum sacrum}\footnote{Joannis Meursii filii \emph{Arboretum sacrum}, sive \emph{De arborum, fruticam et herbarum consecratione, proprietate, usu ac qualitate libr. 3.}, Lugd. Batav. ex officin Elzevir. 1642 ; petit in-12.} ; mais ce chapitre, à peine ébauché et très-superficiel, comme tout l'ouvrage, ne pouvait satisfaire les esprits les moins exigeants.\footnote{Il n'occupe que deux pages et demie, p. 51-53.} Vers le commencement du 18\textsuperscript{e} siècle, un savant dont s'honore l'Allemagne, le docteur Frédéric-Adolphe Lampe,\footnote{Né en 1683 à Detmold ; mort à Brême en 1729.} entreprit des recherches plus étendues sur les cyprès pyramidal ; et, en 1737, huit ans après sa mort, parut à Amsterdam, dans le premier volume de ses œuvres,\footnote{\emph{Dissertation. philologico-theolog. syntagma}, t. 1., p. 574-593.} recueil devenu très-rare, un long fragment d'une dissertation philologique et théologique sur le cyprès. Incomplet sous le point de vue philologique, plus incomplet encore sous le double rapport des monuments de l'art où cet arbre est représenté, et des traditions que nous ont conservées sur le cyprès les écrivains arméniens, arabes ou persans, ce fragment néanmoins suffirait seul pour attester la riche et solide érudition de l'auteur. Après le docteur Lampe, le P. Giorgi, en 1782, dans sa Lettre au chanoine Foggini sur les inscriptions palmyréniennes du Musée capitolin,\footnote{\emph{De inscription. palmyr. quæ in Mas. Capitol. adserv. epistola}, etc. Rom., 1782, in-8°, fig.} s'occupa du cyprès à l'occasion de deux monuments figurés dont je parlerai tout à l'heure. Dominé par la supposition gratuite que la composition de ces deux monuments appartient à des mages de la secte de Zoroastre,\footnote{« \emph{... auctores agnoscunt Magos sacerdotes Solis de secta et schola Zoroastris.} » (\emph{Ibid.}, p. 38.)} il se borne à rapporter quelques-unes des traditions orientales qui concernent les cyprès plantés en divers lieux de la Perse par l'auteur du Zend-Avesta. Plus récemment, dans une belle dissertation lue, en 1834, à l'académie d'Herculanum, mais publiée à Naples en 1841 seulement,\footnote{\emph{Il mito di Ciparisso, Memor. letta all' Acad. Ercolan. in dilucidazione di an dipinto Pompejano} (34 pages, avec une planche). Cette dissertation est insérée dans le t. 3. (p. 265-296) des Mémoires de l'Académie royale d'Herculanum.} feu M. Avellino a su réunir, sur le culte du cyprès, des documents plus complets, des observations plus judicieuses que ne l'avaient fait ses devanciers. On regrette toutefois que cet habile archéologue, dont la mort prématurée\footnote{M. Avellino, né le 14 août 1788, est mort le 9 janvier 1850. Il remplissait avec éclat les fonctions de secrétaire perpétuel de l'académie d'Herculanum.} a causé une pénible sensation dans l'Europe savante, ait cru devoir s'en référer à la Lettre du P. Giorgi, quant au rôle que joua en Orient le symbole du cyprès, sans ajouter sur ce point aucune remarque qui lui soit personnelle, et qui prouve qu'il ait compris, ou même soupçonné que cet arbre était l'emblème vivant de divinités qui ne se confondent ni avec Apollon, ni avec le Soleil.

En 1833, dans un mémoire sur le taureau et le lion considérés comme attributs caractéristiques de Vénus, j'ai entretenu l'Académie de quelques monuments figurés asiatiques, qui me paraissaient établir qu'en Orient le cyprès pyramidal servit à représenter d'une manière symbolique, non-seulement le Soleil et la Lune, mais Vénus elle-même. Ce mémoire, qui fait partie de mes \emph{Recherches sur le culte de Vénus},\footnote{P. 119-240.} n'a pu être imprimé qu'en 1847, et l'a été tel que je l'avais composé et communiqué à l'Académie quatorze ans auparavant. Dans l'intervalle, j'eus l'honneur de lire à la Compagnie, pendant les mois de mars, juin, août et septembre 1843, un travail particulier sur le culte du cyprès chez les peuples civilisés de l'antiquité. Une partie peu considérable de ce travail fut publiée, en 1847, dans les \emph{Annales de l'Institut archéologique}.\footnote{T. 19., p. 34-104.} Depuis cette époque, je me suis livré à de nouvelles investigations ; j'ai mis à profit plusieurs monuments figurés et quelques écrits qui successivement sont venus à ma connaissance ; ils ont fortifié mon opinion sur diverses questions de détail, et m'ont donné lieu de la rectifier sur quelques autres. En même temps, ils m'ont permis de remonter avec plus de certitude à l'origine du culte du cyprès chez plusieurs peuples civilisés de l'Asie, de l'Afrique et de l'Europe, à la signification symbolique qu'eut primitivement cet emblème, et à l'emploi particulier qu'on en fit, soit dans le culte public ou le culte secret des divinités génératrices, soit dans les cérémonies funèbres et dans la composition des monuments funéraires. C'est l'ensemble de mes recherches sur ces divers points qu'aujourd'hui je soumets au jugement de l'Académie. Mon travail embrasse l'Orient et l'Occident ; il se divise en deux mémoires : dans le premier, je considère le cyprès pyramidal comme symbole de vie et, à ce titre, emblème ou attribut des divinités génératrices ; dans le second, je le considère comme symbole funéraire et comme emblème ou attribut des divinités infernales.
\clearpage
\section{Premier Mémoire --- Du Cyprès pyramidal considéré comme symbole de Vie et comme Emblème ou Attribut des Divinités génératrices en Orient et en Occident}
\subsection{Première Section --- Orient}
\subsubsection{Babylonie, Assyrie, Syrie, Phénicie, Arménie, Commagène, et Asie Mineure}
\paragraph{}
Originaire des contrées méridionales\footnote{Le cyprès pyramidal ne croît spontanément que sous un climat chaud et dans des lieux secs (voy. Théophraste, \emph{Hist. plantar.} 4., 1. Cf. Caton, \emph{De R. R.} 48., 151.). « Alibi non nisi in tepore (cupressus) proveniens ... aquas odere cupressi, » a dit Pline (\emph{H. N.} 16., 60., 1 ; éd. Littré). On lit plus loin (17., 39., 2), que non-seulement le cyprès n'aime ni l'eau, ni le fumier, mais que l'eau le fait périr : « quinetiam necatur riguis. » Toute fois l'auteur des \emph{Geoponica} (11., 5.), d'après Didyme ou Démocrite, prétend que cet arbre se plaît surtout dans les lieux humides et couverts, ou ombragés ; il ajoute que la graine de cyprès doit se semer après les calendes de septembre.} de l'Asie antérieure, comme l'attestent la Bible,\footnote{Füller (\emph{Miscell. sacr.} 4., 5.) et Bochart (\emph{Phaleg.} 1., 4) n'ont pas hésité à traduire par cyprès le mot \texthebrew{גֹפֶר}, \emph{gopher}, qui, dans le texte hébreu de la Bible, désigne le bois dont se servit Noé, d'après l'ordre exprès du Seigneur, pour construire l'arche (\emph{Gen.} 6., 14). Les plus habiles philologues de notre siècle et du siècle dernier admettent cette interprétation, rejettent les étymologies proposées par les écrivains grecs, et reconnaissent même que les deux premières syllabes, \textgreek{κυπάρ}, du mot grec \textgreek{κυπάρισσος} ou \textgreek{κυπάριττος}, « cyprès, » représentent indubitablement l'hébreu \emph{gopher} ou \emph{kopher}. En latin, on a d'abord écrit \emph{cyparissus}, puis, par synérèse, \emph{cypressus}, puis enfin \emph{cupressus}, par suite de la conversion de l'\emph{y} en \emph{u} (Servius, \emph{ad} Virgil. \emph{Æneid.} 3., 64 ; ed. Alb. Lion).} les traditions de l'Orient, les écrivains grecs ou latins, et les observations des géographes\footnote{Sur la patrie, la culture ou la diffusion du cyprès pyramidal, il faut lire un article plein de savoir et d'intérêt, que M. le professeur Carl Ritter a placé, en 1844, dans le beau monument qu'il élève aux sciences géographiques, \emph{Die Erdkunde}, 11., 567-582. Cf. l'\emph{Essai} de M. Mirbel sur la distribution géographique des conifères, \emph{Annales du Muséum d'historié naturelle}, t. 13., p. 41, 46, 49 ; tableau, p. 68 et suiv.} et des voyageurs modernes, le bel arbre que nous appelons le \emph{cyprès pyramidal} ou le \emph{cyprès toujours vert},\footnote{\emph{Cupressus fastigiata}, de Cand. --- \emph{Cupressus sempervirens}, Linn.} dut, par sa forme et par quelques particularités qui lui sont propres, attirer de bonne heure l'attention de ces prêtres studieux et méditatifs qui, sur le sol de la Chaldée, s'efforçaient de remonter à l'essence de Dieu par l'étude des êtres, des productions et des phénomènes du monde créé. Préoccupés, dans cette belle étude, du soin de saisir les rapports qui existent entre le Créateur et son œuvre, entre les choses du ciel et celles de la terre, entre les idées métaphysiques ou philosophiques et les objets matériels ou physiques, les Chaldéens avaient fondé sur ces divers rapports un langage symbolique dont les éléments furent empruntés à l'ordre physique. Ces mêmes prêtres, qui, dans leurs conceptions abstraites, avaient eu l'idée d'attribuer à la divinité créatrice la forme d'une pyramide, d'un cône ou d'un obélisque, se trouvèrent conduits, par voie d'analogie, à choisir, parmi les végétaux qui croissaient sous leurs yeux, le cyprès pyramidal pour en faire le représentant vivant et symbolique du Dieu créateur. Cet arbre, au port tout à la fois élégant et majestueux, cet arbre qui s'élance vers le ciel comme la flamme, dont il affect la forme,\footnote{Sidereum gracili spectare cacumine cœlum.\\\hspace*{10mm}(Ovide, \emph{Metamorph.} 10., 3., v. 140.)\\\hspace*{0mm}Théocrite (\emph{Idyll.} 11., v. 45) dit : \textgreek{ῥαδιναὶ κυπάρισσοι}. On trouve \emph{aërias cyparissos} dans Martial (\emph{Epigr.} 12., 50., v. 1) ; \emph{celsa cupressus erigitur} dans Aviénus (\emph{Descript. orb. terr.} v. 1078, 1079 ; \emph{Poet. latin. minor.} t. 5. : ed. Wernsdorf). C'est par inadvertance que le Dr. Lampe (Dissertation citée) attribue à Priscien les deux vers que j'indique ici. Dans la langue arabe, le mot \emph{serv} ou \emph{serb}, qui signifie \emph{cyprès}, s'emploie avec le sens de \emph{majesté}. Jésus, fils de Sirach (\emph{Ecclesiastic.} 24., 27), compare au cyprès la divine sagesse. Les Romains donnaient le nom de \emph{cyparisses} aux feux follets, parce qu'ils affectent la forme du cyprès : « CYPARISSÆ appellantur acies quædam igneæ, quæ noctu apparere solent ad similitudinem cupressi. » (Festus, \emph{De signific. verbor.}, \emph{sub voc.}, p. 64 ; ed. K. O. Müller.) Parfois les écrivains latins comparent aussi le cyprès à une pyramide, à une borne de forme obéliscale : « ... metas imitata cupressus, » dit Ovide (\emph{Metamorph.} 10., 3., v. 106). « Femina (cupressus) sterilis diu : metæ demum aspectu non repudiata ... » (Pline, \emph{Hist. nat.} 16., 60., 1.) Cf. Servius, \emph{ubi infra.} On trouve dans Alciat (\emph{Emblematum Libellus, emblem.} 198) les deux vers que voici :\\\hspace*{10mm}Indicat effigies metæ, nomenque cupressi,\\\hspace*{15mm}Tractandos parili conditione suos.} leur parut joindre à ces avantages, l'avantage non moins précieux de rappeler, par les conditions de sa propre existence, que le Créateur du monde réunit en lui-même, sans cesser d'être un, le sexe masculin et le sexe féminin, ou, en d'autres termes, la puissance active et la puissance passive. On sait qu'en effet le cyprès, arbre de la famille des conifères, appartient à cette catégorie de végétaux que Linné range dans sa \emph{monoécie monadelphie} ; végétaux dont les fleurs ne renferment point dans un même calice les organes des deux sexes, c'est-à-dire les étamines et les pistils, mais dont les chatons à fleurs mâles et les chatons à fleurs femelles se montrent séparément sur un même pied, s'y épanouissent et s'y fécondent.\footnote{Pline (\emph{Hist. nat.} 16., 60., 1, 2 ; Cf. 13., 29., 3 ; 17., 14., 3) et Didyme ou Démocrite, cité par l'auteur des \emph{Geoponica} (11., 6.) sont, avec Servius (\emph{ad} Virg. \emph{Æneid.} 3., 64), si je ne me trompe, les trois seuls écrivains qui commettent la grave erreur de distinguer un cyprès horizontal mâle et un cyprès pyramidal femelle ; ils confondent ainsi, sous ces deux dénominations erronées, deux espèces de cyprès, le \emph{cupressus horizontalis} de Müller et le \emph{cupressus fastigiata} de de Candolle, ou deux variétés du \emph{cupressus sempervirens} de Linné. Le commentateur de Virgile, copiant inexactement le naturaliste romain, va jusqu'à prétendre que le cyprès femelle est stérile, et que le cyprès mâle porte seul des cônes : « Duo ejus sunt genera : nam quæ sterilis est, femina dicitur, ad metæ formam in fastigio convoluta ... contra mas latius spargit ramos, conosque profert nuci pineæ non absimiles, licet minores ... » Pline avait dit seulement : « femina sterilis diu » (16., 60., 1), ajoutant plus loin : « ... e cupresso femina (mas enim, ut diximus, non gignit), pilulæ collectæ ... » (17., 14., 3). C'est de même qu'on lit dans les \emph{Geoponica} (\emph{loc. cit.}) : \textgreek{Κυπάρισσος δὲ, ὁ ἄρσην ἄγονος} « le cyprès mâle est stérile. » On a cru pouvoir inférer d'un passage de Plutarque (\emph{in Phocione}, § 123), que cet auteur considérait le cyprès comme un arbre stérile. Le sens de ses paroles ne me semble pas avoir été bien compris. Nous lisons dans ce passage que Léosthène ayant fait aux Athéniens une harangue pleine d'orgueil et d'insolence, Phocion lui dit : « Jeune homme, vos discours sont semblables aux cyprès : ils sont grands, ils sont hauts, mais ils ne portent pas de fruits ; » \textgreek{Οἱ λόγοι σου, εἶπεν} (\textgreek{ὁ Φωκίων}), \textgreek{ὦ μειράκιον, ἐοίκασι κυπαρίττοις · μεγάλοι γὰρ ὄντες καὶ ὑψηλοὶ, καρποὺς οὐ φέρουσι} ; ce qui, à mon avis, n'implique pas que, chez les Grecs, le cyprès passait pour être stérile, mais simplement qu'il ne porte pas des fruits qu soient bons à quelque chose ; assertion vraie, qui avait donné lieu à ce dicton : \textgreek{κυπαρίττου καρπός}, « fruit de cyprès, » pour dire une chose inutile (Cæl. Rhodigin. \emph{Antiquar. lection.} 25., 2). Aussi Servius (\emph{loc. cit.}), continuant à parler du prétendu cyprès mâle et de ses cônes, s'exprime-t-il ainsi : « Mira inter arbores fœcunditate, quippe quæ trifera est, fructusque habet et januario et majo et septembri maturos : nulli tamen rei utiles ... »} Si l'on doutait qu'une telle disposition eût attiré l'attention des anciens, il me suffirait, pour dissiper toute incrédulité sur ce point, de faire remarquer que les peuples chez qui le palmier croissait à côté du cyprès, n'avaient pu rester étrangers à la connaissance du sexe des plantes, puisqu'au temps de la floraison, ils fécondaient avec les fleurs du palmier mâle celles du palmier femelle. De temps immémorial, les Indiens ont célébré les noces des dieux sous l'emblème de deux palmiers, l'un mâle, l'autre femelle ; ils les plantaient à côté l'un de l'autre, au sommet de quelque montagne. En plusieurs endroits de l'Hindoustan, de semblables plantations se voient encore aujourd'hui sur des collines sacrées, et sont l'objet de la vénération des indigènes. Dans l'Asie occidentale, lorsque les Babyloniens, les Assyriens, les Syriens et les Phéniciens eurent fait de leur dieu créateur, originairement androgyne, deux divinités, l'une mâle, l'autre femelle, ce fut un palmier femelle qui représenta symboliquement Mylitta ou Astarté, leur déesse génératrice. Les monuments figurés en font foi. Les Perses, de leur côté, attribuèrent à Mithra le palmier mâle ; comme nous le verrons plus loin.

D'autres avantages naturels avaient dû concourir à faire du cyprès, dans les idées des Chaldéens d'Assyrie, un emblème expressif du Créateur du monde : telles furent sans doute la longévité de cet arbre très-vivace,\footnote{Voyez, sur la longévité du cyprès pyramidal et sur les grandes dimensions qu'atteint cet arbre, même en Europe, un article inséré, par M. Loiseleur-Deslongchamps, dans les \emph{Annales de la Société d'horticulture} (83\textsuperscript{e} livraison, p. 37-53), et intitulé : \emph{Recherches sur l'histoire des cyprès}. On a longtemps combattu une opinion que, dans ses cours de botanique à la Faculté de médecine de Montpellier, j'ai entendu feu M. Pyrame de Candolle soutenir avec autant de talent que de vrai savoir. Ce célèbre professeur établissait que les arbres ont le privilège de ne pas mourir de vieillesse, et qu'ils périssent seulement de mort accidentelle. Il a depuis, en écrivant sa \emph{Physiologie végétale} (t. 2., p. 1007), présenté ce fait comme une des lois générales de la végétation. Aujourd'hui le fait est théoriquement et pratiquement admis par la grande majorité des botanistes. On reconnaît que certains pieds d'arbres ont pu, dans des circonstances favorables, traverser bien des siècles, comme les plus vieux monuments d'architecture que nous ait légués l'antiquité. Aussi mon savant ami M. Flourens, dans son \emph{Éloge historique d'Aubert du Petit-Thouars} (p. 16), prononcé en 1845, n'a-t-il pas hésité à dire que « le boabab d'Adanson date peut-être de la dernière catastrophe du globe. » « Par sa nature, dit quelque part M. Carl Ritter, le palmier est immortel ; le tronc mort renaît sur sa racine comme un phénix rajeuni ; et celle immortalité lui a valu de bonne heure d'être appelé de ce nom même. »} sa fécondité, la forme tétragonale de ses rameaux,\footnote{Le nombre quatre et la forme carrée ou cubique faisaient allusion aux quatre éléments, et furent consacrés à Vénus et à Mithra (\emph{Zend-Avesta}, t. 2., p. 228 et ailleurs), comme à toutes les divinités génératrices. Selon Plutarque (\emph{De Isid. et Osirid.} § 30, p. 52 ; ed. Parthey), les pythagoriciens exprimaient par le \emph{tétragone} ou le \emph{carré} la puissance de Rhéa, d'Aphrodite, de Déméter, de Vesta et de Héra.} la persistance de son feuillage\footnote{Pline, \emph{Hist. nat.} 16., 33., 1. Plutarque, expliquant ce phénomène suivant les idées de son temps, l'attribue à la résine (littéralement \emph{à la graisse}) et à la chaleur que renferme le cyprès, aussi bien que l'olivier, le laurier et même le lierre : \textgreek{Ἐλαίαν δὲ καὶ δάφνην καὶ κυπάριττον ἀειθαλῆ διαθυλάσσει τὸ λιπαρὸν καὶ τὸ θερμὸν, ὥσπερ ὁ κιττός} (\emph{Sympos.} 3., quæst. 2, ed. Reiske).} toujours vert, toujours vivant, la nature réputée incorruptible de son bois\footnote{Selon Théophraste (\emph{Hist. plantar.} 5., 5.), de tous les bois incorruptibles, le cyprès est celui qui dure le plus. Selon Pline (\emph{H. N.} 16., 78., 1 ; 79., 1-3 ; 80., 2), le bois de cyprès est éternel ; la vétusté, la carie, les vers ou les teignes ne l'attaquent jamais. C'est pourquoi Platon voulait que toutes les lois fussent inscrites sur des tablettes de cyprès (cf. Rhodiginus, \emph{loc. cit.}), comme, assure-t-on, les lois de Solon l'avaient été. Pline cite une statue de Véjovis, de bois de cyprès qui, depuis l'an 661 de la fondation de Rome, s'était conservée intacte jusqu'au temps où il écrivait. Il cite aussi les portes du temple de Diane à Éphèse, qui dataient de près de quatre siècles. Il prétend même (18., 45., 2) qu'il suffit de mêler des feuilles de cyprès avec des semences quelconques pour empêcher celles-ci d'être attaquées par les vers. Il ajoute, d'accord sur ce point avec Théophraste (\emph{loc. cit.}), que le bois de cyprès poli ne perd jamais son éclat (cf. Gregor. Nyssen. \emph{Homil. 4 in Cantic. Canticor.}). Ces divers faits ne l'empêchent pas de tomber dans une erreur et une contradiction manifestes lorsqu'ailleurs il dit que la contexture de ce bois est peu serrée, « materie rara » (60., 1.). De son côté, Virgile (\emph{Georgic.} 2., 442, 443) recommande le cèdre et le cyprès pour la construction des édifices. Martial (\emph{Epigr.} 6., 73.) va jusqu'à dire :\\\hspace*{10mm}... perpetua numquam moritura cupresso.\\\hspace*{0mm}(Cf. \emph{Epigram.} 6., 49. Théocrite, \emph{Idyll.} 1. Horace, \emph{Ars poet.} 330-332. Vitruve, 2., 9. Palladio, \emph{Nov. tit.} 15. Junius, \emph{De pictur. veter.} 3., 11. Füller, \emph{loc. cit.} Isidore de Séville, \emph{ubi infra.} Bochart, \emph{loc. cit.}) --- J'ai eu l'occasion de voir un fragment d'un coffret de cyprès trouvé dans un ancien \emph{tumulus}, près des ruines de Panticapée ; il ne portait aucune trace d'altération. Dans les temps modernes, on a pu quelquefois constater la propriété qu'a le bois de cyprès, comme celui de cèdre, de se conserver plusieurs siècles. Lilius Gyraldus (Giraldi) en cite un exemple remarquable dans sa dissertation \emph{De navigiis} (cap. 7.). Vers le milieu du 15\textsuperscript{e} siècle, les portes de Saint-Pierre de Rome n'étaient nullement vermoulues quand le pape Eugène 4 y fit substituer des portes d'airain. Plus récemment, des planches de cyprès qui étaient entrées dans la construction d'un navire appelé \emph{le Trajan}, furent trouvées intactes lorsque, après être resté treize siècles submergé, ce navire fut retiré de l'eau (Léon Alberti, \emph{De re ædificatoria}, 5., 12).} ; la bonne odeur qu'il exhale\footnote{... \textgreek{καὶ εὐώδης κυπάρισσος}. (Homer. \emph{Odyss.} E, 64.) Ces mêmes mots sont répétés dans les \emph{Homerici centones de Christo}, 1., 28 (\emph{Poetæ gr. christ.} Lutet. Parisior. 1609, petit in-8°). Varron (\emph{apud} Servium \emph{ad} Virgil. \emph{Æneid.} 6., 216) mentionne aussi la bonne et forte odeur qu'exhale le cyprès. Servius (\emph{ibid.} 3., 64) la qualifie de « vehemens ; » et Pline (\emph{Hist. nat.} 16., 60., 1) emploie l'expression « odore violenta. » Isidore de Séville (\emph{Origines}, 17., 7., 34), voulant caractériser la bonne odeur du cyprès, dit : « Jucunditas odoris sui. »} ; les substances inflammables qu'il produit, comme tous les arbres résineux ; la croyance où l'on était que les objets placés dans des coffrets de bois de cyprès ou enveloppés de feuilles de cyprès,\footnote{Voy. \emph{Geoponica}, 2., 18., 4. Columelle, 2., 9., 9.} pouvaient se conserver indéfiniment\footnote{De là l'usage d'ensevelir les morts dans des cercueils de bois de cyprès. De là aussi ces locutions, en parlant d'écrits ou d'objets qui ne méritaient pas d'être conservés : « non coacervanda ligna cupressina, » ou « ligna cupressina ne coacervato. » Horace (\emph{Epist. ad Pisones}, 331, 332, ed. Dœring) a dit :\\\hspace*{15mm}... ... carmina fingi\\\hspace*{10mm}Posse linenda cedro, et levi servanda cupresso ?} ; la vigueur de constitution de cet arbre, qui lui permet de braver à la fois l'intensité de la chaleur, les frimats et les neiges\footnote{Voy. Théophraste (\emph{Hist. plantar.} 4., 1). Les anciens croyaient que le cyprès se couvre trois fois, chaque année, de fruits qui mûrissent aux mois de janvier, de mai et de septembre ; de là l'expression « trifera cupressus » (Pline, \emph{Hist. nat.} 16., 50., 2 ; Servius, \emph{loc. cit.}). Les modernes se bornent à remarquer, quant à la vigueur réelle de cet arbre, que, transplanté de sa patrie en Europe, il ne résiste pas à des froids rigoureux ; mais qu'il ne dégénère nullement, de quelque manière qu'on l'ait semé. Cette dernière observation n'avait point échappé à Pline (\emph{Hist. nat.} 17., 11., 1).} ; la forme enfin de ses fruits, quelquefois assimilée à celle de deux organes qui sont une partie essentielle de l'appareil génital de l'homme.\footnote{Théophraste (\emph{De Caus.} 1., 5.) caractérise le fruit du cyprès en disant : \textgreek{καρπὸν σφαιροειδῆ}. Varron (\emph{De R. R.} 40.) emploie l'expression \emph{galbulus}. Je ne vois nulle part que les anciens se soient arrêtés à remarquer la forme pentagonale des scutelles qui couvrent la surface de la pomme du cyprès. C'est à tort que Pline (\emph{Hist. natur.} 16., 60., 1) désigne ce fruit sous la dénomination de \emph{baies} (\emph{baccæ}). Il dit mieux lorsqu'il emploie (\emph{ibid.} 17., 14., 3) celle de \emph{petites boules} ou \emph{globules} (\emph{pilulæ}). Autrefois, en Europe, le fruit du cyprès figurait dans les pharmacopées et les pharmacies sous la dénomination impropre de \emph{noix de cyprès}.}

Le silence des auteurs anciens sur la plupart des motifs qui présidèrent en Orient au choix du cyprès pyramidal, comme emblème ou comme attribut des divinités créatrices, semblerait, au premier abord, fournir une objection contre la justesse de quelques-unes des considérations par lesquelles je cherche à expliquer un tel choix. Mais, quel que soit le degré de confiance qu'on leur accordera, rien ne peut infirmer les nombreux témoignages qui nous restent de la consécration du cyprès aux dieux générateurs, dès une haute antiquité. Nous allons voir les productions de l'art en Orient, comme en Occident, parfaitement d'accord avec les textes pour mettre ce fait hors de toute contestation.

Les premières antiquités figurées dont je place les dessins sous les yeux de l'Académie, nous transportent en Syrie, dans cette province qui, de même que la Phénicie, rendait un culte solennel à la Vénus des Babyloniens et des Assyriens. Ce culte se naturalisa si complétement chez les Syriens, que cette divinité fut bientôt désignée sous le nom de \emph{Déesse de Syrie}. Aucun temple n'a été plus célèbre que celui qu'elle avait à Hiérapolis. Visité, chaque année, par une multitude de pèlerins qui, de divers points de l'Asie occidentale, de l'Inde, de l'Arabie, de l'Éthiopie, de l'Égypte, y apportaient des offrandes de toute espèce,\footnote{\emph{De Dea syria}, §§ 10, 32.} il renfermait, avec des richesses immenses, deux statues d'or, dont l'une représentait la déesse assise sur des lions et chargée des attributs de Junon, de Minerve, de Vénus Aphrodite, du Soleil, de la Lune, de Rhéa, de Diane, de Némésis et des Parques.\footnote{\emph{Ibid.} §§ 31, 32.} L'autre statue représentait, assis sur des taureaux, Baal ou Jupiter,\footnote{L'auteur du traité \emph{De Dea syria} (§ 31) désigne ce dieu sous le nom de Zeus.} l'époux de la déesse. Dans le même temple, on avait placé un trône du Soleil, sans statue,\footnote{\emph{Ibid.} § 34.} une image en pied d'un Apollon barbu,\footnote{\emph{Ibid.} § 35. --- Je n'hésite pas à rapporter au dieu que l'auteur désigne ici sous le nom d'Apollon barbu, un buste barbu et radié, qui se voit sur un autel de marbre, à quatre faces, dont les dessins m'ont été communiqués par notre obligeant et savant confrère M. Raoul-Rochette. Cet autel, où l'alliance des idées grecques avec les idées asiatiques est évidente, fut donné, il y a peu d'années, par les religieux du Liban, à M. le capitaine de vaisseau de Suin. Il est succinctement décrit dans le mémoire de M. Raoul-Rochette sur l'Hercule assyrien (\emph{Mém. de l'Acad. des inscript.} t. 17., 2\textsuperscript{e} partie, p. 98, 99), et sera prochainement publié et commenté par le docte académicien. Entre autres faits neufs et curieux, il offre, sur trois de ses faces, outre le buste du dieu-soleil barbu, deux images en pied de la \emph{Déesse de Syrie}, images dont une se termine en gaine et semble identifier cette déesse avec la Diane d'Éphèse. La quatrième face est occupée par un buste d'Hercule représenté avec un costume qui, non moins que l'inscription grecque gravée sur une autre des quatre faces, décèle le ciseau d'un sculpteur grec, mais d'un sculpteur de l'époque de la décadence. La tête d'Hercule est coiffée avec le mufle d'un lion dont la peau tombe sur les épaules du buste et vient s'attacher sur le devant de la poitrine, au moyen des deux pattes antérieures de l'animal, qui se nouent l'une avec l'autre. Au-dessus de l'épaule gauche s'élève le gros bout de la massue d'Hercule.} et plusieurs autres simulacres qui ne se rattachent pas directement à l'objet de ce mémoire. Aucun des trois monuments syriens dont j'ai à parler, un autel à quatre faces, un bas-relief, et un trône votif de bronze, ne nous offre la répétition des deux statues principales du temple de Hiérapolis ; mais ils vont nous faire connaître qu'en Syrie le culte de Vénus, adorée sous le symbole du cyprès pyramidal, s'associait tantôt au culte d'une triade suprême et à celui de l'Amour, tantôt seulement au culte du Soleil et de la Lune. Bien que l'autel et le bas-relief eussent été apportés à Rome dès une époque fort ancienne,\footnote{Ce fait remontait à une date si reculée, que Bottari (\emph{Mus. capitolin.} t. 4., p. 77-86), décrivant le bas-relief, se demande ce qui empêche de conjecturer qu'il fut apporté de Palmyre à Rome par l'empereur Aurélien, sectateur zélé, comme on le sait, du culte du Soleil.} et publiés même longtemps avant la mort du docteur Lampe, ils lui étaient restés inconnus. D'autres habiles archéologues en ont fait, avant et après lui, le sujet de savantes dissertations. Toutefois, il faut bien le dire, les interprétations qu'ils ont proposées ne sont ni satisfaisantes, ni complètes, et les dessins qui accompagnent leurs ouvrages manquent d'exactitude. Grâce à l'obligeance de MM. Édouard Gerhard, Émile Braun et G. Henzen, ceux que je joins ici\footnote{Pl. 1., n°\textsuperscript{s} 1 et 2 ; pl. 2., n°\textsuperscript{s} 1 et 2 ; pl. 3., n° 1.} offrent toutes les garanties désirables. Ils ont été exécutés en 1846, à Rome, sous les yeux de ces savants\footnote{Les inscriptions ont été estampées avec le plus grand soin, puis réduites par le procédé de la photographie pour être calquées sur le cuivre de mes planches. Ainsi disparaîtront quelques erreurs qui s'étaient glissées dans la reproduction de ces inscriptions, lorsqu'en 1847 je publiai une première fois les dessins de l'autel et du bas-relief palmyréniens du Musée capitolin (\emph{Monum. inéd. de l'Instit. arch.} t. 4., pl. 38., n°\textsuperscript{s} 11, 11 a, 11 b, 11 c, n° 6).} ; les deux monuments qu'ils reproduisent sont de marbre, proviennent probablement de Palmyre ou de la Palmyrène, et se conservent au Musée capitolin.\footnote{Avant d'être déposé dans ce riche musée, l'autel à quatre faces avait été placé à la villa Bosia, puis dans les jardins du palais Mattéi.}

Sur la face antérieure de l'autel est gravée une inscription latine que Pighius,\footnote{Manuscrits cités par Gruter, \emph{ubi infra}.} le premier, copia pendant son séjour dans cette ville. Vers 1602, elle fut publiée dans la première édition du recueil de Gruter,\footnote{\emph{Inscript antiq.} p. 36., n° 1.} d'après une copie plus complète que celle de l'antiquaire hollandais, restée manuscrite. En 1683, dans ses Recherches curieuses d'antiquité,\footnote{Planche à la page 69.} et, en 1685, dans ses Miscellanées,\footnote{\emph{Miscell. erud. antiq.} planche à la p. 3.} Spon fit graver un dessin peu fidèle du monument vu sous chacune de ses quatre faces. Ce dessin fut reproduit, sans aucune correction, dans le deuxième volume de l'Antiquité expliquée du P. Montfaucon.\footnote{Pl. 179., n° 5.} De 1603 à 1754, plusieurs autres savants\footnote{On en trouvera la liste plus loin, p. 40, note 2.} ont donné des copies plus ou moins exactes de l'inscription bilingue de notre autel et de celle que porte le bas-relief dont je parlerai plus loin ; mais ils n'y ont pas joint les dessins des figures qui ornent ces deux monuments. En 1782, le P. Giorgi, dans la lettre citée plus haut, publia ses observations sur l'autel et le bas-relief palmyréniens, en y insérant les remarques philologiques qu'Adler lui avait communiquées. L'année suivante, 1783, cette lettre fut réimprimée à la fin du quatrième volume de la Description du Musée capitolin.\footnote{P. 371-416.} Dans ce dernier ouvrage, publié avec luxe, on est étonné de trouver l'autel à quatre faces gravé dans une simple vignette,\footnote{\emph{Mus. capitolin.} t. 4., p. 77.} qui reproduit le dessin en miniature et fort inexact, que donne Giorgi\footnote{Lettre citée, planche pour la p. 107.} d'un si précieux monument. Ce dessin, on en aura la preuve plus loin, a induit en erreur, sur un point important, deux archéologues d'un grand mérite, Böttiger\footnote{\emph{Ideen zur Kunstmythol.} p. 239.} et Avellino.\footnote{Mémoire cité, p. 26.} Enfin, M. l'abbé Lanci, après avoir publié, en 1825, un \emph{fac-simile} et une traduction de l'inscription palmyrénienne de notre autel, à la suite de sa dissertation sur la stèle ægypto-phénicienne de Carpentras,\footnote{\emph{Di un egizio monum. con inscriz. fenicia, etc. Spiegazione delle due epigrafi palmir. del Mus. capitol.} (Roma, 1825, petit in-folio, avec deux \emph{fac-simile}), p. 143-152. --- Bianchini et Montfaucon, d'après lui, nous avaient déjà donné un \emph{fac-simile} assez satisfaisant de chacune de ces deux inscriptions.} a inséré, vingt ans plus tard, dans le premier volume de ses \emph{Paralipomenii},\footnote{\emph{Paralipomeni alla illustrazione della sacra Scrittura per monument fenico-assirii ed egiziani} (Paris, 1845, 2 vol. grand in-4°), p. 29-37 ; pl. 2., n°\textsuperscript{s} 1-4.} un commentaire accompagné d'un dessin qui laisse encore beaucoup à désirer, et qui ne reproduit ni l'inscription latine, ni l'inscription en caractères palmyréniens.

Au-dessus de la dédicace latine, qui se compose de six lignes, on voit un buste radié,\footnote{Les rayons solaires dont la tête est ornée sont au nombre de sept, et font allusion tout à la fois, je le présume, aux sept planètes et aux sept conseillers de Baal, types primitifs des sept \emph{amschaspands} d'Ormuzd.} supporté par un aigle. Cet oiseau, attribut caractéristique du dieu solaire asiatique,\footnote{Chez les Perses, l'aigle était le symbole de Mithra, dieu solaire et manifestation d'Ormuzd, le Jupiter du Zend-Avesta.} que les Grecs et les Romains assimilent à Zeus ou Jupiter, dont l'aigle est de même le symbole,\footnote{Remarquons que, sur plusieurs monuments romains, Jupiter, comme ici Baal ou Bel, est représenté par un buste humain que supporte un aigle aux ailes éployées.} cet oiseau, dis-je, nous avertit que nous avons ici sous les yeux l'image de Baal ou Bel, qualifié \emph{Sol sanctissimus}\footnote{Les rapports que la théologie asiatique avait établis entre l'aigle et le soleil ne sont pas moins nettement indiqués dans une croyance populaire fort répandue chez les nations civilisées de l'antiquité : on affirmait que l'aigle est le seul oiseau qui puisse s'élever jusqu'au soleil et soutenir en face l'éclat de ses feux et de sa lumière. L'empereur Julien, au début d'une lettre (\emph{Epistol.} 16.) qu'il écrivait au philosophe Maxime, ou, selon d'autres, à Libanius, paraphrase cette tradition avec le tour d'esprit qu'on lui connaît : « La fable, dit-il, rapporte que l'aigle, pour éprouver si ses aiglons sont légitimes, les enlève tout jeunes et dépourvus de plumes jusqu'à la région éthérée Là, il les présente aux rayons du soleil, comme pour apprendre de ce dieu témoin s'il est vraiment leur père, ou s'ils sont nés d'un coupable adultère. »} dans l'inscription latine, et, à ce titre, appelé dans l'inscription palmyrénienne, comme nous allons le voir, Malackbel, c'est-à-dire, Bélus Roi. Les six lignes latines, sans nommer ce dieu, nous disent que le monument fut consacré au \emph{Soleil Très-Saint}, SOLI SANCTISSIMO, par trois personnages qui portaient des noms romains : \emph{Tiberius Claudius Felix} ; sa femme, \emph{Claudia Helpis}, et leur fils, \emph{Tiberius Claudius Alypus}.\footnote{Félix et son fils Alypus faisaient partie de la troisième cohorte des Calbiens ou Galbiens, CALBIENSES DE COH. 3000 ; ce qui doit s'entendre d'une cohorte fournie par la ville appelée \emph{Calaba} ou \emph{Galaba} et située dans l'Osrhoène, province de la Mésopotamie. Le savant Orelli (\emph{Inscript. latin. select.} t. 1., p. 344, n. 1924) rapporte les six lignes de l'inscription latine, sans autre commentaire que ces mots : « \emph{Calbienses} Grævio \emph{ad Suet. Galb. 3.} sunt milites calbeis s. galbeis donati. »} A droite, la face latérale\footnote{Pl. 2., n° 2.} nous offre l'image du Soleil ascendant, représenté en pied sous les traits d'un jeune homme imberbe, vêtu à l'orientale, et tenant un long sceptre de la main droite ; la déesse de la Victoire lui pose une couronne sur la tête au moment où il monte dans un char qu'emportent vers le ciel quatre griffons ailés, dont il a pris les rênes en main. Au-dessous de ce tableau est gravée, sur trois lignes, une inscription en caractères palmyréniens, que le P. Giorgi,\footnote{Lettre citée, p. 107.} Eichhorn\footnote{\emph{Commentat. societ. reg. scient. gotting. recent.} vol. 6., p. 98-118.} et M. l'abbé Lanci\footnote{Ouvrage cité, \emph{Spiegazione, etc.} (\emph{loc. cit.}).} n'interprètent pas dans les mêmes termes, mais où ces trois savants s'accordent à lire le nom de \emph{Malachbel}, l'un des dieux de Palmyre, et le nom de Tibérius Claudius Félix, qui lui consacra le monument. C'est ce qui résulte aussi de la traduction littérale qu'à ma prière, M. le duc de Luynes a bien voulu faire de cette inscription. La voici en latin, précédée d'une transcription en caractères hébraïques, selon la méthode qui est généralement suivie :

\begin{center}\texthebrew{
עלתא זה למלכבל ולאלהי תדמר

קרב טבריס קלודיס פלקס

ותדמריא לאלהיהן שלם
}
\end{center}
\begin{center}
1\textsuperscript{re} ligne. \emph{Aram\footnote{C'est à tort que M. Kopp (\emph{Bilder und Schrift. der Vorzeit}, t. 2., p. 133) croit trouver dans cette inscription la dédicace d'\emph{une colonne} au Soleil. Le mot \texthebrew{עלתא}, \emph{eltha} ou \emph{ealtha}, qu'il traduit par \emph{colonne}, appartient à la langue syriaque, et signifie constamment \emph{autel} dans la version syriaque de la Bible, \emph{autel} et \emph{sacrifice} (holocauste) dans le dictionnaire de Castell. Les Maronites écrivent \emph{eltho} ou \emph{eletho}, et lui donnent aussi le sens d'\emph{autel}. Enfin \emph{eltha} ou \emph{ealtha} se retrouve, avec ce dernier sens, dans une autre inscription palmyrénienne en l'honneur du Soleil (\emph{Comment. Soc. reg. scientiar. gotting. recent.} t. 6., p. 84).} hanc Malachbelo et diis Palmyræ}

2\textsuperscript{e} ligne. \emph{obtulit} T(i)B(e)RI(u)S CL(a)UDI(u)S F(e)L(i)CS

3\textsuperscript{e} ligne. \emph{et Palmyreni diis suis. Salve} (ou \emph{pax}).
\end{center}
\hspace*{0mm}C'est-à-dire : « Tibérius Claudius Félix a consacré cet autel à Malachbel et aux dieux de Palmyre, et les Palmyréniens à leurs dieux. Salut (ou paix)\footnote{La transcription de M. le duc de Luynes en caractères hébraïques et celle qu'avait publiée, en 1825, M. l'abbé Lanci (ouvrage cité, \emph{Spiegaz.} p. 151), ne présentent entre elles aucune différence ; mais ce dernier ne traduit pas tout à fait de même. Sa version italienne est ainsi conçue : \emph{Quest' Ara a Malachbelo, e agli Dei di Palmira consacrò Tiberio Claudio Felice, e i Palmireni. A' loro Dei pace.}} ! »

\paragraph{}
Cette dédicace, rapprochée de la composition du monument et de la formule latine de consécration, SOLI SANCTISSIMO, prouve, contrairement à l'opinion de Spon et de Montfaucon, que, chez les Syriens, \emph{Malachbel} était la personnification d'un dieu-soleil, et non de la lune, qui est appelée \emph{Aglibol} dans les inscriptions du bas-relief palmyrénien dont il sera bientôt question, et \emph{Selamanès}, dans trois inscriptions lapidaires\footnote{Pococke, \emph{Inscript. antiquæ}, cap. 1., pag. 2, 7, 8 ; p. 3, 9. \emph{Corp. inscr. græc.} vol. 3., nn. 4449, 4450 ; ed. Franz. Cuper, \emph{Dissert. sur div. matières de relig. et de philolog.} (la Haye 1714), t. 2., p. 47. Pococke, \emph{Inscr. antiq.} cap. 1., p. 2, 6. Egmond et Heyman, \emph{Itiner.} t. 2., p. 401. Muratori, t. 1., p. 64. ; t. 4., \emph{Append.} p. 1978, n. 7. \emph{Corp. inscr. græc. loc. cit.} n° 4451. Dans les trois inscriptions, on lit : \textgreek{ΜΑΛΑΒΑΧΩ} et \textgreek{ΜΑΔΡΑΧΟΙ}, au lieu de \textgreek{ΜΑΛΑΧΒΗΛΩ}.} trouvées en Syrie, au sommet du mont \emph{Scheikh Béréquet}, et non loin de l'emplacement de l'ancienne Béroée.\footnote{Aujourd'hui Alep.} La signification du premier de ces trois noms ne peut laisser subsister aucun doute ; car il se compose de deux mots : \emph{malca} ou \emph{malchus}, roi, et \emph{baal} ou \emph{bel}, seigneur, qualifications qui, l'une et l'autre, conviennent parfaitement et au soleil personnifié, et à Baal considéré comme un dieu-soleil. Toutefois M. l'abbé Lanci,\footnote{\emph{Paralipomeni}, t. 1., p. 31-37.} sans s'arrêter à cette considération, et un peu trop préoccupé peut-être du désir de trouver sur les quatre faces de notre autel la personnification des quatre saisons de l'année en usage à Palmyre, ne veut reconnaître sous le nom de Malachbel que le représentant de la première de ces quatre saisons, l'équinoxe du printemps. Il applique ce nom au personnage qui, monté sur le char attelé de quatre griffons, est couronné par la Victoire ; et, pour lui, le buste radié, posé sur un aigle, serait la déification du solstice d'été.

La face latérale gauche du monument qui nous occupe\footnote{Pl. 2., n° 1.} porte un buste voilé, que Spon n'a pas cherché à qualifier, et qui serait, suivant Montfaucon, le portrait de Tibérius Claudius Félix ; selon Eichhorn, mais avec quelque doute, le portrait d'un ministre du culte de Malachbel. Ces trois antiquaires prennent pour une faucille l'instrument placé sur l'épaule droite du buste voilé. Spon penche même à croire que cette faucille, emblème de la saison des moissons, n'est pas sans corrélation avec le char du Soleil, qui est figuré sur l'autre face. Le P. Giorgi ne s'explique pas sur ces deux derniers points, et veut que le buste dont il s'agit représente Ormuzd, dieu des mages. M. l'abbé Lanci\footnote{\emph{Ubi supra.}} le rapporte à l'automne divinisé. Ce buste n'est ni un portrait, ni le dieu des Perses appelé Ormuzd, ni la déification de l'automne, ni Baal, mais bien le dieu suprême des Syriens et des Palmyréniens, que nous trouvons, dans la triade divine des Assyriens et des Perses, sous le nom de \emph{Temps-sans-bornes}, c'est-à-dire l'Éternel. Toutefois il faut reconnaître ici l'influence de l'art grec\footnote{Il faut reconnaître aussi, dans la composition de notre monument, l'influence des idées grecques ; car, de même que, chez les Grecs, Cronos ou Saturne finit par s'effacer pour laisser à Zeus ou Jupiter tous les honneurs divins et le rôle d'un dieu omnipotent, de même nous voyons, sur notre autel palmyrénien, le Temps-sans-bornes ou Saturne relégué au second rang ; c'est le buste de Baal qui occupe la place d'honneur, c'est-à-dire la face antérieure de cet autel ; et c'est à lui, qualifié \emph{Soleil très-saint} (\emph{Sol sanctissimus}), qu'est dédié le monument dans l'inscription gravée au-dessous de son image (pl. 1., n° 1).} ; car, tandis que les Assyriens, les Perses et même les Phéniciens représentent habituellement ce dieu suprême sous le symbole du cercle ou de la couronne,\footnote{Voy. \emph{Nouv. Journ. asiat.} août 1835, t. 16., p. 174, 175 ; \emph{Mém. de l'acad. des inscr.} nouvelle série, t. 17., 1\textsuperscript{re} partie, p. 352-354.} emblèmes d'éternité, nous le voyons, sur l'autel dont je parle, figuré sous une forme humaine, et caractérisé par un voile et une \emph{harpé}\footnote{Je me réserve de montrer ailleurs que la \emph{harpé}, instrument d'origine asiatique, fréquemment placé sur les monuments romains des tauroboles et du culte de Cybèle et d'Atys, a pour type primitif l'arme tranchante qui, dans la légende de Mithra, empruntée aux Chaldéens d'Assyrie, est appelée l'\emph{oreille d'acier} ou plutôt l'\emph{oreille de cuivre rouge.} (Voy. \emph{Zend-Avesta}, t. 2., p. 229 et n. 5. Cf. \emph{ibid.} t. 1., 2\textsuperscript{e} partie, p. 347, 548, n. 3.)} ; ce qui achève de l'assimiler au Saturne (\emph{Cronos}) des Grecs.\footnote{Dans la théologie chaldéenne ou assyrienne, Bélus, comme Ormuzd dans le Zend-Avesta des Perses, est assimilé, non au \emph{Temps-sans-bornes} ou à l'Éternité, mais au \emph{Temps limité}, c'est-à-dire au temps qui doit marquer la durée du monde créé. Cette distinction ayant échappé à l'attention des écrivains occidentaux, et Bélus, en Orient, étant qualifié \emph{Temps} (\emph{Chronos}), quelques auteurs grecs ou latins ont commis la grave erreur de croire que les peuples de l'Asie antérieure confondaient Bélus avec Saturne (\emph{Cronos}). Je puis citer nommément Théophile, évêque d'Antioche, dont le passage mérite d'être textuellement rapporté : ... \textgreek{ἔνιοι μὲν, dit Théophile, σέβονται τὸν Κρόνον, καὶ τοῦτον αὐτὸν ὀνομάζουσι Βὴλ, καὶ Βὰλ, μάλιστα οἱ οἰκοῦντες τὰ ἀνατολικὰ κλίματα, μὴ γινώσκοντες μήτε τίς ἐστιν ὁ Κρόνος, μήτε τίς ἐστιν ὁ Βῆλος · παρὰ δὲ Ῥωμαίοις Σατοῦρνος ὀνομάζεται · οὐδὲ γὰρ αὐτοὶ γινώσκουσι τίς ἐστιν αὐτῶν πρότερον, ὁ Κρόνος ἢ ὁ Βήλ.} \emph{Ad Autolycum}, 3., 29., p. 399 C ; Parisiis, 1742 ; edit. Maran. Voyez aussi Servius (\emph{ad} Virgil. \emph{Æneid.} 1., 733), Damascius (\emph{Vita Isidor. apud} Phot. \emph{Biblioth.} cod. 242, p. 343, ed. Bekker) ; et l'auteur anonyme de l'Exorde du 2\textsuperscript{e} livre de la Chronique d'Eusèbe (\emph{Chronic.} 2., p. 239, edd. Maio et Zohrabo ; p. 14, ed. Aucher).}

Enfin, sur la face postérieure de notre autel,\footnote{Pl. 1., n° 2.} s'élève un grand cyprès pyramidal, chargé de ses fruits globuleux, et non un pin, comme le disent Gruter, Spon et Montfaucon, ou un laurier, comme le veut Eichhorn.\footnote{\emph{Comment. societ. reg. scient. gotting. recent.} t. 6., \emph{class. historic. et philolog.} p. 118.} Vers le sommet de l'arbre est attachée une couronne, emblème d'éternité et de déité, que l'on pourrait aussi prendre pour le ceste de Vénus, si elle était placée moins haut. Des flancs de ce cyprès, image vivante de Vénus, naît l'Amour, qui, sous la forme d'un jeune enfant non ailé, porte sur ses épaules un bélier, symbole tout à la fois de sa mère\footnote{A Amathonte, dans l'île de Cypre, et à Corinthe, on offrait à Vénus un bélier chargé de toute sa toison. (Voy. Jean Lydus, \emph{De mensib.} 4., 45.)} et de son père, Baal ou Bel, époux et frère de la déesse. Les Grecs et les Latins attribuèrent ce symbole à Hermès ou Mercure, époux d'Aphrodite ou Vénus et père de l'Amour, assimilant ainsi ce dieu à Baal, seconde divinité des Assyriens, des Syriens et des Palmyréniens. L'antiquité grecque et l'antiquité romaine ont-elles jamais réussi à représenter, par une composition aussi poétique, aussi ingénieuse, aussi discrète, les amours d'Hermès et d'Aphrodite, de Mercure et de Vénus, et la naissance d'Éros ou de Cupidon ? Je suis tenté d'en douter, dût-on m'accuser de partialité pour le génie symbolique de l'Orient.

Ce génie symbolique, si je ne me fais illusion, se manifeste surtout dans le choix de l'arbre qui est, pour ainsi dire, le mot de l'énigme du tableau placé sous nos yeux. Il n'était pas indifférent, comme on pourrait le croire, de faire naître l'Amour des flancs d'un arbre de telle ou telle espèce. Ne fallait-il pas, puisqu'il s'agissait de représenter sa mère, choisir un arbre qui pût rappeler que primitivement Vénus réunissait en elle les deux sexes ? Or le cyprès n'est-il pas \emph{androgyne} ou \emph{hermaphrodite}, s'il m'est permis de m'expliquer ainsi ? c'est-à-dire, ne porte-t-il pas, sur un même pied, des chatons mâles et des chatons femelles ? Il était donc un emblème éminemment propre à représenter tout à la fois un dieu et une déesse, et à exprimer l'idée de l'union conjugale à laquelle Éros ou l'Amour devait sa naissance. Ne suit-il pas de là que, sur l'autel du Musée capitolin, le cyprès peut être considéré comme le symbole et de la Vénus asiatique et de son époux divin ?

L'attribution de la puissance créatrice à un arbre sacré n'est point ici un fait unique et isolé. Une tradition persane,\footnote{\emph{Zend-Avesta} (\emph{Boun-déhesch}, 15.), t. 2., p. 376 et 377.} empruntée à une source qui nous est restée inconnue, fait naître d'un arbre \emph{Meschia}\footnote{Il est, sans doute, inutile de faire remarquer l'identité de ce nom avec le \emph{Mensch} des langues germaniques.} et \emph{Meschiané}, le premier homme et la première femme, confirmant ainsi la double signification de vie et d'arbre que la langue zende attribue au mot \emph{orouéré}, l'\emph{arvor} ou \emph{arbor} des Latins. Le grand bas-relief à deux faces découvert, en 1832, dans un mithræum, près de Heddernheim, nous montre que la croyance des sectateurs de Zoroastre avait passé, avec le culte de Mithra, chez les Grecs asiatiques, qui, après avoir reçu des mains des Perses les types des monuments consacrés à ce culte, les transmirent, plus ou moins modifiés, aux légions que Pompée et les empereurs romains ou leurs généraux conduisirent dans l'Asie Mineure. Nous voyons, en effet, sur la face antérieure du bas-relief de Heddernheim,\footnote{\emph{Annal. des Vereins für nassauische Alterthumsk.} t. 1., pl. 1. Cette planche est reproduite, sous le n° 90., dans mes \emph{Recherches sur le culte de Mithra}.} la naissance de Meschia, ou la première phase de la vie humaine, représentée par un enfant qui naît de l'arbre appelé \emph{reivas} dans le Boun-déhesch.\footnote{\emph{Zend-Avesta}, t. 2., p. 376.} A la droite de cet arbre sont plantés trois cyprès ; entre le premier et le second, un jeune homme, qui porte sur ses épaules un taureau renversé, la tête en bas, nous offre l'image de la seconde phase de la vie humaine. Plus loin, entre le second et le troisième cyprès, on remarque un groupe de deux figures, où, dans une autre occasion,\footnote{\emph{Nouv. annal. de l'Instit. archéol.} t. 1., p. 474, 475. \emph{Mém. de l'Acad. roy. des inscr.} t. 14., 2\textsuperscript{e} partie, p. 84, 85.} nous avons reconnu Mithra posant une couronne sur la tête du même personnage parvenu à la troisième phase. La tradition qu'attestent tout à la fois ce précieux monument et les livres sacrés des Parses, avait probablement été recueillie chez les Chaldéens ou les Assyriens ; aussi s'était-elle répandue parmi d'autres peuples de l'Asie. Les Annales des Ouigours m'en fournissent la preuve. « A \emph{Coumlandjou}, lieu situé au confluent des rivières \emph{Tougola} et \emph{Sélinga},\footnote{Dans le Syllabaire mandchou-chinois, intitulé : \emph{Thsing-han-touï-in-tseu-che}, on trouve, en mandchou, \emph{Toula} et \emph{Selengge}, et, en chinois, \emph{Thou-la} et \emph{Se-ling-'ge}.} qui prennent leur source dans les monts \emph{Caracorum},\footnote{Sous cette forme, il faut reconnaître la chaîne appelée par Raschid-el-din \emph{Kara-Korum}. Ce nom désigne, en général, toutes les montagnes situées dans le voisinage de deux rivières dont le confluent est à Koumlandjou, et d'une autre rivière nommée \emph{Orkhon}.} il y avait, disent ces Annales,\footnote{Ce sont les Annales mêmes des Ouigours que Raschid-el-din et le vizir Ala-el-din ont eues entre les mains. Le dernier, dans son \emph{Histoire du conquérant du monde} (Djenghis-Khan), en a donné un extrait, qui, traduit en français, a été inséré par M. d'Ohsson, ministre de Suède à la Haye, dans son \emph{Histoire des Mongols} (Paris, 1824, in-8°). C'est à cette traduction que j'emprunte le passage qui est transcrit ici. Les faits qu'il renferme sont aussi rapportés, mais avec quelques variantes, dans l'Histoire de la dynastie mongole, écrite en chinois par \emph{Tchao-yuan-phing} et intitulée : \emph{Su-houng-kian-lou} (section 29., fol. 14 \emph{verso}). Le pays où a lieu le prodige y est nommé \emph{Khorin} ; il n'est question que d'un seul arbre, dont la rotondité est comparée à celle d'une femme enceinte. Au bout de neuf mois et dix jours, l'arbre accouche de cinq garçons ; le cadet reçoit le nom de \emph{Bouka-khan}, et devient roi des Ouigours, après avoir soumis les peuplades voisines. Feu M. Jules Klaproth, à qui l'on doit une traduction française de ce morceau (\emph{Journ. asiat.} 1824, t. 5., p. 268, 269), remarque (\emph{ibid.} p. 272) que si, avec toute raison, l'historien chinois déclare fabuleux le fait qu'il rapporte, on ne doit pas moins conclure de son récit, que les principales circonstances en avaient été empruntées à une même source par lui et par les historiens persans.} deux arbres voisins, l'un appelé \emph{fistouc}, semblable au pin, dont le feuillage ressemble à celui d'un \emph{cyprès}, et dont le fruit a la forme et la saveur de la pomme de pin ; l'autre était un \emph{bouleau}. Les deux arbres s'enflèrent ; ils étaient éclairés par une lumière céleste ; continuant à croître, ils formèrent comme une montagne, d'où l'on entendit sortir des sons harmonieux. Toutes les nuits, elle était entourée d'une vive lumière. Lorsqu'elle fut parvenue au terme de sa grossesse, une ouverture laissa voir cinq chambres, semblables à des tentes et entourées d'un fil d'argent ; dans chacune était assis un enfant allaité au moyen d'un tube suspendu à sa bouche. Les chefs des tribus, frappés d'étonnement, venaient admirer et adorer ces prodiges ... Les cinq enfants étaient traités par les peuples de cette contrée avec le respect qu'on rend aux fils des rois ... » Il n'est pas moins curieux de trouver dans les livres religieux des Thibétains une légende où l'on voit la reine, qui porte dans son sein le \emph{Bôdhisattva}, s'avancer, dans le jardin de Loumbini, jusqu'à un arbre qui s'incline et la salue ; elle saisit une de ses branches, et, au même instant, le Bôdhisattva sort par son côté droit, sans la blesser.\footnote{Voy. le \emph{Rgya Tch'er Rol Pa}, 2\textsuperscript{e} partie, chap. 7., p. 80 et suiv. de la traduction franç. de M. Foucaux (Paris, 1848, in-4°).} Dans cette légende, on le voit donc, un arbre devient l'emblème vivant d'une déesse remplissant, comme Vénus, Junon, Ilithyie, Lucine et Artémis ou Diane, les fonctions d'\emph{obstétrice}.

Des traditions analogues à celles que je viens de rapporter se reproduisent même, mais avec plus ou moins d'altération, dans certaines légendes orientales, importées en Grèce et en Italie. C'est ainsi, par exemple, que les mythographes de ces deux contrées\footnote{Apollodore, 3., 14., 4-7. Ovide, \emph{Metamorph.} 10., 490 sqq. Servius, \emph{ad} Virgil. \emph{Eclog.} 10., 18 ; \emph{ad Æneid.} 5., 72. \emph{Scriptores rerum mythicarum latini tres}, t. 1., p. 60, 85, 239 ; ed. Bode. Conf. M. J. de Witte, \emph{Nouvelles Annales de l'Instit. archéol.} t. 1., p. 539. MM. Ch. Lenormant et J. de Witte, \emph{Élite des monuments céram.} t. 2., p. 220.} font naître Adonis d'une mère métamorphosée en arbre,\footnote{Cet arbre est celui qui produit la myrrhe. Parfois cependant les écrivains de l'antiquité lui substituent le myrte.} et alternativement désignée sous le nom de Myrrha et sous celui de Smyrna. Selon les uns, elle était fille de Cinyras, qualifié roi de Cypre, prêtre d'Aphrodite à Paphos, et même fondateur de cette dernière ville. Selon Panyasis, cité par Apollodore,\footnote{3., 14., 3.} elle avait pour père Théias, roi des Assyriens. Hygin\footnote{Fab. 58., v. 242.} la dit fille de Cinyras, et donne à ce prince le même titre de roi d'Assyrie. La légende grecque qui nous parle de Myrrha punie d'un commerce incestueux par sa métamorphose en un arbre d'où naît Adonis, au bout de dix mois, avait donc été puisée à une source assyrienne ou du moins phénicienne.\footnote{Il paraît que non-seulement l'Asie occidentale est la source des traditions qui avaient répandu chez les Grecs la croyance à des personnages mythologiques nés d'un arbre et à des métamorphoses en arbre, mais qu'une corrélation plus ou moins directe existait entre ces traditions et certaines cérémonies propres à l'initiation aux mystères de la Vénus assyrienne, cérémonies où l'initié paraissait vêtu d'un costume au moyen duquel on pouvait le croire métamorphosé en arbre. Une scène de ce genre est gravée sur un cylindre du cabinet de M. le duc de Luynes, que reproduit le n° 12 de la pl. 54. B de mes \emph{Recherches sur Mithra}.} Elle inspira la composition d'un camée dont l'empreinte a été publiée à Rome, en 1839, par les soins de la direction de l'Institut archéologique.\footnote{Centur. 6., n° 34. Voy. \emph{Bulletino dell' Inst. arch.} 1839, p. 109.} Ce camée représente une nymphe\footnote{Une tradition écrite substitue ici à la nymphe obstétrice un sanglier, le même animal qui devait un jour causer la mort d'Adonis : de ses défenses il frappe l'arbre, et des flancs entr'ouverts de cet arbre sort le jeune enfant (Servius, \emph{ubi supra}).} retirant, par un pied, le jeune Adonis des flancs d'un arbre, qui est la métamorphose de sa coupable mère, et dont le tronc affecte sensiblement les formes d'une femme. À leur tour, les peintures qui ornent quelques vases grecs inédits, dont je dois la connaissance à M. de Witte, nous ont conservé le souvenir de ce mythe, puisqu'elles nous montrent Adonis placé auprès de Myrrha changée en arbre. Le même archéologue m'apprend qu'à Naples, dans la collection Torrusio, il existe une amphore de Nola, qui représente deux personnages, l'un barbu, l'autre imberbe, sortant chacun du tronc d'un arbre. D'autre part, un rite curieux, dont la mention se trouve dans Julius Firmicus, nous autorise à croire que, chez les Phrygiens, dans la célébration de la fête annuelle appelée \emph{Megalesia}, et instituée en l'honneur de la Mère des Dieux, on entendait représenter l'enfantement de la déesse et la naissance d'Atys, en coupant un pin à une certaine hauteur, et fixant au centre de cet arbre l'image d'un adolescent.\footnote{\emph{In sacris phrygiis, quæ Matris Deûm dicunt, per annos singulos} (11. calend. apr.) \emph{arbor pinea cæditur, et in media arbore simulacrum juvenis subligatur.} Jul. Firmicus Maternus, \emph{De err. profan. relig.} 28., p. 103 ; ed. Münter. Cf. Ovide, \emph{Metamorph.} 10., 104. Arnobe, \emph{Advers. Gentes}, 5., 39.} Le même écrivain nous dit aussi que, dans les fêtes consacrées à Proserpine, on coupait un arbre et on le façonnait de manière à lui donner la forme d'une vierge.\footnote{\emph{In Proserpinæ sacris cæsa arbor in effigiem virginis formamque componitur ...} (\emph{loc. cit.} p. 104). À ce sujet, il n'est peut-être pas sans intérêt de remarquer que, dans la langue zende comme dans la langue latine, le mot qui signifie arbre est du genre féminin. Répétons que ce mot, \emph{orouèré}, \emph{arbor} ou \emph{arvor}, est identique dans les deux langues, et ajoutons que les livres sacrés des Perses qualifient l'arbre pur \emph{fille lumineuse de l'eau} (\emph{Zend-Avesta}, t. 2., p. 255).} De son côté, Virgile met dans la bouche d'Évandre deux vers\footnote{Hæc nemora indigenæ Fauni, Nymphæque tenebant,\\\hspace*{5mm}Gensque virûm truncis et duro robore nata.\\\hspace*{15mm}\emph{Æneid.} 8., 314, 315.} qui prouvent que, de son temps, l'Italie conservait une tradition religieuse sur une race d'hommes nés du tronc des arbres et nommément des chênes. Enfin, dans la mythologie des peuples du Nord, l'homme primitif est aussi un arbre ; le tronc de cet arbre reverdit, fleurit, et puise sa nourriture à trois sources : la nuit, la terre et le ciel.\footnote{Mone, dans le tome 5 de la \emph{Symbolique} de M. le professeur Fic. Creuzer (2\textsuperscript{e} édit. allemande), p. 349, 350.}

Toutefois, la naissance de l'Amour portant un bélier sur ses épaules et sortant des flancs d'un cyprès chargé de fruits, est un fait nouveau dans le recueil des mythes ou des traditions hiératiques des peuples de l'antiquité ; et ce fait confère à notre autel palmyrénien le privilège qu'ont souvent les monuments figurés de remplir quelques-unes des lacunes que présentent les écrits qui nous restent des mythographes grecs et des mythographes latins. Mais, si aucun auteur ancien ne fait mention d'un \emph{Amour Criophore}, un passage de Jean Lydus, déjà cité,\footnote{Ci-dessus, p. 19, note 4.} nous apprend que, chez les Cypriens et les Corinthiens, le bélier était consacré à Vénus. D'autre part, le témoignage formel de Pausanias établit que l'attribution du bélier à Hermès ou Mercure, père de l'Amour, remonte à une haute antiquité, se lie primitivement et d'une manière intime aux institutions hiératiques de l'Asie occidentale, et ne doit être considérée, chez les Grecs, que comme une importation. Décrivant Corinthe et ses environs, cette célèbre cité où l'usage de consacrer le bélier à Aphrodite avait été importé d'Amathonte, Pausanias rapporte que, sur la route qui conduit directement à Léchée, il a vu un bélier sculpté auprès d'une statue d'Hermès assis ; et, après avoir cité deux vers de l'Iliade qui ont trait à la préférence accordée au bélier par ce dieu protecteur des troupeaux, il a soin d'ajouter qu'il doit taire ce qu'on lui a révélé du rôle assigné à Hermès et au bélier dans les initiations aux mystères de la Mère des dieux,\footnote{2., 3., 4 : ... \textgreek{τὸν δὲ ἐν τελετῇ Μητρὸς ἐπὶ Ἑρμῇ λεγόμενον καὶ τῷ κριῷ λόγον ἐπιστάμενος, οὐ λέγω}. Ce passage ne sera pas rapproché sans intérêt d'un bas-relief publié par Bartoli (\emph{L'Antr. eleusin. rappresent. in un greco antico basso rilievo del mus. Nani} ; pl. pour la page 3.). On y distingue les Grandes Déesses placées debout sur le seuil de l'antre d'Éleusis. Deux béliers et deux lions sont accroupis sur les rochers qui forment le cintre de cet antre. Mentionnons également une scène d'initiation qui nous reporte dans l'Asie occidentale, et qui est gravée sur une plaque de bronze, du musée royal de Berlin, dont il sera question plus loin. Nous y voyons (pl. 7., n° 6) un des assistants élever le bras droit et tenir dans la main un bâton court, au bout duquel est attachée une grosse tête de bélier. Dans le bas du monument on remarque, au-dessous de cette figure, un animal de même espèce, marchant à la suite d'un taureau. J'appelle aussi l'attention des archéologues sur le rôle important que remplit le bélier dans la composition des types d'un grand nombre de médailles asiatiques, en tête desquelles il faut surtout placer, après les monnaies autonomes attribuées à l'île de Cypre, les monnaies coloniales de Damascus (Cœlé-Syrie). Ces diverses pièces et l'attribution du bélier à Baal, à Hermès, à Mercure, à Éros, à Cupidon, me donnent lieu de soupçonner que cet animal symbolique s'est substitué au taureau par suite d'un emprunt fait à l'Égypte. Cette substitution établit ainsi entre le dieu Ammon, générateur, le Baal ou l'Hermès Criophore des Phéniciens et des Syriens, et l'Hermès ithyphallique des Grecs, certains rapports qui méritent d'autant plus d'être étudiés, que les monuments figurés de la Phénicie et de la Syrie offrent d'autres preuves de l'influence qu'exerça l'Égypte sur les usages religieux de ces deux contrées asiatiques.} divinité dont assurément personne n'est tenté de contester l'origine asiatique. Un autre passage du même voyageur,\footnote{4., 33., 5. Cf. 11., 22., 2.} sur lequel je reviendrai plus loin, nous apprend qu'en Messénie, dans le canton appelé de son temps le \emph{bois Carnasius}, on célébrait ces mystères, et que là, au milieu d'un bois très-épais de cyprès, étaient placées les statues d'Apollon Carnéius, d'Hermès \emph{portant un bélier}, et de Cora, fille de Déméter et surnommée \emph{la Chaste} (\textgreek{ἁγνή}). La suite de mon travail montrera, par d'autres témoignages, que si les Phéniciens et les Syriens faisaient naître d'un cyprès, emblème de sa mère, l'Amour Criophore, et attribuaient ainsi le symbole du bélier tout à la fois à ce jeune dieu et à Baal, son père, ils avaient de même consacré le cyprès pyramidal, non-seulement à Vénus et à l'Amour, mais aussi à Baal, à ce dieu que les Grecs appellent tantôt Zeus, tantôt Hermès.

D'autre part, l'autel palmyrénien du Musée capitolin n'est pas le seul monument qui nous révèle que la Syrie comptait l'Amour Criophore, de même que Baal Criophore, au nombre des dieux qu'elle adorait. Une statuette de bronze, inédite, et récemment trouvée dans cette partie même de l'Asie occidentale, justifie mon assertion. M. le duc de Luynes, à qui elle appartient, et dont l'obligeance et la libéralité égalent le savoir, veut bien me permettre de la publier ici,\footnote{Pl. 4., n° 1.} d'après un dessin exécuté sous ses yeux, et réduit au tiers des dimensions de l'original.\footnote{La hauteur est d'environ 50 centimètres.} Ce bronze représente un adolescent nu et debout, qui porte un bélier sur ses épaules, et qu'à la première vue il est impossible de ne pas déclarer identique avec l'Amour Criophore de notre autel palmyrénien. Il fut découvert en 1849, dans un \emph{sacellum} ou dans une grotte artificielle,\footnote{Tous les détails relatifs à cette intéressante découverte sont rapportés dans une lettre que M. Péretié, chancelier du consulat général de France à Beyrouth, a eu la complaisance de m'adresser le 3 juin 1850. Cette lettre est insérée dans le n° 29 (mai 1851, p. 50-52) de l'\emph{Archäologischer Anzeiger}, que publie à Berlin M. Éd. Gerhard.} que recélaient les flancs d'une assez haute montagne située à environ 20 kilomètres de Seyde, l'ancienne Sidon. Au sommet de cette montagne est bâti un petit village appelé \emph{Rimat}. En face de la statuette étaient placés deux bustes de bronze,\footnote{L'un a 30 centimètres de hauteur, l'autre 8 centimètres seulement.} qui, la tête ornée de douze rayons, représentent chacun un dieu solaire ou le Soleil personnifié. Ainsi, dans cette grotte, chose bien digne d'attention, on avait réuni, de même que sur les faces sculptées de l'autel du Musée capitolin, le culte et les images du Soleil et de l'Amour Criophore. Un rapprochement si plein d'intérêt n'indique-t-il pas que je suis fondé à désigner sous le nom d'Amour Criophore l'enfant qui, sur cet autel, naît des flancs d'un cyprès, et à donner le même nom à l'adolescent que représente la statuette de M. le duc de Luynes ? Ici le bélier,\footnote{M. Éd. Gerhard a enrichi d'un savant article sur l'attribution du bélier à plusieurs divinités la cinquième livraison (n°\textsuperscript{s} 14 et 15, février et mars 1850, p. 149-160 ; pl. 15.) de son recueil intitulé : \emph{Denkmäler, Forschungen und Berichte, als Fortsetzung der archäologischen Zeitung} ; recueil qui ne saurait être trop recommandé à l'attention et aux encouragements de tous les amis de la science. Parmi les monuments inédits que nous fait connaître l'article cité, on doit remarquer, à cause surtout de sa provenance et de son originalité, une terre cuite, coloriée, qui a été rapportée de l'île de Chypre par M. le professeur Ross, et qui, selon M. Gerhard, représente Jupiter Ammon accroupi sur un bélier debout (pl. 15., n° 3). Elle est déposée au musée royal de Berlin, et me paraît confirmer à la fois le fait de l'influence exercée par l'Égypte sur les doctrines religieuses des Phéniciens ou des Cypriens et l'attribution que, depuis peu, on fait à l'île de Cypre de plusieurs médailles classées pendant longues années parmi les \emph{incertaines de Cilicie}. J'entends parler de celles qui portent d'un côté un bélier, de l'autre une croix ansée ou une tête de bélier. La plupart des pièces de la première de ces deux catégories sont figurées sur la planche 1 du savant mémoire de notre confrère M. Raoul-Rochette, intitulé : \emph{De la croix ansée} ou \emph{d'un signe qui y ressemble} (\emph{Mém. de l'Acad. des inscript.} t. 16., 2\textsuperscript{e} partie, p. 286-382). --- Ceci était écrit lorsque j'ai pu prendre connaissance d'une récente publication due au zèle et au savoir d'un habile archéologue dont le nom est cher à tous les amis de la science, M. le duc de Luynes. Profitant des richesses de son propre médaillier, et ne négligeant pas de puiser dans les autres collections particulières et les collections publiques d'Europe, il a réuni, dans un ouvrage intitulé : \emph{Numismatique et inscriptions cypriotes} (Paris, 1852, 1 vol gr. in-4°, avec 12 planches), toutes les médailles qui appartiennent aux deux catégories dont j'entends parler. Il n'hésite pas à les attribuer à l'île de Cypre ; et, par une étude aussi persévérante qu'intelligente, il est même parvenu à en lire les légendes avec assez de certitude pour rapporter nominativement dix de ces pièces à Amathus ou Amathonte (p. 3-8 ; pl. 1., n°\textsuperscript{s} 1-8, 10 et 14), quatre à deux villes alliées, Salamine et Amathus (p. 8, pl. 1., n°\textsuperscript{s} 12 et 13 ; pl. 6., n° 10 ; pl. 12., n° 1), et une (p. 8 ; pl. 1., n° 11) à l'alliance de Salamine avec une ville qu'il croit être Lapathus, sans cependant l'affirmer. Enfin, il attribue à une ville incertaine de la confédération cyprienne (p. 7) la médaille n° 9 de sa planche 1. À l'époque où M. Raoul-Rochette publiait son mémoire sur la croix ansée, on était loin de soupçonner que la numismatique de l'île de Cypre comprît un nombre aussi considérable de médailles au type du bélier, et on désespérait de déchiffrer les légendes gravées sur celles de ces pièces qui étaient alors connues des numismates.} modelé dans des proportions beaucoup plus fortes, est caractérisé de manière à ne pouvoir être confondu avec une autre espèce de quadrupède, comme il est advenu du petit bélier de l'autel palmyrénien.

On objectera peut-être que si les deux figures dont je viens de parler représentent réellement l'Amour, elles devraient avoir des ailes attachées aux épaules. L'objection tombe à la vue de deux médailles autonomes d'argent, frappées à Nagidus, et, par conséquent, dans une province de l'Asie Mineure, la Cilicie, presque aussi renommée que la Syrie, la Phénicie et l'île de Cypre même, par le culte solennel qu'elle rendait à Vénus. Ces deux médailles, déjà publiées dans mes Recherches sur cette déesse,\footnote{P. 194-197 ; pl. 5., n°\textsuperscript{s} 7 et 8.} sont reproduites ici\footnote{Pl. 8., n°\textsuperscript{s} 8 et 9.} ; elles nous montrent ce culte associé, comme il l'était chez les Syriens et les Phéniciens, au culte du dieu que l'Orient adorait sous les noms de Baal ou Bel, d'Adad, de Sabazius, de Temps, et qui s'identifie avec Zeus ou Jupiter et avec Hermès ou Mercure. Au droit de la première des deux pièces,\footnote{\emph{Ibid.} n° 8. Cette médaille était inédite, lorsque je l'ai publiée dans mes \emph{Recherches sur Vénus}.} on reconnaît sans peine ce dieu asiatique, dont le buste radié et supporté par un aigle est sculpté sur une des faces de l'autel palmyrénien ; au revers, comme sur la face postérieure de cet autel, nous trouvons la déesse, épouse de Baal ; mais ici l'anthropomorphisme a repris ses droits. Baltis ou Beltis, appelée aussi Astarté, est représentée sous une forme humaine, et vêtue seulement de la ceinture aux pieds, de même que l'Aphrodite \emph{Génétyllis} des Grecs et la Vénus \emph{Génétrix} des Romains. Assise sur un trône, la tête ceinte d'une couronne crénelée, elle fait, avec la patère qu'elle tient de la main droite, une libation sur un autel placé à ses pieds.\footnote{Une autre médaille autonome de Nagidus, publiée par Pembroke (\emph{Numismat. antiq.} pars 2., tab. 22), offre une variante intéressante : la déesse, dans le même costume et assise sur son trône, fait une libation, non sur un autel, mais sur une fleur qui ressemble à une rose on au \emph{balaustium}. Une Victoire ailée vole dans le champ de la médaille et vient poser une couronne sur la tiare d'Astarté.} Derrière le trône, se tient debout son fils, l'Amour, complétement nu et sans ailes, comme l'Amour Criophore du cabinet de M. le duc de Luynes, comme l'Amour Criophore qui, sur la face postérieure de l'autel palmyrénien, naît des flancs du cyprès, image symbolique de Vénus. Ainsi, nous avons sous les yeux une des médailles de Nagidus composées d'après des traditions religieuses communes aux Syriens, aux Palmyréniens et aux Ciliciens. Elle remonte à une époque où les dominateurs de la Cilicie n'avaient pas encore osé appliquer aux types asiatiques les modifications que, plus tard, le sentiment du beau et du vrai, principe fondamental de l'art grec, devait leur faire subir. Mais à la suite de la série de médailles qui comprend cette pièce, vient se ranger une autre série, où l'on découvre un des exemples les plus frappants que, dans la numismatique asiatique, on puisse citer des heureuses transformations opérées par le burin des artistes grecs. À cette seconde catégorie appartient une élégante médaille autonome, d'argent,\footnote{\emph{Numismat. ant.} pars 2., tab. 22, n° 9.} où nous voyons l'Aphrodite grecque substituée à l'Astarté des Nagidiens. La déesse fait aussi, de la main droite, une libation sur un autel posé à ses pieds ; mais elle passe, avec une tendresse toute maternelle, son bras gauche autour du cou de son fils, et ce fils a des ailes, comme nous sommes habitués à en voir aux épaules de l'Amour sur les monuments figurés des Grecs ou des Romains. La planche où j'ai placé, à côté l'un de l'autre, le dessin de chacune de nos deux médailles nagidiennes, rendra beaucoup plus sensible le contraste qui résulte du rapprochement de deux pièces monétaires, semblables quant à la composition de leurs types, mais différentes par le style propre à chacune d'elles. Ce rapprochement aura de plus l'avantage de mettre hors de doute l'identité des deux Amours Criophores de Syrie, non-ailés, avec l'Éros ailé des Grecs et le Cupidon ailé des Romains.

Cette dernière observation est d'autant mieux placée ici qu'il me paraît intéressant de rapprocher des deux Amours Criophores un très-petit bas-relief de bronze, que je crois inédit et qui fut trouvé, il y a quelques années, dans des ruines romaines, en Transylvanie.\footnote{Je dois la connaissance de ce petit monument aux obligeantes communications de Mme. la baronne Jósika, dont la mort, arrivée en 1850, est un sujet de bien vifs regrets pour ses amis et pour toutes les personnes qui cultivent les sciences et les arts.} Le dessin qui le reproduit, d'après un plâtre que j'en possède, est gravé sous le numéro 2 de ma planche 3. Ici l'Amour est représenté avec des ailes, bien qu'il soit accompagné du bélier. Toutefois il ne le porte pas sur les épaules ; il est, au contraire, étendu ou couché sur le dos de l'animal symbolique,\footnote{Telle était probablement l'attitude d'Aphrodite \emph{Pandémos}, dans le groupe de bronze que Pausanias (6., 25., 2) avait vu à Élis et qui représentait la déesse placée sur un bouc. Ce groupe, ouvrage du célèbre statuaire Scopas, nous explique l'origine du surnom d'\emph{Épitragia} qu'avait reçu Aphrodite. Comme déesse de la mer, nous la trouvons tantôt couchée, tantôt assise sur les mêmes animaux qui lui sont attribués comme déesse de la terre. C'est ainsi, par exemple, que la représente un bas-relief du Musée capitolin (\emph{Mus. capitolino}, t. 4., pl. 62.), où il faut particulièrement remarquer le bouc marin sur le dos duquel est placée Vénus. Ajoutons à ce bas-relief un vase d'argent ciselé, découvert aux environs de Vienne en Dauphiné, et orné de la représentation des quatre saisons. Le printemps y est figuré sous les traits de Vénus assise sur le dos du taureau équinoxial (voy. mes \emph{Recherches sur Vénus}, pl. 14. G, n° 17). --- Mon mémoire était déjà livré à l'impression, lorsque, dans le nouvel ouvrage de M. le duc de Luynes, cité plus haut (p. 28, 29, note 2), j'ai trouvé reproduites (p. 25, 28, pl. 5., n° 3 ; p. 31, pl. 6., n° 5) deux rares médailles autonomes et bilingues de Salamine (île de Cypre), qui appartiennent au Musée britannique, et qui représentent chacune, mais dans un module différent, au droit, un taureau ou une génisse que l'on vient d'immoler, et, au revers, un bélier portant Vénus vêtue d'une longue tunique transparente, et coiffée comme la Vénus de Médicis. On voit quelle corrélation s'établit, d'une part, entre cette Vénus Criopole, Baal ou Mercure Criophore, l'Amour Criophore des Syriens, et l'Amour Criopole du petit bas-relief ici figuré (pl. 3., n° 2) ; d'autre part, entre cette même Vénus Criopole et les autres représentations figurées de Vénus que j'ai indiquées ci-dessus. La substitution du bélier au taureau ou à la génisse dans les images de la déesse, comme dans sa légende, est un fait qui, si je ne me trompe, ajoute quelque intérêt aux observations que renferment la note 1 de la page 26 de ce mémoire et la note 2 de la page 28. De plus, ce fait, rapproché des médailles de l'île de Cypre au type du bélier, et rapproché surtout des monuments figurés qui m'ont autorisé à reconnaître un Amour Criophore et un Amour Criopole, me donne lieu de douter qu'il faille, avec M. Théodore Panofka, rapporter au mythe de Phrixus le joli vase qu'il a publié dans une savante dissertation intitulée : \emph{Atalante und Atlas} (Berlin, 1851, in-4°), p. 10 et 11, n°\textsuperscript{s} 4 et 4 \emph{a} de la planche qui accompagne le texte.} et devient ainsi un Cupidon \emph{Criopole} ; mais cette attitude ne met pas moins en évidence l'intention formelle qu'on avait eue de faire ainsi allusion à la naissance de Cupidon, fruit des amours de Vénus avec Mercure Criophore.\footnote{Dans la mythologie indienne, c'est Agni, le dieu du feu, qui paraît monté sur un bélier. (Voy. M. Ch. Coleman, \emph{The mythology of the Hindus} (Lond. 1832, in-4°), pl. 22., n° 4.)} Par-là, se trouve confirmée une fois de plus l'interprétation que je donne au sujet représenté sur la face postérieure\footnote{Pl. 1., n° 2.} de l'autel du Musée capitolin.

Ce sujet est étrangement défiguré dans les dessins qui en ont été publiés jusqu'à ce jour. Sur la planche jointe à la Lettre citée du P. Giorgi, on a substitué au bélier que porte l'Amour un veau ou une génisse. Le dessinateur ou le graveur dont s'est servi M. l'abbé Lanci, a remplacé le bélier par une chèvre. Un autre érudit a pris ce bélier pour un cheval.\footnote{C'est ce qu'affirme M. l'abbé Lanci (\emph{Paralip.} t. 1., p. 34), en s'abstenant charitablement de nommer l'auteur d'une pareille méprise.} Bien plus, dans le dessin que publia Spon et que reproduisit l'auteur de l'\emph{Antiquité expliquée}, on ne découvre aucune trace d'enfant, de bélier, de veau, de génisse, de chèvre, ni de cheval. On conçoit dès lors qu'aucun de mes savants devanciers n'ait été amené à reconnaître sur notre autel la naissance de l'Amour Criophore, et que tous, sans exception, se soient plus ou moins égarés dans l'interprétation de l'ensemble du monument. Le P. Giorgi,\footnote{Lettre citée, p. 39 et suiv.} par exemple, se préoccupe exclusivement de l'idée que cet autel, comme le bas-relief palmyrénien dont je parlerai plus loin, doit s'expliquer par les doctrines ou les traditions propres aux sectateurs de Zoroastre. De là, il conjecture que l'oiseau sculpté au bas du buste radié est l'\emph{éoroschasp} ou le \emph{simorg}, oiseaux consacrés à Mithra, et que la figure ailée, qui pose une couronne sur la tête du personnage placé dans un char attelé de quatre griffons, est Ormuzd instituant roi des rois Mithra ou le Soleil.\footnote{\emph{Ibid.} p. 110.} Il ne met pas en doute que le cyprès représenté sur chacun des deux monuments ne fasse allusion à Zoroastre, qui, selon le témoignage des auteurs persans, grava sur l'écorce d'un arbre de cette espèce certains dogmes religieux. Le P. Giorgi rappelle à ce sujet que, dans les écrivains orientaux, il est aussi question d'un autre cyprès qui renfermait l'esprit ou le féroüer du faux prophète, et dont les propriétés étaient merveilleuses et surnaturelles. Au nombre des miracles produits par ce second cyprès, le même philologue, sans citer aucune autorité, se croit en droit de comprendre la naissance de l'enfant que nous voyons sortir des flancs d'un cyprès sur une des faces de l'autel du Musée capitolin. Il avance que cet enfant est Zoroastre lui-même, tenant dans ses mains la portion antérieure du corps d'une vache, pour rappeler une des circonstances fabuleuses de sa propre naissance : « Ad indicandum spiritum simul et genitale semen corporis Zoroastris, vaccino lacti a Deo creatore immixtum.\footnote{\emph{De inscription. palmyrenis quæ in Mus. capitol. adserv. interpretand. Epistola}, p. 41 et 42. Dans le passage que je cite, le P. Giorgi fait allusion à une tradition recueillie par Scharistâni (voy. Th. Hyde, \emph{Hist. relig. veter. Persar.} p. 300 ; 2\textsuperscript{e} édition).} » Un habile archéologue de Dresde, feu M. Böttiger,\footnote{\emph{Ideen zur Kunstmytholog.} p. 239.} adoptant avec trop de complaisance l'opinion du philologue romain, poursuit ce même système d'explication, et suppose que l'enfant dont il s'agit est un des vingt-huit \emph{izeds} du Zend-Avesta, « qui, dit-il, élève vers le ciel un jeune taureau, symbole du taureau solaire, ou plutôt de ce taureau du monde que mentionnent fréquemment les livres de Zoroastre, et que des auteurs orientaux postérieurs ont appelé le taureau \emph{Aboudad}. » Ainsi, au lieu de chercher tout naturellement l'explication d'un monument figuré palmyrénien dans la légende des divinités que l'on sait avoir été honorées en Syrie d'un culte particulier, ou dans le rapprochement des divers monuments de l'art qui furent consacrés à ces mêmes divinités, le P. Giorgi et Böttiger ont cru qu'ils devaient la chercher exclusivement dans les livres de Zoroastre ou dans les traditions persanes. Il est cependant avéré que le culte d'Ormuzd et de Mithra ne se répandit point en Syrie, et que le culte beaucoup plus ancien de Baal et de Baltis, Astarté ou Dercéto y resta dominant. Des erreurs analogues à celle que je relève se reproduisent fréquemment, je le dis à regret, dans les dissertations archéologiques où il s'agit, soit d'expliquer des antiquités figurées orientales qui n'appartiennent pas à la Perse, soit de remonter à la source des influences exercées par un art asiatique sur certains monuments figurés de l'Occident. Toutefois Giorgi et Böttiger n'auraient peut-être pas encouru le reproche que je me vois obligé de leur adresser, si le premier de ces deux érudits, examinant avec plus de soin qu'il ne l'a fait le quadrupède que porte sur ses épaules l'enfant qui naît du cyprès, avait reconnu dans ce quadrupède, non point une génisse ou un taureau, mais bien un bélier, animal symbolique, dont les fragments qui nous restent des livres zends ne font aucune mention.

De son côté, le dernier interprète de notre autel palmyrénien, M. l'abbé Lanci, après avoir cru reconnaître sur trois des faces de ce monument, je l'ai déjà dit, la représentation des trois premières saisons de l'année, le printemps, l'été et l'automne, ne met pas en doute que la face postérieure ne représente l'hiver. À cette quatrième saison préside, selon lui,\footnote{\emph{Paralipom.} t. 1., p. 34-37.} le Soleil femelle, le \emph{Soleil pluvieux}, \emph{Aglibol}, roi ou dominateur de la constellation zodiacale du Capricorne ; il se cache ici dans les rameaux du cyprès, symbole de l'exaltation du soleil, de l'exaltation de la gloire\footnote{\emph{Ibid.} p. 57.} ; la petite chèvre (\emph{capretta}), placée sur les épaules de cette divinité féminine, fait allusion au Capricorne, et, ajoute-t-il, au retour du Soleil femelle parvenu à ce signe zodiacal.\footnote{J'ai quelque peine à concilier ces idées avec l'opinion où est M. Lanci (\emph{ibid. ibid.}) que le cyprès était consacré à Aglibol, et que le \emph{gopher} des Hébreux et le \emph{sàru} des Arabes, qui sont deux des noms de cet arbre en Orient, appartiennent à deux racines sémitiques exprimant une même idée, celle de la disposition à la génération ou à l'enfantement. Je laisse aux maîtres en philologie le soin de décider jusqu'à quel point est fondée cette dernière observation.} Une telle interprétation ne se serait sans doute pas présentée à l'esprit de l'ingénieux auteur des \emph{Paralipomeni}, si, regardant d'un peu plus près le monument original, il n'avait pas pris le bélier pour une petite chèvre, et le jeune enfant pour une déesse solaire, Aglibol, que nous allons trouver représentée comme une divinité mâle et lunaire sur un autre monument palmyrénien, non moins authentique.

Les diverses interprétations proposées par mes savants devanciers ne sauraient donc, si je ne me fais illusion, infirmer l'explication que j'ai donnée de chacune des sculptures qui ornent l'autel à quatre faces du Musée capitolin. Cette explication, je ne l'ai pas cherchée ailleurs que dans l'étude des croyances religieuses et du langage symbolique qui furent communs aux Assyriens, aux Syriens et aux Phéniciens. Si, considérant le monument dans son ensemble, nous nous rappelons, d'une part, les témoignages anciens qui assimilent le Soleil à Baal ou Jupiter,\footnote{Voy. plus loin, p. 47, 48, 99.} et, d'autre part, le célèbre passage de Julien\footnote{\emph{Orat. 4. in Solem}, Opp. p. 131 sqq. ed. Spanh. Cf. \emph{Scriptor. rer. mythicar. latini tres}, Mythogr. 2., fab. 18., t. 1., p. 80, ed. Bode. On lit ici ces paroles très-explicites : « Constat autem triplicis esse Apollinem potestatis, et eundem esse Solem apud superos, Liberum patrem in terris, Apollinem apud inferos ... »} sur les trois soleils qu'admettait la théologie syrienne,\footnote{Nous retrouvons des traces de cette croyance chez les Égyptiens, qui, selon Plutarque (\emph{De Isid. et Osir.} p. 372, ed. Xylandro ; p. 93, 94, ed. G. Parthey), sacrifiaient trois fois par jour au Soleil, et chez les Perses, qui donnaient à Mithra une épithète que les écrivains grecs ont traduite par \textgreek{τριπλάσιος}. D'accord avec la signification de cette épithète, la liturgie du Zend-Avesta renferme l'obligation d'adresser trois prières par jour à Mithra.} nous sommes amenés à supposer que notre autel palmyrénien représente, outre Vénus, cette triade solaire, c'est-à-dire le dieu suprême des Syriens et ses deux manifestations. Dans cette hypothèse, le buste orné d'un voile et d'une \emph{harpé},\footnote{Pl. 2., n° 1.} et le buste radié posé sur un aigle,\footnote{Pl. 1., n° 1. N'oublions pas que, chez les Perses, l'aigle était le symbole de Mithra, dieu solaire et manifestation d'Ormuzd. C'est pourquoi un grade de \emph{pèreaigle} avait été institué dans les mystères de ce dieu, et pourquoi aussi nous lisons dans le Zend-Avesta (t. 2., p. 388), que l'aigle se balance entre les deux extrémités du monde, c'est-à-dire entre les deux portes du ciel, le soleil et la lune. --- Depuis que ces remarques ont été placées ici, j'ai trouvé plusieurs autres exemples des rapports directs que la théologie asiatique ou chaldéenne avait établis entre l'aigle et le soleil. Ils me sont fournis par le nouvel ouvrage de M. le duc de Luynes, dont j'ai parlé ci-dessus. Le savant auteur y publie (p. 11 et 14, pl. 2., n° 18 ; p. 14 et 55, pl. 12., n° 6) deux rares et belles médailles d'or, d'Évagoras 1\textsuperscript{er}, l'une du cabinet de S. A. I. le grand-duc de Toscane, l'autre de la collection Hunter, et représentant chacune, au revers de la tête tourellée de Vénus Paphienne, un aigle posé sur la croupe d'un lion et l'astérisque du soleil gravé au-dessus de ce groupe. Un tel type, on le voit, nous ramène à l'autel palmyrénien du Musée capitolin, dont une face (pl. 1. n° 2) nous offre l'image symbolique de Vénus enfantant l'Amour ; une autre face (\emph{ibid.} n° 1) nous montre le buste d'un dieu-soleil, placé sur un aigle. D'autres médailles de l'île de Cypre, également publiées par M. le duc de Luynes, présentent d'un côté une tête de lion tirant la langue, symbole très-significatif du soleil d'été : et, de l'autre côté, tantôt un aigle se balançant dans les airs au-dessus d'un lion couché (ouvrage cité, p. 9, 10 et 12 ; pl. 2., n°\textsuperscript{s} 3-8), tantôt, au lieu d'un aigle, l'astérisque même du soleil gravé au-dessus d'un lion pareillement couché (\emph{ibid.} p. 10 et 12 ; pl. 2., n° 9).} seraient l'un l'image du \emph{Temps-sans-bornes}, le Saturne des Grecs et des Romains ; l'autre, l'image de Baal, c'est-à-dire les images sensibles des deux soleils intellectuels et immatériels qui, selon les expressions de Julien, donnent l'intelligence aux dieux et aux êtres immatériels et invisibles. Le personnage placé sur un char ascendant et attelé de quatre griffons\footnote{Pl. 2., n° 2.} serait et ne peut-être que le troisième soleil de cette théologie : soleil matériel et apparent ou visible, il dispense à la terre la lumière, la chaleur, et, chaque année, vainqueur (\emph{sol invictus}) dans sa lutte avec les ténèbres, le froid et le principe humide, il recommence son ascension vers le ciel, couronné par la Victoire, comme nous le voyons ici. Enfin, sur une quatrième face,\footnote{Pl. 1., n° 2.} le cyprès enfantant l'Amour est l'emblème expressif de la Vénus asiatique, Baltis ou Beltis, épouse et sœur de Baal ou Jupiter, et, par conséquent, fille du Temps-sans-bornes, comme Baal.\footnote{C'est pourquoi, dans les légendes hiéroglyphiques qui accompagnent quelques images de la Vénus assyrienne importée en Égypte sous le nom d'Hathor où Athyr, la déesse est qualifiée \emph{fille du Soleil}. Une note, qui m'a été communiquée par M. le vicomte Emmanuel de Rougé, m'apprend que telle est la véritable signification des signes que Champollion le Jeune traduisait par ces mots : \emph{Grande génératrice du Soleil} (\emph{Mém. de l'Acad. royale des sciences}, t. 13., p. 647-649).}

Que cette interprétation soit fondée ou non, il reste un fait certain, la réunion sur un seul autel de quatre bas-reliefs qui nous offrent l'image d'un dieu suprême qu'il faut nécessairement assimiler à Cronos ou Saturne, l'image de Baal ou le \emph{Temps limité}, qui s'assimile à Jupiter et à Ormuzd, l'image du soleil matériel et ascendant, personnifié, et celles enfin de Vénus et de l'Amour. Par-là, nous constatons que, si dans le célèbre temple d'Hiérapolis, comme chez les Cypriens,\footnote{Julien (\emph{loc. cit.} p. 135 D et p. 144 C) dit expressément que les Cypriens élèvent des autels en commun à Hélios et à Zeus, et qu'Apollon est l'assesseur de Zeus.} on avait associé le culte du Soleil à celui de Baal ou Jupiter et de Baltis ou Vénus, il en était de même à Palmyre.\footnote{L'association du culte et des images du Soleil et de Baal en Cypre et à Palmyre, comme à Hiérapolis, me rappelle un passage de Zosime (\emph{Histor.} 1., 61., 3-5), où il est dit qu'Aurélien, le vainqueur de Zénobie, employa les trésors enlevés de Palmyre à construire dans Rome (d'autres disent dans Palmyre même) un magnifique temple en l'honneur du Soleil, et à l'orner des statues du Soleil et de Bélus (\textgreek{Ἡλίου τε καὶ Βήλου καθιδρύσας ἀγάλματα}). Reitemeier, dans son édition de Zosime (Lipsiæ, 1784), n'a pas hésité à suivre ici le sentiment de Scaliger et à rejeter la leçon proposée par Saumaise. Ce dernier, à cause de l'identité du Soleil et de Bélus chez les Syriens, pensait qu'il fallait lire : \textgreek{Ἡλίου τοῦ καὶ Βήλου} ; ce qui, observe Scaliger, ne peut s'accorder avec la terminaison plurielle du mot \textgreek{ἀγάλματα}. Notre autel confirme cette judicieuse remarque qui, sans doute, n'avait pas échappé au savant Tillemont ; car il traduit : « Aurélien y mit les statues du Soleil et de Bel (Baal). » (Voy. \emph{Hist. des emper.} t. 3., p. 400 ; Paris, 1720-1738.)} Par là aussi nous pouvons, avec toute certitude, remonter à l'origine asiatique du temple érigé, dans l'Acro-Corinthe, en l'honneur d'Aphrodite armée, temple où Pausanias\footnote{2., 4., 7.} avait vu réunis les statues et, par conséquent, le culte d'Aphrodite, d'Hélios et d'Éros. Nous remontons avec non moins de certitude à l'origine également asiatique du type de quelques monnaies autonomes de Dyrrhachium, colonie corinthienne fondée en Illyrie ; elles nous montrent le buste radié du Soleil superposé au groupe d'une vache allaitant un veau\footnote{Voy. \emph{Recherch. sur Vénus}, pl. 14. H, n°\textsuperscript{s} 2 et 4.} ; et ce groupe, comme je l'ai dit ailleurs,\footnote{Voy. \emph{ibid.} p. 165-168.} représente symboliquement la fille du dieu-soleil, la Vénus asiatique, allaitant son fils. À leur tour, ces monnaies de Dyrrhachium et le temple de l'Acro-Corinthe, aussi bien que l'autel du Musée capitolin, nous permettent de croire que, si à Rimat on a découvert, dans un même \emph{sacellum} souterrain, deux bustes du Soleil et une image de l'Amour \emph{Criophore}, ce n'est point accidentellement que ces trois monuments s'y trouvaient réunis.

Le bas-relief palmyrénien que reproduit le n° 1 de ma planche 3, nous offre un second exemple de la représentation de Vénus-Beltis sous le symbole du cyprès pyramidal. Ce marbre, déposé d'abord dans les jardins du palais Mattéi, puis dans la vigne du palais du cardinal Carpi,\footnote{Ms. 3429 du Vatican, fol. 122, cité par Bottari (\emph{Mus. capitolin.} t. 4., p. 78). Gruter, \emph{loc. cit.} Spon, \emph{loc. cit.} --- Eichhorn (\emph{loc. cit.} p. 99) substitue indûment ici le nom de Carpegna à celui de Carpi.} fut ensuite placé, selon les uns, au musée Borgia, selon les autres,\footnote{Voy. \emph{Mus. capit.} t. 4., p. 78. --- C'est par inadvertance, sans doute, que Selden (\emph{ubi infra}) place ce bas-relief dans les jardins du palais Farnèse. Il paraît certain qu'en 1708, il se trouvait au palais Giustiniani (voy. Bianchini, \emph{Opuscula}, t. 1., p. 55-59).} dans les jardins du palais Giustiniani, d'où le pape Benoit 14 l'aurait fait transporter au Musée capitolin. Il fut signalé à l'attention des archéologues, en 1601, par Gruter,\footnote{\emph{Inscript. antiq.} p. 86., n° 8.} qui toutefois n'en donna que les inscriptions et une description succincte d'après les notes de Smeth. Spon\footnote{\emph{Recherch. cur. d'antiquité}, p. 59 et suiv. \emph{Miscellan. erud. antiq.} p. 1, 599. Il cite ici, p. 2, à propos d'Aglibol et de Malachbel, un écrit de François Malaval (\emph{Malavallius Massiliensis}), que je n'ai pu me procurer.} le publia intégralement en 1683 et en 1685. Depuis, il a été reproduit ou seulement commenté par plusieurs autres savants\footnote{Bernard et Smith, \emph{Versio et notæ ad inscript. græc. Palmyr.} (Traj. ad. Rhen. 1698, in-8°), p. 14 ; ou \emph{Inscript. græc. Palmyrenor. cum scholiis et annotationib.} p. 2, n° 3. Selden, \emph{De diis syris}, p. 152, sqq ; edit. Amstelod. 1680. Halley, \emph{Philosophic. transact.} vol. 19. (1695-1697), n° 218, p. 171, 172. Abraham Seller, \emph{The antiquit. of Palmyra} (Lond. 1696, in-8°, fig.), \emph{Append.} chap. 8., p. 352 ; \emph{Acta erudit. lipsiens. Supplem.} t. 3. (1702), sect. 3, p. 102. Th. Hyde, \emph{De relig. veter. Persar.} p. 114, tab. 12., p. 116, edit. 1760. Reland, \emph{Palæstin. illustr.} (Traj. Batav. 1714), t. 2., p. 525, 526. Montfaucon, \emph{L'Antiquit. expliq.} t. 2., 2\textsuperscript{e} part. p. 389-391 ; pl. 174., fig. 3 ; Paris, 1719. Dawkins et Wood, \emph{The ruins of Palmyra}, p. 27, n°\textsuperscript{s} 6 et 9 ; Lond. 1753, grand in-f°. Barthélemy, \emph{Réflex. sur l'alphab. et sur la langue dont on se servait autrefois à Palmyre}, Explic. p. 21 ; Paris, 1754, in-4° ; ou \emph{Mém. de l'Acad. des inscr.} t. 26. (ann. 1759), p. 557 et suiv. Bianchini, \emph{Opuscula}, t. 1., p. 51-70 ; Rom. 1754, in-4°. Swinton, \emph{An explicat. of all the inscript. in the Palmyrene lang. and charact.} cinq lettres adressées à Th. Birch et insérées dans le 48\textsuperscript{e} volume des \emph{Philosophic. transact.} (ann. 1754), n° 87., p. 690-756 ; pl. 24-30. Giorgi, Lettre citée, p. 107, tab. 1. Bottari, \emph{Mus. capitolin.} t. 4., p. 77-86 ; pl. 18. ; Rom. 1782. Eichhorn, \emph{Comment. Societ. reg. scient. gotting. rec.} t. 6., (ann. 1828), \emph{Class. historic. et philologic.} p. 98-105, 117, 118. Boeckh, \emph{Corp. inscr. gr.} t. 3. p. 824, 825, n° 6015. M. l'abbé Michelange Lanci, \emph{Di un egizio monum. con iscriz. fenic.}, \emph{Spiegazione}, etc. p. 143-152 ; \emph{Paralipomeni}, t. 1., p. 29-37, pl. 2., n° 1. Gésénius, \emph{Scriptur. linguæque phœnic. monum.} (Lips. 1837, in-4°) pars 1., p. 80-82. M. le Dr. Stark, \emph{Forschung. zur Geschichte und Alterthumsk. des hellenist. Orients}, t. 1., p. 570, 571 (Jena, 1852, in-8°).} ; et, chaque fois, le sujet qu'il représente et l'inscription bilingue grecque et palmyrénienne, qui lui assigne la date du mois \emph{péritius} de l'an 547 de l'ère des Séleucides,\footnote{Cette date correspond au mois de février 234 ou 235 de l'ère chrétienne.} ont exercé sans beaucoup de succès la sagacité des antiquaires et des philologues. Aucun d'eux, en effet, ne paraît avoir soupçonné que le cyprès sculpté ici sous le portique d'un temple distyle est l'image symbolique de la Vénus orientale,\footnote{On ne peut trouver, sur deux monuments palmyréniens, la Vénus assyrienne représentée symboliquement par un cyprès, sans être amené à supposer que les épaisses forêts de cyprès et de palmiers qui entouraient Palmyre et qui n'ont pas entièrement disparu (voyez Petr. Terz. \emph{Syria sacra}, p. 128, et les voyageurs plus récents) étaient consacrées à la déesse, de même que les cyprès du mont Liban. Les monuments figurés asiatiques prouvent que le palmier femelle, comme le cyprès, était aussi attribué à la Vénus orientale.} et que les deux personnages placés l'un à la droite, l'autre à la gauche de ce cyprès, contractent une alliance conjugale et mystique sous les auspices de cette déesse. Bien plus, Gruter ou plutôt Smeth, Spon, Montfaucon\footnote{Ce savant bénédictin, dans le deuxième volume de l'\emph{Antiquité expliquée} (p. 391, et pl. 179., n° 4), a publié un bas-relief palmyrénien du palais Giustiniani, déjà figuré, en 1651, sur la planche 117. (part. 2.) de la \emph{Galleria Giustiniana}, et presque semblable à celui qu'il reproduit d'après le dessin de Spon. Il le prend pour l'original ; et, en ce cas, le bas-relief du Musée capitolin n'en serait qu'une copie, tandis que l'opinion contraire me paraît bien plus plausible.} et, en dernier lieu, M. l'abbé Lanci,\footnote{\emph{Paralipom.} t. 1., p. 20.} méconnaissant chacun la forme caractéristique de cet arbre pyramidal, l'ont pris pour un pin. Bottari\footnote{\emph{Loc. cit.}} penche fortement à croire qu'au lieu d'un cyprès on a voulu représenter ici un palmier. Le P. Giorgi et M. Avellino n'ont pas commis de telles erreurs ; mais le premier, je l'ai déjà dit, ne veut voir dans le cyprès de nos deux monuments palmyréniens qu'une représentation de ceux qui furent solonnellement plantés en Perse par les mains de Zoroastre. L'archéologue napolitain\footnote{Mémoire cité, p. 25.} se contente de rapprocher du bas-relief le type des médailles de Damascus et d'Héliopolis dont il sera question plus loin et qui portent, celles-ci,\footnote{Voy. ma planche 6., n° 5.} un cyprès planté à l'entrée d'un temple ; celles-là,\footnote{\emph{Ibid.} n° 3.} un cyprès placé entre un cheval et un taureau. À ses yeux, cet arbre n'a jamais représenté que le soleil ; aussi adopte-t-il sans restriction l'opinion où était Eckhel\footnote{\emph{D. N.} 3., p. 532.} que, sur les médailles impériales de Damascus, le cyprès, emblème du soleil, nous apprend pourquoi Cyparisse, cher à Phœbus, fut changé en un arbre de cette espèce. Malgré toute ma déférence pour les jugements portés par l'illustre fondateur de la science numismatique, un tel commentaire, je l'avoue, me semble peu satisfaisant. Un savant académicien de Göttingue\footnote{Eichhorn, \emph{loc. cit.} t. 6., p. 98-105.} a trouvé plus simple de passer sous silence le cyprès de notre bas-relief palmyrénien ; en conséquence, il cherche à interpréter le sujet de ce marbre sans s'occuper des rapports qui peuvent exister entre les deux personnages mythologiques qu'on a voulu y représenter et l'arbre sacré qui s'élève au milieu d'eux. Quant à M. l'abbé Lanci, aux yeux de qui ces deux figures sont la personnification du commencement et de la fin de l'année palmyrénienne, il considère le prétendu pin placé entre elles, comme l'emblème de la vie perpétuelle du soleil.\footnote{\emph{Paralipom.} t. 1., p. 30.} Il est cependant impossible, dès le premier abord, de ne pas reconnaître, sous les traits du personnage mâle, sculpté à droite du cyprès, le dieu Men ou Lunus parfaitement caractérisé par le grand croissant attaché à ses épaules. Dès lors, la figure féminine, dont la main droite s'enlace avec la main gauche de ce dieu, ne peut être que la personnification du soleil devenu, chez plusieurs peuples de l'Orient, une divinité femelle le jour où la Lune fut déclarée un dieu mâle. Si je répète que ce couple adelphe s'unit ici en présence de l'image symbolique de Vénus,\footnote{Dans les temps anciens, les institutions civiles de l'Orient, calquées sur les institutions religieuses, voulaient que le roi et la reine fussent assimilés aux dieux créateurs du monde, et qu'ils parussent au milieu de leur cour revêtus, le roi, des insignes du Soleil ; la reine, des insignes de la Lune. La loi exigeait, de plus, que, le Soleil et la Lune, comme Bel et Beltis ou Mylitta, comme Jupiter et Junon en Occident, étant réputés frère et sœur et unis par un lien conjugal, le roi choisît parmi ses sœurs sa femme légitime. Ce mariage, ainsi sanctionné par les prescriptions hiératiques, n'avait rien d'incestueux aux yeux des peuples de l'Orient, et se célébrait chez eux en présence et sous les auspices, soit de Mithra, soit de la Vénus assyrienne, comme l'hyménée du Soleil avec la Lune représenté sur notre bas-relief palmyrénien.} c'est pour dire, par anticipation, que plusieurs autres monuments nous montrent la déesse placée tantôt sous l'emblème du cyprès, tantôt sous une forme humaine, entre le Soleil et la Lune représentés, à leur tour, soit sous l'emblème de deux cyprès, soit sous leur forme d'astres, soit enfin sous les traits d'Apollon et de Diane. Giorgi\footnote{Lettre citée, p. 62-72, 99-102, 110, 111, 114-117.} commet donc une grave méprise en assimilant à Ormuzd, dieu des Perses, la première des deux figures de notre bas-relief palmyrénien, et à Mithra la seconde. Eichhorn ne tombe pas dans une erreur moins grave lorsqu'il prend ces deux figures pour deux personnages mâles, et transformes-en un simple suppliant, qui implore le dieu Lunus, la figure féminine placée à gauche de l'arbre.\footnote{\emph{Loc. cit.} p. 99, 100.} Le peu de mots que j'ai dits de l'opinion de M. l'abbé Lanci a suffi, sans doute, pour montrer à l'Académie combien il s'est écarté de l'idée qu'attachaient les Palmyréniens au sujet sculpté sur le monument qui nous occupe. Par-là, il s'est trouvé dans l'impossibilité de proposer une explication admissible du rôle qu'y jouent le personnage femelle et le personnage mâle sous les traits de qui nous devons reconnaître le Soleil et la Lune. Bien qu'ils n'eussent pas compris le véritable sens de cette scène, Spon, Montfaucon, Saumaise,\footnote{\emph{In Flav. Vopiscum notæ} ; voy. \emph{Histor. August. scriptor.} p. 370.} Zoëga\footnote{\emph{Abhandlung.} p. 175 ; ed. Welcker.} et Avellino\footnote{Mémoire cité, p. 25.} n'avaient pas hésité à y reconnaître aussi la présence de ces deux astres personnifiés. C'est, en effet, à la Lune et au Soleil divinisés qu'est consacré le monument par une inscription bilingue, qui leur donne le nom d'\emph{Aglibol} ou \emph{Aglibolos} et de \emph{Malachbel} ou \emph{Malachbélos}, en les qualifiant, mais dans le texte grec seulement, \emph{dieux de la patrie}, \textgreek{ΠΑΤΡΩΟΙ · ΘΕΟΙ}.\footnote{Voy. sur cette qualification les observations de M. le Dr. Stark, ouvrage cité, t. 1., p. 570 et suiv.} Les quatre premières lignes se lisent ainsi : \textgreek{ΑΓLΙΒꞶΛꞶ · ΚΑΙ · ΜΑΛΑΧΒΗΛꞶ · ΠΑΤΡꞶΟΙϹ · ΘΕΟΙS · ΚΑΙ · ΤΟ · ϹΙΓΝΟΝ · ΑΡΓΥΡΟΥΝ · ϹΥΝ · ΠΑΝΤΙ · ΚΟΖΜꞶ · ΑΝΕΘΗΚΕ · Τ}\footnote{M. l'abbé Lanci lit : \textgreek{Λ · ΑΥΡ ·} (\emph{Spiegazione}, etc. p. 147).} \textgreek{· ΑΥΡ · ΗΛΙΟΔꞶΡΟϹ · ΑΝΤΙΟΧΟΥ · ΑΔΡΙΑΝΟϹ · ΠΑΛΜΥΡΗΝΟϹ · ΕΚ · ΤꞶΝ · ΙΔΙꞶΝ · ΥΠΕΡ · ϹꞶΤΗΡΙΑϹ. ΑΥΤΟΥ. ΚΑΙ · Τ · ϹΥΜΒΙΟΥ · ΚΑΙ · Τ · ΤΕΚΝꞶΝ · ΕΤΟΥϹ · Ζ · Μ · Φ · ΜΗΝΟϹ · ΠΕΡΙΤΙΟΥ} ; c'est-à-dire : « Tite Aurèle Héliodore Hadrien, palmyrénien, fils d' Antiochus, a consacré, à ses frais, à Aglibolos et à Malachbélos, dieux de la patrie, (ce monument) et une statue d'argent avec tous ses ornements, pour la conservation de sa vie et de celle de sa femme et de ses enfants, année 547, mois de péritius. »

Les deux lignes de caractères palmyréniens qui, sur le marbre, suivent immédiatement cette inscription, n'en sont pas la traduction exacte. C'est ce qui arrive fréquemment dans les inscriptions bilingues ou trilingues ; et, faute d'avoir tenu compte de ce fait, d'habiles philologues se sont parfois égarés en s'obstinant à vouloir trouver entre les mots de chaque texte une parfaite correspondance. Le premier essai tenté pour déchiffrer les caractères palmyréniens de notre bas-relief remonte à Joseph Scaliger. Il dut lui en coûter beaucoup d'avouer qu'il n'avait pu les lire, « lui qui cependant se glorifiait de savoir assez de langues pour pénétrer par terre jusqu'à la Chine, sans le secours d'aucun interprète.\footnote{Barthélemy, \emph{Mém. de l'Acad. des inscript.} t. 26., p. 578.} » En 1632, Samuel Petit\footnote{\emph{Epistol. ann. 1632 ad Peireskium, apud} Spon. \emph{Miscell. erud. antiq.} p. 2, 3. On trouve là, de plus, une lettre écrite à Spon par Galland, qui se moque agréablement de Samuel Petit, et lui reproche d'avoir fait une \emph{exposition} tout aussi énigmatique que les caractères orientaux qu'il a cherché à expliquer.} donna une version latine de l'inscription palmyrénienne. Ce second essai fut plus malheureux encore : on mit en doute le bon sens du traducteur.\footnote{\emph{Loc. cit.} Cf. la Lettre citée du P. Giorgi, p. 6.} Je ne m'arrêterai pas à raconter ici toutes les autres tentatives, plus ou moins infructueuses, qui se succédèrent depuis 1632 jusqu'en 1754. L'abbé Barthélemy en a fait le récit,\footnote{\emph{Loc. cit.} p. 578 et suiv. À l'opuscule qu'il cite de l'orientaliste Rhenferd, il faut ajouter une lettre où Gisbert Cuper transmet à Bianchini la traduction faite par Rhenferd de l'inscription palmyrénienne dont il s'agit. Cette lettre est insérée dans le tome 1 (p. 60-70) des \emph{Opuscula} de Bianchini (Rom. 1754). L'écrit de Rhenferd, publié en 1704, sous le titre de \emph{Periculum phœnicium}, a été réimprimé en 1722, à Franecker, dans le recueil in-4° des œuvres de cet auteur.} et s'est acquis l'honneur d'avoir, le premier, restitué l'alphabet palmyrénien et transcrit l'inscription de notre bas-relief et celle de l'autel à quatre faces avec assez d'exactitude pour ne laisser à ses successeurs qu'un très-petit nombre de rectifications à opérer ou de lacunes à remplir. En 1825, M. l'abbé Lanci, très-versé dans la connaissance des idiomes sémitiques ou araméens, et fort peu satisfait, ainsi qu'Akerblad,\footnote{M. l'abbé Lanci, ouvrage cité (\emph{Spiegaz.}), p. 145.} du travail de Giorgi, a proposé\footnote{\emph{Ibid.} p. 149.} une transcription en dialecte chaldaïque et une version italienne, qui se rapprochent beaucoup de la lecture de l'académicien français. Je les rapporte ici l'une et l'autre :

\begin{center}\texthebrew{
לעגלבול ומלככל וסמיתא די כספא ותצביתה

עבר מן כיסה ירחי בר חליפי בר

ירחי בר לשמש־שער ועל חיוהי וחיא

בנוהי בירח שבט שנת (הקככהב)
}
\end{center}

\begin{center}
\emph{Ad Aglibolo e Malachbelo, e il simulacro di argento, e gli ornamenti suoi, fece a sue spese Iarchi, figlio di Chaliphi, figlio di Iarchi, figlio di Lascèmesc-sêad, per la salute sua, e la salute de' figli suoi, nel mese di Scebat dell' anno 547.}
\end{center}

\paragraph{}
À ma prière, M. le duc de Luynes a bien voulu revoir, sur l'estampage qui m'avait été envoyé de Rome, la transcription et la traduction de M. l'abbé Lanci. Voici comment il transcrit en caractères hébraïques et traduit en latin le texte palmyrénien que reproduit fidèlement cet estampage :

\begin{center}\texthebrew{
לצגלבּול ומלכבּלו בּמיתא די כספא ותצבּיתהּ עבּר מן כיסהּ ירחי בּר־חליפי בּר־

ירחי בּר־לשמששערו על חיוהּ יוחיהּ בּנוהּי בּירחּ שבּטּ שנת $\svgAAA$
}
\end{center}
\begin{center}
1. \emph{Aglibolo et Malachbelo sacellum argenteum et decorationem ejus fecit ex propria pecunia Iarchi,\footnote{\texthebrew{ירחי}, \emph{Iarchi}, de \texthebrew{ירח}, « luna, mensis. » (Note de M. le duc de Luynes.)} filius Chalifi,\footnote{\texthebrew{חליפי}, \emph{Chalifi}, de \texthebrew{חלף}, « successit, vires obtinuit ; » cf. \emph{chalife}, $\arabicAAG$. \emph{Idem.}} filii}

2. \emph{Iarchi, filii Lasemesceari,\footnote{\texthebrew{לשמששערו}, \emph{Lasemescear}, ou de \texthebrew{לשמש}, \emph{Lasemes}, « solis ; » \texthebrew{שׁער}, \emph{scear}, « porta. » \emph{Id.}} pro salute Ieochaiæ,\footnote{\texthebrew{יוחיה}, \emph{Ieochaia}, de \texthebrew{יוּ}, \emph{Iao} ; \texthebrew{חיה}, « vivere fecit. » \emph{Idem.}} filii sui, mense scebat, anno 547.}\footnote{Les lettres numérales de la date sont curieuses, et je ne les retrouve pas absolument semblables dans les autres inscriptions palmyréniennes que je connais : $\svgAAB$ 500 + $\svgAAC$ 40 + $\svgAAD$ 5 + $\svgAAE$ 2 = 547. $\svgAAE$ est le 2 arabe. \emph{Id.} --- Ce sont ces lettres numérales que Rhenferd (Bianchini, \emph{Opuscula}, t. 1., p. 69, 70) avait prises pour un nom de prince ou de magistrat.}
\end{center}
\hspace*{0mm}C'est-à-dire : « Iarchi, fils de Chaliphe (ou Kalife), fils de Iarchi, fils de Lasemescear, a consacré à Aglibol et à Malachbel, à ses propres dépens, une chapelle d'argent avec tous ses ornements, pour la conservation de la vie de son fils Iéochaia, dans le mois de scébat de l'année 547. »

\paragraph{}
Le déchiffrement de M. le duc de Luynes et celui de M. l'abbé Lanci présentent entre eux, on le voit, quelques légères différences. Le premier de ces deux savants a désiré, avec sa modestie habituelle, que son travail fût soumis à un juge dont personne ne peut contester la compétence, M. Salomon Munk. J'ai dû déférer à ce désir, et il m'est très-agréable d'avoir à dire que la transcription de M. le duc de Luynes en caractères hébraïques a obtenu l'assentiment de M. Munk, sauf les mots rendus par \emph{pro salute Ieochaiæ, filii sui}, qui, selon lui, comme selon M. l'abbé Lanci, doivent se lire : \texthebrew{על חיוהי וחיה בּנוהי}, c'est-à-dire, \emph{pro vita sua et vita filiorum suorum}.

La traduction de M. le duc de Luynes et celle de M. l'abbé Lanci s'accordent à nous montrer combien le texte palmyrénien diffère du texte grec : l'habitant de Palmyre, ou de la Palmyrène, qui a composé la dédicace dans l'idiome national, reprend pour son père et pour lui-même, dans cette dédicace, le nom qu'ils portaient avant de se laisser imposer ou d'adopter des noms grecs ou romains ; au lieu d'une statue d'argent, il est question ici, selon M. de Luynes, d'une petite chapelle d'argent, ce qui nous donnerait à comprendre que la statue dont parle le texte grec était placée dans un \emph{sacellum}, ou dans une niche. Enfin, selon les mœurs de l'Orient, l'auteur de la dédicace palmyrénienne ne fait aucune mention de sa femme, et ne désigne parmi ses enfants que ses fils (si l'on suit la traduction de M. Lanci et celle de M. Munk), ou que son fils \emph{Iéochaia} (comme traduit M. le duc de Luynes), parce que vraisemblablement ses autres enfants étaient des filles.

Dans les deux textes nous lisons le nom de Malachbel. Il désigne ici le Soleil. Sur l'autel décrit plus haut, il s'applique à Baal ou Bel lui-même, représenté sous les traits d'un dieu-soleil et qualifié \emph{sol sanctissimus}. Par-là, nous acquérons une nouvelle preuve de la confusion qui avait fini par s'établir entre Baal et le Soleil, chez les Syriens, comme entre Mithra et le Soleil chez les Perses et les Romains. Malgré l'avis contraire du savant Selden, qui veut,\footnote{\emph{De Diis syriis}, syntagm. 2., cap. 1., p. 152, 153.} dans Aglibol, reconnaître le dieu-soleil Alagabal ou Élagabal, Aglibol, dans l'inscription bilingue de notre bas-relief palmyrénien, est donc nécessairement le nom\footnote{Ce nom, qui a été lu tantôt \emph{Agalibalus}, tantôt \emph{Iaribolus}, n'a jamais été interprété d'une manière satisfaisante. Plusieurs érudits, au nombre desquels on est étonné de trouver Selden (\emph{loc. cit.} p. 149), ont proposé de le traduire par \emph{Deus} ou \emph{Dominus rotundus}, attendu, disent-ils, que l'antiquité identifiait Dieu avec la sphère du monde. Je vois avec peine qu'à l'exemple de Selden M. le Dr. Stark, dans son savant ouvrage sur l'Orient hellénistique, cité plus haut, identifie (t. 1., p. 571) \emph{Aglibol} avec \emph{Élagabal}, et en fait le dieu du soleil. Avellino ne commet pas cette erreur ; il reconnaît (Mémoire cité, p. 25) le dieu Lunus sous le nom d'Aglibol, comme Halley et Barthélemy l'avaient reconnu sous le nom d'\emph{Iaribolus}.} qu'à Palmyre on donnait à la Lune ou plutôt au dieu Lunus. Dans le texte grec, comme dans le texte palmyrénien, Aglibol est nommé le premier ; et, bien que la dédicace s'adresse simultanément à Aglibol et à Malachbel, c'est-à-dire à Lunus et au Soleil, Tite Aurèle Héliodore Hadrien, en présentant pour offrande une statue d'argent, accomplissait sans doute un vœu particulier fait à Aglibol ; car l'argent, personne ne l'ignore, était partout le métal consacré à la lune, comme l'or au soleil.\footnote{Si le vœu s'était adressé au Soleil, c'est-à-dire à Malachbel, l'offrande aurait été une statue d'or ou dorée. Telles étaient les statues consacrées à Apollon, à Mithra ; et nous apprenons de Pausanias (5., 19., 3) que, dans l'Altis, on voyait un \emph{xoanon} d'Apollon dont la tête était dorée.} Cette supposition semble être confirmée par le soin que, sur notre bas-relief, on avait eu d'orner d'une couronne à pointes la tête d'Aglibol et de le revêtir du costume militaire des empereurs romains, tandis que Malachbel, vêtu à l'orientale, s'y montre sans armes et sans couronne. C'est une flatterie qui s'adresse aux dominateurs de la Palmyrène, et qui ne peut nous laisser aucun doute, ce me semble, quant à la prééminence accordée ici à Aglibol sur Malachbel, conséquence naturelle du changement de sexe qu'une prescription hiératique, passablement extraordinaire, avait fait subir au Soleil et à la Lune.\footnote{J'ai déjà dit ailleurs (\emph{Recherch. sur Vénus}, p. 20) que cette prescription hiératique me paraît avoir été elle-même la conséquence forcée d'une révolution politique qui, ayant amené sur le trône d'Assyrie une femme ambitieuse et douée d'une haute intelligence, Sémiramis peut-être ou toute autre, avait causé une grave perturbation dans les institutions civiles en faisant descendre à une position subordonnée le mari de la reine. Dès lors, un des rapports les plus essentiels qu'on avait précédemment établis entre les institutions civiles et les institutions religieuses se trouvait tout à coup rompu par la force et la violence. Pour rétablir l'assimilation de la reine et du roi aux dieux créateurs du monde, il fallut sans doute faire intervenir un décret sacerdotal, qui transporta à la divinité féminine la prééminence jusqu'alors attribuée à la divinité mâle, et qui, en même temps, déclara la lune mâle et le soleil femelle, changeant ainsi le sexe qu'auparavant on avait assigné à ces deux astres. Un nombre considérable de monuments figurés attestent cette innovation religieuse et confirment sur ce point le témoignage de notre bas-relief palmyrénien. Et, chose aussi remarquable que conforme aux règles suivies par le sacerdoce dans l'antiquité, le changement de sexe du soleil et de la lune se révèle avec non moins d'évidence dans les langues d'Orient et dans celles d'Occident. En Orient, selon que les peuples se soumirent ou non à la révolution politique et à la réforme religieuse dont j'entends parler, nous voyons uniformément, dans leurs idiomes respectifs, le mot lune prendre le genre masculin, ou conserver le genre féminin, et les mots soleil et amour, qui souvent sont synonymes, prendre le genre féminin, ou conserver le genre masculin. En Occident, la langue grecque, la langue latine et les idiomes qui en sont dérivés ont conservé le mot lune au féminin, et les mots soleil et amour au masculin ; tandis que les langues germaniques et les langues slaves, dont la formation et l'importation d'Asie en Europe sont d'une époque beaucoup moins ancienne, affectent à la lune le genre masculin, et aux mots soleil et amour le genre féminin.} Toutefois, si après avoir remarqué que, dans la dédicace bilingue, Aglibol prend le pas sur Malachbel, on était tenté de croire que celui-ci, dans les sculptures du bas-relief, occupe la place d'honneur, puisqu'il est placé à la droite, et Aglibol à la gauche de Vénus, représentée sous l'emblème du cyprès,\footnote{Pl. 3., n° 1.} les médailles asiatiques dont je parlerai plus loin nous prouveraient qu'on ne peut attacher aucune importance à cette disposition, ni en tirer une conséquence quelconque ; car elles nous montrent le soleil et réciproquement la lune, tantôt à la droite, tantôt à la gauche de Vénus. Malgré cette variante, les artistes ne croyaient pas sans doute violer le dogme qui assigne à Mylitta une place habituelle au ciel entre le soleil et la lune, les deux agents de la création et de la reproduction, les deux astres chargés de donner à la terre la lumière, la chaleur, l'eau et la fécondité, les deux astres que l'hérésiarque Bardesane, au rapport d'Abou'lpharage,\footnote{Pag. 125.} appelait : \emph{le père et la mère de la vie}. Un dogme semblable est formellement énoncé à l'égard de Mithra dans les livres sacrés des Perses.\footnote{\emph{Zend-Avesta}, t. 1., 2\textsuperscript{e} partie, p. 28 ; t. 2., p. 13.} De là ces nombreux monuments qui représentent le dieu ayant à sa droite le buste ou le char du Soleil, et à sa gauche le buste ou le char de la Lune. De là aussi ce commandement de la liturgie du Zend-Avesta, qui, certainement emprunté aux Chaldéens d'Assyrie, veut que le soleil et la lune soient invoqués en même temps que Mithra dans les prières composées en l'honneur de ce dieu.\footnote{\emph{Ibid.} t. 2., p. 15, 16 ; 204-232.} De plus, le Zend-Avesta renferme deux prières particulières (\emph{néaeschs}) qui s'adressent, lune au soleil,\footnote{\emph{Ibid.} p. 8-14.} l'autre à la lune,\footnote{\emph{Ibid.} p. 16-19.} et il impose aux mazdéiesnans l'obligation de réciter entre ces deux prières le \emph{néaesch} et l'\emph{iescht de Mithra}.\footnote{\emph{Ibid.} p. 15, 16.} Ne nous étonnons donc pas de voir un habitant de Palmyre qualifier le soleil et la lune \emph{dieux de la patrie}. Ils reçoivent la même qualification, non-seulement dans une autre inscription lapidaire trouvée à Palmyre,\footnote{Wood, \emph{Rud. Palmyr.} p. 27, n° 4 (cf. \emph{Philosoph. Transact.} vol. 48., p. 691, n° 4). \emph{Corp. inscript græcar.} vol. 3., n° 4480.} mais aussi dans les trois inscriptions citées,\footnote{Ci-dessus, p. 17.} qui proviennent d'une montagne de Syrie, appelée \emph{Scheikh Béréquet}. Au témoignage des monuments de tout genre qui attestent combien était répandu, dans l'Asie occidentale, le culte de ces deux divinités, vient s'ajouter le témoignage, pour ainsi dire, officiel qui résulte d'une médaille impériale, frappée à Tripolis, en Phénicie,\footnote{Pellerin, \emph{Mélang. de div. méd.} 1., p. 343. Eckhel, \emph{D. N.} 3., p. 374. Mionnet, \emph{Descript. de méd.} 5., 400, 401, n° 426.} médaille dont le type nous offre, au revers de la tête d'Antonin le Pieux, les images debout du Soleil et de la Lune personnifiés et entourés de la légende grecque : \textgreek{ΗΛΙΟϹ · ϹΕΛΗΝΗ · τΡΙΠΟΛιτων}. Les formules latines de consécration, SOLI • INVICTO • ET • LVNAE • AETERNAE\footnote{Gruter, \emph{Inscr. antiq.} p. 33., nn. 5, 6.} ; AETERNITATI • SACR • SOLI • ET • LVNAE\footnote{\emph{Id. ibid.} p. 32., n. 9.} ; SOLI • AETERNO • LVNAE,\footnote{\emph{Id. ibid.} n. 10.} ou simplement, SOLI • ET • LVNAE,\footnote{\emph{Id. ibid.} p. 31., nn. 11, 12, 13. Maffei, \emph{Mus. veron.} p. 81, n. 10.} se rencontrent bien souvent sur les monuments lapidaires, et attestent, à leur tour, que les Romains, comme les Grecs, avaient adopté, dans l'Asie occidentale, l'usage d'adresser nominativement leurs vœux au Soleil et à la Lune divinisés.

De même que notre bas-relief palmyrénien, le troisième monument de Syrie dont il me reste à parler nous montre ces deux divinités formant une triade secondaire avec la Vénus asiatique représentée sous le symbole du cyprès pyramidal. Mais ici le Soleil et la Lune, au lieu d'être figurés sous une forme humaine, sont eux-mêmes représentés chacun sous l'emblème de cet arbre sacré, symbole de vie, preuve nouvelle et irrécusable que le cyprès sculpté sur le bas-relief et sur l'autel palmyrénien du Musée capitolin est bien réellement, comme je l'ai avancé, l'image symbolique de la Vénus qu'adoraient les Syriens. Le monument qui nous fournit cette preuve est un petit trône votif, de forme asiatique et de bronze, que M. Henri Guys, ancien consul de France à Beyrouth, a rapporté de Syrie, en 1848, et qu'il a cédé au cabinet des médailles et antiques de la Bibliothèque impériale. Ce trône n'a que 28 à 29 centimètres de haut, sur 12 à 13 de large. Il est supporté par quatre griffes de lion, et orné, en relief, sur chacune de ses faces latérales, d'un lion debout, marchant de gauche à droite et formant une saillie très-prononcée. La face antérieure est décorée de trois cyprès pyramidaux également en relief, placés sur un même plan. Celui qui s'élève au milieu des deux autres les dépassent d'un tiers environ. Ce monument votif\footnote{Les témoignages ne manquent pas pour prouver que souvent les anciens consacraient, dans les temples, des trônes d'or, d'ivoire ou de bronze. Je désigne celui-ci comme un monument votif, parce qu'il ne m'est pas démontré que, malgré la dépression circulaire qui se fait remarquer dans l'intérieur, une statuette ait jamais été placée sur ce petit trône.} ne doit remonter qu'à la dernière période de la domination romaine sur la Syrie. Le style en est lourd, incorrect, et entaché de tous les défauts qui accusent une époque de décadence. L'Académie en jugera par les deux dessins que j'ai placés sous les n°\textsuperscript{s} 1 et 2 de la planche 5, et qui représentent ce petit trône, l'un de face, l'autre de profil. Mais, à mes yeux, il rachète toutes ses imperfections par le fait nouveau qu'il apporte à la science archéologique ; et ce fait, bien qu'il soit unique jusqu'à ce jour, est néanmoins en parfaite harmonie avec les témoignages qui résultent des documents écrits. Lucien, ou plutôt l'auteur inconnu du célèbre Traité sur la Déesse de Syrie,\footnote{Si je suis bien informé, un savant helléniste, M. le professeur Immanuel Bekker, qui prépare une édition critique des œuvres de Lucien, a lu récemment à l'Académie royale de Berlin une dissertation composée dans le but d'établir que ce satyrique grec n'est pas l'auteur du traité \textgreek{Περὶ τῆς συρίης θεοῦ}.} la Vénus assyrienne, ne nous apprend-il pas\footnote{\emph{De Dea syria}, 31, 32 ; ed. Lehmann.} que, dans le temple d'Hiérapolis, la statue de cette antique divinité était assise sur des lions, et celle du dieu, son époux, assise sur des taureaux ? Et ces deux espèces de quadrupèdes ne représentent-elles pas, dans le langage symbolique, l'une le soleil, l'autre la lune\footnote{Voyez mes \emph{Recherches sur le culte de Vénus}, p. 220-227.} ? De plus, la déesse n'avait-elle pas la tête ceinte des rayons du soleil, en même temps qu'elle portait, parmi d'autres attributs, ceux d'Artémis ou Diane, et de Séléné ou la Lune\footnote{\emph{De Dea syria, loc. cit.}} ? De son côté, Macrobe a soin de nous dire que, chez les Assyriens, des lions étaient placés aux pieds d'Atergatis, par la même raison que, chez les Phrygiens, on attelait des animaux de cette espèce au char de la \emph{Mère des Dieux}, qui est la Terre.\footnote{« Sub eodem (Atargatis) simulacro species leonum sunt, eadem ratione, terram esse monstrantes, qua Phryges finxere Matrem Deûm, id est terram leonibus vehi. » (\emph{Saturnal.} 1., 23. ; edit. Varior.)} Or Atergatis, divinité également tellurique, est la même que la Déesse de Syrie. Strabon\footnote{... \textgreek{ἡ Βαμβύκη, ἣν καὶ Ἐδεσσαν καὶ Ἱεραν πόλιν καλοῦσιν, ἐν ᾗ τιμῶσι τὴν Συρίαν θεὸν τὴν Ἀταργάτιν}. (\emph{Geogr.} 16., p. 748, Casaub.)} et Pline,\footnote{« Bambycen, quæ alio nomine Hierapolis ... ibi prodigiosa Atargatis ... colitur. » (\emph{H. N.} 5., 19., 1.)} aussi bien que Macrobe, ne nous laissent aucun doute sur cette identité. Le passage cité du traité attribué à Lucien reçoit même une éclatante confirmation du témoignage d'un moyen bronze\footnote{Pour me conformer à l'usage suivi jusqu'à présent, j'emploie, dans ce mémoire, les dénominations vulgaires de \emph{petit bronze, moyen bronze, grand bronze, médaillon}, quoique je reconnaisse combien M. le comte Borghesi est fondé à proposer de leur substituer les dénominations scientifiques qui résultent des divisions et subdivisions légales de la monnaie chez les Romains. Voyez la lettre de ce savant archéologue à M. l'abbé Cavédoni, datée du 13 août 1849, et insérée par ce dernier dans sa \emph{Numismatica biblica} (Modène, 1850, in-8°), p. 111-136.} frappé en l'honneur d'Alexandre Sévère à Hiérapolis même,\footnote{Ce nom, sur les médailles, est écrit : \emph{Hiéropolis}.} dans la Cyrrhestique. Il représente,\footnote{Cette belle et rare médaille, qui appartient au cabinet impérial de Vienne, est figurée sous le n° 1 de la planche 3 B de mes \emph{Recherches sur Vénus}. Les numismates ne l'ont pas toujours interprétée avec une connaissance suffisante de la mythologie orientale, bien que Eckhel, leur illustre maître, eût tracé la route à suivre, en rapprochant de cette médaille le passage cité du traité \emph{De Dea syria} (Voy. \emph{D. N.} 3., p. 262, 263). Il ne lui avait pas échappé non plus (\emph{ibid.} p. 296) que la belle tétradrachme de Caracalla, dont je parle quelques lignes plus bas, se rapporte au même passage. Toutefois le revers de cette tétradrachme n'est pas absolument semblable à celui de la médaille impériale de Hiérapolis, quoiqu'on pût conclure le contraire de ces paroles d'Eckhel (\emph{ibid.} p. 263) : « In quo (tetradrachmo) eadem omnino est aversa, nisi quod pro leone inferne conspicitur aquila. » C'est une légère inadvertance ; car plus loin (\emph{ibid.} p. 296) il décrit ainsi ce même revers, d'après Pellerin : « Sacellum, intra quod signum militare, numine hinc et inde adsidente, quorum uni adstat leo, alteri taurus ... »} au revers de la tête radiée de ce prince, et entourés de la légende \textgreek{ΘΕΟΙ · ϹVΡΙΑϹ · ΙΕΡΟΠΟΛΙΤΩΝ}, la Déesse de Syrie assise sur une couple de lions, et son divin époux, Baal ou Bel, assis sur un couple de taureaux. Entre ces deux divinités s'élève un \emph{sacellum} surmonté d'une colombe, oiseau de Vénus ; un lion placé au bas de cet édifice achève de le caractériser comme un sanctuaire particulièrement consacré à la déesse.\footnote{Ce lion marche de gauche à droite, comme chacun des deux lions qui ornent le petit trône votif figuré ici sous les n°\textsuperscript{s} 1 et 2 de la pl. 5.} Ce type, à quelques légères différences près, se reproduit au revers d'une belle tétradrachme de Caracalla, frappée à Antioche de Syrie, décrite premièrement par Pellerin, puis par Eckhel\footnote{\emph{D. N.} t. 3., p. 296. --- Cette tétradrachme paraît être restée unique, et j'ignore même entre les mains de qui elle se trouve ; le médaillier de la Bibliothèque impériale ne la possède pas.} ; mais, à mon grand étonnement, omise par Mionnet dans sa Description de médailles et dans son Supplément.

D'autre part, quelques médailles impériales, frappées dans l'île de Cypre\footnote{Voy. mes \emph{Recherches sur Vénus}, pl. 1., n° 12.} et à Sardes,\footnote{Mionnet, \emph{Supplém.} 7., 422, n° 481 ; \emph{Descript. de méd.} 4., 125 et 126, n° 714.} ont pour type un cône, symbole de Vénus, placé au milieu du temple de la déesse, et ce temple est surmonté de l'astérisque du soleil, superposé au croissant de la lune. Un cône, d'origine asiatique, que j'ai publié\footnote{\emph{Recherch. sur Vénus}, p. 32-118 ; pl. 1., n°\textsuperscript{s} 1 et 1 \emph{a}.} il y a déjà plusieurs années, nous montre la Vénus orientale, hermaphrodite, tenant de la main droite un serpent, symbole de vie,\footnote{\emph{Ibid.} p. 35 et 36.} la tête couronnée des rayons du Soleil, et de la main gauche un autre serpent, celui-ci surmonté d'un croissant. Une pierre gravée, que j'ai également publiée,\footnote{Voyez \emph{Recherch. sur Vénus}, pl. 14. G, n° 4.} représente Vénus assise sur un trône, ayant à sa droite le soleil, à sa gauche la lune.\footnote{Les monuments de l'art établissent de même des rapports directs entre d'autres divinités génératrices et le soleil et la lune. Les représentations figurées de Zeus ou Jupiter, de Héra ou Junon, en fournissent des exemples aussi bien que celles de Pan, le dieu créateur par excellence. Parmi ces dernières, je citerai un vase peint d'Apulie, très-curieux et publié par M. Éd. Gerhard (\emph{Lichtgottheiten}, p. 8 ; pl. 3., n° 3) ; il représente Hélios et Séléné, tous deux debout sur un quadrige placé dans une barque, selon un usage hiératique, que l'Asie occidentale peut revendiquer tout aussi bien que l'Égypte. Le char est précédé par le dieu Pan, qui lui sert de guide. Sur un autre vase (cratère) peint, M. Th. Panofka (\emph{Vases Blacas}, n°\textsuperscript{s} 17 et 18) croit reconnaître le même dieu placé entre le quadrige du Soleil et la Lune personnifiée.} Plusieurs médailles impériales d'Apollonia d'Illyrie, dont il sera question plus loin, portent, à leur revers, l'image de la déesse pareillement assise sur un trône, mais placée au milieu d'un temple flanqué de deux cyprès, qui sont, n'en doutons pas, le cyprès du soleil et celui de la lune, et surmonté de cinq autres cyprès, qui ne peuvent être que les représentants des cinq autres planètes.\footnote{Le n° 4 de la planche 8 ci-jointe offre un exemple de ce type. Ailleurs Vénus, comme déesse à la fois solaire et lunaire, est représentée assise sur un trône, ayant à ses pieds deux griffons, symboles du soleil et de la lune, et portant sur la main droite une chouette, emblème de la nuit. (Voyez mes \emph{Recherches sur Vénus}, pl. 14. H, n° 10.)} Bientôt nous verrons sur des bas-reliefs d'époque romaine Mithra placé de même entre sept cyprès, emblèmes des sept planètes, ou simplement entre deux cyprès plantés, l'un à côté du buste de la Lune, l'autre à côté de celui du Soleil. Enfin, au revers de deux médailles impériales frappées à Perga (Pamphylie), en l'honneur d'Aurélien, et figurées sous les n°\textsuperscript{s} 1 et 2 de ma planche 14, nous voyons à l'entrée du temple d'Artémis, divinité qui se confond avec Vénus, un cône, image symbolique de chacune de ces deux déesses, placé entre deux cyprès plantés, l'un sous l'astérisque du soleil, l'autre sous le croissant de la lune.\footnote{Quelquefois le cône, emblème d'Artémis, est placé entre le Soleil et la Lune représentés tout à la fois sous leur forme d'astres et sous une forme humaine (voy. \emph{Recherches sur Vénus}, p. 48 et 49, pl. 1., n° 13). C'est aussi sous l'emblème d'une pierre conique que le dieu \emph{Men} ou \emph{Lunus} était adoré à Carrhes, en Mésopotamie. Nous en avons la preuve sur des médailles coloniales, frappées dans cette ville en l'honneur de Septime Sévère, et longtemps attribuées à \emph{Ælia Capitolina}, l'ancienne Jérusalem. Lorsqu'en 1847, M. Adolphe de Rauch faisait insérer dans le 19\textsuperscript{e} volume des \emph{Annales de l'Institut archéolog.} (p. 282 ; pl. P, n° 5) une description et un dessin d'une monnaie coloniale frappée à Carrhes, sous le règne de Septime Sévère, il la croyait inédite, et pensait que, pour la première fois, on avait sous les yeux une représentation du temple érigé dans cette ville en l'honneur du dieu Lunus. Mais il ignorait que, dès l'année 1837, j'avais publié, sous le n° 9 de la pl. 15 de mes \emph{Recherches sur Vénus}, une pièce coloniale, inédite, de la même ville et à l'effigie du même empereur, qui faisait connaître la forme de ce temple. Par une circonstance fortuite, dix ans plus tard, je publiais, en même temps que M. de Rauch, dans le volume cité des \emph{Annales} (p. 45 ; pl. C, n° 3), une autre monnaie coloniale de Carrhes, également inédite, très-analogue à celle que possède cet habile numismate, mais d'un coin différent, et dont le revers, mal frappé, offre en entier la facade du temple de Lunus, sans permettre de lire la légende. Ce dernier défaut se retrouve à peu près sur la pièce que j'avais publiée en 1837 et reproduite en 1847, sous le n° 10 de la planche 38 des \emph{Monuments inédits de l'Institut archéolog.} Il nous explique pourquoi ces deux pièces, qui appartiennent au riche médaillier de la Bibliothèque impériale, y sont restées classées à \emph{Ælia Capitolina}, jusqu'au moment où M. de Rauch, ayant sous les yeux un exemplaire parfaitement conservé, l'a judicieusement attribué à Carrhes devenue colonie romaine sous le nom d'\emph{Aurelia Carrhenorum}. C'est avec leur fausse attribution à \emph{Ælia Capitolina} que j'avais publié les deux pièces que possède la Bibliothèque impériale ; et, de plus, trompé, comme mon dessinateur, par l'état fruste du revers d'une de ces monnaies, j'ai commis l'erreur de prendre pour des cyprès les deux enseignes militaires entre lesquelles est placé l'emblème conique du dieu Lunus. Le bel exemplaire publié par M. de Rauch m'a fait reconnaître ma méprise ; et si je reproduis ici, sous le n° 4 de la planche 6, celui dont j'avais donné un dessin en 1837 et en 1847, c'est afin que l'Académie puisse juger de la fausse apparence qui a causé cette méprise. À mon grand regret, l'erreur s'est reproduite, en dernier lieu, dans une note que j'avais remise à mon savant ami M. le vicomte de Santarem au sujet de \emph{l'arbre du soleil} et de \emph{l'arbre de la lune} qui servent à caractériser l'Asie ou l'Inde sur plusieurs cartes géographiques du moyen âge. Cette note a été imprimée, sans que j'en aie pu voir et corriger les épreuves, dans le 3\textsuperscript{e} volume (p. 506-513) de l'important ouvrage de M. de Santarem, intitulé : \emph{Essai sur l'hist. de la cosmogr. pendant le moyen âge}, etc. Paris 1852, in-8°, avec atlas gr. in-fol.}

Il n'est donc pas douteux que, sur la face antérieure de notre trône votif,\footnote{Pl. 5., n°\textsuperscript{s} 1 et 2.} orné latéralement des deux lions de la Déesse de Syrie, les trois cyprès représentent symboliquement la déesse elle-même, le soleil et la lune, qui forment ici, comme sur le bas-relief palmyrénien, une triade secondaire, à la fois céleste et tellurique. Remarquons que l'auteur de ce petit monument, fidèle aux règles de l'art asiatique, a eu soin d'indiquer, dans cette triade, la supériorité des fonctions et du rang de la Déesse de Syrie, en donnant au cyprès qui la représente des proportions colossales, et aux deux cyprès qui représentent le soleil et la lune, de très-humbles proportions. Tel est aussi le contraste que nous offrent, sur les médailles citées de Perga, le cône colossal, qui tient lieu de l'image d'Artémis, et les deux petits cyprès plantés auprès de ce cône, et surmontés, l'un, de l'astérisque du soleil, l'autre, du croissant de la lune. Tel est encore le contraste qui se manifeste sur une médaille d'Ephèse,\footnote{Mionnet, \emph{Suppl.} 6., 156, n° 502.} où, au revers de la tête de Septime Sévère, nous trouvons le simulacre de Junon \emph{Pronuba} placé entre deux petites figures nues et assises à terre, au-dessus desquelles on voit, dans le champ, le soleil et la lune. Les graveurs cependant ne se sont pas toujours assujettis à la règle que je constate. C'est du moins ce qu'il est permis de croire, si l'on admet comme exact le dessin, plus que médiocre, que Pembroke\footnote{\emph{Numism. antiq.} pars 2., tab. 73.} a publié d'une monnaie autonome d'Apamée. Cette pièce, qui nous reporte en Syrie et qui me semble être restée unique,\footnote{Elle me paraît avoir échappé à l'attention de Mionnet.} aurait pour type, au revers de la tête laurée d'un personnage mâle, mais imberbe, trois cyprès\footnote{Ces trois cyprès, tels que les a reproduits le dessinateur de Pembroke, présentent chacun, au-dessous de la naissance de leur feuillage, une saillie ou une espèce de crochet que forme un court rejeton et qui se retrouve à la même place, soit sur le tronc du plus grand des trois cyprès qui ornent le trône votif de la Déesse de Syrie (pl. 5., n° 5), soit sur le tronc d'un cyprès planté à côté d'Apollon, au milieu d'un disque d'argent dont il sera question plus loin (voy. ma planche 20., n° 5).} de taille égale, plantés sur un même rang. Je n'ai pas vu l'original, et je ne puis dire si, comme il est arrivé quelquefois, le dessinateur n'a pas pris ici trois épis de blé pour trois cyprès. Dans le cas où il ne se serait pas trompé, il y aurait un rapprochement intéressant à faire entre ces trois cyprès gravés au revers d'une monnaie frappée dans une ville de Syrie, et les trois cyprès placés sur la face antérieure du trône votif trouvé dans la même contrée et consacré à la divinité qui s'appelait la Déesse de Syrie.

Les trois monuments syriens, où Vénus vient de se révéler à nos yeux sous l'emblème du cyprès pyramidal, me conduisent tout naturellement à rechercher si quelque trace d'une représentation semblable, ou, du moins, de l'attribution du cyprès à Vénus, ne se découvre pas dans la Babylonie et l'Assyrie, ces deux contrées célèbres\footnote{Strabon (\emph{Geogr.} 16.) comprend Babylone dans l'Assyrie ; mai les inscriptions gravées en caractères cunéiformes sur les monuments antiques de la Perse donnent constamment des noms distincts à l'Assyrie et à la Babylonie.} qui furent le berceau du culte de la déesse, et qui voyaient des forêts de cyprès croître spontanément sur leur sol. Je ne m'arrêterai pas à examiner quelle présomption, à cet égard, il serait permis de tirer d'un passage où Strabon,\footnote{\emph{Geogr.} 16., p. 738.} décrivant l'Atacène, rapporte qu'aux environs d'Arbèles on trouve un \emph{cyparissôn} ou \emph{cupressetum} (\textgreek{ὁ κυπαρισσών}), une source de naphte, des feux sacrés (\textgreek{τὰ πυρά}) et un temple d'\emph{Anæa} (\textgreek{τὸ τῆς Ἀναίας ἱερόν}). Cette divinité est la même que l'\emph{Anahid} ou la Vénus-Mylitta des Babyloniens et des Assyriens ; mais, bien qu'un illustre géographe allemand, M. Carl Ritter,\footnote{\emph{Die Erdkunde}, 11., p. 577, 578.} n'ait pas hésité à croire qu'il s'agit ici d'un bois de cyprès consacré, comme le temple, à la déesse Anæa, le texte du géographe d'Amasie ne me semble pas établir explicitement le fait. Observons toutefois qu'à l'exemple des autres écrivains grecs, Strabon attache ici au mot \textgreek{κυπαρισσών} un sens qui implique l'idée d'un bois sacré.\footnote{On ne peut appliquer le même sens aux plantations de cyprès qu'Ammien Marcellin (24., 6., 3) vit dans les environs de Ctésiphon et dont il parle en ces termes : « In agro consedimus opulento, arbustis et vitibus et cupressorum viriditate lætissimo. »} C'est un pareil \emph{cupressetum} que nous offre un bas-relief qui décore une portion de la face sud-est du palais assyrien découvert à Khorsabad par M. Botta.\footnote{\emph{Monum. de Ninive}, pl. 107., n° 13, et pl. 114.} On y remarque une colline plantée de cyprès, au milieu desquels est placé un pyrée. Quatre colombes voltigent entre ces arbres, et nous rappellent que le nombre quatre et la colombe étaient consacrés à Mylitta, comme à Vénus. Un autre bas-relief de Khorsabad\footnote{\emph{Ibid.} pl. 141., salle 13., n° 4.} nous fournit une preuve bien plus certaine de l'attribution du cyprès à la déesse ; mais la scène qu'il représente nous transporte évidemment hors de l'empire assyrien ; car elle fait partie d'un grand tableau destiné à perpétuer le souvenir de la prise d'une ville ou d'une forteresse par l'armée du roi d'Assyrie. Dans cette scène, nous voyons des soldats assyriens occupés à démolir et à piller plusieurs édifices, parmi lesquels on distingue un petit temple, à fronton triangulaire, qui, sans aucun doute, était consacré à la Vénus asiatique, puisque, devant cet édifice, est placé, sur un socle, le groupe d'une vache allaitant son veau. La plupart des archéologues reconnaissent aujourd'hui avec moi que, chez les peuples de l'Asie occidentale et particulièrement chez les Phéniciens et dans leurs colonies, ce groupe symbolique représentait Vénus et l'Amour\footnote{Voy. mes observations à ce sujet dans le recueil des \emph{Mémoires de l'Académie}, nouvelle série, t. 15., 2\textsuperscript{e} partie, p. 80, note 6 ; et dans mes \emph{Recherches sur le culte de Vénus}, p. 165-168 ; Atlas, pl. 14. G, n°\textsuperscript{s} 9, 10, 12, 13, 18 ; pl. 14. H, n°\textsuperscript{s} 2-6, 9. L'obélisque grec découvert dans la vallée du Xanthus, en Lycie, par M. Ch. Fellows (\emph{An account of discoveries in Lycia}, p. 169 et suiv. pl. pour la p. 170), met hors de toute contestation mon interprétation du groupe dont il s'agit. Un des beaux bas-reliefs qui ornent cet obélisque nous montre, sur la face occidentale, deux images de Vénus placées en regard l'une de l'autre. La première nous offre, sous une forme purement humaine, l'Aphrodite des Grecs, assise et entièrement vêtue. La seconde, posée sur un cippe, représente Astarté figurée, à la manière orientale, par le groupe symbolique d'une vache allaitant son veau. A ce témoignage, si decisif, il ne me reste plus qu'à ajouter celui qui résulle d'un fragment de cylindre assyrien, où l'on voit aussi une vache allaitant son veau. Ce cylindre provient des secondes fouilies de M. Layard.} ; de plus, des boucliers coniques, dont le centre est orné en saillie d'une tête de lion,\footnote{Cet ornement nous rappelle les têtes de lion placées en saillie sur les murs extérieurs du temple de Dendérah consacré à Vénus-Hâthôr.} animal consacré à la déesse, sont suspendus aux fenêtres du temple. Ne nous étonnons donc pas si cet édifice, comme le trône votif de la Déesse de Syrie,\footnote{Ci-dessus, p. 51 et suiv. pl. 5., n°\textsuperscript{s} 1 et 2.} fut décoré de trois cyprès pyramidaux : l'un est placé au sommet du fronton, les deux autres s'élèvent derrière les deux personnages qui gardent l'entrée du temple.\footnote{Ces deux personnages nous révèlent peut-être l'origine d'un usage qui nous est attesté par les médailles sassanides où nous voyons l'autel du feu également gardé par deux personnages, sous les traits de qui on a même cru reconnaître le roi et la reine de Perse.} Quelques-unes de ces particularités, de même que l'ornement en échiquier ou en treillis\footnote{Un semblable échiquier ou treillis sert fréquemment de soubassement aux sujets religieux représentés sur des cônes, des \emph{métroïdes} et des scarabées de travail phénicien.} qui couvre toute la surface du tympan, me portent à conjecturer que la ville ou la forteresse qui renfermait ce temple était située en Phénicie ou, du moins, sur un territoire soumis à la domination phénicienne.

Si nous ne reconnaissons avec certitude le cyprès sur aucun des bas-reliefs qui ornent les palais assyriens découverts à Nimroud et à Koyoundjuk par M. A. H. Layard\footnote{Ce zélé voyageur désigne, avec raison, sous le nom de sapins (\emph{fir-trees}) les arbres plantés sur le sol d'une contrée montagneuse, traversée par une armée assyrienne dont la marche est représentée sur un des bas-reliefs qu'il a découverts à Koyoundjuk (\emph{Monuments of Nineveh}, pl. 81). Il remarque judicieusement (\emph{ibid.} p. 18) que cette contrée devait être située à une grande distance de Ninive, et au nord de l'empire assyrien, puisque le sapin ne croît pas au-delà des limites septentrionales de l'Assyrie.} ; et si les bas-reliefs du palais de Khorsabad nous offrent les deux seuls exemples cités de l'emploi du cyprès avec un sens religieux,\footnote{Les cyprès que l'on voit sur d'autres bas-reliefs du palais assyrien de Khorsabad (\emph{Monum. de Ninive}, pl. 108., 109., 111-113.), sont plantés, comme ornement, dans un parc où le roi se livre à l'exercice de la chasse. Peut-être cette scène fait-elle partie d'une fête analogue à la \emph{fête des rois}, qui, en Perse, selon Firdousi, nous le verrons plus loin, se célébrait anciennement dans la partie des jardins ou du parc des palais des rois d'Irân qui était ornée de \emph{hauts cyprès}.} n'oublions pas que nous sommes loin de posséder la série entière des sculptures qui décoraient ces vastes et somptueux édifices\footnote{Au moment même où ces lignes s'impriment, je reçois de M. Layard un exemplaire d'un beau volume qu'il vient de publier (Londres, 1853, gr. in-8°, fig.) et qui contient la relation de sa seconde expédition dans l'Asie occidentale. Cet important ouvrage, intitulé : \emph{Discoveries in the ruins Of Nineveh and Babylon}, etc. est orné des dessins d'un grand nombre de monuments assyriens, parmi lesquels je remarque (p. 232) deux bas-reliefs tirés des ruines de Koyoundjuk et représentant chacun un jardin royal, planté de cyprès et d'une autre espèce d'arbre, moins facile à reconnaître. Je trouve en même temps, à la p. 604, le dessin du cylindre assyrien dont j'ai parlé ci-dessus (p. 59 et 60, note 4).} ; n'oublions pas non plus qu'il nous reste à interroger les petits monuments exhumés des ruines de Babylone et de Ninive. Parmi ceux-ci, plusieurs cylindres et plusieurs coupes, dont le témoignage n'a jamais été invoqué dans la question qui nous occupe, présentent des exemples certains de l'attribution du cyprès à la déesse Mylitta. Ils peuvent ainsi, on va le voir, suppléer aux documents écrits et aux grands monuments de sculpture qu'ont fait disparaître, depuis tant de siècles, les révolutions politiques et les bouleversements physiques qui détruisirent de fond en comble l'empire assyrien, ses institutions, ses belles et vastes cités, ses édifices publics, réputés si longtemps, les uns et les autres, les plus étonnantes merveilles de l'art et de la magnificence de l'Orient. Un passage important, recueilli par Moïse de Khoren, mais jusqu'à ce jour resté inaperçu et sans commentaire, nous permettra aussi de constater que, dès une époque très-reculée, le culte de Mylitta et du cyprès pyramidal fut porté du territoire assyrien dans la partie de l'Asie antérieure qui successivement reçut les noms de Haïgasdan et d'Arménie.

Je place sous les yeux de l'Académie les dessins de quatre cylindres assyriens,\footnote{Pl. 9., n°\textsuperscript{s} 1, 2, 4 et 5.} où la présence du cyprès au milieu de scènes d'initiation,\footnote{Ces initiations, en Orient, comme chez quelques peuples d'Occident, avaient lieu dans des gtottes ou des antres ; et si je fais remarquer ici que les cyprès, dont l'Odyssée (E, 63-65) vante la bonne odeur, ombrageaient l'entrée d'une grotte, c'est que ce poème renferme plus d'une allusion à des initiations ou à des métamorphoses qui faisaient partie d'une institution de mystères d'origine asiatique. Je reviendrai plus loin sur cette observation.} sur lesquelles je me réserve de m'expliquer ailleurs, nous autorise à reconnaître, comme nous l'avons fait sur les trois monuments syriens décrits plus haut, que le cyprès est l'image symbolique de la déesse qui présidait aux mystères. Le premier de ces cylindres appartenait autrefois à M. J. Robert Steuart. J'ignore en quelles mains il est passé depuis la mort bien regrettable du possesseur et la vente publique de sa précieuse collection. Deux autres\footnote{Pl. 9., n°\textsuperscript{s} 2 et 5.} sont déposés au Musée britannique ; le quatrième\footnote{\emph{Ibid.} n° 4.} se conserve au cabinet des médailles et antiques de la Bibliothèque impériale. Après ces quatre cylindres, je pourrais en produire ou en citer beaucoup d'autres qui offrent, comme ici, la représentation de Vénus-Mylitta sous l'emblème d'un cyprès, dont la forme est plus ou moins conventionnelle.\footnote{Les cyprès des jardins royaux représentés sur les bas-reliefs de Khorsabad, sur les coupes de bronze ou d'argent doré et sur les plats d'argent doré dont il sera question plus bas, affectent aussi des formes conventionnelles. Ceux qui sont sculptés sur les murs de Persépolis se rapprochent un peu de la nature.} Il me suffira, sans doute, de dire qu'il en existe un grand nombre, dans les collections publiques\footnote{Le Musée royal de la Haye en possède, à lui seul, sept ou huit de cette catégorie. Il s'en trouve plusieurs aussi dans le musée du Louvre et dans le Musée britannique.} ou privées, depuis que les fouilles et les voyages en Orient ont enrichi l'Europe d'une masse considérable de cylindres assyriens, babyloniens, phéniciens ou persépolitains, petits monuments qui étaient encore très-rares à l'époque où feu M. Cl. Rich en recueillait une trentaine à Bagdad et dans les ruines de Babylone et de Ninive. Toutefois, parmi ceux qu'on possède, il ne serait pas facile d'en indiquer plusieurs qui pussent présenter un second exemple d'une particularité intéressante que nous offre un cinquième cylindre, figuré sous le n° 3 de la planche 9. Celui-ci, qui, de même que le n° 4, appartient à la Bibliothèque impériale, porte deux cyprès au lieu d'un seul. Au-dessus de ces deux arbres, plantés à côté l'un de l'autre, est gravé un astérisque cruciforme, qui représente ici, comme sur beaucoup d'autres cylindres, l'étoile de Vénus-Mylitta.\footnote{\emph{Mém. de l'Acad. des inscript.} t. 17., 1\textsuperscript{re} partie, p. 370, 371 ; p. 3., n°\textsuperscript{s} 2 et 5.} Si nous nous reportons aux divers monuments figurés où déjà nous avons trouvé la déesse placée, ainsi qu'Artémis et Mithra, entre le cyprès du soleil et celui de la lune, nous ne pourrons mettre en doute que, sur notre cylindre, les deux cyprès plantés au-dessous de l'étoile de Vénus-Mylitta ne soient également les images symboliques des deux astres qui caractérisent la puissance génératrice de la Reine des cieux.

Sur un grand nombre de coupes de bronze, travaillées au repoussé et découvertes dans les nouvelles fouilles que M. Layard a faites dernièrement à Nimroud et à Koyoundjuk, pendant sa seconde expédition,\footnote{Je tiens ce renseignement de M. Layard lui-même, dont je ne saurais trop louer l'empressement à faire toutes les communications qui peuvent être utiles aux progrès de l'archéologie asiatique. Ce zélé voyageur publiera prochainement deux nouveaux ouvrages, qui feront connaître en détail au monde savant les résultats inespérés de ses secondes fouilles sur le sol assyrien. --- Depuis que ceci a été écrit, l'un de ces deux ouvrages a paru. Voy. ci-dessus, p. 61, note 2.} on voit, comme sur les cylindres ici figurés, des cyprès plantés au milieu de scènes d'initiations, qui appartiennent à la célébration des mystères de Mylitta. Le même fait se reproduit sur deux belles coupes d'argent doré, trouvées dans l'île de Chypre, et sur deux plats de même matière exhumés d'un des hypogées d'Étrurie. Je parlerai plus loin de ces quatre monuments, qui, sans aucun doute, ont une origine assyrienne ou plutôt phénicienne.

Quelques cylindres asiatiques, sans nous offrir le cyprès pyramidal planté à la place même que devrait occuper la statue de Mylitta, de Beltis, d'Astarté ou de Mithra, nous montrent ce symbole porté à la main par un prêtre, ou par une prêtresse, au milieu d'une scène d'initiation. Dans mes \emph{Recherches sur le culte de Mithra},\footnote{Atlas, pl. 36., n° 11.} j'ai publié un petit monument de cette catégorie. Je me réserve d'en parler dans le paragraphe de ce mémoire où j'aurai à entretenir l'Académie d'un bas-relief de Dendérah qui donne lieu à des rapprochements dignes d'attention.

Il serait fort difficile de préciser l'époque où, chez les Babyloniens et les Assyriens, commença l'usage d'introduire le symbole du cyprès dans la liturgie propre au culte des dieux qu'ils adoraient. Les premiers pas à faire dans une telle investigation seraient d'assigner une date certaine à l'institution primitive de ce culte, des mystères qui s'y rattachaient, et du culte de la divinité féminine qui, plus tard, sous le nom de Mylitta, présida à ces mystères. Or les renseignements nous manquent ; mais les traditions recueillies par les historiens de Syrie nous révèlent un fait qui, joint à la mention que nous trouvons de la déesse Astarté ou de la Reine des cieux, dans quelques parties anciennes de la Bible et dans les fragments qui nous restent de la théologie phénicienne rédigée par Sanchoniathon, atteste que, chez les Assyriens, le culte du cyprès remontait à une haute antiquité. Dans les extraits que Moïse de Khoren\footnote{Né vers l'an 370, mort vers 450.} nous a conservés du livre présenté à Valarsace,\footnote{Valarsace, premier roi de la branche des Arsacides d'Arménie, régna depuis l'année 149 avant notre ère jusqu'à l'année 127, selon Saint-Martin (\emph{Memoire sur l'Armenie}, t. 1., p. 410).} roi d'Arménie, par le Syrien Mar Iba Cadina ou Mar Ibas Katina,\footnote{On trouve aussi ce nom écrit \emph{Mar Apas Gadina, Mar Abas de Catina}, et, par abréviation, \emph{Mar Ibas} ou \emph{Maribase}.} et composé, assure-t-il, d'après les chroniques chaldéennes,\footnote{Voy. sur ce point les observations de M. Quatremère (\emph{Journal des Savants}, juin 1850, p. 364, 365), et celles de M. Éd. Dulaurier (\emph{Journ. asiatique}, janvier 1852, p. 19-21, note 1).} nous lisons\footnote{Mos. Choren. \emph{Histor. armen.} 1., 15., 19. ; édit. des frères Whiston.} que Sémiramis, à la tête d'une armée assyrienne, ayant porté la guerre chez les Arméniens et vaincu leur roi, Ara le Beau, qui fut tué dans le combat, confia à Gartos, fils de ce prince, le gouvernement de l'Arménie, et lui fit prendre le nom de son père. Quelques années après, Ara 2, fils d'Ara le Beau, se révolta contre Sémiramis, et périt, à son tour, dans une bataille que lui livra cette belliqueuse princesse. Il laissa un fils appelé Anouschavan, qui se rendit célèbre par son courage et sa sagesse, non moins que par ses paroles et ses hauts faits. Il était surnommé \emph{Sôs}, ajoute Mar Iba Cadina, parce qu'il avait été sacré au pied des cyprès\footnote{... « perchè secondo il rito era stato iniziato nei cipressi di Aramanec, » selon la traduction italienne de Moïse de Khoren par M. l'abbé Cappelletti (p. 49). Un usage analogue se retrouve chez les Israélites, à une époque qui, selon la chronologie de la Vulgate, correspond à l'an 1235 avant la naissance de J. C. Nous lisons, en effet, dans le livre des Juges (9., 6) : « Congregati sunt autem omnes viri Sichem el universæ familiæ urbis Mello : abieruntque et constituerunt regem Abimelech juxta quercum quæ stabat in Sichem. » Toutefois, à moins d'admettre que le couronnement d'Abimelech au pied d'un chêne était un emprunt illicite, fait, comme plusieurs autres, par les Israélites à quelque peuple païen, il est difficile de concilier ce passage du livre des Juges avec ceux du Deutéronome (16., 21) et de l'Exode (34., 13), qui contienneni la défense expresse de planter un bois (\emph{lucus}) ou seulement un arbre quelconque auprès de l'autel du Seigneur, et l'ordre non moins exprès de couper les bois sacrés dont les gentils entouraient les autels ou les statues des faux dieux.} jadis plantés par Arménag ou Aramanéag,\footnote{Ce prince, fils de Haïg, le chef de la première dynastie des rois d'Arménie, monta sur le trône l'an 2026 avant J. C. selon Saint-Martin, ouvrage cité, tom. 1., p. 407.} à Armavir, l'antique capitale de l'Arménie.\footnote{Voy. Moïse de Khoren, \emph{Hist. armen.} lib. 1., cap. 11., p. 32. Jean Catholicos, \emph{Hist. d'Arménie}, chap. 8., mss. de la Bibliothèque impériale, n° 91, fol. 18 ; trad. franç. de J. Saint-Martin, p. 10 et 11. J. Saint-Martin, \emph{Mém. sur l'Arménie} t. 1., p. 123, 124, 207 et 296. --- Armavir était située au nord-est de la ville moderne d'Érivan (voy. M. Carl Ritter, \emph{Die Erdkunde}, 11., 467, 571).} Pendant un long temps, poursuit l'historien syrien, les prêtres arméniens conservèrent l'usage de tirer des cyprès d'Arménag des augures favorables ou sinistres, selon qu'un vent doux ou un vent violent agitait les rejetons et les rameaux de ces arbres.\footnote{« Aræus Aræades, cum Semiramide bellum gerens, interficitur ; relicto filio, fortitudine et prudentia, cum dictis tum factis præcellentissimo, Anusavano illo, qui Sosius appellabatur, quippe qui pro eorum cærimoniis apud Armenaci cupressos in Armaviro consecratus fuerat ; quarum cupressorum surculis ramisque seu leni sive violento vento agitatis armenii flamines ad longum tempus in auguriis uti consueverunt. » (Mos. Choren. \emph{Hist. armen.} lib. 1., cap. 19.). Voy. la note ci-après.} Ici les frères Whiston, éditeurs de Moïse de Khoren, ont eu soin de placer une note\footnote{Mos. Choren. \emph{loc. cit.} not. 1 : « Vox hæc \emph{sos} in sermone armenio \emph{cupressum} arborem sonare videtur. » Cette note est reproduite par M. Joseph Cappelletti dans sa version italienne de Moïse de Khoren (Venez. 1841), où il traduit également \emph{sôs} par \emph{cyprès} (p. 49, note 2). Le mot \emph{sôs}, dans les anciens auteurs arméniens, s'emploie indifféremment pour désigner deux espèces d'arbres, chacune de forme pyramidale, le cyprès et le peuplier blanc (voy. le grand Dictionnaire arménien de l'Académie de Saint-Lazare. Venise, 1836). Mais, dans certaines parties de la version arménienne de la Bible, il n'a que le seul sens de cyprès. Dans quelques autres, on lui substitue le mot \emph{nodj} ou \emph{nouedj}. Aujourd'hui les Arméniens désignent le cyprès pyramidal sous le nom de \emph{selvi} ou \emph{servi}, altération du mot \emph{serv}, qui a la même signification dans les vocabulaires persans, et qui est aussi en usage chez les Arabes et chez les Turcs. Les Mékhitaristes, dans leur version italienne de Moïse de Khoren (\emph{Storia di Mose Corenese.} Venezia, 1841, p. 61), revue par M. N. Tommaseo, traduisent \emph{sôs} par platane. Je ne connais pas les raisons qui les ont portés à faire une substitution aussi arbitraire ; elle n'a point obtenu l'assentiment de M. Carl Ritter. Ce savant géographe s'en tient, comme moi, à la traduction des frères Whiston (voyez \emph{Die Erdkunde}, 11., 571). De son côté, M. le Vaillant de Florival, dans sa version française de Moïse de Khoren, imprimée avec le texte, la même année 1841, à la typographie arménienne de Saint-Lazare, sous les yeux et avec le concours des Mékhitaristes, traduit également \emph{sôs} par platane, à l'exemple de ces RR. PP. Mais, pressé de s'expliquer sur les motifs de la substitution de ce nom à celui de cyprès, il n'a pas fait difficulté de m'avouer qu'elle n'a point pour elle l'autorité d'un manuscrit de Moïse de Khoren, qui serait resté inconnu aux frères Whiston. S'il s'est déterminé à l'adopter, c'est dans la seule pensée que le passage où Mar Iba Cadina parle de la plus ou moins grande agitation des branches et des rejetons des arbres plantés par Arménag, s'applique plus convenablement au platane qu'au cyprès, dont le feuillage est peu mobile. M. Édouard Dulaurier, qui se livre avec autant d'ardeur que de succès à l'étude de la langue arménienne, a bien voulu, à ma prière, revoir ce passage sur le texte ; il reconnaît au mot \emph{sôs} le double sens de cyprès et de peuplier, et me fait remarquer que l'expression \emph{sôsavioun} ou \emph{sôsap'hioun} indique le bruissement on le murmure, et non l'agitation que cause dans le feuillage le souffle de l'air ou du vent. Le même orientaliste m'apprend que, dans l'\emph{Histoire de la croix d'Abarank}, saint Grégoire de Nareg (Œuvr. compl. Venise, 1827, in-8°, page 285), voulant désigner les deux espèces de bois dont Salomon s'était servi pour reconstruire le portique du temple, emploie les mots \emph{nodj} et \emph{sôs}, qui représentent ici le cyprès et le cèdre, conformément aux passages de la Bible où il est dit que, dans la réédification du temple, on avait fait entrer concurremment ces deux sortes de bois tirées du mont Liban. Mais aucune autorité ne justifie la traduction du mot \emph{sôs} par platane.} qui n'ajoute pas peu à l'importance du passage, en nous apprenant que, dans la langue arménienne, le mot \emph{sôs} signifie \emph{cyprès} ; car, chez les Grecs et les Romains, nous trouvons Vénus appelée d'un nom, \textgreek{Κύπρις} ou \emph{Cypris}, qui, je dois le répéter ici, est le même que ceux sous lesquels furent connus l'île phénicienne que tout le monde sait avoir été consacrée à la déesse, et l'arbre que je place au nombre des emblèmes qui servirent à la représenter symboliquement. Remarquons, en même temps, que si l'île de Vénus s'appelle \textgreek{Κύπρος} en grec, \emph{Cyprus} en latin, et l'arbre de Vénus, \textgreek{κυπάρισσος} ou \textgreek{κυπάριττος} et \emph{cyparissus} ou \emph{cupressus}, le cuivre, métal consacré à la déesse, comme le prouve la monture des objets qui appartiennent à son culte,\footnote{Les montures antiques des cylindres et des cônes assyriens que je rapporte aux mystères de Mylitta, sont toutes de cuivre, sans aucune exception. L'arme de Mithra s'appelle, dans le \emph{Zend-Avesta}, l'\emph{oreille de cuivre}. Saint Jérôme (\emph{in vita S. Hilarionis}, § 21), racontant comment saint Hilarion délivra une jeune fille des enchantements pratiqués par un jeune homme des environs de Gaza, qui n'avait pu la séduire, nous dit que ce jeune homme s'était rendu à Memphis, pour y apprendre des prêtres du temple d'Esculape l'art de faire succomber la vertu d'une vierge. Il revint au bout d'un an, prononça certaines paroles magiques sur le seuil de la porte de son amante, et y enfouit des lames d'une espèce de cuivre appelée \emph{æs cyprium}. Ces lames étaient couvertes de figures extraordinaires, « et portentosas figuras sculptas in æris cyprii lamina defodit ... » De telles superstitions avaient sans doute été importées autrefois à Memphis par les Phéniciens, avec le culte d'Esculape ou de la Vénus assyrienne.} est désigné en grec par les mots \textgreek{κύπριος χαλκός}, en latin par \emph{cuprum}.\footnote{Le latin \emph{cuprum}, l'allemand \emph{Kupfer} et l'anglais \emph{copper}, qui signifient également le cuivre, nous ramènent à la forme \emph{gopher} ou \emph{kopher} qu'avait reçue le nom du cyprès dans la langue hébraïque (voy. ci-dessus, p. 4 et 5, note 2).} De plus, l'île de Chypre, le cyprès et le cuivre ont retenu, dans toutes les langues de l'Europe, leur nom primitif, dérivé du grec ou du latin ; et ce nom, identique avec celui de la déesse Cypris,\footnote{Remarquons que la forme \emph{cypris} se retrouve dans les noms \emph{cipris, tziprus}, que porte le cyprès, chez les peuples slaves, et dans les adjectifs qui en dérivent, \emph{ciprisowi, ciprisni, tziprusi}. On emploie, substantivement aussi, \emph{ciprisowi}, et il signifie alors un \emph{cyparissôn} ou \emph{cupressetum} (voy. le \emph{Lexicon slavicum} de Bernolák ; Budæ, 1825 ; 6 vol. in-8°).} nous donne ainsi lieu de constater que l'antique culte de Vénus fut importé de l'Orient en Occident avec l'usage d'imposer le nom même de la déesse aux objets qui lui étaient consacrés. Ajoutons que les Étrusques avaient reçu de l'Asie occidentale le culte d'une divinité féminine qui porte le nom très-remarquable de \emph{Cupra} ou \emph{Cypra}\footnote{Strabon, \emph{Geogr.} 5., p. 241. Servius, \emph{ad} Virgil. \emph{Æneid.} 1., 426.} ; et si les auteurs latins assimilent à Junon cette déesse, il ne faut pas oublier que primitivement Junon et Vénus étaient une seule et même divinité. Rappelons-nous aussi qu'un des arbres consacrés à Apollon s'est appelé \emph{Daphné} ; qu'un ornement de tête et une coiffure ont retenu le nom de \textgreek{μίτρα}, \emph{mitra}, et une pierre précieuse, originaire de Perse, celui de \emph{mithrax}.\footnote{Pline, \emph{H. N.} 37., 63., 1.} Rappelons-nous enfin que, chez les Égyptiens, selon le témoignage de Champollion le jeune,\footnote{\emph{Panthéon égyptien}, explication de la pl. 22 et ailleurs.} les animaux consacrés aux dieux générateurs portent, dans les légendes hiéroglyphiques, les noms mêmes des divinités dont ils sont les emblèmes. C'est ainsi que le taureau s'appelle \emph{Apis} ; le bélier, \emph{Amon} ; l'ibis, \emph{Thoth} ; le chacal, \emph{Anébo} ; le bouc, \emph{Mendès} ; le crocodile, \emph{Souk}, etc. Il est permis d'espérer que, lorsqu'on sera plus avancé que nous ne le sommes dans l'interprétation des hiéroglyphes idéographiques, on parviendra également à constater que, chez les Égyptiens, certaines plantes, certaines pierres précieuses, certains métaux, portaient aussi les noms des dieux à qui on les avait consacrés.

Bien qu'aucune tradition ne fasse directement mention d'une divinité appelée \emph{Sôs},\footnote{Il est bien probable que le nom de cette divinité était entré dans la composition de celui de deux rois d'Assyrie appelés, l'un \emph{Sôsarès} (\textgreek{Σωσάρης}), l'autre \emph{Sôsarmus} (\textgreek{Σώσαρμος}), dans la \emph{Chronique} d'Eusèbe (1., p. 45 ; 2., 241, 290, 294 ; edit. Mediolan.) ; \emph{Sôsmarès} et \emph{Sôsarmus}, dans la version arménienne (1., p. 99 ; 2., p. 119, 129 ; ed. Venet.) La même remarque peut s'appliquer au nom de \emph{Sôsarmus}, roi des Mèdes (Euseb. \emph{Chronic.} 1., p. 46).} les diverses observations qui précèdent nous autorisent à supposer, avec quelque vraisemblance, que ce nom était très-analogue, sinon semblable, à l'un de ceux sous lesquels les Arméniens adoraient la Vénus assyrienne ou babylonienne, puisque \emph{sôs} est le nom du cyprès, c'est-à-dire d'un arbre qui, sur les monuments de l'Asie antérieure, représente symboliquement Mylitta et Astarté ; d'un arbre qui, transplanté du sol asiatique sur le sol européen, n'a été et n'est encore connu dans tout l'Occident, je le répète, que sous l'un des anciens noms de Vénus. Lorsque le roi d'Arménie Anouschavan prenait le surnom de Sôs, il entendait donc probablement s'attribuer le nom même de la déesse en l'honneur de qui un de ses prédécesseurs, Arménag, fils de Haïg, avait établi le culte du cyprès, vers les bords de l'Araxe, sur la colline même où, peu d'années après, s'élevèrent les murs de l'antique Armavir. En agissant ainsi, Anouschavan ne faisait que suivre l'exemple de plusieurs rois d'Assyrie, qui, on le sait, se firent appeler du nom même d'une des divinités adorées dans leur empire ; et cet exemple, l'histoire nous l'apprend, fut suivi, non-seulement par leurs successeurs, mais par les rois de Perse, par les Arsacides qui régnèrent sur l'Arménie et le Pont, et par les princes qui gouvernèrent la plupart des autres provinces de l'Asie occidentale. Que ma conjecture à l'égard du surnom de Sôs soit fondée ou non, il est certain que, dès une époque voisine, sinon contemporaine, de la fondation du royaume d'Arménie, la première dynastie, celle des Haïganiens,\footnote{Selon Saint-Martin (\emph{Mém. sur l'Armén.} t. 1., p. 407), elle commença avec l'année 2107 avant J. C. Jean Catholicos recule cette date jusqu'en 2297 ou 2295.} avait établi le culte du cyprès, emblème vivant d'une divinité créatrice. Il n'est pas moins avéré que ce culte continuait d'être en vigueur, chez les Arméniens, au moment où Anouschavan, neuvième descendant de Haïg, avait reçu, avec la couronne d'Arménie,\footnote{En l'année 1725 avant J. C. (Saint-Martin, \emph{loc. cit.}).} comme prince feudataire de l'empire assyrien,\footnote{Sur les liens politiques qui très-anciennement existaient entre l'empire assyrien et le Haïgasdan ou l'Arménie, voy. Moïse de Khoren, \emph{Histor. armen.} 1., 12.} le surnom de Sôs ou Cyprès. Et si je suis porté à admettre que la divinité représentée par cet arbre symbolique était la Vénus assyrienne, une pareille supposition n'a rien assurément qui doive nous étonner, lorsque nous nous rappelons que Haïg et ses compagnons d'armes, avant d'aller chercher un refuge dans la contrée septentrionale qui successivement s'est appelée le Haïgasdan et l'Arménie, habitaient le territoire d'Assyrie, et avaient dû y vivre soumis aux institutions religieuses et civiles du pays,\footnote{Parmi les faits qui justifient cette assertion, je me bornerai à rappeler que jusqu'au temps de Mesrob, c'est-à-dire jusqu'au commencement du 5\textsuperscript{e} siècle de l'ère chrétienne, les Arméniens n'eurent point, pour écrire leur propre langue, un alphabet qui leur fût particulier ; ils se servaient des caractères syriaques.} bien que, comme les Chaldéens d'Assyrie, ils ne fussent ni sémites, ni araméens. Or personne n'ignore que, dès une haute antiquité, les Chaldéens avaient établi, dans les pays situés entre le Tigre et l'Euphrate, le culte et les mystères d'une divinité que les Assyriens adoptèrent en lui donnant le nom sémitique ou araméen de Mylitta. Nous pouvons donc croire que ce culte fut apporté dans le Haïgasdan ou l'Arménie par Haïg lui-même ; et si nous trouvons dans cet antique royaume l'arbre symbolique de la déesse appelé d'un nom qui n'appartient point aux langues des enfants de Sem, n'oublions pas que, selon le témoignage formel de la Bible\footnote{\emph{Genes.} 10., 3.} et de tous les écrivains arméniens,\footnote{Voy. Saint-Martin, ouvrage cité, t. 1., p. 253 et suiv.} la maison de Thogormah ou Thorgom, dont Haïg fut le chef, était issue de la race japhétique.\footnote{Un savant dont l'Allemagne s'honore, M. Görres, soutient même l'opinion que la race japhétique tirait son origine de l'Arménie (voyez l'ouvrage qu'il a publié sous le titre suivant : \emph{Die Völkertafel des Pentateuchs, oder die Japhetiden und ihr Auszug aus Armenien.} Regensburg, 1845).} C'est à cette même race que se rattachent les Chaldéens, instituteurs des Babyloniens, des Assyriens, et fondateurs d'un culte où le symbole du cyprès jouait un bien grand rôle.

Lorsqu'on voit un des successeurs de Haïg se faire couronner roi d'Arménie au pied de l'arbre sacré qui était un des emblèmes de Mylitta et d'Astarté, la pensée se reporte involontairement vers l'un des monuments figurés les plus anciens de l'Asie occidentale, le grand bas-relief découvert par M. Charles Texier\footnote{\emph{Description de l'Asie Mineure}, 1\textsuperscript{re} partie, t. 1., pl. 78.} à Yazili-kaïa, sur le sol de l'antique Phrygie ; bas-relief où l'on a représenté un prince assyrien ou phrygien recevant des mains d'Astarté le sceptre et la couronne. La présence de la déesse, figurée ici sous une forme humaine, explique suffisamment pourquoi, dans cette scène, nous ne trouvons pas le symbole du cyprès ; mais le lion, sur lequel Astarté est debout, et le taureau, qu'on n'a pas oublié de placer à ses côtés, ne nous ramènent-ils pas aux monuments figurés que j'ai produits dans mon Mémoire sur le taureau et le lion considérés comme attributs caractéristiques de Venus\footnote{\emph{Recherch. sur Vénus}, p. 119-240.} ? Tout à l'heure, sur quelques médailles de Syrie et de Phénicie, ne verrons-nous pas, planté entre un lion et un taureau, un cyprès pyramidal, image symbolique de la déesse ? En Perse, plusieurs bas-reliefs, taillés sur des faces de rochers, comme l'est celui d'Yazili-kaïa, et conçus dans le même esprit que ce dernier monument, représentent, à leur tour, un prince sassanide au moment où il reçoit tantôt des mains d'Ormuzd, en présence de Mithra,\footnote{Porter's \emph{Travels in Georgia, Persia}, etc. vol. 2., pl. 66.} tantôt des mains de la déesse Anaïs,\footnote{\emph{Id. ibid.} vol. 1., pl. 19.} la couronne du monde, c'est-à-dire la couronne royale de Perse. Un bas-relief d'époque romaine va bientôt nous montrer un initié recevant des mains du dieu Mithra, au pied de trois cyprès, la couronne mystique ou céleste.

La domination de Sémiramis sur l'Arménie, loin d'affaiblir la ferveur des rois de ce pays pour le culte de Vénus, l'aurait au contraire excitée, s'il en avait été besoin ; car qui ne sait combien l'histoire de cette trop célèbre princesse est intimement liée à la légende de la Vénus asiatique ? Et ne dois-je pas faire remarquer que, par une coïncidence qui ne saurait avoir été fortuite, Sémiramis, née en Syrie, où Vénus était représentée sous l'emblème du cyprès, trouva le culte de la déesse établi, chez les Arméniens, avec le culte du cyprès, comme elle l'avait certainement trouvé institué avec le culte de ce même arbre dans la célèbre contrée située entre le Tigre et l'Euphrate, qui vit naître cette antique théologie chaldéenne dont il n'est pas impossible, encore aujourd'hui, de retrouver les principaux dogmes et de constater l'influence sur les idées théologiques et cosmologiques des peuples les plus civilisés des autres parties du monde ancien ?

Cette remarque me conduit à faire observer qu'un des traits saillants de la théogonie et de la cosmogonie chaldéennes se découvre dans la tradition que Mar Iba Cadina avait puisée à une source quelconque, et qui se rapporte au culte du cyprès chez les Arméniens. Le rôle de l'air ou du vent, dans cette tradition, et l'association intime de cet agent physique au culte de l'arbre qui était l'emblème de Mylitta, ne nous rappellent-ils pas que, selon le système chaldéen, la déesse médiatrice entre l'homme et la divinité suprême était assimilée à l'air, cosmologiquement parlant ? Et ce fluide ne remplit-il pas, à son tour, entre l'eau et le feu, le rôle de médiateur, c'est-à-dire le rôle d'un agent intermédiaire destiné à faciliter la combinaison entre eux de deux éléments d'une nature opposée ? Je dirai même, à cette occasion, que le mot qui signifie proprement \emph{air} dans les langues sémitiques ou araméennes, n'appartient à aucune d'elles, tandis qu'il a son radical dans les idiomes japhétiques. C'est un point sur lequel je reviendrai ailleurs, lorsque je serai amené, par le cours de mes travaux, à réunir dans un mémoire toutes les observations qui m'ont permis d'avancer que les Chaldéens descendent des enfants de Japhet, et sont les inventeurs du culte et des mystères de la divinité qu'ils nommaient Mithra, et que les Assyriens ont appelée Mylitta.

L'historien arménien Jean Catholicos, dans l'abrégé très-rapide qu'il s'est contenté de nous donner du livre de Mar Iba Cadina, n'a pas, à l'exemple de Moïse de Khoren, conservé le passage relatif au culte du cyprès établi dans Armavir par Arménag ; mais, du moins, après avoir, à son tour, affirmé que l'historien syrien avait puisé à des sources chaldéennes, il ne néglige pas de dire qu'Anouschavan reçut le nom de \emph{Sôs}.\footnote{\emph{Hist. d'Armén.} mss. arméniens de la Bibliothèque impériale, n° 91, chap. 8., fol. 23 ; trad. fr. de Saint-Martin, p. 13.} Il supplée même sur un point au silence de Moïse, en nous apprenant que le père de ce prince s'appelait Gartos avant d'avoir pris le nom d'Ara, d'après le désir ou l'ordre de Sémiramis.

Il serait sans doute intéressant de connaître, avec quelque certitude, la date de l'introduction du culte du cyprès en Arménie, contrée généralement élevée et froide, où cet arbre ne croît pas spontanément.\footnote{Le palmier-dattier ne croît pas non plus spontanément sur le sol de l'ancienne Arménie ; et cependant, selon le témoignage d'Hérodote (1., 194), le vin de palmier était une des principales denrées que les Arméniens transportaient par eau à Babylone. Il faut donc supposer qu'avec la culture du cyprès ils avaient aussi importé chez eux celle du palmier-dattier.} Je suis disposé à croire que là ce fait, de même que l'institution de la \emph{fête des roses}\footnote{Cette fête, en arménien, s'appelait \emph{Vartavar}, mot composé, dont l'interprétation présente quelques difficultés (voyez Moïse de Khoren, \emph{Œuvres compl.} p. 344 ; Venise, 1843). Selon quelques-uns, il signifierait littéralement \emph{éclat} ou \emph{splendeur des roses}. M. l'abbé Cappelletti le rend par les mots : \emph{Fiammegiature delle rose}. Vartavar est aujourd'hui le nom que donnent les Arméniens à la fête de la Transfiguration.} en l'honneur d'Anahid, ne doit pas être séparé de celui de l'importation du culte de la Vénus assyrienne,\footnote{J'ai précédemment publié plusieurs monuments figurés, dont le témoignage s'ajoute à celui des médailles citées plus haut, pour prouver que le culte du \emph{palmier-doum}, comme celui du cyprès, se liait à la légende de Mylitta ou Astarté, à celle d'Hermès ou Mercure, et, par suite, au culte de Mithra, du soleil et de la lune. Remarquons que si les prêtres arméniens tiraient des augures du cyprès pyramidal, chez les Grecs, on en tirait du palmier (\emph{Orphic. Fragm.} 40., p. 496 ; ed. Hermann) ; la Sibylle écrivait même ses oracles sur des feuilles de ce dernier arbre.} et qu'il est permis d'attribuer à Haïg cette importation. Toutefois je pense qu'à défaut de traditions antérieures au témoignage qui nous a été conservé par Mar Iba Cadina, il convient de ne pas remonter au-delà du règne d'Arménag, fils de Haïg, puisque ce prince est le premier roi d'Arménie dont le nom se trouve associé à une mention expresse du culte du cyprès dans cette contrée ; mais l'époque de ce règne ne saurait elle-même être déterminée d'une manière rigoureuse. La chronologie des historiens orientaux ne repose pas sur des bases assez solides pour que nous admettions avec une entière confiance les dates qu'elle fournit. Moïse de Khoren, l'auteur le mieux informé des faits qui appartiennent à l'histoire ancienne de l'Arménie, ne fixe même aucune date pour les règnes des rois d'Arménie antérieurs à la conquête de la Perse par Alexandre le Grand ; et si feu mon savant ami M. Saint-Martin a placé en l'année 2026 avant J. C. l'avénement d'Arménag,\footnote{\emph{Loc. cit.}} il a eu soin d'avertir ses lecteurs\footnote{\emph{Ibid.} p. 404 et 405.} que toutes les dates qui, dans ses tables chronologiques de l'Histoire d'Arménie, se rapportent à des rois de la dynastie des Haïganiens, sont empruntées à l'historien Michel Tchamtchéan,\footnote{Ce sont aussi les dates fournies par le P. Tchamtchéan qui servent de base à la chronologie des rois arméniens dans l'ouvrage de M. l'abbé Joseph Cappelletti intitulé : \emph{Armenia} (Florence, 1841 ; trois vol. in-8°).} et paraissent être déterminées d'une manière fort arbritaire. Il déclare même qu'il ignore sur quelles autorités le P. Tchamtchéan se fonde pour les donner. L'an 2026 de l'ère ancienne n'est donc que la date plus ou moins approximative de la première mention du culte du cyprès chez les Arméniens. En adoptant, sous toutes les réserves convenables, ce point de départ, nous trouvons, avec le P. Tchamtchéan,\footnote{\emph{Loc. cit.} p. 407. (Cf. M. Cappelletti, \emph{Armenia}, t. 2., p. 9.)} que le roi Ara le Beau ou Ara 1\textsuperscript{er}, qui fut tué dans un combat contre Sémiramis, était monté sur le trône d'Arménie mille sept cent soixante-neuf ans avant la naissance de J. C., que son fils, Ara 2, qui périt de la même manière, avait reçu de Sémiramis, en 1743, l'investiture du royaume d'Arménie ; et qu'il eut pour successeur immédiat son fils Anouschavan, couronné en 1725 sous le nom de Sôs, à Armavir, au pied des cyprès plantés par Arménag. Ces dates, bien qu'incertaines, fournissent trois nouveaux termes qu'il conviendrait peut-être d'ajouter aux diverses données dont on s'est servi jusqu'à ce jour pour assigner au règne de Sémiramis une date approximative. Personne n'ignore que les circonstances les plus importantes de ce dernier événement restent encore plongées dans l'obscurité qui dérobe à nos yeux les origines de l'empire d'Assyrie et l'histoire des premiers siècles de son existence. Aussi l'avénement de cette princesse a-t-il été placé par les écrivains modernes à des époques si diverses, qu'elles varient depuis l'année 737 jusqu'à l'année 2200 avant l'ère chrétienne. Moïse de Khoren,\footnote{\emph{Hist. armen.} 1., 19.} en disant qu'Abraham mourut pendant le règne de Ninyas, qui avait assassiné Sémiramis, sa mère, nous autorise à supposer qu'il assignait au meurtre de cette princesse une date antérieure à l'année 1821, époque de la mort d'Abraham, selon la chronologie de la Vulgate ; mais cette chronologie ne fait dater que de l'an 1215 le règne de Sémiramis. Les savants auteurs de l'\emph{Art de vérifier les dates} ont adopté l'année 1916. Fréret penchait pour cette même date ; M. Daunou, pour celle de 1196 ou 1195.\footnote{\emph{Biogr. univers.} de Michaud, t. 41., p. 550-555.} Saint-Martin,\footnote{\emph{Biogr. univers.} de Michaud, article \textbf{Sanchoniathon}, t. 40., p. 305, 1\textsuperscript{re} col.} s'écartant ici des calculs du P. Tchamtchéan, fait commencer en 1997, et finir en 1957, les quarante années que dura le règne de Sémiramis. D'après l'historien arménien et d'après M. l'abbé Joseph Cappelletti, dont il est l'unique guide, cette princesse aurait conquis l'Arménie après 1769 et avant 1743. Quel que soit le chiffre qu'il faille admettre, on est amené à reconnaître que l'importation du culte du cyprès chez les Arméniens remonte à une haute antiquité, puisque nous venons de constater qu'il y était établi dès le règne d'Arménag, fils et successeur de Haïg, chef de la première dynastie royale d'Arménie. Cette époque est donc antérieure à la domination de Sémiramis.

Le culte de la Vénus assyrienne, partout sans doute associé à celui du cyprès pyramidal, se répandit de la capitale des rois arméniens dans les différentes parties de leur royaume. Nous apprenons même de Pline,\footnote{\emph{H. N.} 5., 20., 1 ; cf. 33., 24., 1.} de Strabon\footnote{\emph{Geogr.} 11., p. 532.} et de Dion Cassius,\footnote{36., 31, 36.} qu'un canton de la Grande Arménie, situé non loin des sources de l'Euphrate, avait reçu le nom de \emph{contrée Anaïtique} (\emph{Anaïtica regio}), et même le nom d'\emph{Anaïtis} (\textgreek{Ἀναΐτις}).\footnote{Strab. \emph{loc. cit.}} Pline ajoute\footnote{\emph{Hist. nat.} 16., 64., 1.} que dans les environs se trouvait le \emph{lac Anaïtique} (\emph{Anaïticus lacus}). Ces diverses dénominations se rapportent, on le voit, à la déesse appelée \emph{Anahid, Anaïs} ou \emph{Anaïtis}. Nous savons, de plus, parle double témoignage des textes et des monuments figurés, que, chez les Arméniens, cette déesse continua d'être adorée jusqu'au moment où l'Arménie fut entièrement convertie au christianisme, c'est-à-dire plusieurs siècles après la naissance de J. C.

De l'Arménie, le culte de la déesse et celui de son arbre symbolique pénétrèrent chez les Albaniens et s'étendirent même jusque chez les Huns. Nous en trouvons des preuves plus ou moins directes dans l'Histoire des Aghouans par Moïse Gaghagantouatsi, ouvrage arménien inédit, dont M. Édouard Dulaurier a bien voulu, à ma prière, extraire et traduire plusieurs passages importants. Dans deux de ces passages,\footnote{Ms. de la bibliothèque du couvent patriarcal d'Edchmiadzin, liv. 2., ch. 41., fol. 122, 124. M. Dulaurier a pu obtenir une copie de ce précieux manuscrit, qui est unique.} on voit que, vers la fin du 7\textsuperscript{e} siècle de notre ère,\footnote{Vers l'an 685.} le culte d'Anahid ou d'Aphrodite et celui des arbres se pratiquaient encore chez les Huns. Deux autres passages, que je vais rapporter, font une mention expresse des cyprès qui, chez les Aghouans,\footnote{C'est le nom original du peuple qui habitait l'ancienne Albanie du Caucase.} vers la même époque,\footnote{Vers l'an 667.} entouraient une chapelle chrétienne, et d'un prodige qui se manifesta au-dessus de ces arbres. L'historien arménien, ayant à raconter la découverte de la croix de saint Mesrob dans le village de Kis,\footnote{Ce village est situé dans la province d'Oudie, qui comprend une partie de l'ancienne Albanie.} où s'était rendu Israël, évêque des Aghouans, s'exprime en ces termes\footnote{Liv. 2., chap. 33., fol. 104.} : « Le saint prêtre de Dieu ne pouvait découvrir avec certitude l'endroit où était enfouie cette croix ; et tandis qu'il était occupé à le chercher, un prodige admirable éclata sur des cyprès touffus qui, espacés d'un côté et d'autre, renfermaient dans leur enceinte la sainte chapelle. Au-dessus de ces arbres apparurent des rayons d'un feu brillant, de forme circulaire, lesquels, se concentrant à l'extrémité la plus élevée de cette coupole, se mouvaient en tout sens et au-dessus des arbres, semblables à l'arc qui se montre dans les nuages.\footnote{\textarmenian{Եւ ոչ գիտելով ստուգել՝ և ՚ի նշմարելզտեղին, նոյն ժամայն նշան զարմանալի լինէր ՚ի վերայ սաղարթապատ ծառոցն նոճիցն, որք երկաքանչիւր աստի և անտի միջաւորեալ՝ ունէին յինքեանս զկոչարանն սուրբ․ որ իբր եթէ ծագումն երևեալբոցաճաճանչ վերոյ ունել կամարաձև․ և ՚ի նոյն միացեալ հաւասար գերակատար գմբէթին՝ բաւանգակապէս սահեալ ճառագայթէր ՚ի ծայրս ծառոցն նոճիցն ճըշգըրտապէս աղեղանդ որ յամբս երևի։}} » Un peu plus loin, dans le même chapitre,\footnote{Fol. 107.} l'écrivain arménien continue son récit en disant : « Le saint prêtre de Dieu, Israël, s'étant acquis la vénération du catholicos et du prince, ainsi que des grands (du pays des Aghouans), obtint d'eux la permission de faire librement tout ce qu'il voudrait. Au bout de quelques jours, désirant perpétuer le souvenir du prodige merveilleux et éclatant qui s'était opéré un grand nombre de fois sur ces cyprès, il voulut les consacrer à la sainte croix, en leur donnant la forme de celle de Notre-Seigneur ; et ayant appelé des ouvriers habiles en menuiserie, il leur ordonna de faire une croix convenablement travaillée. Il la fit orner d'incrustations représentant la passion de Jésus-Christ.\footnote{\textarmenian{Յայնմհետէ մեծարու եղեալքահանայն Աստուծոյ Իսրայէլ առաջի կաթողիկոսին և իշիանին և յաչս մեծամեծացն, առնոյր և հրաման ՚ի նոցանէ գործել զամենայն համարձակապէս։ Եւ յետ սուղ ինչ աւուրց վասն առաւել հրաշիցն և մեծամեծ սքանչելեացն՝ որ բազում անգամ եղեալ էր ՚ի վերայ ծառոցն նոճից, կամեցաւ զի ՚ի նուէր նշանի ձաւնեսցէ զնոսա ՚ի պատկեր տեառնանման խաչին։ Եւ առժամայն կոչեալ վարձաւորս արս ճարտարս հիւսանց, պատշաճաբար յաւրինել զնշան խաչին․ քանդակերպ զարդու դրուագելով պատրաստութեամբ զամենայն ըստ գործոյ Տեառն։}} » Ces deux passages, rapprochés du témoignage qui se rapporte au culte de Vénus chez les Huns, nous permettent de supposer que la chapelle chrétienne placée au milieu d'un \emph{cupressetum}, chez les Aghouans, avait été élevée là sur les ruines d'un ancien \emph{sacellum} d'Anahid. Ils nous prouvent du moins que ce dernier peuple n'avait pas cessé d'attacher au cyprès des idées religieuses.

Si du royaume d'Arménie nous descendons dans une contrée voisine, la Commagène,\footnote{Ce nom est écrit par deux \emph{m} sur les médailles et aussi dans deux inscriptions lapidaires, latines, qui ont été trouvées en Algérie, depuis peu d'années, et qui font une mention expresse de la sixième cohorte des Commagéniens (voy. l'intéressant rapport de M. Léon Renier à M. le ministre de l'Instruction publique, daté de Lambèse (l'ancienne \emph{Lambæsis}), le 2 avril 1851, p. 38-40 du tirage à part des quatre Rapports de ce zélé voyageur).} nous y trouvons la preuve que les habitants de ce pays avaient reçu, avec le culte de Vénus, celui du cyprès. Cette preuve résulte du témoignage, pour ainsi dire, officiel d'une médaille frappée en l'honneur de Julia Domna dans la ville que les Romains appelaient \emph{Germanica Cæsarea}. Sur cette médaille, dont le n° 7 de ma planche 8 reproduit un dessin fidèle, nous voyons, au revers de la tête de l'impératrice, une galère qui est renfermée dans un cercle ou dans une couronne, et qui nous rappelle la galère d'Astarté si fréquemment placée sur les monnaies romaines de Phénicie ; au-dessous est gravé un taureau, symbole affecté à cette même déesse ; et, à gauche, on voit un cyprès\footnote{Ici cet arbre est de l'espèce vulgairement appelée \emph{cyprès de Montpellier} (\emph{cupressus horizontalis}. De Candolle et Miller. \emph{Cupressus sempervirens}, var. \textgreek{β}. Linn.), parce que, selon une tradition locale, elle couvrait autrefois la colline sur laquelle est bâtie la ville. Comme le cyprès pyramidal, cette espèce ou variété croit sur le Liban.} planté sur un autel carré. Un cylindre du musée du Louvre\footnote{Ce cylindre et plusieurs autres qu'a récemment acquis le musée du Louvre seront décrits dans le Catalogue d'antiquités assyriennes dont M. de Longpérier va publier une seconde édition.} nous montre un cyprès également placé sur un autel. Deux personnages accomplissent un acte d'adoration devant cet arbre symbolique, et nous fournissent ainsi un exemple irrécusable de l'usage de représenter, sous l'emblème du cyprès, la divinité asiatique qui préside aux mystères.

Dans un autre pays voisin de l'Arménie et de la Commagène, l'antique Phrygie, dont les origines ne sont pas mieux connues que celles de l'empire assyrien et du royaume fondé dans la Babylonie, le culte du cyprès, dès une époque antérieure à la guerre de Troie, s'y montre uni à celui de Rhéa et de Vénus, divinités primitivement identiques, qui, dans la suite des temps, conservèrent chacune, au nombre de leurs symboles ou attributs, le lion, le taureau, et non-seulement le cyprès, mais aussi le pin, deux arbres mystiques que nous retrouvons, avec le lion et le taureau, dans la légende de Mithra. Au moment de fuir des murs d'Ilion embrasé, Énée indique, pour point de ralliement, aux malheureux Troyens qui veulent partager son sort, un antique cyprès que, depuis longues années, leur dit-il, conserve la piété de nos pères. Cet arbre vénéré était planté hors de la ville, à côté d'un temple abandonné de Cérès ou Rhéa :

\begin{quotation}
Est urbe egressis tumulus, templumque vetustum

Desertæ Cereris, juxtaque antiqua cupressus

Relligione patrum multos servata per annos.\footnote{Virgile, \emph{Æneid.} 2., 713-715.}
\end{quotation}

\paragraph{}
C'est ainsi que Virgile, d'après une de ces traditions hiératiques qu'il avait recueillies avec le plus grand soin, fait parler le fils d'Anchise et de Vénus. Nous verrons bientôt que le pieux Troyen, après s'être embarqué au port d'Antandre, aborde sur les rivages de la Thrace, et y célèbre les funérailles de l'infortuné Polydore en faisant dresser des autels funéraires que l'on couvre de branches de cyprès.\footnote{Cette circonstance et le rôle important que joue ailleurs le cyprès dans les rites funéraires, ont entraîné Servius (\emph{ad} Virgil. \emph{Æneid.} 2., 714) à faire sur les vers cités de l'Énéide, un commentaire qui prouve jusqu'à quel point il ignorait que cet arbre avait été placé au nombre des emblèmes funéraires, précisément parce qu'il était l'image symbolique des divinités qui, comme Cérès ou Rhéa, président tout à la fois au ciel, à la terre et aux enfers : « Antiqua cupressus] Ad vetustatem retulit, dit-il ; cupressum autem funebrem arborem bene ante templum deæ lugentis esse confirmat. »}

Si l'immortel chantre de l'Iliade, par le rôle qu'il fait jouer à Vénus, et par l'arrivée des troupes assyriennes envoyées au secours de Priam, nous révèle à grands traits les liens religieux et politiques qui existèrent entre le roi des rois, chef de l'empire d'Assyrie, et ses feudataires les rois de Phrygie, le poète latin confirme le témoignage du poète grec en nous donnant des détails qui nous prouvent que le culte du cyprès était passé de chez les Assyriens chez les Troyens avec le culte d'une déesse que les écrivains grecs, lorsqu'ils parlent des divinités assyriennes ou babyloniennes, désignent tantôt sous les noms d'Urania,\footnote{Hérodote, 1., 131.} de Mylitta,\footnote{\emph{Id. ibid.}} d'Alitta,\footnote{\emph{Id. ibid.}} d'Alilat,\footnote{\emph{Id.} 3., 162.} d'Anæa\footnote{Strabon, \emph{Geogr.} 16., p. 738.} ou Nanea,\footnote{2. \emph{Machab.} 1., 13 et 15.} et d'Anaïtis,\footnote{Strabon, \emph{Geogr.} 11., p. 512, 532 ; 12., p. 559 ; 15., p. 733. Plutarque, \emph{in Vita Artax.} § 27. Pausanias, 3., 16., 6. Clément d'Alexandrie, \emph{in Protreptic.} 5., Opp. t. 1., p. 57 ; ed. Potter. Dion Cass. p. 104 ; ed. Reimar. Agathias, 2., 24 ; ed. Bonn. Eustathe, ad Homer. \emph{Iliad.} 14., v. 295 ; in Dionysium, v. 846. --- \textgreek{Ἀναΐτις, Ταναΐτις}, \emph{Anæa, Nanea}, ne sont que des altérations de la forme orientale \emph{Anahid}.} tantôt sous le nom de Rhéa,\footnote{Diodore de Sicile, 2., 9 ; ed. Dindorf.} tantôt enfin sous ceux de Déesse de Syrie, et de Héra\footnote{\emph{De Dea syria}, Lucian. \emph{Opp.} t. 9., p. 81.} ou Junon. Nous verrons bientôt Diodore de Sicile attester, à son tour, que, dès une époque fort reculée, le culte du cyprès s'associait à celui de Rhéa chez les Crétois, tout comme chez les Phrygiens. Nous verrons bientôt aussi Claudien, dans son poème sur l'enlèvement de Proserpine, attribuer directement à Cérès, comme Virgile, cet arbre symbolique. Ajoutons enfin que ces deux poètes\footnote{Virgile, \emph{Georgic.} 2., 84. Claudien, \emph{De raptu Proserp.} 3., 371, 372.} célèbrent, chacun de leur côté, les beaux cyprès qui couvraient le mont Ida.

Après tant de témoignages directs ou indirects recueillis en faveur de l'attribution du cyprès à la divinité polynome que nous connaissons aussi sous les noms de Mère des dieux, de Cybèle, de \emph{Dea bona}, il me reste cependant à invoquer ceux que peuvent fournir quelques médailles asiatiques dont, à dessein, j'ai différé d'entretenir l'Académie. À tort ou à raison, j'ai conçu l'espoir que les observations qui précèdent disposeraient la Compagnie à m'absoudre du reproche de témérité, obligé que je suis de combattre, au sujet de ces médailles, une opinion soutenue par Eckhel, le prince de la science numismatique, et par ses émules. Les pièces dont il s'agit furent frappées en Phénicie et en Syrie pendant la domination romaine. Si elles appartiennent ainsi à une époque peu ancienne, elles ont du moins, comme la plupart de celles que j'ai déjà décrites ou que je décrirai plus loin, l'avantage de faire partie d'une série qui fréquemment nous offre la reproduction de types hiératiques, dont l'invention et l'usage remontent à une haute antiquité. C'est à la politique adoptée par les Grecs dans les provinces conquises de l'Asie occidentale et suivie par les Romains, leurs successeurs, que nous sommes redevables de la conservation ou de la restauration de ces types et d'une multitude d'autres représentations religieuses, non moins intéressantes à étudier. N'est-il pas permis de penser que les conquérants grecs, comme les conquérants romains, pour le dire en passant, durent éprouver d'autant moins de répugnance à faire d'une telle conduite la base principale de leur politique, que dans l'Asie occidentale ils retrouvaient, sous d'autres noms, il est vrai, plus d'une divinité dont jadis le culte et les symboles avaient été importés de cette contrée en Grèce et en Italie ?

En tête des médailles dont j'entends parler, se place une suite nombreuse de pièces qui, sous la domination romaine, furent frappées dans l'île phénicienne d'Aradus. Elles ont toutes pour revers un cyprès planté entre un lion et un taureau. Les n°\textsuperscript{s} 1 et 2 de ma planche 6 en reproduisent deux : la première est à l'effigie de Caracalla ; la seconde, à l'effigie d'Élagabale. Eckhel et tous les numismates qui ont écrit après lui ne veulent voir ici, dans le cyprès, que l'emblème du soleil, et dans les deux animaux, que les signes distinctifs des enseignes de deux légions romaines. Mais ils ne tiennent compte ni de l'autel et du bas-relief palmyréniens,\footnote{Pl. 1., n°\textsuperscript{s} 1 et 2 ; pl. 2., n°\textsuperscript{s} 1 et 2 ; pl. 3., n° 1.} où le cyprès est indubitablement l'image symbolique de la Vénus asiatique, ni des nombreuses monnaies autonomes de l'île d'Aradus, dont le type est un taureau bondissant, gravé au revers de la tête d'une divinité femelle qui ne peut être que Vénus-Astarté ; ni enfin des documents qui attribuent à cette déesse le symbole du taureau et celui du lion. Or, si nous rapprochons des deux marbres palmyréniens le trône votif syrien de la Bibliothèque impériale\footnote{Pl. 5., n°\textsuperscript{s} 1 et 2.} et les divers monuments écrits ou figurés qui, dans mon mémoire cité,\footnote{\emph{Recherches sur Vénus}, p. 119-240.} établissent d'une manière incontestable, ce me semble, la double attribution du taureau et du lion à Vénus en Orient et en Occident, nous sommes nécessairement conduits à reconnaître que, sur les médailles impériales d'Aradus, le cyprès qui s'élève entre ces deux animaux est l'image d'Astarté placée entre deux symboles, dont l'un, le taureau, représente la lune et le principe humide ou passif, et dont l'autre, le lion, est l'emblème du soleil et du principe igné ou actif.\footnote{Je suis heureux de pouvoir dire ici que mon explication des diverses significations symboliques du taureau et du lion a obtenu le plein assentiment de deux juges très-compétents, notre savant confrère M. Raoul-Rochette (voy. \emph{Journal des Savants}, janv. 1850, p. 39 et 40) et M. Christian Walz, habile professeur à l'université de Tubingue. Ce dernier, dans une belle dissertation (\emph{De Nemesi Græcorum.} Tubing. 1852, in-4°, fig.), m'a fait même l'honneur de transcrire en entier (p. 16, not. 90) le passage de mes \emph{Recherches sur Vénus} où je résume (p. 222, 223) les opinions de l'antiquité sur le lion et le taureau.} Ici, comme sur le bas-relief palmyrénien du Musée capitolin et sur la face antérieure du trône votif syrien de la Bibliothèque impériale, comme sous la base du cône asiatique cité\footnote{Ci-dessus, p. 54.} et sur la pierre gravée également citée,\footnote{Ci-dessus, p. 55.} Vénus forme donc une triade avec le soleil et la lune, avec le principe igné et le principe humide, ces grands agents célestes et terrestres de la création et de la reproduction. Remarquons de plus qu'à Yazili-Kaïa, dans le grand bas-relief sculpté sur le roc et découvert en 1834 par M. Ch. Texier,\footnote{Voy. \emph{Recherch. sur Vénus}, p. 126-130 ; pl. 2. M. Ch. Texier, \emph{Descript. de l'Asie Mineure}, 1\textsuperscript{re} part. pl. 78. --- Ce voyageur donne le nom de \emph{Pterium} aux ruines antiques qui se voient non loin d'Yazili-Kaïa. La Commission chargée, en 1831, de faire à l'Académie des belles-lettres un rapport sur le voyage de M. Texier croit reconnaître dans ces ruines \emph{Soandus} au lieu de \emph{Pterium} (\emph{Journal des Savants}, juin 1831, p. 370). M. J. W. Hamilton, qui a visité Yazili-Kaïa deux ans après M. Texier, n'adopte ni l'un ni l'autre de ces noms, et leur substitue celui de \emph{Tavium}, métropole de la Galatie (\emph{Research. in Asia Minor, Pontus and Armenia}, vol. 1., p. 391. \emph{Journ. of Royal Geogr. Society}, vol. 7., p. 74).} Astarté est représentée debout sur un lion, ayant tout auprès d'elle un taureau. Remarquons aussi que sur un fragment de frise antique, publié dans mes \emph{Recherches sur Vénus},\footnote{Pag. 170-218 ; pl. 8., n° 1.} la déesse est placée entre des taureaux et des lions ; et n'oublions pas qu'à Persépolis\footnote{MM. Flandin et Coste, \emph{Voyage en Perse}, Perse ancienne, pl. 155. (Voy. mes \emph{Recherches sur Mithra}, atlas, pl. 4.)} Mithra, divinité primitivement identique avec Vénus, est de même placée (sous l'emblème du \emph{mihr} ou de la colombe) entre des taureaux et des lions, formant ainsi avec ces deux espèces de quadrupèdes une triade symbolique, évidemment empruntée au système religieux des Chaldéens d'Assyrie.

À côté des médailles dont il vient d'être question, je n'hésite pas à placer plusieurs monnaies impériales ou coloniales frappées à Damascus, dans la Cœlésyrie ; elles sont à l'effigie d'Élagabale,\footnote{\emph{Mus. Theupol.} p. 720.} de Trébonien Galle,\footnote{Mionnet, \emph{Descript. de méd.} 5., 295, n° 83.} ou de Volusien.\footnote{\emph{Id. ibid.} 296, n° 89.} Leur type, presque identique avec celui des monnaies citées d'Aradus, a certainement la même signification symbolique ; car, si le cyprès pyramidal, au lieu de s'élever entre un taureau et un lion, est planté ici entre un taureau et un cheval, il faut se rappeler que le cheval est un autre symbole solaire, qui, comme le lion, fut consacré à Vénus chez les Grecs et chez les Romains. J'ai déjà eu l'occasion de le dire dans une dissertation que l'Académie a jugée digne de faire partie du recueil de ses Mémoires.\footnote{Tome 15., 2\textsuperscript{e} partie, pages 292-294. --- On connaît maintenant des médailles impériales de Trébizonde qui font de Mithra une divinité \emph{équestre}, comme Vénus, comme chacun des deux Dioscures, comme le Soleil, comme la Lune.} Une des monnaies citées de Damascus, celle qui porte l'effigie de Trébonien Galle, est figurée ici sous le n° 3 de la planche 6. Le docte Eckhel ne fait aucune difficulté de reconnaître que le cheval et le taureau sont les symboles du Soleil et de la Lune ; et, avec toute raison, il cite, à l'appui de cette interprétation, le témoignage des médailles autonomes grecques de la même ville, où se voient, d'un côté, le buste du Soleil, et de l'autre, le buste de la Lune.\footnote{Voyez Mionnet, \emph{Descript. de méd.} 5., 283, n° 10. Sestini, \emph{Lett. numism. continuaz}, 6., 86, n° 4.} On s'étonne qu'après ce judicieux rapprochement, l'illustre numismate, négligeant de rapprocher des monnaies de Damascus celles d'Aradus, affirme que, sur ces dernières, le lion et le taureau placés en regard d'un cyprès sont, non point les emblèmes du Soleil et de la Lune, mais les emblèmes de deux légions romaines.\footnote{\emph{D. N.} 3., 394.} Il avait dit, quelques pages plus haut,\footnote{\emph{Ibid.} p. 331, 332.} que le cyprès fut consacré tout à la fois au Soleil, d'après la raison qu'il en avait donnée précédemment, et à la Lune, divinité identique avec Hécate ou Proserpine, parce que, selon le témoignage formel de Pline, le cyprès était consacré à Pluton. On n'est pas moins surpris qu'Eckhel ait pu écrire ces mots sans se trouver amené à supposer que primitivement le cyprès avait dû être l'emblème sacré d'une divinité androgyne et créatrice, dont le Soleil et la Lune étaient les agents et la manifestation. N'avait-il pas, dans la Phénicie même, le témoignage de plusieurs monuments monétaires où l'on voit, à l'entrée du temple tétrastyle d'Astarté, le feu sacré, emblème de cette divinité, briller sur un autel posé entre la statue d'Apollon ou du Soleil et la statue d'Artémis ou de la Lune, en même temps que parfois, au milieu du tympan d'un fronton triangulaire, est placé, je l'ai déjà dit, le buste radié d'Astarté ? Le n° 3 de ma planche 8 fournit un exemple de ce dernier type,\footnote{D'autrefois la statue d'Astarté est placée sous le portique d'un temple, entre deux \emph{sacellum} où l'on n'aperçoit ni figures, ni cyprès ; tel est le type d'un moyen bronze d'Élagabale, frappé aussi à Tripolis de Phénicie, et figuré sous le n° 1 de ma pl. 8. Ici les deux petites chapelles sont certainement dédiées au Soleil et à la Lune ; et les trois colonnes qui décorent le portique de chacune d'elles font allusion, je le suppose, au triple Soleil et à la triple Hécate.} au revers d'une pièce frappée à Tripolis de Phénicie, en l'honneur de Caracalla. On sait que les médailles de ce prince, plus peut-être que celles d'aucun autre empereur romain, nous offrent, pour l'histoire des religions de l'Asie occidentale, un grand nombre de types anciens qui, sans cette circonstance, seraient restés inconnus aux archéologues. Le n° 2 de cette planche 8 nous montre, au revers de la tête du même empereur, sur une autre monnaie de Tripolis, un type analogue à celui que représente le n° 3 ; mais, dans le fronton triangulaire\footnote{Deux astérisques sont gravés dans le champ de la médaille, l'un à droite, l'autre à gauche de ce fronton, et représentent sans doute ici l'étoile du matin et l'étoile du soir, ou la planète Vénus sous ses deux aspects.} du temple d'Astarté, on a substitué au buste radié de la déesse un petit globe, emblème connu de la divinité qu'adoraient les Phéniciens. Il nous rappelle, en effet, qu'à Aphaques, dans le Liban, le jour de la fête d'Astarté, on faisait paraître dans les airs un globe de feu.\footnote{Zosime, \emph{Historiæ}, 1., 58., 4 ; ed. Reitemeier.} Je ne puis prononcer le nom d'une montagne si célèbre par le culte qu'on y rendait à la Vénus surnommée \emph{Libanitis}\footnote{Lucien, \emph{Advers. Indoctum}, § 3.} et \emph{Libanéis},\footnote{Nonnus, \emph{Dionysiac.} 43., v. 105 ; ed. Græfe.} sans remarquer que là, dès une haute antiquité, le temple et les sanctuaires de la déesse avaient dû se trouver entourés de cyprès, l'arbre symbolique qui lui était consacré. Le troisième et le quatrième livre des Rois,\footnote{3. \emph{Reg.} 5., 8-10 ; 6., 15, 34 ; 9., 11 ; 4. \emph{Reg.} 19., 23. --- Dans la construction du temple de Jérusalem, on employa concurremment du bois de cyprès et du bois de cèdre, tirés l'un et l'autre des forêts du mont Liban. Un passage du Cantique de Salomon (1., 16) nous prouve que dans les édifices particuliers les solives étaient de cèdre et les lambris de cyprès.} le deuxième livre des Paralipomènes,\footnote{2., 3., 5.} l'Ecclésiastique, Isaïe,\footnote{37., 24 ; 60., 13.} Ézéchiel,\footnote{27., 5.} Zacharie,\footnote{11., 1, 2.} le Cantique de Salomon,\footnote{1., \emph{v. ultim.}} saint Jérôme\footnote{\emph{Comment. in Esa.} 60. ; \emph{in Ezech.} 27.} et saint Cyrille\footnote{\emph{Super Isaiam}, lib. 5.} font une mention expresse des belles forêts de cyprès du mont Liban. Diodore de Sicile\footnote{19., 58., 3 ; ed. Ludov. Dindorf. --- L'historien grec fait ici une mention expresse de l'admirable beauté et de la hauteur des cyprès el des cèdres que, peu après la mort d'Alexandre le Grand, on tira du Liban pour construire la flotte d'Antigone.} ajoute son témoignage à celui des écrivains sacrés ; et Jésus, fils de Sirach, désigne nominativement, dans l'Ecclésiastique,\footnote{24., 13. Cf. 50., 11-14.} la partie du Liban qu'on appelait \emph{Hermon}. Je montrerai ailleurs comment ce nom, qui avait été substitué à celui de \emph{Sénir}, se lie plus intimement que peut-être on ne le pense à la légende de la Vénus asiatique ; mais tout à l'heure je produirai deux médailles et quelques autres monuments figurés qui prouvent que le symbole du cyprès était attribué à Hermès ou Mercure, tout aussi bien qu'à Vénus.

Les observations placées par M. Movers dans son savant ouvrage sur les Phéniciens\footnote{\emph{Die Phönizier}, t. 1., ch. 15., p. 575-584.} établissent avec certitude, ce me semble, non-seulement que sur le Liban le cyprès était consacré à cette déesse, mais que là, comme aussi à Aphaques, à Byblos, à Béryte et au lieu appelé maintenant \emph{Kalaat-Fakra}, la Vénus du mont Liban portait le même nom que cet arbre. Sur ce point, l'opinion de l'habile professeur de Breslau n'a pu être influencée par le témoignage des monuments figurés phéniciens, syriens ou palmyréniens ; il n'en cite aucun, et ne paraît même pas avoir soupçonné que l'antiquité asiatique nous a légué plus d'un exemple de la représentation de Vénus sous la forme d'un cyprès pyramidal. Il prend texte d'un passage où Sanchoniathon,\footnote{\emph{Fragm.} p. 18 ; ed. Orelli.} en même temps qu'il place au nombre des montagnes déifiées le Casius, le Liban et l'Antiliban, nous fait connaître une divinité locale appelée \emph{Brathy} (\textgreek{Βραθύ}). Or Dioscoride,\footnote{1., 88.} Hiller en avait déjà fait la remarque,\footnote{\emph{Hierophyticon}, pars 1., cap. 40., p. 375.} nous apprend que l'une des deux espèces de sabine portait le nom de \textgreek{Βραθύ}, \emph{brathy}, à cause de la ressemblance de ses feuilles avec celles du cyprès. Pline\footnote{\emph{Hist. nat.} 24., 61., 1.} confirme le fait et ajoute que, pour cette raison, quelques personnes appellent la sabine \emph{cyprès de Crète} (\emph{cupressus cretica}). Il nous dit, de plus,\footnote{\emph{Ibid.} 12., 39., 1. --- C'est un des arbres, observe Pline, dont l'empereur Claude, dans ses Histoires, parle comme d'une merveille. Il dit que les Parthes mêlent ses feuilles à leurs boissons ; que son odeur rappelle celle du cèdre, et que la fumée qui s'exhale de son bois, quand on le brûle, détruit les qualités pernicieuses des exhalaisons de quelques autres espèces d'arbres. On le trouve sur le mont Zagrus. J'ajoute que le \emph{cupressus horizontalis} croît spontanément aussi sur le Liban ; là il est mêlé avec le cyprès pyramidal.} qu'un arbre appelé \emph{bratus}, \emph{brathus} ou \emph{brathys}\footnote{Saumaise (\emph{Exercit. Plinian.} p. 260) a proposé de lire \emph{brathyn} dans le passage cité de Pline (12., 39., 1). Quelques éditions récentes portent, les unes, \emph{bratum}, les autres, \emph{brathum}. M. Movers (ouvrage cité, t. 1., p. 575) et M. Littré lisent \emph{bratum}.} croît dans l'Élymaïde et ressemble à un cyprès dont les rameaux descendraient vers la terre, particularité qui me porte à penser qu'il s'agit ici de l'espèce de cyprès nommée par les botanistes \emph{cupressus horizontalis}. Ces divers renseignements conduisent M. Movers à considérer \emph{brathy} comme un mot qui, dans un dialecte mixte, sémitique et araméen, servait à désigner tout à la fois le cyprès et la sabine. En conséquence, il n'hésite pas à identifier le nom de la déesse \emph{Brathy} avec \emph{brathy}, un des noms que portait le cyprès. D'autre part, cet arbre, dans la Bible, étant fréquemment nommé \emph{béroth, bérouth} ou \emph{barouth}, l'écrivain allemand a toute raison de se croire autorisé à déclarer identique avec la déesse \emph{Brathy} du Liban une divinité que Sanchoniathon\footnote{\emph{Fragm.} p. 24 ; ed. Orelli.} nomme \emph{Bérouth} (\textgreek{Βηρούθ}),\footnote{L'identité de ce nom avec celui de la ville appelée \emph{Béryte}, \textgreek{Βηρυτός}, nous permet de constater ici un second exemple d'éponymie dans la légende de Vénus.} et dont l'époux habitait les environs de Byblos. Dès lors ces deux déesses, appelées chacune du même nom que le cyprès, se confondent à leur tour avec la Vénus du mont Liban adorée à Aphaques, à Béryte,\footnote{Nonnus (\emph{Dionysiac.} 41., 83, 143 sqq. ed. Græfe), après avoir célébré la haute antiquité de la ville appelée d'abord \emph{Béroë}, puis \emph{Béryte}, énumère les divinités qu'on y adore, et qualifie cette ville de \emph{maison de Vénus} et \emph{demeure des Amours} (\emph{ibid.} v. 146) : ... \textgreek{Παφίης δόμος, οἶκος Ἐρώτων}.} à Byblos et au lieu nommé aujourd'hui \emph{Kalaath-Fakra}. Ainsi nous avons, dans ces divers rapprochements, une nouvelle preuve de la consécration du cyprès à la Vénus asiatique, et un second exemple d'un usage qui déjà nous était révélé par l'identité du mot \emph{cyprès} avec le nom de la déesse \emph{Cypris}. N'y a-t-il pas lieu de s'étonner que ce dernier rapprochement ait échappé à l'attention de M. Movers ? Ce léger oubli, je me hâte de le dire, est amplement racheté par les nombreux témoignages qu'il a su recueillir\footnote{\emph{Loc. cit.} p. 560-584.} touchant la coutume, très-répandue chez les peuples de l'Asie occidentale, de représenter leurs divinités féminines sous la forme d'un arbre coupé ou d'une colonne de bois, et de leur consacrer, non-seulement le cyprès, mais aussi le pin, le grenadier et l'amandier. Rapprochés de l'autel et du bas-relief palmyréniens du Musée capitolin, ainsi que du trône votif de la Bibliothèque impériale et d'un passage où Zosime\footnote{\emph{Historiæ}, 1., 58., 7 ; ed. Reitemeier.} dit expressément que la déesse adorée à Palmyre était l'Aphrodite surnommée \emph{Aphacitis},\footnote{Ce surnom a échappé à l'attention du savant Larcher.} c'est-à-dire la \emph{Vénus d'Aphaques}, ces témoignages me portent à croire que, dans cette dernière ville, où Vénus était appelée \emph{Brathy}, c'est-à-dire \emph{Cyprès}, la déesse devait être habituellement représentée sous la forme même d'un cyprès,\footnote{Voyez ce que dit M. Movers (\emph{loc. cit.}) au sujet de Vénus-\emph{Aschéra}.} comme elle l'est sur les trois monuments palmyréniens ou syriens ici figurés.

Je suis loin, quant au cyprès, d'avoir épuisé les précieux secours que M. Movers a complétement négligé d'emprunter à la nombreuse série des monnaies asiatiques qui attestent la consécration de cet arbre aux divinités génératrices, soit mâles, soit femelles. Plusieurs pièces coloniales frappées à Damascus,\footnote{Je ne crois pas devoir y comprendre une monnaie coloniale de la même ville, qui porte l'effigie de Valérien père et dont le revers est décrit en ces termes par l'abbé de Tersan, dans le Catalogue des médailles du cabinet d'Ennery : « Un cabas ou espèce de panier de jonc, d'où sort une branche de cyprès ou un thyrse. » Bien que Mionnet ait admis sans rectification cette monnaie dans son \emph{Suppl.} (8., 206, n° 59), et bien que je n'aie pas vu l'original, je suis porté à penser qu'il y a une erreur dans l'indication d'une branche de cyprès.} dans la Cœlésyrie, méritent l'attention de l'Académie, non moins que les monnaies de la même ville dont il a été question plus haut. Elles sont à l'effigie de Sévère Alexandre,\footnote{Sestini, \emph{Descript. num. var.} p. 528. Mionnet, \emph{Descript. de méd.} t. 5., p. 292, n° 61.} de Philippe père,\footnote{Vaillant, \emph{Num. in colon. percuss.} t. 2., p. 232. Mionnet, \emph{loc. cit.} n° 62.} de Philippe jeune,\footnote{Mionnet, \emph{loc. cit.} p. 294, n° 77.} ou de Trébonien Galle,\footnote{\emph{Id. ibid.} p. 295 et 296, n° 85. \emph{Mus. Sanclem. Numism. select.} p. 512, tab. 34., n° 371.} et nous montrent, en adoration devant un cyprès pyramidal, la figure nue de Silène ou d'un Faune qui, sur l'épaule gauche, porte une outre gonflée par le liquide qu'elle contient. Le n° 1 de la planche 7 ci-jointe offre un exemple de ce type. Le même personnage mythologique, sur quelques monnaies coloniales de Tyr, à l'effigie de Caracalla,\footnote{Voy. le n° 5 de la planche 7 ci-jointe. Eckhel, \emph{Catalog. Mus. cæsar. vindob.} t. 1., p. 244, n° 20 ; cf. n° 19.} d'Élagabale,\footnote{Sestini, \emph{Mus. Hederv.} 3., p. 97, n° 34. Mionnet, \emph{loc. cit.} p. 434, n° 650.} de Sévère Alexandre,\footnote{Mionnet, \emph{ibid.} p. 438, n° 674.} de Philippe père,\footnote{\emph{Recherches sur Vénus}, pl. 12., n° 5.} ou de Trébonien Galle,\footnote{Mionnet, \emph{loc. cit.} p. 443 et 444, n° 703.} se retrouve placé, dans une attitude semblable, devant un palmier que nous reconnaîtrons ailleurs pour un autre emblème de la Vénus orientale, et qui est accompagné ici de l'étoile d'Astarté et d'un \emph{murex} que, dans ce cas, je prends pour l'équivalent du \emph{cteis} gravé sur plusieurs cônes ou cylindres asiatiques, consacrés à cette déesse.\footnote{Voy. mes \emph{Rech. sur Vénus}, p. 52-54, pl. 1., n°\textsuperscript{s} 1 et 2.} Je dirai, dès à présent, que si l'on pouvait douter que, sur les diverses médailles citées, Silène ou Faune Ascophore fût en adoration devant l'image symbolique d'Astarté, il suffirait, pour acquérir une entière conviction à cet égard, de jeter les yeux sur un nombre considérable d'autres médailles asiatiques, où ce même personnage, portant, comme ici, une outre sur l'épaule gauche, accomplit un acte d'adoration aux pieds d'une statue colossale d'Astarté. Les monuments monétaires de Sidon, de Tyr et de Bostra, qui se rapportent aux règnes de Caracalla,\footnote{Eckhel, catalogue cité, t. 1., p. 244, n°\textsuperscript{s} 19 et 20. Mionnet, \emph{Descript. de méd.} t. 5., p. 429, n° 626.} d'Élagabale,\footnote{Vaillant, \emph{Num. in colon. percuss.} Sestini, \emph{Mus. Hederv.} 3., p. 97, n°\textsuperscript{s} 30 et 37 ; C. M. H. n°\textsuperscript{s} 6148 et 6100. Mionnet, \emph{loc. cit.} p. 384, n°\textsuperscript{s} 319 et 320.} de Sévère Alexandre,\footnote{Sestini, \emph{Descript. num. var.} p. 539, n° 32. Mionnet, \emph{loc. cit.} p. 437, n° 673.} de Gordien Pie,\footnote{Mionnet, \emph{ibid.} p. 438, n° 676.} de Philippe père,\footnote{\emph{Id. ibid.} p. 439 et 440, n° 685 ; p. 441, n° 692. --- Ajoutez une médaille de la même série, à l'effigie d'Otacilie, \emph{ibid.} p. 441 et 442, n° 695.} de Philippe jeune,\footnote{\emph{Id. ibid.} p. 442, n° 698.} de Trajan Dèce,\footnote{Sestini, \emph{Descript. num. vet.} p. 539. Mionnet, \emph{loc. cit.} p. 584 et 585, n° 35.} de Trébonien Galle,\footnote{Mionnet, \emph{ubi supra}, p. 443 et 444, n° 703 ; p. 444, n° 704.} de Volusien,\footnote{Vaillant, \emph{ubi supra.} Mionnet, \emph{loc. cit.} p. 445, n°\textsuperscript{s} 714 et 716. --- \emph{Mus. Theupol.} p. 760.} de Valérien père\footnote{Mionnet, \emph{loc. cit.} p. 447, n° 724.} et de Gallien,\footnote{\emph{Id. ibid.} p. 449 et 450, n° 739. --- Ajoutez une médaille de la même série, à l'effigie de Salonine, et décrite par Sestini (\emph{Mus. Hederv.} 3., p. 98, n° 45).} fournissent des exemples de ce sujet ; j'en ai choisi trois que je place sous les yeux de l'Académie.\footnote{Pl. 7., n°\textsuperscript{s} 2 et 4 ; pl. 8., n° 10.}

Je n'ignore point que, selon le témoignage irrécusable des monuments,\footnote{Voy. Mionnet, \emph{Descript. de méd.} t. 5., p. 237, n°\textsuperscript{s} 19 et 20 ; p. 582, n° 24 ; t. 2., p. 643 et 644, n° 104 ; p. 645, n° 110 ; p. 647, n° 124 ; p. 652 et 653, n° 159 ; t. 3., p. 493, n° 8 ; t. 2., p. 179 et 180, n° 233. Vaillant, \emph{Num. in colon. percuss.} t. 2., p. 114. Christ. Ramus, \emph{Catal. num. vet. reg. Daniæ}, 1., 346, n° 8. Sestini, \emph{Descript.} p. 79, n° 5 ; \emph{Lett. numism.} t. 4., p. 93 ; t. 9., p. 18. D'Ennery, \emph{Catal.} p. 558. Mionnet, \emph{Suppl.} t. 3., p. 165, n° 1073, pl. 9., fig. 1.} le dieu Faune ou le dieu Silène, portant une outre sur l'épaule, figure seule et sans Astarté ou Vénus sur les médailles coloniales romaines, frappées à Alexandria-Troas, à Antioche de Pisidie, à Bostra en Arabie, à Cœla dans la Chersonèse de Thrace, et à Corinthe. Je n'ignore pas non plus que la plupart des numismates, et nommément l'illustre Eckhel,\footnote{\emph{D. N.} t. 4., p. 493-496.} le considèrent comme un simple signe colonial, même lorsque, sur les médailles romaines citées de la Cœlésyrie, de la Phénicie et de l'Arabie, nous le voyons adorer Astarté tantôt sous l'emblème du cyprès pyramidal, tantôt sous une forme humaine. Mais je ne puis me défendre de penser que les Romains, en le plaçant dans cette attitude auprès de la déesse, avaient entendu par-là indiquer qu'il existait des rapports particuliers entre elle et le personnage mythologique dont il s'agit, personnage ambigu qui était resté une énigme pour les écrivains latins les plus instruits.\footnote{« Nam Fauni vocem equidem numquam audivi ; tibi, si audisse te dicis, credam ; etsi Faunus omnino quid sit, nescio. » (Cicéron, \emph{De natur. Deor.} 3., 6., p. 216 ; ed. Schütz.) --- Personne n'ignore que les Romains adoraient aussi une déesse appelée \emph{Fauna}. (Lactance, \emph{De falsa relig.} 1., 22., ed. Walch. Cf. Macrobe, \emph{Saturn.} 1., 12., et Arnobe, 1., p. 15 ; 5., p. 168). On l'identifie avec Cybèle ou Cérès, avec la \emph{Dea bona} ; mais sa légende est tout aussi obscure que celle du dieu Faunus.} La supposition qu'il me suggère ne pourrait-elle pas se justifier par le souvenir des relations que l'Italie, comme nous le verrons plus loin, avait anciennement établies entre le symbole du cyprès et le dieu Silvain, qui a tant d'analogie avec le dieu Faune ? Et cette attribution du cyprès ne nous autoriserait-elle pas à rechercher si la déesse \emph{Fauna} n'avait, de son côté, aucun trait de ressemblance avec l'Astarté des peuples de l'Asie occidentale ? Ce n'est point ici le lieu rie discuter ces questions incidentes. Je me contenterai de faire remarquer qu'au revers d'une belle médaille, qui fut frappée à Thessalonique, en l'honneur de Trajan Dèce,\footnote{Voy. Mionnet, \emph{Suppl.} t. 3., p. 165, n° 1073, pl. 9., fig. 1.} on voit, devant la statue colossale d'une divinité hermaphrodite,\footnote{Cette figure attend un commentaire.} à deux têtes, vêtue à mi-corps comme la \emph{Vénus Génitrice}, et vaguement désignée sous le nom de \emph{Janus bifrons}, une petite figure semblable en tout au Faune ou au Silène Ascophore qui, sur les monnaies d'Asie et d'Arabie, est en adoration tantôt aux pieds d'Astarté, tantôt devant l'arbre symbolique de la déesse.

Les médailles que j'ai décrites plus haut, celles surtout d'Aradus et de Damascus, concourent à établir, comme les trois monuments syriens nous avaient déjà permis de le faire, que, chez les Assyriens, les Syriens, les Arméniens et les Phéniciens, Vénus fut adorée sous le symbole du cyprès. Toutefois, nous n'avons pas encore interrogé les antiquités de Cypre, et il importe d'autant plus de ne point les laisser dans l'oubli que cette île, tout aussi phénicienne qu'Aradus, est bien autrement célèbre par le culte qu'y recevait, dès les temps les plus reculés, la Vénus assyrienne. Personne n'ignore que, selon les témoignages les plus dignes de foi,\footnote{Hérodote, 1., 105. Pausanias, 1., 14., 6.} ce culte fut porté d'Ascalon en Cypre par les Phéniciens ; et nous pourrions \emph{a priori}, et sans craindre de nous être trop avancé, affirmer que le culte du cyprès, inséparable de la légende et des rites de la déesse, fit partie d'une si mémorable importation. Les monuments de l'art, loin de nous démentir, prouvent que cette assertion est parfaitement fondée. Ici, cependant, nous ne trouvons à invoquer le témoignage d'aucune médaille frappée dans l'île de Cypre ; mais, à défaut de médailles, je citerai deux monuments précieux, récemment exhumés du sol même de cette île et encore inédits. Ce sont deux belles coupes d'argent doré, qui jadis furent probablement en la possession de deux rois, à l'usage de qui tout annonce qu'elles étaient destinées.\footnote{Ces deux coupes faisaient partie d'un trésor enfoui dans la terre, qui comprenait dix autres coupes du même genre et un grand nombre d'objets ou de bijoux d'or ou d'argent doré. Il paraît certain que tout a été fondu, à l'exception des deux coupes apportées à Paris. C'est pour la science une perte à jamais regrettable ; mais cette perte est en partie compensée par les belles coupes de bronze ciselées qui proviennent des secondes fouilles de M. Layard sur le sol assyrien.} L'une, il y a deux ans, a été rapportée de Beyrouth par M. de Saulcy, qui libéralement en a fait don au musée du Louvre ; elle est gravée en creux. L'autre, ciselée en relief, vient d'être acquise par le même musée. Je me propose de les publier incessamment avec les commentaires nécessaires. En attendant, je me bornerai à dire ici que je rapporte aux cérémonies instituées dans les mystères de la Vénus assyrienne les sujets mystiques représentés dans l'intérieur de ces deux coupes royales, et que, sur les deux coupes, les initiations s'accomplissent en présence d'un cyprès, ou de deux cyprès accouplés. Je ne puis donc douter que, chez les Cypriens, comme chez les Aradiens, comme chez les habitants de la Phénicie continentale, de la Syrie et de l'empire assyrien, le cyprès n'eût été consacré à la déesse dont il porte le nom, aussi bien que l'île\footnote{Je n'ignore point que je heurte ici une opinion généralement admise : on veut que l'île de Cypre ait donné son nom à la Vénus asiatique, tandis que, à mon avis, elle l'aurait reçu de la déesse elle-même, appelée \emph{Cypris}, nom identique avec celui de l'arbre qui servait à la représenter symboliquement ; mais il est vrai de dire que jusqu'à ce jour personne n'avait soupçonné un tel usage, ni, par conséquent, été amené à remarquer l'identité du nom de Cypris et du nom de Cypre avec le mot \emph{gopher} ou \emph{kopher} qui, dans les langues sémitiques, signifie \emph{cyprès}, et peut-être avec le mot \emph{Kupfer} qui, en allemand, désigne le métal consacré à la déesse, le cuivre. Le plus savant géographe de notre temps, M. Carl Ritter, semble, indirectement et à son insu, me donner raison, lorsqu'il dit (\emph{Die Erdkunde}, 11., p. 577, 578) que le nom de l'île appelée \emph{Cypre} lui paraît venir des forêts de cyprès (\emph{gopher}) qui devaient couvrir une grande partie de cette île et fournir aux Phéniciens les bois de construction dont ils avaient besoin pour leurs flottes. De son côté, mais également à son insu, M. Movers justifie d'avance mon opinion en établissant, comme je l'ai dit plus haut (p. 88-90), que, chez les Phéniciens, Vénus s'appelait aussi du nom de \emph{Bérouth}, qui, comme \emph{gopher}, signifie cyprès, et qui était, en même temps, le nom que la ville phénicienne de Béryte avait sans doute reçu de la déesse du Liban.} où une tradition fait aborder Cypris au moment de sa naissance. On verra, dans la suite de mon travail, que le culte de cet arbre symbolique fut importé par les Phéniciens, non-seulement dans l'île d'Aradus, dans l'île de Cypre et en Étrurie, mais aussi chez les Grecs et dans la péninsule hispanique.

Si, en Asie, nous trouvons ce culte intimement lié à la légende de Vénus, et cette déesse, le soleil, la lune et l'Artémis de Perga représentés chacun sous l'emblème du cyprès, nous ne pouvons-nous défendre de la présomption que les peuples de l'Orient avaient dû attribuer ce même symbole à la divinité mâle, solaire, qu'ils adoraient sous divers noms. Bien plus, cette présomption se change en certitude à la vue des médailles frappées dans une ville de la Cœlésyrie où chacun sait quels honneurs religieux y recevaient le soleil et le dieu asiatique dont j'entends parler, alternativement confondus l'un avec l'autre. À défaut d'autres témoignages, le nom de Baal-Hamon, et ceux d'Héliopolis et de Baalbeck, que porta successivement cette ville, confirmeraient la confusion ou l'assimilation que je juge nécessaire de rappeler ici. Je rappellerai, en même temps, que la cité connue sous ces trois noms était située non loin de la pointe culminante de ce mont Liban qui, dès une époque très-reculée, fut, dans l'Asie antérieure, un des principaux sièges du culte de la Vénus assyrienne. Les monnaies coloniales d'Héliopolis, qui attestent l'association du culte du cyprès avec celui de ses deux divinités éponymes, Hélios et Baal, nous offrent, au revers de la tête de Philippe père\footnote{Voy. la médaille figurée sous le n° 5 de la planche 6. Elle est décrite dans le 5\textsuperscript{e} vol. de Mionnet, p. 302, n° 123.} et au revers de la tête d'Otacilie,\footnote{Vaillant, \emph{Numism. in colon. percuss.} 2., 262.} son épouse, la façade d'un temple dodécastyle, où l'on monte par un escalier de six marches. Sous le péristyle de ce majestueux édifice, s'élève un cyprès pyramidal, planté à la place même où, sur d'autres monnaies asiatiques, nous voyons tantôt une pierre conique, emblème d'Astarté et d'Artémis, tantôt la statue en pied de la première de ces deux déesses, ou de quelque autre divinité d'un ordre également supérieur, tantôt enfin un buste colossal d'Astarté. Il n'est donc pas douteux qu'ainsi placé, le cyprès représente symboliquement une des divinités principales qu'on adorait à Héliopolis. Les numismates, et un grand nombre d'autres archéologues, ne veulent ici encore considérer cet arbre que comme l'emblème du soleil.\footnote{J'ai le regret d'être obligé de dire que feu notre confrère M. Émeric David n'a pas su éviter cette erreur dans son savant ouvrage intitulé \emph{Jupiter} (tome 2., page 437).} Toutefois la série des monuments monétaires d'Héliopolis et un passage de Macrobe bien souvent cité, mais toujours à l'occasion de monuments peu judicieusement interprétés, permettent de croire que les Héliopolitains attachaient au symbole du cyprès, non pas seulement l'idée de l'astre qui répand sur la terre la lumière et la chaleur, mais l'idée d'une divinité solaire, ou tout à la fois solaire et lunaire. Car le philosophe néoplatonicien affirme que le Soleil et Jupiter sont un même dieu, une même puissance, et que, pour cette raison, les Assyriens rendent au soleil, dans la ville d'Héliopolis, un culte solennel sous un nom que l'on traduisait en grec par la dénomination de \emph{Zeus Héliopolite}.\footnote{« Nec ipse Jupiter, rex deorum, solis naturam videtur excedere, sed eumdem esse Jovem ac solem claris docetur indiciis ... unde utrumque constat una potestate censendum. Assyrii quoque sub nomine Jovis, quem \emph{Dia Heliopoliten} cognominant, maximis ceremoniis celebrant in civitate quæ Heliopolis nuncupatur ... » (\emph{Saturnal.} 1., 23.)} Ils donnaient aussi à ce dieu le nom d'\emph{Adad}, c'est-à-dire \emph{unique}, et l'honoraient comme la divinité suprême ; c'est le même écrivain qui le dit. Mais, ajoute-t-il, les Assyriens associent à Adad une déesse nommée \emph{Atargatis} ; ils les considèrent comme les arbitres souverains de toutes choses. Sous ces deux noms, ils entendent le Soleil et la Terre ; et, au lieu d'énoncer par une multitude d'épithètes les diverses espèces de puissance et de prééminence attribuées à ces deux divinités, ils les expriment par les symboles dont ils ornent leurs images. Ces symboles, poursuit Macrobe, se rapportent au soleil, car la tête d'Adad est ceinte de rayons inclinés, qui indiquent que la force du ciel réside dans ceux que le soleil envoie à la terre. Les rayons de la tête d'Atargatis tendent, au contraire, vers le ciel, montrant ainsi que la force des rayons qui viennent d'en haut fait naître tout ce que la terre produit.\footnote{« Deo enim, quem summum maximumque venerantur (Assyrii), \emph{Adad} nomen dederunt ; ejus nominis interpretatio significat, \emph{unus}. Hunc ergo ut potentissimum adorant deum ; sed subjungunt eidem deam nomine \emph{Adargatin} ; omnemque potestatem cunctarum rerum his duobus attribuunt, Solem Terramque intelligentes ; nec multitudine nominum enuntiantes divisam eorum per omnes species potestatem, sed argumentis quibus ornantur, significantes multiplicem præstantiam duplicis numinis. Ipsa autem argumenta solis rationem loquuntur ; namque simulacrum Adad insigne cernitur radiis inclinatis, quibus monstratur vim cœli in radiis esse solis, qui demittuntur in terram. Adargatis simulacrum sursum versum reclinatis radiis insigne est, monstrando radiorum vi superne missorum enasci quæcunque terra progenerat. » (Macrobe, \emph{ubi supra}.)} Déjà nous savions, par le témoignage de Strabon\footnote{\emph{Geograph.} 16., p. 748.} et de Pline,\footnote{\emph{Hist. nat.} 5., 19., 1.} qu'Atargatis, Adargatis ou Adergatis, était un des noms de la Déesse de Syrie. Déjà aussi Hérodien, parlant d'une divinité locale (\textgreek{τοῦ ἐπιχωρίου θεοῦ}), dont l'oracle avait promis la victoire aux habitants d'Aquiléia, ville de la Pannonie, nous avait appris que les Aquiléiens donnaient à cette divinité le nom de \emph{Bélis}, l'identifiaient avec Apollon et lui rendaient un culte solennel.\footnote{\textgreek{Βέλιν δὲ καλοῦσι τοῦτον, σέβουσί τε ὑπερφυῶς, Ἀπόλλωνα εἶναι ἐθέλοντες}. (Hérodien, \emph{Histor.} 8., 3 ; ed. Imm. Bekker.) --- Le manuscrit de Venise, au lieu de \textgreek{Βέλιν}, porte \textgreek{Βέλεν}.} Hésychius, de son côté, dit\footnote{\emph{Sub voc.} \textgreek{Βέλα}.} que, chez les Laconiens, le soleil est adoré sous le nom de \emph{Béla}. D'autre part, Servius, dans ses savants commentaires sur l'\emph{Énéide} de Virgile, affirme que \emph{Bel, Hel} et \emph{Hélios}, sont des dénominations identiques,\footnote{\emph{Ad Æneid.} 1., 646.} et que, chez les Assyriens, par une raison qui tient à certains dogmes religieux, Bel, Saturne et le soleil\footnote{« ... Apud Assyrios autem Bel dicitur, quadam sacrorum ratione, et Saturnus et sol. » (\emph{Ad Æneid.} 1., 733.)} sont une seule et même divinité. Plus tard, Damascius\footnote{\emph{In Vita Isidori, apud} Phot. \emph{Biblioth.} cod. 242, p. 343 ; ed. Imm. Bekker.} affirme à son tour que les Phéniciens et les Syriens identifient Saturne, El, Bel et une divinité qu'il nomme \emph{Bôlathé}.\footnote{... \textgreek{ὅτι Φοίνικες καὶ Σύροι τὸν Κρόνον Ἦλ καὶ Βὴλ καὶ Βολαθὴν ἐπονομάζουσιν}.} Enfin, un mythographe latin anonyme, des derniers temps du paganisme, nous dit qu'on identifiait Adonis avec le soleil,\footnote{\emph{Scriptor. rer. mythic. latini tres ; Mythogr.} 3., fab. 11., 17.} et nous savons qu'Adonis et Bélus étaient un même dieu. Ces diverses traditions ne contribuent pas peu, avec les autres témoignages que j'ai produits plus haut, à nous montrer que, sur les quatre faces de l'autel palmyrénien\footnote{Pl. 1. ci jointe, n°\textsuperscript{s} 1 et 2 ; pl. 2., n°\textsuperscript{s} 1 et 2.} du Musée capitolin, il faut réellement reconnaître, comme j'ai proposé de le faire, les images du Temps-sans-bornes, de Baal ou Bel et du Soleil réunies à l'image symbolique de Vénus, dont le nom asiatique \emph{Beltis} ou \emph{Baaltis} semble facile à reconnaître sous la forme \emph{Bôlathé}, qui en est probablement une altération, ainsi que \emph{Belethé} et \emph{Belthès}. En même temps, nous pouvons, par un passage très-curieux attribué à Ptolémée, constater le rapport intime qui liait au dogme théologique, fondement de cette association de divinités, une croyance astrologique répandue dans l'Orient. Le sens formel de ce passage est que, chez les Assyriens, les Babyloniens, les Perses, etc. qui vivent sous le triangle \emph{notapéliotique} et qui adorent Vénus, Saturne, et Mithra, qui est le soleil, les organes de la génération sont sacrés, à cause des aspects du soleil, de Saturne et de Vénus, qui sont \emph{séminaux}.\footnote{... \textgreek{καθιεροῦταί τε παρ' αὐτοῖς τὰ γεννητικὰ μόρια, διὰ τὸν τῶν προκειμένων ἀστέρων σχηματισμὸν, ὄντα σπερματικὸν φύσει}. (Ptolémée, \textgreek{Τετράβιβλος σύνταξις μαθηματική}, liv. 2., ch. 3., p. 17 édit. de Nuremb. 1535, in-4°.)}

Sestini a publié\footnote{\emph{Descriz. delle med. ant. gr. del mus. Hederv.} 3., 82, n° 10 ; pl. 31., fig. 1.} une monnaie coloniale d'Héliopolis, qui, très-différente de celle dont je viens de parler, mérite néanmoins d'en être rapprochée. Au revers du buste de Valérien, elle représente deux athlètes assis et vus de face ; ils posent chacun sur leur tête une couronne et soutiennent l'urne des jeux, ornée de deux palmes. Auprès de l'un est planté un chêne ; à côté de l'autre s'élève un cyprès. Il est difficile de ne pas supposer que ces arbres sacrés sont les emblèmes des deux divinités en l'honneur de qui furent institués à Héliopolis les jeux gymniques où nos deux athlètes avaient obtenu les palmes de la victoire ; et ces divinités étaient, je le présume, Baal ou Jupiter, à qui nous savons que le chêne fut consacré, et Beltis, Atergatis ou Junon, que déjà, plus d'une fois, nous avons vue représentée sous l'emblème du cyprès pyramidal. Mais rapprochons de cette seconde monnaie coloniale d'Héliopolis les médailles impériales frappées à Éphèse, et nous aurons, au revers de celles-ci, la preuve directe et irrécusable que si, dans l'Asie antérieure, on consacrait le chêne à Zeus ou Jupiter, on lui consacrait aussi le cyprès. Or, sous la domination des Grecs et des Romains, le culte de cette divinité ne devait-il pas être la continuation, pour ainsi dire, du culte de Baal ou Bel, dieu créateur qu'adoraient les Babyloniens, les Assyriens, les Syriens, les Phéniciens ? C'est, en effet, un type composé dans un goût asiatique, qui, sur deux beaux médaillons de bronze, frappés à Éphèse en l'honneur d'Antonin le Pieux, nous offre un cyprès pyramidal, placé, comme symbole de vie, auprès de l'image de Jupiter et au milieu d'édifices dont l'architecture offre un caractère tout particulier. On en jugera par les dessins que je joins ici de ces deux médaillons.\footnote{Pl. 4., n°\textsuperscript{s} 4 et 5.} Mionnet\footnote{\emph{Suppl.} 6., 141, 142, n° 413.} décrit en ces termes le revers du premier : « \textgreek{ΠΕΙΩΝ · ΕΦΕϹΙΩΝ}.\footnote{Sur le nom \textgreek{ΠΕΙΩΝ}, il faut lire les observations très-instructives que notre savant confrère M. Hase avait communiquées à feu M. Mionnet et que ce dernier s'était empressé de publier (\emph{loc. cit.} note \emph{a}) ; elles rectifient une opinion erronée de Seguin, reproduite par Eckhel lui-même.} Jupiter-\emph{Pluvius} tourné à droite, assis sur un siége placé sur le sommet du mont Pion ; il verse la pluie de la main droite et tient un foudre de la gauche ; sur le côté droit du mont, on voit deux habitations à trois étages ; sur le côté gauche est un autre édifice de même architecture ; au pied du mont, à droite, est un temple distyle, vu de côté, derrière lequel est un cyprès ; à gauche, un fleuve barbu, couché. » Le revers du second médaillon\footnote{Pl. 4., n° 5.} ne diffère de celui-là que par la suppression des trois édifices à trois étages et crénelés, que Mionnet indique par ces mots : « habitation à trois étages. » L'habile numismate, en décrivant ce second médaillon d'Antonin le Pieux,\footnote{\emph{Supplém.} 6., 142, n° 414.} commet l'erreur de désigner sous le nom de peuplier l'arbre que, avec toute raison, il reconnaît pour un cyprès sur le premier. Lors même que le témoignage de ces deux monuments monétaires nous manquerait, nous ne pourrions révoquer en doute l'usage où furent les Éphésiens de rendre un culte quelconque au cyprès. Artémidore, né à Éphèse, sous le règne même d'Antonin le Pieux, comprend expressément cet arbre au nombre des objets dont l'apparition dans les songes à une signification digne d'être remarquée. Loin de le considérer comme un présage de mort prochaine, il dit\footnote{\emph{Oneirocritic.} 2., 25. ; 4., 12. ; ed. Reiff.} que le cyprès, à cause de la hauteur qu'il atteint, est le signe de la longanimité et de la temporisation.

Nos deux médaillons servent, pour ainsi dire, de commentaire à cette interprétation, puisqu'ils attribuent le symbole du cyprès à Jupiter, dispensateur suprême de la vie. Ils nous montrent de plus qu'à Éphèse, comme ailleurs, le Jupiter qu'on adorait était un dieu tout à la fois solaire et lunaire ; d'une main, il lance le feu céleste, source des feux du soleil ; de l'autre main, il répand sur la terre des torrents de pluie qui, selon les doctrines communes à tous les peuples de l'antiquité, ont leur source dans la lune. Ces deux attributs nous rappellent un passage de Clément d'Alexandrie\footnote{\emph{Admon. ad gentes}, p. 43 ; ed. Potter.} où il est dit que les Mages considèrent l'eau et le feu perpétuel comme les images de la divinité. C'est aussi pourquoi, au revers de trois monnaies d'Alexandria-Troas, dont j'aurai bientôt à parler, on voit Apollon faire une libation sur un trépied allumé. Parfois, il nous apparaît avec l'attribut de la foudre, et, dans ce cas, sa chevelure est nouée sur son front de manière à figurer une foudre. Tel le représente un buste placé dans la belle galerie de M. le comte de Pourtalès-Gorgier. Deux médailles autonomes grecques d'Alexandria-Troas, décrites par Mionnet,\footnote{Mionnet, \emph{Supplém.} 5., 510, n°\textsuperscript{s} 76 et 77.} nous révèlent la même attribution, à l'aide d'un type qui de plus exprime une idée fort analogue à celle que nous présentent le revers des trois autres médailles citées de la même ville et le revers des deux médaillons d'Antonin le Pieux ; ici, au revers de la tête laurée d'Apollon, nous voyons un cheval paissant, emblème du soleil qui, selon les stoïciens, se nourrit de l'humidité de la terre ; au-dessous du cheval est gravé un foudre. Et puisque la numismatique d'Éphèse et d'Alexandria-Troas nous permet de constater l'assimilation de Jupiter au Soleil ou à Apollon, et \emph{vice versa}, ainsi que le caractère à la fois solaire et lunaire de ces deux divinités, il ne sera pas inutile de dire ici que, non loin d'Éphèse, selon Strabon,\footnote{\emph{Geograph.} 14., p. 639 et 640.} était situé un bois sacré nommé Ortygie et principalement planté de cyprès. Parmi plusieurs temples, ce bois en renfermait un où l'on voyait, auprès de la statue de Latone portant un sceptre, un groupe composé de ses deux enfants, Apollon et Artémis, que leur nourrice tenait par la main. Ajoutons que, dans la ville d'Éphèse, les portes du temple d'Artémis ou Diane étaient de bois de cyprès.\footnote{Théophraste, 5., 5. Pline, \emph{Hist. nat.} 16., 79., 1 et 2. --- Sur les portes de bois de cyprès, voyez Sagittaire, \emph{De januis}, cap. 23.} Vitruve\footnote{2., 19.} mentionne même une statue de la Diane d'Éphèse, faite avec du bois de cette espèce.\footnote{Il ne peut s'agir ici de la statue qui se voyait dans le temple même d'Éphèse. Celle-ci n'était pas de bois de cyprès ; mais on varie beaucoup sur la matière qu'on y avait employée. Selon Xénophon (\emph{de Cyri expeditione}, 5., 3., 12), cette statue était d'or ; selon Pline (\emph{H. N.} 16., 79., 1 et 2), la plupart des écrivains s'accordaient à la dire de bois d'ébène ; mais, ajoute-t-il, Mucien, l'écrivain le plus récent parmi ceux qui ont vu la statue, affirme qu'elle est de bois de vigne, et nous apprend même le nom du sculpteur qui l'avait faite, Pandémion. Cette statue, poursuit Pline, n'a jamais été changée, bien que le temple qui la renferme ait été rebâti jusqu'à sept fois.} Xénophon\footnote{\emph{Loc. cit.}} dit pareille chose d'une statue d'Artémis dont il avait orné le temple érigé par ses soins, à Scillonte, ville d'Élide, située dans la Triphylie. De son côté, Virgile nous apprend\footnote{\emph{Æneid.} 3., 678-680.} que des bois de chênes et de cyprès étaient consacrés tantôt à Jupiter, tantôt à Diane. Il nous ramène ainsi à la monnaie coloniale d'Héliopolis que je viens de citer d'après Sestini, et qui nous a offert un chêne et un cyprès plantés à côté l'un de l'autre, comme images vivantes de Baal, le Jupiter des Syriens, et de Beltis, qui se confond avec Artémis ou Diane.

Après ces divers faits, et surtout après le témoignage qui résulte des médailles citées de Perga,\footnote{Pl. 13., n°\textsuperscript{s} 1 et 2.} quant à l'attribution du cyprès à Artémis, ne doit-on pas s'attendre à trouver dans les textes, ou sur les monuments figurés, la preuve qu'en Orient, sous la domination des Grecs et des Romains, cet arbre symbolique fut nominativement consacré à Apollon, comme il l'était à Artémis ou Diane, au soleil et à la lune ? En effet, nous apprenons de plusieurs écrivains grecs ou latins\footnote{Philostrate, in \emph{Vit. Apollon.} 1., 16. Libanius, \emph{De vita sua}, p. 76 et 77 ; in \emph{Antiochic.} p. 380, 381 ; ed. Morell. Saint Jean-Chrysostome, \emph{Homil.} 17., \emph{ad popul. Antioch. de statuis evers.} Sozomène, \emph{Hist. eccl.} 5., 19. Procope, \emph{Persicor.} 2., 11., 14. Cf. Claudien, \emph{De raptu Proserp.} 3., 372, 373 ; et les commentaires de Godefroy sur le Code Théodosien (\emph{ad libr.} 10., 1., 12). --- Ce savant jurisconsulte a soin de faire remarquer que, sur la Table de Peutinger, le bois sacré de Daphné est indiqué par des cyprès.} qu'il existait, dans le célèbre faubourg d'Antioche de Syrie appelé Daphné, un bois de très-grands cyprès et de lauriers, qui, plantés entre des rochers et sur un terrain accidenté, entouraient un temple dédié à Apollon, divinité dont là, comme ailleurs, les Grecs avaient probablement substitué le nom à celui de Baal ou du dieu solaire adoré par les Syriens sous quelque autre nom. Selon Malala,\footnote{\emph{Chronic.} p. 204 ; edit. Bonn.} ce serait Melkarth, dieu solaire des Tyriens, qui aurait planté les cyprès de Daphné. Là, dit Philostrate,\footnote{\emph{Loc. cit.}} la terre, à cause de Cyparisse réputé assyrien, passait même pour produire le germe ou la semence du cyprès. Couper un des arbres du bois sacré de Daphné, c'était une grave offense envers Apollon.\footnote{Libanius, \emph{De vita sua}, p. 77.} Aussi la plantation s'était-elle conservée jusqu'au temps du bas-empire, comme le prouvent à la fois le témoignage des historiens contemporains,\footnote{Procope, \emph{Persic.} 2., 14.} et les dispositions du code Théodosien\footnote{Lib. 10., tit. 1., \emph{De jure fisci}.} et du code Justinien\footnote{\emph{De cupressis ex luco daphnensi vel perseis per Ægyptum non excindendis vel vendendis} (lib. 11., tit. 77).} qui eurent pour but de mettre le \emph{cupressetum} de Daphné à l'abri des dévastations que, selon la remarque de M. Avellino,\footnote{Mémoire cité, p. 35.} on pouvait craindre de la part des néophytes chrétiens.

Les monuments monétaires de la Troade confirment le témoignage des textes ; ils attribuent le cyprès à Apollon, et par là se lient intimement aux représentations figurées qui déjà, en Syrie et à Perga, nous ont montré le soleil tantôt sous l'emblème d'un des trois cyprès qui ornent le trône votif de la Déesse de Syrie,\footnote{Pl. 5., n° 1.} tantôt sous l'emblème d'un cyprès surmonté d'un astérisque et placé en regard d'un autre cyprès surmonté du croissant de la lune.\footnote{Pl. 14., n°\textsuperscript{s} 1 et 2.} Déjà Virgile nous avait permis de constater que, dès une haute antiquité, le culte du cyprès s'était introduit à Troie même, associé au culte de Cérès ou Rhéa,\footnote{Ci-dessus, p. 80, 81.} divinité primitivement identique avec la Vénus assyrienne. C'est maintenant dans la ville d'Alexandria-Troas que nous allons trouver le culte du cyprès associé à celui d'Apollon, de ce dieu dont Lucien avait vu la statue placée à Hiérapolis, dans le temple de la Déesse de Syrie.\footnote{Ci-dessus, p. 11 et 38.} Au revers des monnaies autonomes coloniales de cette ville, l'image d'Apollon \emph{Sminthius}, vêtu de la stole et vu de face, est placée sur un socle, entre un cyprès pyramidal et un trépied allumé. De la main gauche le dieu tient son arc, et de la main droite une patère, dont il se sert pour faire une libation sur le feu sacré, acte solennel, qui, je le répète, lui donne le même caractère que nous avons reconnu à Jupiter Pluvius, au revers des médailles impériales d'Éphèse.\footnote{Pl. 4., n°\textsuperscript{s} 4 et 5.} Je place sous les yeux de l'Académie\footnote{Pl. 13., n° 1. --- Cette pièce inédite appartient au cabinet des médailles de la Bibliothèque impériale ; elle n'a pas été publiée par Mionnet. Celle qu'a décrite (\emph{Supplém.} 5., 512, n° 90) ce laborieux et très-regrettable numismate offre au droit, comme toutes les autres monnaies coloniales de la même ville, un \emph{vexillum} placé derrière la tête tourrelée de la ville personnifiée. Ce \emph{vexillum} manque sur la pièce que je publie.} un exemple inédit du type que je viens de décrire. À quelques variantes près, on retrouve ce type sur les monnaies impériales coloniales d'Alexandria-Troas. Je joins ici\footnote{\emph{Ibid.} n°\textsuperscript{s} 2 et 3.} les dessins de deux de ces pièces que je crois inédites aussi. L'une, à l'effigie de Commode,\footnote{N° 2. On conserve cette pièce au cabinet des médailles de la Bibliothèque impériale. Elle ne paraît pas avoir été connue de Mionnet ; mais elle a beaucoup d'analogie avec celles qu'il décrit sous le n° 115, aux pages 645 et 646 du tome 2 de sa \emph{Description de médailles}, et sous le n° 141, dans son \emph{Suppl.} t. 5., p. 519. Un revers à peu près semblable à celui de ces deux dernières monnaies se trouve déjà sur les pièces coloniales de la même ville, frappées sous le règne de Marc Aurèle. (V. Mionnet, \emph{Descr. de méd.} 2., 644, 645, n° 109.)} nous offre la statue d'Apollon vu de profil seulement ; le dieu solaire, vêtu d'une longue stole et placé debout sur un piédestal, est tourné à droite vers un trépied allumé. Comme sur les monnaies autonomes coloniales, il porte son arc de la main gauche, et tient de la main droite une patère au-dessus du feu sacré. Mais ici, autour du trépied s'enroule un serpent, symbole de vie\footnote{Sur la signification symbolique du serpent, je demande à l'Académie la permission de me référer à ce que j'en ai dit dans mes \emph{Recherches sur Vénus}, p. 35 et 36. Ce qu'il importe de remarquer ici, c'est que le reptile qui s'enroule autour du trépied d'Apollon est \emph{le bon serpent}, que nous retrouvons sur d'autres monuments qui représentent, soit ce dieu solaire, soit Esculape, soit Hygie. Il sert ainsi à caractériser ces deux dernières divinités, dont les fonctions sont de présider à la conservation de la vie et de la santé. Dans la légende d'Apollon, il est opposé au \emph{mauvais serpent}, appelé Python, que ce dieu tue à coups de flèches. De même, dans les livres sacrés des Perses, Mithra combat avec son arc et ses flèches l'ennemi du ciel, Ahriman, qui a revêtu la forme d'une grande couleuvre ; et, sur les monuments mithriaques d'époque romaine, on voit même le \emph{bon serpent} attaquer résolument le serpent d'Ahriman, c'est-à-dire le \emph{mauvais serpent}. Ajoutons que, selon Damascius (\emph{in Vita Isidori, ap.} Phot. \emph{Bibl.} cod. 242, p. 352), l'Esculape qu'on adorait à Béryte n'était ni grec, ni égyptien, mais d'origine phénicienne.} ; sa tête s'élève au-dessus des flammes de cet autel. Derrière Apollon est planté un cyprès pyramidal, autre symbole de vie. La seconde monnaie impériale coloniale\footnote{Pl. 13., n° 3. Cette médaille a été dessinée par M. Muret, lorsqu'elle était en vente à Paris chez un marchand d'antiquités. Il la croit authentique, mais il ignore quel en est le possesseur actuel. On ne connaissait, avant l'apparition de cette pièce, aucune monnaie coloniale d'Alexandria-Troas qui, au revers de la tête de Maxime, offrît le type que nous avons ici sous les yeux. Ce type peut maintenant être rapproché de celui d'une autre pièce coloniale de la même ville et à l'effigie du même prince, où l'on voit, placée dans un temple, la statue d'Apollon tenant un arc à la main. (Eckhel, \emph{Catalog. Mus. cæs. vindob.} 1., 162, n° 17.)} présente, au revers de la tête de Maxime, le même sujet, avec ces deux seules différences, qu'ici Apollon porte un carquois attaché derrière l'épaule gauche, et qu'autour du trépied ne s'enroule pas un serpent. Il sera intéressant pour nous de constater plus loin, par le témoignage d'un précieux monument d'argent, trouvé dans les Gaules et déposé au musée du Louvre,\footnote{Pl. 20., n° 5.} qu'un type analogue à celui qui est gravé au revers des médailles autonomes coloniales d'Alexandria-Troas fut importé en Occident dès une époque assez rapprochée du temps où Auguste avait érigé cette ville en colonie romaine.

Apollon, Baal ou le Temps, Adad, Zeus ou Jupiter, Beltis, Mylitta, Astarté, la Déesse de Syrie, Rhéa ou Cérès, et Artémis ou Diane ne furent sans doute pas les seules divinités, non plus que le soleil, la lune et les planètes, qui, chez les peuples de l'Asie antérieure, eurent pour emblème ou pour attribut le cyprès pyramidal. Hermès ou Mercure doit être ajouté à cette liste déjà si nombreuse. Et ne serait-il pas surprenant que l'arbre qui tient son nom de Cypris et qui devient l'image symbolique de la déesse au moment même où elle donne le jour au fils de Mercure, n'eût pas été consacré à un dieu dont la légende est inséparable de celle de Vénus ; à un dieu qui représente l'énergie mâle d'une divinité primitivement androgyne, dont Vénus est l'énergie femelle ; à un dieu qui préside, non-seulement à la vie terrestre, mais à la vie future ou céleste, puisqu'il est revêtu de la haute fonction de psychopompe ou conducteur des âmes ? Déjà un passage de Pausanias, cité plus haut,\footnote{Ci-dessus, p. 26, 27.} nous a fait connaître qu'une statue d'Hermès Criophore avait été placée dans un bois sacré de cyprès, à Carnasius en Messénie. Un autre passage du même auteur\footnote{8., 17., 1 et 2.} nous apprend qu'en Arcadie, au sommet du Cyllène, on voyait les ruines d'un temple dédié à Hermès Cyllénien et une statue colossale de ce dieu, faite avec du bois de citrus. Or, c'est dans une grotte de cette montagne célèbre que Maïa avait donné le jour à Hermès ; et Ovide,\footnote{\emph{Fast.} 5., 87.} en caractérisant le Cyllène par l'épithète de \emph{cupressifère} (\emph{cupressifera Cyllene}), nous autorise à croire, malgré le silence de Pausanias sur ce point, que là le temple et la statue d'Hermès avaient aussi été érigés au milieu d'un bois de cyprès. Les médailles de Lydie que je vais décrire, celles d'Achaïe\footnote{Pl. 7., n° 3.} et un fragment de patère\footnote{Pl. 18., n° 2.} dont je parlerai plus loin, achèveront de mettre hors de doute l'attribution du cyprès à Mercure en Orient comme en Occident. Les dessins joints ici reproduisent quatre pièces frappées à Mostène de Lydie. La première\footnote{Pl. 3., n° 3. Mionnet, \emph{Descript. de méd.} 4., 89, n° 482.} est une monnaie autonome où l'on voit, au revers de la tête du Dème, une amazone à cheval, la tête ceinte d'une couronne murale et un bipenne sur l'épaule gauche ; elle s'avance vers un autel allumé et vers un cyprès planté à côté de cet autel. L'intention religieuse d'un tel sujet est manifeste. Nous retrouvons ce type au revers de plusieurs monnaies impériales de la même ville, les unes à l'effigie d'Hadrien,\footnote{Vaillant, \emph{Num. græc.} Mionnet, \emph{Suppl.} 7., 393, n°\textsuperscript{s} 350 et 351.} les autres à l'effigie de Commode,\footnote{Mionnet, \emph{Descr. de méd.} 4., 91, n° 493.} de Philippe père,\footnote{Sestini, \emph{Letter. numism.} t. 8., 94.} ou de Gallien.\footnote{Mionnet, \emph{Descript. de méd.} 4., 92, n° 498 ; \emph{Supplém.} 7., 394, n° 355.} Je donne ici\footnote{Pl. 3., n° 4 ; pl. 4., n° 2.} un exemple de celles de ces pièces qui furent frappées en l'honneur d'Hadrien et de Commode. Une quatrième\footnote{Pl. 4., n° 3. Mionnet, \emph{Descript. de méd.} 4., 92, n° 497.} représente, au revers de la tête de Gallien, non-seulement un cyprès, un autel allumé et une amazone à cheval, mais Hermès ou Mercure, qui, nu et placé à droite de l'arbre sacré, tient d'une main son caducée, et saisit de l'autre la bride du cheval de l'amazone. Par ce mouvement et par toute son attitude, il nous indique, clairement que le but de son intervention est de conduire cette amazone au-delà de l'autel et du cyprès, c'est-à-dire dans un sanctuaire où, sans doute, on célébrait des mystères fondés ici, comme partout, sur le dogme de la descente et de l'ascension des âmes. C'est ainsi que, dans l'Odyssée,\footnote{10., v. 275 sqq.} Hermès se présente à Ulysse au moment où le héros, arrivé dans l'île d'Éa,\footnote{Ou Ææa, île du Pont-Euxin, selon les uns ; île de la Mer tyrrhénienne, selon d'autres. Pline (\emph{H. N.} 25., 5., 2) assigne l'Italie pour patrie à Circé.} va franchir le seuil du redoutable palais de Circé : il lui donne, avec la plante appelée \emph{moly}, le pouvoir d'échapper aux enchantements qui ont métamorphosé ses compagnons en porcs, en loups, en lions. Déjà, par l'ordre suprême de Zeus ou Jupiter, Hermès avait rempli auprès d'Ulysse la mission d'un dieu sauveur en mettant fin à la captivité où le retenait Calypso dans une grotte de l'île d'Ogygie,\footnote{Ile de la Mer ionienne.} entourée d'aunes, de peupliers et de cyprès.\footnote{5., v. 28 sqq.} Or, à mes yeux, cette grotte et surtout le palais de Circé étaient aussi des sanctuaires consacrés à la célébration des mystères de la Vénus asiatique ; et si Calypso et Circé se présentent à nous avec le double caractère de grandes prêtresses et de déesses, si, dans les vers du chantre de l'Odyssée, elles reçoivent même la qualification de déesses,\footnote{Cet usage était passé en Occident. Tacite (\emph{De morib. Germanor.} 8.) parle d'une prêtresse nommée \emph{Véléda} qui, chez les Germains, au temps de Vespasien, était regardée comme une divinité.} c'est que la haute dignité hiératique dont elles étaient revêtues impliquait nécessairement que, de leur vivant, elles avaient obtenu les honneurs de l'apothéose. Les fonctions de psychopompe attribuées à Hermès et le passage cité\footnote{Ci-dessus, p. 25-27.} de Pausanias sur le rôle qu'il remplissait dans les mystères de la Mère des dieux justifient, ce me semble, mes diverses interprétations. Dans mon ouvrage sur le culte de Mithra, j'aurai prochainement l'occasion de montrer que les amazones jouent un grand rôle parmi les personnages initiés aux mystères de la Vénus asiatique. Nous les verrons même accomplir, en présence du cyprès et du palmier femelle, certaines cérémonies propres à ces mystères. Cette circonstance me donne lieu de supposer que si, au revers d'une monnaie autonome de Mastaura,\footnote{Pembroke, \emph{Num. antiq.} pars 2., tab. 39. Mionnet, \emph{Descript.} 4., 83, n° 453.} autre ville de Lydie, nous voyons un cyprès planté à côté d'un autel, sans être accompagné d'aucune figure, nous pouvons cependant rapporter ce type au culte de la Vénus asiatique ou à celui de Mercure\footnote{Remarquons cependant qu'une monnaie impériale de Mastaura, à l'effigie de Caracalla, offre un revers que Mionnet (\emph{ibid.} 85, n° 464) décrit en ces termes : « Apollon nu, le pallium sur l'épaule, debout devant un autel près duquel est un arbre entouré d'un serpent, et tenant dans la main droite une patère, la gauche sur le flanc ; près de lui un cippe. » Je n'ai pas vu cette pièce, qui appartenait autrefois à Cousinéry ; mais je suis très-porté à croire que l'arbre non dénommé par Mionnet est un cyprès. S'il en est ainsi, nous avons là une variante curieuse de la médaille impériale d'Alexandria-Troas (pl. 13., n° 2), où Apollon est placé debout entre un cyprès et un trépied allumé, autour duquel s'enroule un serpent. Dès lors, le type de la monnaie autonome de Mastaura que je cite pourrait tout aussi bien se rapporter au culte d'Apollon qu'à celui de Vénus ou de Mercure. La numismatique de cette ville prouve que la première de ces trois divinités y était en grande vénération ; car il faut ajouter à la médaille citée de Caracalla deux monnaies autonomes (Mionnet, \emph{Descript.} 4., 83, n°\textsuperscript{s} 454 et 455) qui nous offrent l'image d'Apollon. En même temps, la série des monnaies impériales nous montre (\emph{ibid.} p. 87, n° 477. \emph{Mus. Theup.} 2., p. 1336) que les habitants de Mastaura rendaient aussi un culte particulier à Vénus et à Mercure.} ; car, au revers d'une monnaie impériale de la même ville, frappée en l'honneur de Tibère et de Livie,\footnote{Du Mersan, \emph{Descr. des méd. du cab. Allier de Hauteroche}, p. 99.} nous retrouvons une amazone à cheval ; et, bien que le cyprès ne se montre pas dans la composition de ce type, nous sommes tout naturellement ramenés aux médailles citées de Mostène, et, par conséquent, à un culte lydien, où Vénus, le cyprès et les amazones jouaient un rôle particulier.

Ces remarques me portent à penser que si, au revers d'une médaille frappée en l'honneur de Commode à Héraclée de Bithynie et publiée par Sestini,\footnote{\emph{Letter. num.} 6., 42, 43 ; pl. 2., n° 5.} on voit, à côté d'un cyprès pyramidal, Hercule debout, armé d'un arc et d'une massue, ce cyprès n'est ni l'arbre des Hespérides, comme le veut le numismate italien, bien qu'il le reconnaisse pour un cyprès, ni « l'arbre consacré au Soleil, et à ce titre l'arbre le plus propre à la construction du bûcher d'Hercule, » comme le conjecture notre savant confrère M. Raoul-Rochette.\footnote{\emph{Mém. de l'Acad. des inscr.} nouv. série, t. 17., 2\textsuperscript{e} part. p. 294. --- Dans le savant Mémoire que je cite ici, M. Raoul-Rochette, adoptant sans restriction l'opinion d'Eckhel, semble ne considérer le cyprès que comme l'emblème d'une seule divinité, le Soleil. Il pense qu'à ce titre le cyprès devait entrer dans la construction du bûcher (\emph{pyra}) d'Hercule, et faisait allusion à l'apothéose de ce personnage mythologique. Par-là, il se trouve disposé à reconnaître cet arbre auprès d'Hercule sur quelques monuments où personne avant lui ne l'avait reconnu. C'est ainsi, par exemple, qu'au revers de deux monnaies autonomes de Sardes, publiées, l'une par Sestini (\emph{Descr. numism. veter.} p. 436, n° 15), l'autre par Hunter (\emph{Mus. Hunter.} tab. 47, n° 18.), Hercule nu, assis sur un rocher, lui paraît être appuyé sur un cyprès (\emph{loc. cit.} p. 210). Je suis obligé de dire qu'ici Hercule s'appuie non sur un arbre, mais sur sa massue renversée ; il en est de même au revers des médailles impériales de la même ville (voy. Vaillant, \emph{Num. gr.} Sestini, \emph{Descriz. d'alc. med. gr. del mus. Fontan.} t. 1., p. 112, n° 4 ; pl. 3., p. 74, n° 5. Mionnet, \emph{Suppl.} 7., 421, n° 476), médailles où Hercule est représenté debout, au lieu d'être assis.} Je pencherais plutôt à croire qu'au lieu d'avoir un sens funéraire, le cyprès est ici, de même que sur les médailles citées de Mostène, l'image symbolique de la Vénus orientale, de cette déesse aux mystères de qui Hercule, comme chaque amazone, avait été initié. L'habile archéologue que je viens de nommer me fournit lui-même, à son insu, deux arguments qui peuvent justifier ma supposition. D'une part, il soutient avec toute raison\footnote{Mémoire cité, p. 290.} que l'Hercule à qui les habitants d'Héraclée rendaient des honneurs divins, était non l'Hercule grec, mais l'Hercule phénicien. D'autre part, il n'hésite pas à reconnaître\footnote{\emph{Ibid.} p. 291, 292.} la tête d'Astarté, la Vénus phénicienne, sur plusieurs monnaies d'argent, de petit module et de fabrique archaïque, qui appartiennent à la numismatique de cette ville. Il publie quatre de ces pièces,\footnote{\emph{Ibid.} pl. 5., n°\textsuperscript{s} 1-4.} et n'a pas de peine à établir leur authenticité, longtemps mise en doute. Si, comme je le crois avec lui, leur type nous offre la tête de la Vénus phénicienne, représentée avec un caractère de physionomie étranger à l'art grec, ne devient-il pas très-probable que les Grecs, comme plus tard les Romains, à leur exemple, ne se bornant pas à figurer cette divinité, sur les médailles, d'après un tel type, la représentèrent aussi sous un des emblèmes symboliques qui lui étaient consacrés dans les sanctuaires des Phéniciens ? Et le cyprès pyramidal n'était-il pas un de ces emblèmes ? Un fait semblable à celui que me paraît présenter la série des monnaies d'Héraclée ne se reproduit-il pas dans la numismatique de l'île phénicienne d'Aradus et de la ville syrienne de Damascus ? Ne nous fournit-elle pas la preuve que, dans ces deux localités, on représentait la déesse Astarté tantôt sous une forme purement humaine, tantôt sous le symbole du cyprès pyramidal ? Ailleurs, ne trouvons-nous pas, sur les monuments d'une même contrée, d'une même île, d'une même cité, Vénus et Diane figurées chacune alternativement sous une forme humaine, sous une forme moitié humaine, moitié animale, et sous l'emblème d'un cône ?

Je ne sais à quelle divinité, à quelle province de l'Asie occidentale rapporter nominativement et avec certitude le monument dont il me reste à parler pour achever la revue de toutes les antiquités asiatiques qui, à ma connaissance, offrent quelque trace du culte du cyprès. Le style du monument, le sujet qu'il représente, le costume des personnages que nous y voyons figurer, le choix et la disposition des symboles ou des accessoires qui les accompagnent, tout nous avertit, au premier abord, que de nouveau nous avons devant les yeux une de ces compositions religieuses dont la conception appartient à l'Orient et dont l'exécution décèle les derniers temps de la domination romaine. Celle-ci est gravée en creux sur une petite plaque de bronze, carrée, que M. Éd. Gerhard acquit à Rome, en 1835, et qu'il déposa au musée royal de Berlin. Le dessin que j'en donne,\footnote{Pl. 7., n° 6.} d'après un calque dont je suis redevable à ce savant ami, la représente avec les dimensions de l'original.\footnote{Deux trous carrés, pratiqués, l'un dans le haut, l'autre dans le bas de cette plaque, nous montrent qu'elle avait dû être fixée sur les parois d'un édifice, ou sur un objet quelconque destiné à un usage religieux.} Une explication complète du sujet qui en couvre la surface m'entraînerait beaucoup trop loin. Elle trouvera plus convenablement sa place dans mon travail sur les mystères institués en Orient, et n'est pas nécessaire pour comprendre le rôle symbolique que jouent ici les deux cyprès plantés, l'un à droite, l'autre à gauche de la scène principale. Deux serpents naissent chacun du centre de ces arbres, et s'élancent vers une tête de lion, vue de face et gravée, dans le haut du tableau,\footnote{Nous retrouverons sur un bas-relief mithriaque romain, dont il sera question plus loin, un mufle de lion, vu de face, et placé entre le buste du Soleil et le buste de la Lune superposés chacun à un cyprès (voy. ma pl. 15.).} entre deux astérisques à six branches, qui me paraissent tenir ici la place de la planète Vénus considérée sous son double aspect, c'est-à-dire le matin et le soir. À gauche de l'un des deux astérisques, on voit le buste du Soleil, et, à droite du second, le buste de la Lune. Ces deux puissants agents de la création et de la reproduction, ainsi représentés sous une forme humaine, occupent chacun un des angles supérieurs du monument ; ils se trouvent donc superposés ici à deux cyprès, comme le sont également, mais sous leur forme d'astres, le soleil et la lune au revers des deux médailles citées de Perga.\footnote{Pl. 14., n°\textsuperscript{s} 1 et 2.} Une disposition différente, dont toutefois la signification est la même, se fait remarquer sur trois bas-reliefs mithriaques, de travail romain, où nous verrons bientôt deux cyprès plantés tantôt à côté des bustes du Soleil et de la Lune, tantôt entre ces deux bustes. Mais sur notre plaque de bronze, je le répète, du centre de chacun des deux cyprès naît un serpent, et cette circonstance achève de caractériser, comme symboles du principe actif et du principe passif de la vie, ces deux arbres sacrés, l'arbre du soleil et celui de la lune ; car nous ne pouvons oublier que le serpent est lui-même un symbole de vie.\footnote{Voy. ci-dessus, p. 106, 107, note 5.} Ici les deux reptiles semblent mordre chacun une des oreilles du mufle de lion. Leur tête est ainsi liée ou soudée à celle de cet animal solaire, en même temps que leur queue, comme une racine dans la terre, reste engagée dans le centre de l'arbre où elle puise la vie. De cette double disposition, il résulte que la tête de lion devient le point culminant d'un encadrement formé par les deux cyprès et par les deux serpents. Ainsi les personnages de la scène d'initiation gravée au centre de cet encadrement se trouvent placés entre la région supérieure ou la lumière, et la région inférieure ou les ténèbres ; la première, représentée ici tout à la fois par le Soleil, la Lune, Phosphore, Hesper, et la tête de lion, animal solaire ; la seconde, caractérisée dans le bas du tableau par un bélier, un taureau, un poisson et une oie, tous symboles du principe humide. À la vue de cette tête de lion, pour ainsi dire mordue de chaque côté par un serpent, on se rappelle involontairement les statues ou les bas-reliefs que j'ai restitués à Mithra \emph{Léontocéphale},\footnote{\emph{Mém. de l'Acad. des inscript.} nouvelle série, t. 15., 2\textsuperscript{e} partie, p. 201-306.} et un petit bas-relief assyrien où je me suis cru autorisé à reconnaître l'image de Vénus \emph{Léontocéphale}.\footnote{\emph{Ibid.} p. 230 et suiv.} Les figures de Mithra dont j'entends parler représentent ordinairement ce dieu des Perses le corps entouré par les replis d'un serpent qui, s'élevant graduellement, place sa tête au sommet et sur le devant d'une tête de lion implantée sur les épaules de Mithra au lieu d'une tête humaine.\footnote{\emph{Recherch. sur Mithra}, atlas, pl. 70. ; pl. 72., n°\textsuperscript{s} 1 et 2.} D'autrefois le serpent se replie autour du cou de ce dieu, et paraît vouloir mordre le bas de la tête de lion\footnote{\emph{Ibid.} pl. 73., n° 1 ; \emph{Mém. de l'Acad. des inscript. loc. cit.} pl. 1., n° 1.} ; ou bien, le serpent est double et alors ses deux têtes viennent se placer, en regard l'une de l'autre, sur la poitrine de Mithra Léontocéphale.\footnote{\emph{Rech. sur Mithra}, atlas, pl. 71., n° 3.} D'autrefois encore deux serpents s'enroulent autour des ailes de cette divinité et s'élancent de là vers sa tête de lion.\footnote{\emph{Ibid.} n° 2.} De son côté, le petit monument assyrien représente Vénus-Mylitta à tête de lion, tenant un serpent de chaque main.\footnote{\emph{Mém. de l'Acad. des inscript. loc. cit.} pl. 2., n°\textsuperscript{s} 1 \emph{a}, 1 \emph{b} et 1 \emph{c} ; \emph{Recherch. sur Vénus}, pl. 17., n°\textsuperscript{s} 1, 1 \emph{a} et 1 \emph{b}.} C'est toujours, quant aux rapports du serpent avec le lion et à la signification symbolique de ces deux animaux, la même idée exprimée, on le voit, par des procédés différents. Un second rapprochement va nous révéler une nouvelle communauté d'idées, qui donne à la plaque de bronze du musée de Berlin un intérêt tout particulier dans la question de la signification symbolique du cyprès sur les monuments figurés que nous a légués l'antiquité asiatique. Nous ne pouvons, en effet, voir ici la vie, sous le symbole du serpent, sortir du sein d'un cyprès, comme ailleurs, sous le même symbole, elle s'échappe de la ciste sacrée des mystères dionysiaques, sans nous rappeler la face antérieure du grand bas-relief mithriaque découvert, en 1832, à Heddernheim : là,\footnote{\emph{Annal. des Vereins für nassau. Alterthumsk.} t. 1., pl. 1. \emph{Recherches sur Mithra}, atlas, pl. 110.} du sein d'un arbre, et sous la forme d'un enfant, sort la vie humaine, de même que, sur une des faces de l'autel palmyrénien du Musée capitolin,\footnote{Pl. 1., n° 2.} l'Amour Criophore naît des flancs mêmes d'un cyprès. J'expliquerai plus loin comment, dans les sculptures du bas-relief de Heddernheim, trois arbres de cette dernière espèce servent à caractériser les trois grandes phases de la vie humaine. Dès à présent, je crois en avoir dit assez pour montrer combien peu sont fondés les archéologues qui ne veulent voir dans le cyprès qu'une image du soleil\footnote{C'est donc à tort que feu M. Émeric David (\emph{Jupiter}, t. 1., Introd. pag. 269. ; p. 107 ; t. 2., p. 436, 437), partant de l'idée que l'obélisque est l'image des rayons du soleil, considère exclusivement le cyprès comme « un obélisque naturel qui s'emploie souvent, dans les compositions allégoriques, à la place de l'obélisque artificiel. » Il est à regretter que M. Avellino ait circonscrit la signification symbolique du cyprès au point de donner son plein assentiment à cette opinion (voy. Mém. cité, p. 29, 30).} et qu'un emblème funéraire.

\subsubsection{Arabie}
\paragraph{}
Hérodote nous apprend que les Arabes avaient reçu des mains des Assyriens le culte de Vénus-Uranie, et qu'ils la nommaient \emph{Alitta}\footnote{Hérodote, 1., 131. --- Selden (\emph{De diis syr.} syntagm. 2., cap. 2., p. 179) identifie avec l'\emph{Alitta} d'Hérodote la divinité qui, dans le Koran (sur. 53., v. 19), suivant les diverses manières de lire le nom, est appelée \emph{Allat} (\emph{Lat}), \emph{Alleth}, ou \emph{Alletto}. M. Kopp (ouvrage cité, p. 269) estime que ce passage du Koran a été mieux interprété par Selden que par aucun autre traducteur.} et \emph{Alilat}.\footnote{Hérodote, 3., 162.} Ce fait important est confirmé, soit par les nombreuses médailles impériales frappées à Bostra,\footnote{Mionnet, \emph{Descript. de méd.} 5., 579-584, n°\textsuperscript{s} 7-10, 14, 29, 35. Sestini, \emph{Descr. num. veter.} p. 548, n° 1.} à Esbus\footnote{Mionnet, \emph{loc. cit.} 585, n° 38.} et à Moca,\footnote{\emph{Id. ibid.} 586, n° 40.} sur lesquelles est gravée l'image de la Vénus asiatique, soit par les inscriptions himyarites, soit par le témoignage de saint Hilarion, soit enfin par les écrits mêmes de plusieurs Arabes musulmans et par certains usages, certaines superstitions que l'islamisme n'a pu abolir.

Les inscriptions himyarites dont j'entends parler ont été découvertes, il y a peu d'années, par M. Arnaud,\footnote{Voy. la relation du voyage de M. Arnaud et les pièces publiées par les soins de M. Jules Mohl dans le \emph{Journal asiatique}, 4\textsuperscript{e} série, t. 5., p. 208 et suiv. t. 6., p. 169 et suiv.} sur le sol de l'Arabie. M. Fulgence Fresnel\footnote{Voy. ses Lettres à M. J. Mohl, \emph{ibid.} p. 194-237.} a essayé de les interpréter, et croit y trouver la preuve que les anciens habitants de cette contrée adoraient une divinité du nom d'\emph{Athtor} ou \emph{Othtor}. Il n'hésite pas à reconnaître sous ce nom la déesse que les Sidoniens et les Hébreux appelaient \emph{Aschtôreth} ; les habitants de Saba, \emph{Aschtor} ; les Égyptiens, \emph{Ator} ou \emph{Athor}, \emph{Athyr} ou \emph{Athyri} ; les Grecs, \emph{Astarté}. Par-là se trouverait confirmée l'origine asiatique que, malgré le silence d'Hérodote sur ce point, j'attribue à la déesse égyptienne en l'assimilant à la Vénus des Assyriens et des Phéniciens. Si l'on remarque que le nom du pays appelé \emph{Aschour} et \emph{Atour}, dans la Bible,\footnote{\emph{Aschour}, dans le texte hébreu ; \emph{Atour}, dans les versions araméennes.} \emph{Assyrie}, chez les Grecs et les Romains, est écrit \textgreek{Ἀτυρία}, \emph{Aturie}, dans quelques manuscrits de Dion Cassius\footnote{\emph{In Trajano} (68., 14., 26). --- Reimar (p. 1141) et Sturz (t. 4., p. 334) substituent \textgreek{Ἀσσυρία} à la forme \textgreek{Ἀτυρία} ou \textgreek{Ἀττυρία}. Dans Strabon (16., 737), \textgreek{Ἀτουρία} ne désigne que le territoire de Ninive.} ; \emph{Athour}, $\arabicAAA$, dans Massoudi\footnote{\emph{Notic. et extr. des Mss.} t. 8., p. 159.} ; \emph{Athurá}, dans les inscriptions en caractères cunéiformes du système persique, qui sont gravées sur les murs de Persépolis,\footnote{M. le professeur Lassen, \emph{Über die Keilinschriften der erst. und zweit. Gattung}, p. 46, 48, 52, 92, 99, 175, 176, 178.} ou sur le roc, à Bi-sutoun,\footnote{M. le lieutenant-colonel Rawlinson, \emph{Journ. of the roy. asiat. Soc.} vol. 10., part. 1., p. 1, col. 1, l. 14, 15 ; p. vj, col. 11, l. 7.} et enfin \emph{Assur} dans les inscriptions en caractères cunéiformes du système assyrien ou babylonien,\footnote{\emph{Id. ibid.} vol. 14., part. 1., col. 1, l. 5, l. 40. --- \emph{Aschour} et \emph{Assur} nous révèlent peut-être l'étymologie d'\emph{Aschéra}, nom qui, dans la Bible, paraît désigner Vénus.} on sera peut-être même tenté de supposer avec moi que Vénus, chez les Arabes et chez les Égyptiens, avait reçu un nom ou un surnom qui rappelait l'origine assyrienne de son culte.\footnote{Remarquons toutefois que le nom du pays commence par un \emph{aleph}, tandis que le nom de la déesse a pour première lettre un \emph{ayin}.} N'est-ce pas pour la même raison que chez les Grecs on l'appelait \textgreek{Θεὰ ἀσσυρία},\footnote{Nonnus, \emph{Dionysiac.} 31., 203.} la \emph{Déesse assyrienne} ou d'\emph{Assyrie}, et, plus tard, \textgreek{Θεὰ συρία},\footnote{Lucien, \emph{Opp.} t. 9. --- Cicéron (\emph{De nat. Deor.} 3., 23) compte \emph{Astarté} au nombre des quatre Vénus qu'il trouve dans les auteurs anciens ; il la déclare \emph{syrienne} et née à Tyr. --- Larcher (\emph{Mém. sur Vénus}, p. 23), sans citer ses autorités, dit que Vénus avait à Rome un temple où elle était adorée avec Adonis, « suivant le rite assyrien. »} la \emph{Déesse syrienne} ou \emph{de Syrie} ? Mais M. Fresnel,\footnote{\emph{Loc. cit.} p. 199.} après avoir dit que le nom \emph{Athtor} ou \emph{Othtor} « ne coïncide exactement avec aucune racine hébraïque ou arabe, » constate que, par une transformation dont il cite d'autres exemples, \emph{Athtor} s'identifie avec l'\emph{Aschtôreth} (Astarté) des Hébreux et des Phéniciens, sauf l'absence de la terminaison féminine dans le premier de ces noms\footnote{M. Fresnel (\emph{ibid.} pag. 200, 237) dit qu'\emph{Athtor}, la Vénus des Arabes, n'est point une divinité femelle, comme l'était \emph{Aschtôreth} ou \emph{Astarté}, mais qu'elle est une des trois divinités dont le nom, dans les inscriptions himyarites, « a une forme masculine, » grammaticalement parlant. Cette observation est d'autant plus précieuse à recueillir, que le nom d'\emph{Athtor} devient ainsi, avec celui de \emph{Mithra}, un second exemple de l'assimilation d'une divinité mâle ou androgyne avec la déesse femelle qu'Hérodote (1., 131) appelle \emph{Uranie} et \emph{Mylitta}. Par-là se trouve indirectement confirmée l'interprétation que je donne au passage si controversé, où cet historien identifie Mithra avec ces deux dernières divinités et avec Alilat ou Alitta (voy. mes \emph{Recherches sur Vénus}, p. 183-190).} ; et il propose une étymologie qui rapprocherait \emph{Athtor} d'une racine arabe qu'on trouve dans le mot $\arabicAAB$ (\emph{atar}). Je me réserve d'examiner ailleurs, avec toute l'attention que méritent les opinions de ce savant, la valeur de cette étymologie.\footnote{J'examinerai alors aussi l'étymologie qu'on a proposée du nom d'Hâthôr, \emph{demeure} ou \emph{maison d'Horus}, d'après les signes hiéroglyphiques qui servaient à écrire ce nom sur les monuments antiques de l'Égypte. On peut déjà pressentir que je ne suis pas disposé à l'adopter, et que je penche plutôt à considérer la forme \emph{Hâthôr} ou \emph{Athyr} comme la transcription égyptienne d'un adjectif qui, dans les idiomes asiatiques, signifiait l'\emph{Assyrienne}. Une note que me communique en ce moment M. Adrien de Longpérier, et que je publierai dans mes \emph{Recherches sur Vénus}, me prouve que les racines fournies par l'arabe, par l'hébreu et par le dialecte chaldaïque, loin de contrarier ma supposition, servent puissamment à la justifier. Car, s'il résulte des observations philologiques de ce savant que le nom d'Athor ou Hâthôr emporte avec lui, comme je l'ai toujours conjecturé, la double idée ou signification de \emph{taureau} et d'\emph{Assyrie}, je suis amené à faire remarquer que la province de la Grèce ancienne appelée \emph{Béotie} avait vu Thèbes, sa capitale, bâtie sous les auspices de la Vénus assyrienne ou phénicienne, symboliquement représentée, comme dans tout l'Orient, par une \emph{vache}. Car c'est la déesse elle-même qu'il faut reconnaître dans la vache qui, ainsi que l'avait annoncé l'oracle, servit de guide à Cadmus lorsqu'il débarqua sur les côtes de la Grèce pour y fonder une colonie. À côté de ce fait vient se placer le témoignage d'Hésychius. Ce lexicographe (\emph{voc.} \textgreek{Ἀθύρ} ; ed. Alberti) nous dit que, chez les Égyptiens, \emph{Athyr} était à la fois le nom d'un mois et le nom d'une vache. Remarquons cependant que, dans les textes hiéroglyphiques des monuments égyptiens, la vache sacrée, emblème d'Hâthôr, reçoit (Champollion le jeune, \emph{Panth. égypt.} expl. de la pl. 23 \emph{E}) le nom d'\emph{Éhé} ou \emph{Ahé}, probablemment emprunté aux idiomes sémitiques ou araméens, qui lui attribuent la double signification de \emph{vie} et de \emph{serpent} (\emph{Recherch. sur Vénus}, p. 35, 36). Mais ajoutons que cet emprunt est une nouvelle preuve de l'origine asiatique qu'avaient en Égypte le culte de Vénus Hâthôr et les symboles de la vache et du taureau.} Ce qu'il importe de faire remarquer ici, c'est que le cuivre, comme je l'ai dit plus haut, était le métal consacré à Vénus, et qu'on a découvert, dans la presqu'île du Sinaï, un grand nombre de stèles égyptiennes, d'ancien style, qui représentent la déesse Hâthôr\footnote{Voy. la note ci-après.} et mettent sous sa protection particulière les riches mines de cuivre situées dans cette presqu'île célèbre.\footnote{Les gisements du cuivre sont dans la vallée que les Arabes appellent \emph{Ouadi-Magarah} et à \emph{Serbout el-quadim}. Les inscriptions hiéroglyphiques des stèles consacrées dans ces lieux à la déesse Hâthôr fournissent la preuve que, dès une époque très-reculée, et principalement sous les 4\textsuperscript{e}, 5\textsuperscript{e}, 12\textsuperscript{e}, 18\textsuperscript{e}, 19\textsuperscript{e} et 20\textsuperscript{e} dynasties, ces mines étaient exploitées par les Égyptiens. Sur les stèles, le nom du pays est écrit \emph{Mawek}, selon la transcription de M. le vicomte Emmanuel de Rougé ; \emph{Mafek}, selon celle de Champollion le jeune, qui reconnaît que ce nom, dans le groupe $\hieroAAAA\:\hieroAAAB$, est en même temps celui du métal exploité, le cuivre. Presque tous les proscynèmes trouvés dans la presqu'île s'adressent à la déesse Hâthôr. Les légendes hiéroglyphiques qui là accompagnent son image lui donnent le nom de $\hieroAAAC\hieroAAAD\:\hieroAAAE\:\hieroAAAB\:\hieroAAAF$, \emph{Dame de Mawek}, ce qui implique à la fois l'idée du cuivre et celle du pays, mais surtout cette dernière. Ici, la déesse est représentée sous une forme humaine ; elle est vêtue à l'égyptienne, et sa tête est surmontée d'un disque placé entre deux grandes cornes de vache. Quelques-uns de ces monuments ont été indiqués par les premiers voyageurs européens qui ont visité le mont Sinaï. Burton en a publié plusieurs dans ses \emph{Excerpta hieroglyphica}. M. le comte Léon de Laborde a donné de quelques autres, dans son \emph{Voyage en Arabie}, des croquis faits un peu à la hâte. Mais M. Lottin de Laval (voyez \emph{Revue archéol.} septième année, n° du 15 juillet 1850 ; huitième année, n° du 15 mai 1851), a rapporté et déposé au musée du Louvre les empreintes de toutes les stèles de cette espèce qu'il a moulées sur les lieux mêmes. Enfin, dans le grand et bel ouvrage qui nous fait connaître les découvertes et les travaux de la commission envoyée en Égypte par S. M. le roi de Prusse, M. le docteur R. Lepsius vient de publier presque tous ceux de ces monuments qui offrent un intérêt particulier. --- J'extrais la plupart de ces détails d'une note savante qu'a bien voulu me remettre M. de Rougé et que j'insérerai en entier dans mes \emph{Recherches sur Vénus}. Elle se termine par l'observation que, dans quelques textes hiéroglyphiques sacrés, \emph{Mawek} est pris pour le type de l'Orient, ce qui, par rapport à l'Égypte, semble naturel, eu égard à la position topographique de la contrée appelée \emph{Mawek}.} A ce témoignage il convient d'ajouter celui de saint Hilarion\footnote{Saint Jérôme, \emph{in Vita S. Hilarionis}, § 25 ; cf. \emph{in Amos}, 5. Cf. Gordien le moine, \emph{in S. Placidi martyr. Vita}, n. 61. --- Sous le règne de l'empereur Arcadius (395 à 408) le culte de Vénus était encore en grand honneur à Majuma, port de Gaza, ville peu éloignée du bourg de Thabatha où était né saint Hilarion.} et celui de saint Théodule,\footnote{Voyez un écrit attribué sans preuves suffisantes à saint Nil et intitulé : \emph{De cæde monachorum in monte Sina} (capp. 4., 14.). Il a été inséré, par le P. Combefis, dans l'\emph{Auctar. novissim. Biblioth. Patrum}, et, par les Bollandistes, dans les \emph{Acta Sanctorum}, au 14 janvier. Cet écrit et la Vie de saint Hilarion m'ont été indiqués par notre savant confrère M. Reinaud. --- Voyez aussi, dans le recueil des Bollandistes (\emph{loc. cit.}), un fragment qui a pour titre : \emph{De S. Theodulo presbytero, monacho in monte Sina.} --- Les renseignements fournis par saint Théodule sont d'autant plus précieux que ce religieux, enlevé par les Sarrasins nomades, vécut assez longtemps au milieu d'eux pour connaître en détail les rites qu'ils observaient dans les prières, les sacrifices humains et les sacrifices de chameaux blancs qu'ils offraient à l'étoile du soir et à l'étoile du matin, c'est-à-dire à Vénus.} fils de saint Nil. Nous apprenons de ces deux pieux personnages que, de leur temps, c'est-à-dire au 4\textsuperscript{e} et au 5\textsuperscript{e} siècle de notre ère, les Sarrasins nomades qui vivaient dans les déserts d'Arabie, les Iduméens ou Ismaëlites, et nommément les habitants de la ville d'Élusa\footnote{Cette ville, située dans l'Idumée, sur les confins de l'Arabie Pétrée, s'est appelée aussi \emph{Elusium} et \emph{Eolesa}.} et de la presqu'île du Sinaï, rendaient un culte particulier à l'étoile du soir et à l'étoile du matin, qui, pour ces peuples, étaient la manifestation de la déesse que nous appelons Vénus.

À leur tour les écrivains arabes musulmans rapportent certaines traditions, malheureusement peu nombreuses et très-succinctes, qui sont relatives à une divinité féminine, identique avec la Vénus-Allat ou Alitta d'Hérodote, et honorée d'un culte qui, avant l'islamisme, paraît avoir été fort répandu dans l'Arabie. Ils la désignent sous les noms d'\emph{Allat} ou \emph{Ellat} et d'\emph{el-Samora}, dont le sens littéral n'est pas bien connu, et sous celui d'\emph{el-Uzza} ou \emph{el-Ozza}, qui signifie à peu près \emph{la Glorieuse}. Il y a plus d'une raison de croire qu'elle était adorée à la Mecque, dans le temple même qui est devenu la mosquée la plus révérée des musulmans, et un lieu célèbre par le pèlerinage que, chaque année, ils y font en grande pompe. Scharistâni, cité par Pococke,\footnote{\emph{Spec. hist. Arab.} p. 120 ; ed. Oxon. 1806.} nous apprend que la déesse avait un temple à \emph{San'a} ou \emph{Sanaa}, dans le Yémen. Ce temple, qui s'appelait \emph{Bayt-Ghomdan, maison} ou \emph{demeure de Ghomdan}, fut détruit par le khalife Othman. Les Sarrasins surnommaient \emph{el-Khabar, la Grande},\footnote{Voyez, dans mes \emph{Recherches sur Vénus} (p. 106, 107), mes observations sur l'identité de ce nom avec celui de \emph{Cabire}.} la divinité qu'on y adorait ; et, jusqu'au temps de Mahomet, le culte de cette divinité continua à être pratiqué parmi eux. Nous en avons la preuve dans la formule d'un anathème qu'était obligé de prononcer contre la déesse \emph{el-Khabar} tout Sarrasin qui se convertissait au christianisme.\footnote{Voy. \emph{Rech. sur Vénus}, p. 107.} De nos jours encore, les Arabes donnent à la planète Jupiter le surnom de \emph{Grande Fortune}, et à la planète \emph{el-Zahara},\footnote{Ce nom ou ce surnom emporte avec lui l'idée de \emph{splendeur, de lumière éclatante}.} qui est Vénus, le surnom de \emph{Petite Fortune}. Ajoutons que les écrivains orientaux confessent hautement qu'\emph{el-Zahara} est la \emph{planète de l'islam}, et nous comprendrons combien sont restées profondes les racines que le culte assyrien de Vénus avait jetées dans le sol de l'Arabie.

D'autre part, si le zèle fanatique des sectateurs de Mahomet n'a laissé subsister dans cette contrée aucun des livres sacrés des Arabes, adorateurs de Vénus, aucun monument architectonique ou figuré qu'on puisse avec toute certitude rapporter au culte national de cette antique divinité, les poètes et les historiens orientaux nous fournissent du moins la preuve qu'avec le culte de Mylitta les Assyriens avaient transmis aux Arabes l'usage de consacrer à la déesse le palmier et le cyprès. Remarquons d'abord que si le dernier de ces deux arbres n'est pas indigène en Arabie, comme on l'assure, il y est du moins cultivé de temps immémorial,\footnote{Les voyageurs modernes citent des plantations de cyprès notamment à Ouadi-Bischeh et sur la route d'Aakik à Thérad (voy. M. Carl Ritter, \emph{Die Erdkunde}, t. 1., p. 957, 998).} et joue, dans les poètes nationaux, le même rôle que lui attribuent les poètes persans. Les écrivains arabes l'appellent de quatre noms, \emph{serv} (\emph{serb}) ou \emph{sarou},\footnote{On écrivait autrefois \emph{sarvei}.} \emph{div-dar} ou \emph{div-darou}\footnote{Voy. sir William Ouseley, \emph{Travels in var. countr. of the East}, vol. 1., p. 387 ; et \emph{ibid.} note 75.} et \emph{arar}. Les trois premiers n'appartiennent ni à la langue arabe, ni à aucun idiome sémitique ou araméen ; mais ils se trouvent dans les lexiques persans. \emph{Arar} est indiqué par le \emph{Borhan-katti}, et M. Carl Ritter\footnote{\emph{Die Erdkunde}, 11., 569.} paraît disposé à l'identifier avec le mot \emph{abra}, qui, en pehlevi, signifie également cyprès. Ne nous étonnons donc point si Ferdousi, dans le \emph{Livre des Rois} (\emph{Schah-nameh}),\footnote{T. 1., p. 121 ; trad. franç. de M. Jules Mohl.} donne le nom persan de \emph{Serv, Cyprès}, au roi du Yémen dont les trois filles épousèrent les trois fils de Féridoun, le héros de l'Irân. Ce nom, rapproché de celui de \emph{Sôs, Cyprès}, qu'avait reçu un des premiers rois de l'Haïgasdan,\footnote{Ci-dessus, p. 65-71.} nous révèle, entre les anciens usages de l'Assyrie, de l'Arabie et de l'Arménie, une communauté qui s'explique tout naturellement par la notion de l'importation du culte de la Vénus assyrienne dans ces deux dernières contrées. Le témoignage de Firdousi acquiert quelque poids, lorsque nous voyons d'autres écrivains musulmans attester que le culte des arbres était anciennement répandu en Arabie et dans le Yémen en particulier. Divers auteurs arabes, cités par Pococke,\footnote{\emph{Specim. hist. arab.} p. 90 sqq. --- Conf. le Discours préliminaire (p. 23.) que Sale a mis en tête de sa traduction du Koran (édit. de 1795).} disent, en effet, que plusieurs tribus adoraient, sous l'emblème d'un arbre, la divinité qu'ils nomment tantôt \emph{Allat}, tantôt \emph{el-Samora}, tantôt \emph{el-Uzza} ou \emph{el-Ozza}. Ce dernier nom est même celui que portait un palmier ou un dattier qu'adorait, comme emblème sans doute de la déesse, la tribu de \emph{Khozaa}.\footnote{Azraki cité par Burckhardt, \emph{Voyage en Arabie}, trad. française d'Eyriès, t. 1., ch. 9., p. 220.} Un rocher appelé \emph{Allat} ou \emph{el-Lat}, comme la divinité dont l'image y avait très-probablement été sculptée,\footnote{Ce fait mérite d'être rapproché de la découverte des deux figures colossales que M. le lieutenant Burnes a trouvées sculptées dans le roc, sur le territoire de Bamiyan, dans la contrée la plus sauvage de l'Hindou-khou, vers la source du Surkhab, un des affluents de l'Oxus. Ces deux énormes colosses, dont il a publié la description et le dessin dans l'intéressante relation de son voyage (\emph{Travels into Bokhara}, etc. 2\textsuperscript{e} éd. vol. 2., pag. 159 et suiv. et pl. pour la pag. 159), sont encore aujourd'hui l'objet de la vénération des habitants du pays, qui leur donnent, à l'un, le nom de \emph{Lat}, à l'autre, celui de \emph{Munat}, probablement le \emph{Ménat} du Koran (sur. 53., v. 20).} était en grande vénération parmi les \emph{Béni-Thékif}.\footnote{Burckhardt, \emph{loc. cit.} --- Ce zélé voyageur ajoute ici qu'un grand arbre, nommé \emph{zat arouat}, était particulièrement vénéré par les \emph{Koreïsch}. On peut croire qu'en général le culte des arbres était fort répandu chez les Arabes. Nous en retrouvons des traces évidentes jusque dans le Koran et dans la légende de Mahomet. (Voyez Pococke, \emph{Specim. hist. Arab.} notes, p. 192.)} Mahomet, ajoutent les mêmes auteurs, abolit le culte de cette déesse, fit renverser et brûler les arbres et les temples qui lui étaient consacrés. Selon Firouzabadi,\footnote{Cité par Pococke, \emph{Specim. hist. Arab.} p. 92, 93.} un personnage nommé \emph{Dhâlem} avait, le premier, dans la tribu des \emph{Ghatfan} ou \emph{Ghafan}, consacré un arbre à \emph{Allat} ou \emph{el-Samora}, et placé sur cet arbre une maison ou chapelle qu'il avait appelée \emph{boss} ; on ne pouvait y entrer sans entendre un son particulier. Un officier de Mahomet, nommé Chaled, détruisit le petit édifice, fit jeter l'arbre dans le feu, après qu'il eut été scié, et perça même de son épée la prêtresse d'\emph{Allat}, qui remplissait l'air de ses cris de désespoir, à la vue de cet acte de profanation. Une partie de ce récit rappelle les traditions persanes ou arabes qui nous apprennent que Zoroastre avait fait construire un palais, ou du moins une grande salle, sur le cyprès planté par lui à Kaschmer. Tabary, de son côté, rapporte\footnote{Dans un passage de sa Chronique, cité par Sir William Ouseley (\emph{Travels}, vol. 1., appendix, p. 369-371).} que, dans le Yémen, les habitants de \emph{Nédjran} conservaient, dans la partie orientale de leur ville, un superbe palmier autour duquel ils célébraient chaque année une fête solennelle : ils s'y rendaient processionnellement, dit-il, après l'avoir couvert de riches étoffes et entouré de leurs idoles ; ils déposaient là des offrandes, et ils récitaient des prières jusqu'à ce qu'un esprit ou démon leur eût fait entendre sa voix du milieu de l'arbre. Ils se prosternaient alors devant le palmier, après quoi ils se retiraient. À la suite de ces détails curieux, qui nous reportent aux monuments figurés des Assyriens, des Syriens et des Phéniciens, où le palmier est l'emblème de Mylitta et d'Astarté, l'historien arabe raconte comment un descendant des apôtres de Jésus-Christ, venu de Syrie dans le Yémen, abattit le palmier de Nédjran, et se servit de cette circonstance pour convertir au christianisme les habitants. Selon Djennabi,\footnote{Voy. Pococke, \emph{ubi supra}, p. 100.} ce fut Mahomet lui-même qui brisa et fit jeter hors de l'édifice une \emph{colombe de bois} qu'on avait anciennement placée dans le temple de la Mecque qui depuis s'est appelé la \emph{Kaaba}. Il est sans doute permis de supposer que cet oiseau symbolique était de cyprès, bois réputé incorruptible, et qu'il avait été consacré à la Vénus des Arabes, qui, comme la Vénus des Assyriens, des Phéniciens et de tous les peuples d'Occident, avait pour emblème ou pour attribut la colombe.

La narration de Firouzabadi, à quelques légères différences près, est confirmée par Zakaria Cazwini,\footnote{Cité par Sir William Ouseley, \emph{ubi supra}, p. 371.} qui écrivait sa Chronique dans le 13\textsuperscript{e} siècle de notre ère. Ce dernier auteur fait précéder son récit de cette observation, que les Arabes qui primitivement suivaient la religion d'Abraham, étaient ensuite tombés dans une grande idolâtrie, et adoraient les uns une pierre,\footnote{Voy. mes \emph{Recherch. sur le culte de Vénus}, p. 61, note 4.} les autres un arbre. Nous avons à regretter qu'à l'exemple des écrivains cités par Pococke, Zakaria ne dise pas si cet arbre était un cyprès ou un palmier. Mais nous devons remarquer, à l'appui des observations qui précèdent, que le cyprès, chez les Arabes, ne peut avoir porté des noms étrangers, ni servi de nom propre à un roi du Yémen et de comparaison aux poètes nationaux, pour exprimer les idées de majesté et de beauté, sans avoir été là, comme chez les Assyriens, le symbole de la déesse qui réunissait au suprême degré ces deux qualités.

\subsubsection{Perse}
\paragraph{}
Le cyprès pyramidal croît spontanément sur le sol de la Perse, du golfe Persique jusqu'à la mer Caspienne, sans en excepter le Khorassan.\footnote{Voy. les nombreuses citations que M. Carl Ritter (\emph{Die Erdkunde}, 11., 568-574) a réunies pour montrer que le cyprès pyramidal est indigène dans toute la Perse et, en particulier, dans les forêts du Khorassan ; mais il a soin d'ajouter (\emph{ibid.} p. 575) que l'Irân ne fut pas la seule patrie de cet arbre et qu'on le trouvait également indigène dans l'empire assyrien, dans le pays des Cananéens, sur le Liban et l'Hermon, de même que sur les montagnes de la Judée. J'ai cité plus haut un passage de Pline qui nous apprend qu'une autre espèce de cyprès (\emph{cupressus horizontalis}) croissait spontanément dans l'Élymaïde.} Les témoignages à cet égard abondent ; ils sont même unanimes pour établir que là cet arbre atteint les plus grandes dimensions et forme un des plus beaux ornements de certaines localités. Thomas Herbert,\footnote{\emph{Some yeares Travels into divers parts of Asia and Afrique} (London, 1638), p. 134, 135.} qui voyageait en Perse dans les années 1627 et 1628, fait une mention particulière des cyprès majestueux qu'il vit dans les forêts, les parcs ou les jardins des environs de Schiraz. Déjà, en 1622, un autre voyageur, Pietro della Valle,\footnote{\emph{Viaggi}, t. 2., La Persia, parte 2a, lett. 16., de' 27 di luglio 1622, p. 311 ; Rom. 1658 ; t. 2., p. 449 ; Venetia, 1661.} avait été frappé de l'incomparable beauté d'un groupe de cyprès qu'il remarqua sur sa route, à l'est de cette ville. Plus tard, Kæmpfer,\footnote{\emph{Amœnit. exotic.} fascic. 2., p. 366, 367.} faisant la description de Schiraz, parle avec admiration des arbres de même espèce\footnote{« Telluris genio earum (\emph{cupressorum}) vigori singulariter favente ... Audeo affirmare, nusquam terrarum extare, vel tantam uno loco cupressorum multitudinem, vel tam alto fastigio, tam læto et concordi elegantia luxuriantium. » (\emph{Ibid.} p. 367.)} qui ornaient les magnifiques jardins dont la ville était entourée. Il compare enfin à de superbes hallebardiers les cyprès qui bordaient la route royale.\footnote{« Tot desuper intuemur cupressorum densas series (\emph{in via regia}), cum ad viæ latera, tum per viridariorum margines consitas, quæ velut hastatas satellitum copias, et nescio qualem ineffabilis magnificentiæ speciem imaginationis nostræ objiciunt ; adspectu, quo nullum elegantiorem, collatis in unum studiis, ars et natura possint efficere. » (\emph{Ibid.})} En 1811, sir William Ouseley,\footnote{\emph{Travels}, vol. 2., p. 74, 90-94.} traversant le territoire de Schiraz, trouva encore debout, à l'est de la ville, une douzaine de ces cyprès qui autrefois avaient attiré l'attention de Pietro della Valle. La station où ils se sont conservés s'appelle \emph{Servistân} ou \emph{Selvistân} « le jardin des cyprès, » vraisemblablement à cause de la quantité d'arbres de cette espèce qui jadis couvraient ce canton.\footnote{Plus à l'est de Schiraz, à \emph{Fassa} ou \emph{Passa}, le voyageur anglais remarqua aussi de beaux cyprès (\emph{loc. cit.}).}

Des traditions écrites, des monuments figurés, vont nous apprendre qu'au 6\textsuperscript{e} siècle avant l'ère chrétienne, l'usage de faire du cyprès un emblème religieux fut porté de la Babylonie chez les Perses avec le culte de Mithra, comme il l'avait été chez les Arabes avec le culte d'Alitta, Alilat ou Allat. Nous acquerrons ainsi une preuve nouvelle de l'authenticité des renseignements qui avaient permis à Hérodote, deux siècles plus tard, d'affirmer que le culte de Mithra, de même que celui d'Alitta, dérivait du culte assyrien de Mylitta ou Vénus-Uranie.

Le \emph{Schah-nameh} de Firdousi, le \emph{Muntekhab-el-tewarikh} ou \emph{Schemschir-khani},\footnote{Écrit en persan, au 17\textsuperscript{e} siècle de notre ère, par Tawakkol-Beg.} le \emph{Schah-nameh-nasr},\footnote{Ouvrage anonyme et composé, par quelque destour ou prêtre parse, d'après le \emph{Schah-nameh} de Firdousi.} le \emph{Rauzet-el-safa}\footnote{On prononce : \emph{assafa} ou \emph{essafa}.} de Mirkhond, le \emph{Fehrenghi-Djihanguiri}, le \emph{Borhan-katti}, un écrit de Boundâry ou Bondâry, et les autres ouvrages orientaux ou fragments cités par Thomas Hyde, d'Herbelot,\footnote{Cf. Golius, \emph{in Alfergan.} p. 231. Elmacin, \emph{Histor. Saracen.} livr. 2., p. 186, 191 et suiv. de la trad. lat. d'Erpénius ; p. 160 de la trad. franç de Vatier.} Anquetil du Perron, ou publiés par MM. Jules Mohl et Olshausen,\footnote{\emph{Fragm. relatifs à la relig. de Zoroastre, extr. des mss. persans de la Biblioth. du roi.} Paris, 1829, in-8° (textes sans traduction).} et par M. le Dr Vullers,\footnote{\emph{Fragm. über die Relig. des Zoroaster} (Bonn, 1831, in-8°), p. 70 et suiv.} s'accordent en effet à nous dire, avec des circonstances plus ou moins croyables, plus ou moins merveilleuses, que Zoroastre, puissamment secondé par le roi Gustasp, introduisit dans la Bactriane et dans le pays que nous appelons aujourd'hui le Khorassan le culte et le pèlerinage du cyprès, en même temps qu'il y établissait le culte de Mithra. À peine arrivé à la cour de Gustasp, à peine admis à lire devant ce prince le livre de la loi ou le \emph{Zend-Avesta}, le disciple des Chaldéens d'Assyrie ou de Babylone planta dans la ville de Bactres, à côté du palais du roi, un cyprès qui, selon l'auteur du \emph{Schah-nameh} en prose,\footnote{\emph{Schah-nameh-nasr}, cité par Th. Hyde, \emph{Relig. veter. Pers.} p. 324 ; edit. 2a. Anquetil du Perron, \emph{Zend-Avesta} (Vie de Zoroastre), t. 1., 2\textsuperscript{e} partie, p. 33.} acquit en peu de jours de telles dimensions, que dix grandes cordes, dont à la vérité on ne nous indique pas la longueur, suffisaient tout juste à embrasser sa circonférence, et que l'heureux planteur put faire construire une grande salle sur les branches les plus élevées de l'arbre miraculeux. Tawakhol-Beg\footnote{Dans son \emph{Muntekhab-el-tewarikh}, cité par Anquetil, \emph{ubi supra}, note 2.} ajoute que les feuilles de ce cyprès donnaient de l'intelligence ou de l'esprit à ceux qui en mangeaient. Les traditions qui se rapportent à un second cyprès, également planté par Zoroastre, sont plus détaillées. Ici je laisserai parler Anquetil, qui les avait recueillies avec soin, sans se douter pourtant que le culte du cyprès eût originairement été institué, avec le culte de Mylitta, par les Chaldéens ou les Assyriens ; sans se douter non plus que cet arbre mystique jouât un rôle quelconque dans les monuments figurés mithriaques. « Il y avait, dit-il,\footnote{Ouvrage cité, t. 1., 2\textsuperscript{e} partie, p. 46 et 47. --- Ce passage d'Anquetil du Perron est tiré d'une histoire inédite des sulthans Seldjoukydes, par Emad-eldyn, intitulée : $\arabicAAC$ (c'est-à-dire, \emph{La crême du secours et le choix du temps}), et abrégée par l'iman Alfath, fils d'Ali-el-Boundâry d'Ispahan, qui écrivait vers l'an 1230 de notre ère. La Bibliothèque impériale de Paris possède un exemplaire de cet abrégé (\emph{manuscr. arab.} ancien fonds, n° 767 A).} dans Kaschmer (ou Kischmer), village du Khorassan,\footnote{Ce village était situé dans le canton de Nischapour ; son nom s'écrit ou se prononce tantôt \emph{Kaschmer}, tantôt \emph{Kischmer}, tantôt même \emph{Kaschmir}.} un \emph{atesch-gah}\footnote{Autel du feu ou pyrée.} célèbre. Près de la porte de ce temple, Zoroastre planta un cyprès, et grava sur l'écorce du tronc de cet arbre, que Gustasp avait embrassé sa Loi.\footnote{Ce dernier fait, dans la traduction que nous donne Thomas Hyde (\emph{Hist. relig. veter. Persar.} p. 321 ; edit. Oxon. 1760) d'un passage emprunté a un auteur musulman, est rapporté en ces termes : « in cujus trunco scripsit (\emph{Zoroaster}), quod Guschtasp acceptaverat Dei Religionem, quod testaretur ista cupressus. »} Après plusieurs années, ce cyprès étant devenu grand, épais et garni de branches, on bâtit dessus un palais qui avait quarante coudées de haut et en carré. Il renfermait deux salles, dont le toit était d'or, le plancher d'argent, les murs d'ambre et ornés de pierres précieuses. On y forma (\emph{sic}) les portraits de Djemschid et de Féridoun. Gustasp se retira dans ce palais, pour de là s'élever au ciel lorsque son heure serait venue. Ce prince, continue Anquetil,\footnote{Ouvrage cité, t. 1., 2\textsuperscript{e} partie, p. 47.} dépêcha des courriers aux extrémités de son empire, et écrivit aux gouverneurs de venir à pied visiter le cyprès, écouter Zoroastre, et d'abandonner le culte des idoles du Touran et de Tchin. Boundâri ajoute qu'on lui obéit moitié de gré, moitié de force ; et ce fut ce zèle ardent qui occasionna en partie les guerres sanglantes que Gustasp eut à soutenir. » Anquetil termine en disant, dans une note,\footnote{\emph{Ibid.} p. 46 et 47, note 2. \emph{Mém. de l'Acad. roy. des inscriptions}, t. 37., p. 728.} que, selon le \emph{Fehrenghi-Djihanguiri},\footnote{Composé, en 1605, par Djémal-eldyn-Hussein-Andjou.} Zoroastre planta deux branches de cyprès qu'il avait apportées du ciel, l'une à \emph{Kaschmer},\footnote{\emph{Kischmer} dans le \emph{Schah-nameh} de Firdousi et ailleurs. Le \emph{Borhan-katti} et un autre dictionnaire, intitulé : \emph{Les Sept mers}, donnent à ce village, comme le \emph{Fehrenghi-Djihanguiri}, le nom de \emph{Kaschmer}.} village du canton de \emph{Tarschiz},\footnote{M. Kleuker écrit ce nom \emph{Tarschitz} ; M. Vullers, \emph{Terschis} ; M. Carl Ritter, \emph{Turschiz}.} dans le Khorassan ; l'autre à \emph{Féroumad},\footnote{Hyde (ouvrage cité, p. 332) fait observer que ce lieu est aussi désigné sous les noms de \emph{Pharouyad} et de \emph{Pherdid}.} qui était un des villages du canton de \emph{Tous}, dépendant aussi du Khorassan. Cette dernière tradition avait déjà été puisée à la même source par le Dr Thomas Hyde,\footnote{\emph{Ibid.} p. 332.} qui n'a pas oublié, comme l'a fait Anquetil, d'ajouter que, suivant le \emph{Fehrenghi-Djihanguiri}, les deux cyprès furent plantés sous une étoile heureuse.\footnote{M. le Dr. Vullers (\emph{Fragm. über die Relig. des Zoroaster}, p. 113, 114, note 20) a publié le texte du \emph{Borhan-katti} relatif à ces deux cyprès.} Le savant Anglais cite un passage de Scharistâni,\footnote{Hyde, ouvrage cité, p. 300.} où nous lisons que Dieu, après avoir placé l'esprit ou l'âme de Zoroastre dans un arbre qui croît sur les montagnes les plus élevées, planta cet arbre dans l'Aderbaïdjân, sur le mont \emph{Asmuîdacher}. Hyde n'hésite pas à dire qu'il s'agit ici du cyprès.\footnote{\emph{Ibid.} p. 332.}

Firdousi, dans son \emph{Schah-nameh} ou \emph{Livre des rois},\footnote{Pag. 1067 et 1068 ; éd. de Calcutta ; trad. franç. de M. Jules Mohl, partie inédite.} ajoute, au sujet du cyprès de Kischmer, quelques détails qui ont été négligés par les écrivains que je viens de citer, et qui méritent d'être rapportés. C'est ainsi que, selon le poète, le temple du feu, devant lequel Zoroastre planta cet arbre, s'appelait \emph{Mihr-Berzin}, dénomination qui, pour nous, a le précieux avantage de montrer le culte du cyprès aussi intimement lié, chez les Perses, au culte de Mithra, qu'il l'était, chez les Assyriens et les Phéniciens, au culte de Mylitta ou d'Astarté. L'expression dont se sert Firdousi, en parlant du palais que fit construire le prophète, est assez vague pour qu'il soit possible de comprendre que ce palais fut édifié, non sur les branches du cyprès, mais bien autour de l'arbre. Diverses variantes, recueillies par M. Mohl, qui a eu la complaisance de me les communiquer, présentent ce dernier sens. On lit ensuite, dans le \emph{Schah-nameh}, que Zoroastre fit peindre, sur les murs du palais, Gustasp en adoration devant le soleil et la lune, Féridoun armé de sa massue à tête de taureau, et les portraits de tous les autres rois de Perse. Quand l'édifice fut achevé, Gustasp envoya dans toutes les parties de son empire un message conçu en ces termes : « Qu'y a-t-il de comparable dans le monde au cyprès de Kischmer ? Dieu me l'a envoyé du ciel, et m'a dit que de là j'irais dans le ciel. Écoutez maintenant, vous tous, mon conseil : rendez-vous à pied auprès du cyprès de Kischmer ; suivez tous la voie de Zoroastre et tournez le dos aux idoles de la Chine. Au nom du pouvoir et de la gloire des Iraniens, je vous ordonne de vous ceindre tous du kosti, d'abandonner les coutumes de vos pères et de vous réunir sous l'ombre de ce cyprès. Venez au temple du Feu, d'après l'ordre du prophète véridique. » « Tous ceux qui portaient des diadèmes, reprend Firdousi,\footnote{\emph{Fragm. relatifs à la relig. de Zoroastre}, texte persan de la vie de Gustasp, p. 20. \emph{Fragm. über die Relig. des Zoroaster}, trad. allem. de M. le Dr. Vullers, p. 72.} se rendirent, sur l'ordre du roi, auprès du cyprès de Kischmer ; par leur présence, le temple devint semblable au paradis, et Zoroastre y enchaîna les divs. Donne à ce cyprès l'épithète de \emph{paradisiaque}, si tu ne sais pas pourquoi il se nomme le \emph{cyprès de Kischmer}. Pourquoi ne l'appellerais-tu pas un \emph{rejeton du paradis} ? Dans ce monde voit-on s'élever un cyprès comme celui de Kischmer\footnote{Hamdallah Kazwini rapporte, au sujet du cyprès de Kaschmer, quelques particularités qui s'écartent de celles que présente le récit de Firdousi. (Voy. Sir William Ouseley, \emph{Travels}, vol. 1., p. 388 et 389.)} ? » Plus loin, le poète raconte\footnote{\emph{Schah-nameh}, 3\textsuperscript{e} djeld, cité par Anquetil du Perron (\emph{Vie de Zoroastre}, dans le \emph{Zend-Avesta}, t. 1., 2\textsuperscript{e} partie, p. 54).} qu'au bout de huit années d'un second séjour à la cour de Gustasp, Zoroastre, voyant que le pèlerinage du cyprès s'était ralenti, conseilla au roi de secouer le joug du Touran. Son but probablement était d'essayer de ranimer le zèle de ses sectateurs par une guerre de religion.

Le \emph{Dabistân}\footnote{\emph{The Dabistan} or \emph{School of manners}, translated from the original Persian by David Sea and Antony Troyer, vol. 1., p. 150 (Paris, 1843, 3 volumes in-8°).} décrit, d'après le code de Mâhâbâd, les différentes régions dont se compose le paradis. Les minéraux appartiennent à la première ; les végétaux, à la seconde. Celle-ci renferme deux arbres : le platane et le cyprès. L'auteur du \emph{Dabistân}\footnote{\emph{Ibid.} p. 309.} ne fait mention que d'une seule branche de cyprès qui, selon les livres des Behdinians, aurait été rapportée du paradis\footnote{Cette tradition, qui, comme le passage ci-dessus de Firdousi, place dans le paradis céleste le cyprès, symbole de vie, a probablement sa source dans les témoignages bibliques relatifs à l'arbre de vie, à l'arbre du bien et du mal, que le Seigneur avait mis au milieu du paradis où vivaient Adam et Ève. Ce rapprochement ne pouvait échapper à la pénétration de M. Carl Ritter (\emph{Die Erdkunde}, 11., 571). Il faut y ajouter celui que suggère l'arbre merveilleux placé par Mahomet dans le paradis du Koran.} par Zoroastre et serait devenue un arbre, après avoir été plantée de ses mains devant l'\emph{atesch-gah} ou pyrée de Kischmer ; mais le même ouvrage nous apprend que,\footnote{\emph{The Dabistan} or \emph{School of manners}, p. 309.} selon quelques Yezdénians, Zoroastre avait imploré en faveur de ce rejeton la protection particulière d'\emph{Azerwan} (l'ized du feu), surveillant supérieur de tous les arbres de l'espèce du cyprès. On voit ici comment l'idée du feu, exprimée à la fois par un pyrée et par le génie qui préside à cet élément, se rattache à la signification symbolique de l'arbre dont on avait fait l'emblème ou l'attribut des divinités génératrices, et dont la forme pyramidale rappelle la forme de la flamme, qui s'élance de la terre vers les cieux.\footnote{Je me félicite de me trouver également d'accord sur ce point avec M. Carl Ritter ; et je ne puis résister au plaisir de rapporter textuellement ici les judicieuses remarques que lui suggèrent, en termes élégants, les divers faits qui, chez les Perses, se rattachent au culte du cyprès : « Kann man sich wundern, dit-il (\emph{ubi supra}), wenn die Cypresse, ein Paradiesbaum, durch ihren pyramidalen Wuchs, wie die Flamme, majestätisch in hohen Lüften steigend, an die Pforte des heiligsten Feuertempels gepflanzt, mit der von Zoroaster eingegrabenen Schrift des Gesetzes, durch ganz Iran die Begleiterin jedes Heiligthums, jedes Ateschghas, jedes Paradeisos, jeder Königsresidenz der Diener Ormuzds wurde ? »} L'auteur anonyme du \emph{Desâtir},\footnote{P. 9, § 57.} s'il est permis de citer une pareille composition, renchérit sur cette alliance d'idées : parmi les nombreux emprunts qu'il fait au \emph{Dabistân}, sans indiquer la source où il puise, nous retrouvons la tradition qui caractérise la région des végétaux par le platane et le cyprès ; mais il ajoute au nom d'\emph{Azerwan} celui de \emph{Nouzerwan} (l'ized de la lumière), qui aurait été le second gardien de ces deux arbres.

Le docteur Hyde\footnote{\emph{Hist. relig. veter. Pers.} p. 332.} n'a pas omis de rapporter, d'après le \emph{Fehrenghi-Djihanguiri}, que, par l'ordre de Motawakkel, dixième khalife abbasside, le cyprès qu'avait planté Zoroastre à Kaschmer (Kischmer), dans le Khorassan,\footnote{C'est à tort qu'Anquetil du Perron (\emph{ubi supra}) reproche à Th. Hyde d'avoir confondu ici Kaschmer avec Kachemîr.} fut coupé et transporté à Baghdâd, malgré les instances et les offres considérables d'argent\footnote{On porte la somme à cinquante mille dinars, environ six cent mille francs.} que firent en corps les mages ou destours pour obtenir la révocation des ordres du khalife. Ces faits sont racontés fort au long dans un chapitre du \emph{Dabistân},\footnote{\emph{The Dabistan}, etc. vol. 1., p. 306 et 308.} et dans un article du \emph{Borhan-katti}, cité par Anquetil du Perron,\footnote{\emph{Mém. de l'Acad. royale des inscript.} t. 37., p. 728-730.} par Sir William Ouseley\footnote{\emph{Travels}, vol. 1., p. 387 et 388.} et par M. le docteur Vullers.\footnote{\emph{Loc. cit.}} Ils s'y trouvent accompagnés de quelques détails qui nous apprennent de plus que, lorsque la volonté de Motawakkel fut connue à Kischmer, la population du canton et celle de tout le Khorassan se rassemblèrent au pied du cyprès sacré, suppliant, avec des lamentations et des larmes, le gouverneur arabe\footnote{Il s'appelait \emph{Taher, fils d'Abdallah}.} d'épargner cet objet de leur vénération. Le refus de ce gouverneur fut suivi d'une scène de désolation générale, et l'arbre, en tombant sous les coups redoublés de la hache des Arabes, causa d'immenses dégâts aux bâtiments et aux fontaines du voisinage. Une grande catastrophe devait marquer la chute de ce cyprès sacré et empêcher qu'il ne fût mis en la possession du khalife qui l'avait fait abattre. Le convoi qui transportait à Baghdâd le tronc et les dépouilles de cet arbre était parvenu, le 12 décembre 861, à une journée du palais appelé \emph{Djafari} ou \emph{Djafariah},\footnote{M. le Dr. Vullers (\emph{Fragmente}, p. 114, note) dit que ce palais était ainsi nommé parce qu'il avait été bâti par le khalife Motawakkel, qui s'appelait aussi \emph{Djafar}. Selon le \emph{Borhan-katti}, cet édifice était situé dans l'Irak arabique, sur la rive occidentale du Tigre, à \emph{Samira} ou \emph{Samrah}, ville dont le nom, ajoute le savant orientaliste allemand, s'écrit de plusieurs manières. Il renvoie ici à d'Herbelot (\emph{Bibloth. orient.} art. \textbf{Sermenrai}) et à Abou'lféda (\emph{Annal. moslem.} 2., 204, 207, 220).} lorsqu'on apprit que, pendant la nuit et à la suite d'une débauche, Motawakkel avait été assassiné dans cette résidence par les chefs de la garde turque.\footnote{Sur cet assassinat, voyez Abou'lféda, \emph{Annal. moslem.} 2., p. 204. --- D'après la version rapportée par Anquetil (\emph{ubi supra}, p. 729), le khalife aurait été coupé en morceaux par ses esclaves.} Selon le \emph{Dabistân}, les Yezdénians prétendaient que le gardien céleste des cyprès était apparu à un docteur nommé Hakim Mirza, pour lui annoncer d'avance qu'il avait ordonné que l'on mît à mort Motawakkel, coupable du crime d'avoir fait abattre le cyprès de Kischmer. Le compilateur persan, voulant nous donner une idée des dimensions gigantesques qu'avait atteintes cet antique cyprès, porte à la somme exorbitante de cinq cent mille dinars\footnote{C'est aussi la somme qu'indiquent le \emph{Fehrenghi-Djihanguiri} et le \emph{Borhan-katti}. Le dinar est une monnaie de compte, par laquelle on entend ordinairement la valeur du sequin de Venise ou de Hongrie, c'est-à-dire un peu moins de douze francs. Quelquefois cependant on prend le dinar pour l'équivalent de treize francs.} le montant des frais occasionnés par le transport du tronc seul, et raconte que les branches formèrent la charge de treize cents chameaux. Quelques écrivains, ajoute-t-il, disent que le tronc n'avait pas moins de trente-trois coudées et trois quarts de circonférence.\footnote{Ces dimensions, sans doute, sont fort exagérées. Cependant il semble avoir été donné à la famille des cyprès de produire des espèces capables d'atteindre de grandes proportions. Ainsi, par exemple, feu M. Dupratz attestait avoir vu, près de la Nouvelle-Orléans, un pied de \emph{cyprès de la Louisiane} (\emph{cupressus disticha.} Linn.) qui avait douze brasses de circonférence et une hauteur extraordinaire. (Cubières, \emph{Mém. sur le cyprès de la Louisiane}, lu à l'Institut le 24 avril 1809, p. 5.) Il n'en précise pas la hauteur, mais on sait que, dans les états de la Delaware et du Maryland, il n'est pas rare de rencontrer des pieds de cyprès distique qui ont vingt-sept à trente mètres de haut. En Orient, le cyprès pyramidal atteint presque cette dernière dimension.} Le \emph{Fehrenghi-Djihanguiri} prétend que dix mille bœufs pouvaient se reposer à l'ombre de ce cyprès. Le \emph{Borhan-katti}, moins porté ici à l'exagération, réduit ce nombre à deux mille. Mais les deux dictionnaires s'accordent à dire avec assurance que la chute de l'arbre ébranla tellement la terre, que les canaux et les bâtiments voisins en furent très-endommagés. Selon le \emph{Fehrenghi-Djihanguiri} et le \emph{Dabistân}, quatorze cent cinquante ans s'étaient écoulés depuis le jour où il avait été planté jusqu'à l'année 232 de l'hégire.\footnote{« Le \emph{Borhan-katti} ne marque pas la date, dit Anquetil (\emph{loc. cit.} p. 729) ; et, si le calcul du \emph{Fehrenghi-Djihanguiri} est juste, il faut que l'on n'ait transporté le cyprès de Kaschemer à Sermen-rai que treize à quatorze ans après l'avoir abattu. »} Cette dernière date, qui correspond à l'an 846-847 de notre ère, ferait ainsi remonter la plantation du cyprès de Zoroastre jusqu'en l'année 604 ou 590 avant J. C. si le nombre 1450 doit s'entendre d'années solaires, et seulement à l'an 562 ou 548, s'il s'agit d'années lunaires.\footnote{\emph{Zend-Avesta}, t. 1., 2\textsuperscript{e} partie (Vie de Zoroastre), p. 61.} Quelle que soit la date qu'il convienne d'adopter, nous pouvons tenir pour certain que, si le culte du cyprès n'avait pas déjà été introduit dans ce pays au temps de Djemschid\footnote{L'époque de l'arrivée de Djemschid en Perse est fort incertaine. Langlès la fixait au 8\textsuperscript{e} siècle avant J. C. Saint-Martin la reculait jusqu'au 15\textsuperscript{e}.} ou de Féridoun, comme le donne à entendre Firdousi, il y fut institué au plus tard vers le milieu du 6\textsuperscript{e} siècle avant J. C. par Zoroastre, l'élève des Chaldéens d'Assyrie. Ce fait mérite d'autant plus d'attirer notre attention que, avec le culte du cyprès, l'auteur du \emph{Zend-Avesta} apportait en Perse, je le répète, le culte de Mithra, divinité androgyne, primitivement identique avec la Vénus babylonienne ou assyrienne, dont nous avons vu que le cyprès, dès une époque très-reculée, fut un des symboles ou un des attributs caractéristiques.

Remarquons aussi que Zoroastre, en plantant un cyprès à côté du pyrée de Kischmer, ne faisait qu'imiter un usage dont nous avons trouvé des exemples analogues, soit sur des médailles asiatiques qui reproduisent des types propres à la religion des Assyriens, des Syriens ou des Phéniciens, soit dans un récit emprunté aux traditions religieuses des Troyens ou des Phrygiens. D'autres exemples, comme nous le verrons plus loin, se rencontrent sur des médailles frappées en Occident, ou dans la description qui nous reste de plusieurs édifices religieux de la Grèce.

Remarquons enfin que l'auteur du \emph{Schah-nameh}, qui partout se montre très-initié à la connaissance des anciens usages religieux ou civils de la Perse, dit que Féridoun planta des cyprès et des roses à la place des herbes sauvages\footnote{\emph{Le Livre des rois}, traduct. franç. de M. J. Mohl, t. 1., p. 119. --- Au début de ce poème (\emph{ibid.} p. 9), Firdousi, dans un beau passage, compare la parole à un arbre dont la cime touche aux cieux et dont les racines ne sont pas accessibles à l'homme. La pensée qu'exprime ici le poète persan nous rappelle un vers des Oracles attribués à Zoroastre, dont le sens est que l'arbre de la vérité n'appartient pas à la terre :\\\hspace*{5mm}\textgreek{Οὐ γὰρ ἀληθείης φυτὸν ἐνὶ χθονί}.\\\hspace*{10mm}(\emph{Oracula Zoroastr.} 10., 282 ; ed. Stanley.)} ; il compare fréquemment au cyprès, au soleil, à la lune, les princes et princesses des premières dynasties persiques ; il emploie même, avec intention sans doute, certaines images qui font une allusion plus ou moins directe à quelque tradition antique ou à quelque ancien monument de l'art. C'est ainsi, par exemple, qu'il nous représente, dans la salle du trône, le roi d'Irân, Féridoun, tantôt placé comme un grand cyprès, au-dessus duquel brille la lune, entre ses deux épouses, Schehrinaz « à la taille de cyprès » et Arnewaz « à la face de lune\footnote{\emph{Le Livre des rois}, p. 103 ; cf. p. 73, 77, 79.} ; » tantôt entouré de ses trois fils « à stature de cyprès\footnote{Jésus, fils de Sirach (\emph{Ecclesiastic.} 50., 11-14), emploie une image analogue dans le portrait qu'il fait du grand prêtre Simon et de ses frères.} » et de ses trois belles-filles,\footnote{\emph{Le Livre des rois}, p. 119 et 137.} de même que le soleil est entouré des six planètes. Poursuivant le récit de la vie et des actions de son héros, Firdousi\footnote{\emph{Ibid.} p. 163.} raconte que, dans un acte de violent désespoir, Féridoun, qui venait de recevoir la nouvelle de la mort de son troisième fils, Iredj, roi d'Irân, mit le feu au palais et au jardin ou parc de ce prince, et brûla notamment « les hauts cyprès » plantés dans la partie du jardin où se célébrait \emph{la fête des rois}. Ces plantations de cyprès dans de tels lieux, pour le dire en passant, semblent être un nouvel emprunt fait par les Perses aux Assyriens, car plusieurs bas-reliefs du palais de Khorsabad représentent une chasse royale dans un parc ou jardin planté de beaux cyprès.\footnote{\emph{Monum. de Ninive}, pl. 108., 109., 111-113.} Précédemment\footnote{\emph{Le Livre des rois}, p. 75.} le poète avait appelé « cyprès argenté » la fille de Djemschid, qu'il nomme peu après Arnewaz « à la face de lune. » Plus loin,\footnote{\emph{Ibid.} p. 269.} Zal, dans un entretien avec son amante Roudabèh « au visage de lune, » l'interpelle en ces termes : « Ô cyprès au sein argenté et parfumé de musc\footnote{Après Firdousi, plusieurs autres poètes orientaux continuent à faire l'éloge de la beauté en prenant le cyprès pyramidal pour terme de comparaison : la tête d'une belle femme est comparée à la lune, et sa taille à un cyprès ; son amant l'interpelle ou la supplie en ui disant : « Ô mon Cyprès, ô ma lune ! » (Voy. les fragments de poésie persane cités et traduits par M. le baron de Hammer-Purgstall dans l'intéressant recueil intitulé : \emph{Schöne Redekünste Persiens}). Il n'est pas jusqu'aux historiens persans ou arabes qui ne comparent au cyprès la taille de leurs héros. Mirkhond nous en fournit un exemple dans son \emph{Histoire des Seldjoukydes} (p. 133 de la traduction allemande publiée à Giessen, en 1837, par M. le Dr. Vullers).} ! ... » Par-là Firdousi nous permet de croire que, puisque le cyprès de la lune était d'argent, le cyprès du soleil devait être d'or. Cette remarque, rapprochée de la tradition persane qui nous a appris qu'à Bactres le palais élevé auprès du cyprès de Zoroastre avait une toiture d'or et un plancher d'argent, ne nous montre-t-elle pas que ces deux métaux étaient tout aussi bien consacrés à Mithra qu'à Mylitta, à Apollon et à toutes les divinités créatrices qui, primitivement androgynes, avaient pour manifestation, dans l'ordre de la création du monde, le soleil et la lune, agents du pouvoir générateur actif et du pouvoir générateur passif ? Le palais d'or et d'argent érigé auprès du cyprès, image symbolique de Mithra, comme d'Apollon,\footnote{Ci-dessus, p. 104-107.} le cyprès d'or du soleil et le cyprès d'argent de la lune ne nous ramènent-ils pas ainsi à un antique usage qui, de l'Asie occidentale, était passé chez les Grecs, et sur lequel, dans une autre occasion, j'ai appelé l'attention des savants,\footnote{\emph{Nouv. Annal. de l'Instit. archéol.} t. 1., p. 204-209. \emph{Recherches sur le culte de Vénus}, p. 108-116.} l'usage d'ériger aux divinités génératrices des statues composées d'or et d'argent, ou d'or et d'ivoire ?

Le temps, qui ne respecte rien, les tremblements de terre et le fanatisme aveugle des musulmans, qui ont couvert de ruines le sol de la Perse, semblent avoir cependant laissé subsister une partie des antiques sculptures de Persépolis pour servir de confirmation au double témoignage que, dans la question de l'importation en Perse du culte du cyprès, nous fournissent les traditions conservées par les écrivains musulmans que j'ai cités, et les monuments mithriaques, d'époque romaine, dont il sera question tout à l'heure. Au milieu des groupes de l'imposante panégyrie qui décore le grand escalier de l'édifice improprement appelé \emph{Tchéhelminar, les quarante colonnes}, on voit plusieurs cyprès,\footnote{Chardin, \emph{Voyage en Perse}, t. 2., pl. 58. ; édition d'Amsterdam, 1735. Corneille le Bruyn, \emph{Voyage par la Moscovie en Perse}, t. 2., pl. 126. ; éd. d'Amsterdam, 1718. Niebuhr, \emph{Voyage en Arabie, en Perse}, etc. t. 2., p. 103, 105, 106, pl. 22., 23. ; édition d'Amsterd. 1780. Porter's \emph{Travels in Georgia, Persia}, etc. vol. 1., pl. 34 et 38. MM. Eug. Flandin et Pascal Coste, \emph{Voyage en Perse}, Perse ancienne, planche 105. --- Les cyprès pyramidaux des bas-reliefs de Persépolis et ceux des sculptures de Khorsabad ont entre eux une grande analogie de style, quant à leur forme conventionnelle et à la disposition non moins conventionnelle des branches et du feuillage ; mais les premiers, comme la plupart des figures et des objets divers qu'on voit représentés à Persépolis, sont lourds et ramassés. Leur tronc est très-court, leur feuillage pyramidal fort massif, tandis qu'à Khorsabad les cyprès sculptés sur plusieurs bas-reliefs sont très-élancés ; leur tronc s'élève à une grande hauteur, et leur feuillage est beaucoup plus épais ou touffu qu'a Persépolis. Ici tout accuse un art imité de l'art assyrien à une époque de décadence.} dont la présence justifie Firdousi d'avoir placé des cyprès dans la partie des jardins du palais d'Iredj où se célébrait \emph{la fête des rois}\footnote{Les rois de Perse ont conservé jusqu'à ce jour l'usage fort ancien de faire planter des cyprès dans leurs jardins et dans leurs parcs. Gmelin (\emph{Reise durch Russland}, th. 4., s. 67, 111, 465 ; Pétersb. 1771-1774) en cite un exemple qui se rapporte au règne de Schah Abbas le Grand, dans la première moitié du 17\textsuperscript{e} siècle.} ; car les archéologues et les voyageurs s'accordent à reconnaître que, sur les faces des quatre rampes du grand escalier de Tchéhelminar, on avait voulu représenter la fête du \emph{neu-rouz} ou nouvel an, instituée tout à la fois en l'honneur d'Ormuzd et de Mithra, et en l'honneur des rois. Le neu-rouz aujourd'hui n'est qu'une cérémonie civile, qui se célèbre dans le palais du prince régnant ; mais, comme autrefois, remarquons-le bien, elle se célèbre à l'équinoxe vernal,\footnote{Selon le témoignage des auteurs arabes, l'usage d'offrir au souverain des présents, à l'époque du neu-rouz ou nouvel an, s'était perpétué à la cour des khalifes de Baghdad jusqu'au temps des Abassides (voy. \emph{Ibn Khallikan's Biographical Dictionary}, translated from the Arabic by baron Mac Guckin de Slane. Paris, 1842, in-4° ; vol. 1., p. 203, et p. 340, note 13). Cet usage était certainement la continuation d'une coutume qui se pratiquait à la cour des anciens rois de Babylone et de Ninive, avec cette différence cependant que, sous les Abassides, le neu-rouz se célébrait à l'équinoxe d'automne et non à l'équinoxe vernal. C'est ainsi que nous verrons ailleurs la fête de Mithra (le \emph{Mihirdjân} ou \emph{Mihrgân}) transportée de l'équinoxe du printemps à l'équinoxe d'automne.} à l'époque où une nouvelle vie se répand sur la terre. Les sculptures de Persépolis présentent le double caractère, religieux et civil, qu'on observe dans la composition de tous les autres monuments antiques de la Perse, comme dans la composition des bas-reliefs qui ornent les édifices assyriens dont les ruines ont, de nos jours, été découvertes à Khorsabad, à Nimroud, à Koyoundjuk. Et si, à l'angle de chacune des quatre rampes du grand escalier de Tchéhelminar, on trouve, non loin des cyprès et de la panégyrie royale, le groupe d'un lion dévorant un taureau, emblème énergique du phénomène solaire propre à l'équinoxe vernal,\footnote{Voy. \emph{Mémoires de l'Acad. des inscript.} nouvelle série, t. 15., 2\textsuperscript{e} partie, p. 65-69.} quelques bas-reliefs qui ornent d'autres parties des édifices de Persépolis nous montrent Mithra, sous le symbole de la colombe ou du \emph{mihr}, placé au milieu d'une rangée de taureaux et au milieu d'une rangée de lions superposées l'une à l'autre,\footnote{Chardin, \emph{Voyage en Perse}, tome 2., pl. 63 et 64. Corneille le Bruyn, \emph{Voyage par la Moscovie en Perse}, t. 2., pl. 153. Niebuhr, \emph{Voyage en Arabie}, t. 2., pl. 29 et 30. --- Dans mes \emph{Recherches sur le culte public et les mystères de Mithra} (atlas, pl. 4), j'ai publié, d'après un dessin beaucoup plus fidèle, habile-exécuté sur les lieux par M. Eug. Flandin, le bas-relief que j'indique ici.} et figurant symboliquement la lune et le soleil, le principe humide ou passif et le principe igné ou actif.\footnote{\emph{Recherches sur Vénus}, p. 156-158.} Ces diverses particularités nous ramènent, on le voit, aux médailles asiatiques citées,\footnote{Ci-dessus, p. 83-85 ; pl. 6, n°\textsuperscript{s} 1 et 2 ; pl. 14, n°\textsuperscript{s} 1 et 2.} qui ont pour type Vénus-Astarté représentée sous l'emblème d'un cyprès planté entre un taureau et un lion, ou Artémis figurée sous l'emblème d'un cône placé entre deux cyprès surmontés, l'un du croissant de la lune, l'autre de l'image du soleil. En même temps, nous découvrons à quelle source avait été puisé le dogme du \emph{Zend-Avesta},\footnote{T. 2., p. 13. Voy. \emph{Mém. de l'Académie des inscript.} nouv. série, t. 14., 2\textsuperscript{e} part. p. 98.} qui place au ciel, entre le soleil et la lune, la résidence habituelle de Mithra.

Une preuve moins directe, mais non moins authentique, des rapports que le système religieux emprunté par Zoroastre aux Chaldéens d'Assyrie avait établis entre Mithra, le soleil, la lune et le cyprès, se tire de cinq bas-reliefs mithriaques, d'époque romaine, dont je me réserve de parler avec détail dans le cours de ce mémoire, et où nous trouverons le cyprès du soleil et le cyprès de la lune sculptés, l'un à la droite, l'autre à la gauche du dieu des Perses.

Les témoignages que nous fournissent les sculptures de Persépolis et ces cinq bas-reliefs mithriaques romains nous permettent de croire que les cylindres persiques, gravés pour les sectateurs du culte secret de Mithra, reproduisaient, à leur tour, l'emblème du cyprès, tout comme les cylindres des mystères assyriens de Mylitta. Je ne puis en citer aucun ; mais je dois dire que si, au lieu d'attribuer au culte secret de Mithra les cinq cylindres reproduits sur la planche 9 ci-jointe, je les ai compris au nombre de ceux que revendiquent les mystères de la Vénus assyrienne, je ne m'y suis trouvé déterminé que par l'appréciation du style propre à chacun de ces cinq petits monuments, et non par des considérations tirées de tel ou tel détail caractéristique. J'en excepte cependant le cylindre n° 4, qui porte une légende où l'on reconnaît un système d'écriture cunéiforme qui n'est pas celui des Perses. Je ferai voir ailleurs que, lorsque les cylindres et les cônes ne sont accompagnés d'aucune légende en caractères cunéiformes ou en caractères propres aux écritures nationales des peuples sémitiques ou araméens, il est impossible de distinguer, autrement que par l'appréciation toujours difficile et plus ou moins arbitraire du style de la gravure, ceux de ces petits monuments figurés qu'il conviendrait de rapporter aux mystères de Mylitta ou d'Astarté, et ceux qui réellement appartiennent aux mystères de Mithra. Presque toujours ils se confondent les uns avec les autres par l'identité de leurs sujets, de leur style et des divers symboles sous lesquels y sont représentées ces deux divinités, qui jamais, à ma connaissance, ne s'y trouvent figurées sous une forme humaine.

Un passage très-laconique de Dinon, que nous a conservé Athénée,\footnote{\emph{Deipnosoph.} 11., p. 503, ed. Casaubon ; t. 2., p. 1132, ed. G. Dindorf.} est tout ce qui nous reste des détails que les auteurs anciens avaient dû recueillir sur l'emploi du cyprès, en Perse, dans les cérémonies religieuses ou civiles. Ce passage nous apprend que, chez les Perses, on tressait des \emph{couronnes de cyprès}, mais il ne nous dit pas à quel usage on les faisait servir ; et nous manquons des renseignements nécessaires pour décider si là elles étaient employées, comme ailleurs, dans les cérémonies propres à l'apothéose et dans les rites funéraires.

Avant de quitter la terre classique où il faut étudier les doctrines, les symboles et les monuments de l'art propres au culte public et aux mystères de Mithra, examinons si elle ne fournirait pas quelque témoignage non équivoque, qui pût être ajouté à celui que déjà nous ont fourni les sculptures de Persépolis. Examinons si, en même temps que, dans une province d'Arménie limitrophe de la Perse, l'entrée d'une grotte creusée dans le roc, aux environs de Vân, a retenu, d'âge en âge jusqu'à nous, le nom de \emph{Porte de Mithra},\footnote{Le docteur Frédéric Schulz (\emph{Journal asiat.} 3\textsuperscript{e} série, t. 9., p. 302 et suiv.) nous apprend, en effet, que cette entrée s'appelle encore aujourd'hui \emph{Meher-Capoussi}.} le souvenir du culte du cyprès et la vénération des Perses pour cet arbre symbolique ne se seraient pas également perpétués parmi les habitants actuels de la Perse méridionale.

À l'ouest et à trois heures de distance des sépultures royales de Nakhsch-i-Roustem, s'élèvent trois rochers qu'on désigne communément sous le nom de forteresses d'\emph{Istakhr} ou \emph{Istakhar}. Autrefois fortifiés par l'art, ils durent servir à défendre l'entrée de la plaine où sont situées les ruines de Persépolis ; ils dominent cette plaine ; et, comme si le culte du cyprès, dans des lieux qui furent chers aux sectateurs de Zoroastre, n'avait jamais dû cesser d'être attesté par un témoin vivant, on voit aujourd'hui encore, sur la plate-forme du rocher du milieu, un vieux cyprès planté à côté d'une ruine que les Persans appellent \emph{Kaleh-serv}, c'est-à-dire le \emph{Fort du cyprès}.\footnote{Voy. un rapport de M. Pascal Coste, inséré dans le journal \emph{l'Institut}, n° 77, mai 1842, p. 69.} Cet arbre, objet de leur vénération, provient, sans doute, des cyprès qui autrefois avaient dû être plantés devant les portiques de Persépolis,\footnote{J'ai déjà dit ci-dessus (p. 126) qu'au temps où Thomas Herbert voyageait en Perse, il avait eu l'occasion de remarquer que le cyprès pyramidal croît abondamment dans le Farsistân, province où sont situées les ruines antiques appelées \emph{Tchéhelminar}. Il a soin d'ajouter que les arbres représentés sur quelques bas-reliefs qu'on voit dans ces ruines sont précisément de cette même espèce.} à l'exemple des cyprès que les Assyriens, les Syriens, les Phéniciens, les Lydiens, les Phrygiens, plantaient devant ou dans les sanctuaires de Baal et de Mylitta ou Astarté ; et à l'exemple aussi des deux cyprès plantés par Zoroastre, l'un à Bactres, devant le palais du roi Gustasp, l'autre à Kischmer, devant un pyrée. Le tronc n'a pas moins de quatre mètres de circonférence ; il touche à un grand réservoir, qui reçoit de plusieurs autres réservoirs supérieurs les eaux pluviales, et qui semble destiné à nous rappeler que le culte de Mithra devait toujours se célébrer auprès d'une source naturelle ou artificielle.\footnote{Eubule cité par Porphyre, \emph{De antr. nymphar.} 7. ; cf. 17. ; ed. van Goens.} Plus au midi de la Perse, à l'ouest de Persépolis, dans les montagnes des Bahméi, au lieu nommé \emph{Tenghi-Saoulek}, et situé à six ou sept parasanges de Bahbèhan, M. le baron Bode, en 1841, a trouvé dans une forêt de chênes et de cyprès deux bas-reliefs antiques qui, s'il est permis d'en juger d'après une description très-sommaire,\footnote{Voyez la Lettre de M. Eugène Boré, qui a été publiée dans le \emph{Journal asiatique}, 3\textsuperscript{e} série, t. 13., 1842, p. 328 et suivantes.} ont dû appartenir à quelque monument consacré au culte de Mithra. Ajoutons, sur la foi de tous les voyageurs, que dans le midi de la Perse,\footnote{Voy. ci-dessus, p. 126, 127.} comme dans l'ancienne Babylonie, comme aux environs des ruines de Palmyre, les forêts de cyprès ne sont pas rares, et que cet arbre y atteint des proportions majestueuses. Ajoutons aussi, d'après le témoignage d'un passage de Firdousi,\footnote{Manuscrits persans de la Bibliothèque impériale, n° 229, fol. 3 v°.} cité par mon savant ami M. Jules Mohl,\footnote{\emph{Le Livre des Rois}, trad. franç. tom. 1., préface, p. 47.} qu'à Merv, sous le règne du sultan Mahmoud, vivait un personnage nommé \emph{Serv-Azad}, qui se disait de la famille de Sam, fils de Nériman. Il s'appelait ainsi parce que probablement le nom de Serv-Azad, que j'expliquerai dans mon second mémoire, était en usage au temps des ancêtres dont il prétendait descendre. Ce qui n'est pas douteux, c'est que ce nom composé, dont la première partie signifie cyprès, nous ramène à ceux de \emph{Sôs} et de \emph{Serv}, que nous avons vus portés, l'un par un ancien roi d'Arménie, l'autre, par un roi du Yémen, contemporain de Féridoun.

Personne n'ignore que, de tout temps, le culte des arbres, malgré l'invasion des doctrines exclusives de l'islamisme, n'a pas cessé d'être très-répandu dans les diverses provinces de la Perse.\footnote{Les relations des voyageurs européens sont unanimes à cet égard. Voyez surtout Chardin, \emph{Voyage en Perse}, t. 2., p. 44-49, 86, 87, 200 ; édit. d'Amsterdam, 1735, in-4°. W. Ouseley, \emph{Travels}, vol. 1., p. 313 ; appendix, n° 9., p. 359 et suiv. vol. 2., p. 83.} On trouvera plus loin la preuve que l'usage d'employer le cyprès comme emblème funéraire s'y est également conservé, de même que dans toute l'Asie. Le poète Saadi, dans son \emph{Gulistan} ou \emph{Jardin des roses}, nous montre un personnage persan qui, chaque nuit, se rend au pied d'un arbre sacré, s'y agenouille, et prie Dieu de lui accorder un fils. Cet arbre, dit le poète,\footnote{Cité par Ouseley, \emph{Travels}, t. 1., p. 377.} était journellement visité par une foule de pèlerins qui avaient quelque grâce à demander. Barbaro, envoyé à la cour de Perse vers l'année 1471, par le gouvernement de Venise, fut étonné, en traversant les provinces persanes, d'y trouver de vieux arbres qui recevaient une espèce du culte.\footnote{Voy. la traduction latine de son itinéraire dans l'ouvrage de Bizarus, intitulé : \emph{Rerum Persicarum historia}, p. 469 sqq.} Plus tard, en 1622, Pietro della Valle\footnote{\emph{Loc. cit.}} éprouve le même étonnement lorsqu'en présence d'un vieux cyprès resté debout à Passa ou Fasa, auprès des ruines antiques que l'on croit être celles de Pasargarde, il apprend quelles dévotions viennent accomplir les Persans au pied de cet arbre, et avec quelle foi ils croient que ce cyprès renferme l'esprit ou l'âme de quelque personnage bienheureux. Plus tard encore, Chardin\footnote{\emph{Voyage en Perse}, t. 2., p. 44, 47, 49, 86, 87 et 200.} et le frère Angélo,\footnote{\emph{Gazophyl. ling. Pers.} sub. voc. \emph{Platanus}, p. 293.} pendant leur voyage en Perse, signalent, en plusieurs lieux, le culte qu'ils voient les Persans rendre à de vieux arbres, appelés du terme générique de \emph{dirakht-i-fâzel}, c'est-à-dire les \emph{arbres excellents}, expression que l'on croirait empruntée au langage de Zoroastre. Le voyageur français, faisant une mention particulière d'un platane planté sur une des places d'Ispahan et journellement visité dans un but religieux, remarque que, parmi les Persans musulmans, les dévots aiment mieux faire leurs prières sous un vieux arbre que dans une mosquée. On sait aussi, par les récits des écrivains musulmans et des voyageurs européens,\footnote{Voy. Francklin, \emph{A Tour from Bengal to Persia} (Calcutta, 1788), p. 26. Ouseley's \emph{Travels}, vol. 1., appendix, p. 400 et suiv.} qu'à Schiraz, pendant bien des siècles, deux superbes cyprès, appelés l'\emph{amant} et sa \emph{maîtresse},\footnote{\emph{Aaschouk} et \emph{maaschouka}.} furent l'objet d'une espèce de culte de la part des habitants de la ville et de la province. Lorsqu'on les abattit, il y a quelques années, une indignation générale et presque une révolte éclatèrent contre les auteurs de cet acte, qui fut considéré comme un sacrilège. De nos jours encore, et je puis apporter ici mon témoignage personnel, on rencontre fréquemment en Perse des arbres de diverses espèces, tout chargés de morceaux d'étoffes que des pèlerins sont venus y suspendre avec une intention superstitieuse.\footnote{Dans les temps anciens, un semblable usage avait été importé de l'Asie occidentale chez les Étrusques. (Voyez \emph{Annal. de l'Instit. archéolog.} lettre citée, t. 5., p. 97.)} Sir William Ouseley, dans l'appendice de l'intéressante relation de son Voyage en Perse,\footnote{\emph{Travels}, vol. 1., p. 384, 385. --- Dans ce même volume, on trouve des détails pleins d'intérêt sur le culte des arbres en Égypte et dans quelques autres parties de l'Afrique (p. 391 et 394), comme aussi sur le platane appelé \emph{Ménélaïs} (p. 389).} rapporte même une anecdote qui prouve que, depuis Saadi jusqu'à nous, la Perse a vu plus d'une fois se renouveler, au pied des \emph{arbres excellents}, une prière inspirée par le désir d'obtenir le bonheur de la paternité.

On se tromperait fort si l'on croyait que l'usage de suspendre des offrandes aux branches de ces arbres est né en Perse dans les temps modernes. Des traditions certaines nous apprennent qu'à l'époque des Achéménides, il y était déjà établi. Sous le règne de l'un de ces princes, Xerxès 1\textsuperscript{er}, c'est le roi lui-même qui, parvenu à une journée de Sardes, s'arrête au pied d'un superbe platane, suspend aux branches de cet arbre de riches chlamydes, des bracelets d'or, plusieurs autres offrandes, et prépose à la garde de l'objet de sa dévotion un des guerriers du \emph{corps des immortels}. Le fait nous est attesté par Hérodote\footnote{7., 31. --- Le même historien (7., 27) fait une mention particulière d'un platane d'or et d'un cep de vigne, également d'or, que Pythius, seigneur lydien qui habitait la Phrygie, avait donnés en présent à Darius, fils d'Hystaspe. Voy. les autres autorités citées par M. Roulez, dans ses savants commentaires sur Ptolémée Hephæstion (p. 82). Le même commentateur rappelle (\emph{ibid.} p. 37) que Priam possédait aussi un cep de vigne d'or. Antigone en trouva un dans le trésor des rois de Perse, à Suze, après avoir défait Eumène et s'être emparé de la citadelle (Diodor. Sicul. 19., 48., 7 ; ed. Ludov. Dindorf). C'est probablement celui que Darius avait reçu de Pythius.} et par Ælien.\footnote{\emph{Var. histor.} 2., 14. : cf. 9., 39. Eustath. ad Homer. \emph{Iliad.} 2., 307.} Ce dernier écrivain place dans son récit quelques détails que ne contient pas celui d'Hérodote, et qui nous permettent de croire qu'il avait puisé ailleurs que dans le livre de son célèbre devancier. Xerxès, en agissant comme on le raconte, ne faisait qu'ajouter de puériles démonstrations aux préceptes tracés dans le \emph{Zend-Avesta}, préceptes que Zoroastre disait avoir reçus d'Ormuzd lui-même, et qui s'accordent parfaitement avec le culte des arbres en général et avec le culte du cyprès en particulier. Le \emph{Vendidâd},\footnote{Le \emph{Vendidâd} est le vingtième nosk du \emph{Zend-Avesta}.} par exemple, ordonne de prier, d'invoquer les arbres, et les qualifie de \emph{purs et saints} : « Approchez-vous des arbres qui croissent, dit Ormuzd à Zoroastre dans ce livre.\footnote{\emph{Zend-Avesta}, t. 1., 2\textsuperscript{e} partie (\emph{Vendidâd}, farg. 19.), p. 416.} Prononcez bien ces paroles près des arbres qui croissent : je prie les arbres purs et saints qu'Ormuzd a donnés.\footnote{Dans le \emph{Bahman iescht} pehlvi, qui passe pour être la traduction ou le précis seulement d'un office ou d'une prière dont l'original zend n'existe nulle part et semble même n'avoir jamais existé, on trouve la description de deux arbres merveilleux et fatidiques qu'Ormuzd montre à Zoroastre : l'un porte quatre branches de métal, et l'autre, sept. La pièce a été traduite en parsi ou persan ; la version pehlvie est rare dans l'Inde. Les plus habiles destours du Kirman, de Surate et de Bombay ne croient pas à l'authenticité de ces deux traductions, et ils ont raison. Mais la version parsie n'en est pas moins entre les mains de la plupart des Parses ou des Guèbres de l'Inde et du Kirman. (Voy. Anquetil du Perron, \emph{Zend-Avesta}, t. 1., 2\textsuperscript{e} partie, pag. 18 et 19.) --- Le Rév. John Wilson (\emph{The Parsi Religion}, p. 252, 259, 262, 267 ; Bombay, 1843) a réuni divers passages qui se rapportent au culte des arbres parmi les Parses de l'Inde.} L'abondance et le béhescht\footnote{Ce mot, qui équivaut à \emph{gorotman} et à \emph{albordj}, sert à désigner le lieu du ciel où résident Ormuzd, Mithra, les amschaspands, les izeds et les féroüers des bienheureux.} sont pour le juste qui est pur. Celui-là est pur qui est saint, qui fait des œuvres célestes et pures. Tirez de ces (arbres) le barsom long d'une fois la largeur (de l'arbre). Qu'il n'y ait que l'homme pur qui coupe le barsom ; et que, le tenant de la main gauche, il fasse izechné à Ormuzd, aux amschaspands,\footnote{Le soleil et la lune, dans le \emph{Zend-Avesta}, sont qualifiés du titre d'amschaspands, comme les sept conseillers d'Ormuzd.} au hom de couleur d'or, grand et très-pur,\footnote{On n'est pas d'accord, et on ne peut l'être, sur l'espèce à laquelle appartiennent le \emph{hom} des Perses et le \emph{hom} (\emph{soma}) des Indiens. Dans le \emph{Zend-Avesta}, ce n'est certainement qu'un nom générique ; car, sur les monuments figurés des Assyriens, des Phéniciens et des Perses, le \emph{hom} est constamment représenté avec des formes conventionnelles et très-variées, dont aucune ne saurait être rapportée à une espèce d'arbre connue. Quant au cyprès, il n'est pas une seule fois nommé dans les fragments qui nous restent du \emph{Zend-Avesta}. Le \emph{Boun-dehesch} (§ 22.) n'en fait mention qu'en énumérant quelques-uns des arbres dont les fruits ne peuvent servir de nourriture à l'homme.} au pur Bahman\footnote{Bahman est la manifestation d'Ormuzd et de Mithra comme rois du ciel (voy. \emph{Mém. de l'Acad des inscr.} nouv. série, t. 15., 2\textsuperscript{e} part. p. 220-227).} qu'Ormuzd a établi chef du behescht. » Aussi trouvons-nous dans le deuxième hâ de l'izeschné\footnote{\emph{Zend-Avesta}, t. 1., 2\textsuperscript{e} part. p. 96.} cette formule, plusieurs fois répétée ailleurs : « Avec ce zour,\footnote{Le \emph{zour} est une eau consacrée, dont la liturgie persique prescrit un fréquent usage.} avec ce barsom, je prie tous les arbres donnés d'Ormuzd, purs, et je leur fais iescht. » Ces paroles se prononcent à la suite d'une invocation dans laquelle le mazdéïesnan adresse sa prière à Mithra, au soleil, à la lune, au feu, fils d'Ormuzd, et à l'eau pure et excellente, donnée d'Ormuzd.\footnote{\emph{Zend-Avesta}, \emph{loc. cit.} p. 95.} Bien plus, on retrouve la prière aux arbres dans le néaesch du soleil\footnote{\emph{Zend-Avesta}, t. 2., p. 14.} et dans le néaesch de la lune,\footnote{\emph{Ibid.} p. 19. --- Dans cette même prière (p. 18) on récite ces paroles : « Lorsque la lumière de la lune répand la chaleur, elle fait croître les arbres de couleur d'or, elle multiplie la verdure sur la terre. »} qui, je le répète, doivent se réciter l'un avant, l'autre après le néaesch de Mithra. De telle sorte que, si les bas-reliefs romains nous montrent Mithra placé entre le cyprès du soleil et le cyprès de la lune,\footnote{Ci-dessus, p. 141.} la liturgie du \emph{Zend-Avesta} nous montre la prière à Mithra placée entre deux prières, l'une au soleil et aux arbres, l'autre à la lune et aux arbres. Dans l'iescht de Mithra, il est dit\footnote{15\textsuperscript{e} cardé, \emph{Zend-Avesta}, t. 2., p. 215.} que ce dieu « augmente l'eau et les arbres qui sont les délices des keschvars\footnote{Selon la cosmogonie du \emph{Zend-Avesta}, la terre fut divisée, par Ormuzd, en sept parties appelées \emph{keschvars}.} ; que les keschvars sont placés sous sa surveillance et qu'il est le germe des germes.\footnote{\emph{Zend-Avesta} (iescht de Mithra, 16\textsuperscript{e} cardé), t. 2., p. 215, 216.} » Le quatrième temps de la création, qui s'appelle le \emph{Gahanbar Eïathrem},\footnote{Selon la transcription d'Anquetil.} doit, d'après l'ordre de Zoroastre, se célébrer comme il se célèbre encore maintenant, dans le \emph{mois de Mithra}. Or c'est pendant ce quatrième temps qu'Ormuzd, aidé des sept amschaspands, a créé les arbres.\footnote{\emph{Zend-Avesta}, t. 2., p. 85.} Enfin, on lit dans la prière déjà citée, que « Mithra donne à la terre l'eau, les arbres, les féroüers.\footnote{\emph{Ibid.} (iescht de Mithra, 25\textsuperscript{e} cardé), t. 2., p. 223.} » Les rapports que ce passage établit entre les arbres et les féroüers méritent d'être remarqués ; ils nous font comprendre comment, chez les Perses, le culte des arbres se liait à celui des féroüers, c'est-à-dire au culte des mânes ou des ombres des saints et des héros : « Si de la terre, dit Ormuzd à Zoroastre,\footnote{\emph{Ibid.} (iescht favardin, c'est-à-dire office des féroüers, cardé 1.), t. 2., p. 249.} croissent en abondance les arbres qui donnent d'eux-mêmes la vie, c'est pour la gloire et l'éclat des féroüers (des saints forts et bien armés). » Plus loin,\footnote{\emph{Zend-Avesta}, (iescht favardin), t. 2., p. 257.} Zoroastre prie en ces termes : « Je fais izeschné aux purs, forts et excellents féroüers des saints, qui montrent aux arbres créés à croître purement ; qui, placés au-dessus (des arbres) donnés en abondance (comme) sur un trône, sont occupés à les bénir pendant le temps long (de la durée du monde) : par-là, ils répandent l'abondance pour les hommes donnés d'Ormuzd, sur (le monde qui est) la voie aux deux destins, sur les grains produits avec profusion, sur les arbres, l'amour d'Ormuzd, l'amour des amschaspands. » On a fait remarquer avant moi que l'auteur de cette prière avait dû puiser à la même source où les Grecs prirent l'idée de leurs Dryades et de leurs Hamadryades, devenues, sous la plume des poètes chrétiens du moyen âge et de la renaissance, des forêts et des arbres enchantés. Mais je crois avoir, le premier, il y a plusieurs années,\footnote{Lettre citée, \emph{Annal. de l'Inst. archéol.} t. 5., p. 97.} signalé l'importation en Étrurie de l'usage asiatique dont je parlais tout à l'heure, celui de suspendre des péplus, des écharpes ou des morceaux d'étoffes aux branches des arbres sacrés. L'exemple incontestable que j'en ai cité a été tiré des belles et curieuses peintures de la grotte Marzi.

\subsubsection{Pays situés entre la Perse et l'Indus, Inde, Thibet, Chine et Japon}
\paragraph{}
Sur les plateaux habités de l'Afghanistân, et du royaume de Kaboul en particulier, comme aussi dans les vallées dépendantes des chaînes de montagnes qui traversent ou qui bordent ces deux pays, on trouve, mêlée avec des cèdres, des chênes, des châtaigniers, des oliviers sauvages et d'autres arbres, une grande espèce de cyprès. Si je ne me trompe, elle fut remarquée pour la première fois, en 1808, par M. Mount Stuart Elphinstone.\footnote{\emph{An Account of the kingd. of Caubul} (Lond. 1815, in-4°, fig.), p. 146.} M. Carl Ritter\footnote{\emph{Die Erdkunde}, 11., 568.} n'hésite pas à la rapporter à l'espèce appelée en Europe \emph{pyramidale}, bien que le voyageur anglais ne s'explique pas sur ce point et semble faire ses réserves en disant : « A sort of gigantic cypress are also among the natives of the mountains. » L'habile géographe allemand n'hésite pas davantage à affirmer\footnote{\emph{Ibid.} p. 568, 569.} que le cyprès pyramidal est du nombre des arbres dont se composent les forêts situées aux environs de Hérat, dans l'Afghanistân, mais qu'il ne croît pas spontanément dans les vallées du Kachemîr.

On peut du moins regarder comme certain que la culture du cyprès pyramidal a été importée dans ce dernier pays. Elle y est même fort répandue, selon les témoignages unanimes que j'ai recueillis de la bouche de plusieurs voyageurs instruits. Elle remonte à une époque qu'on ne saurait déterminer, mais qui peut être suivit de près, si elle ne l'avait précédée, l'arrivée de Zoroastre dans un pays voisin, la Bactriane, et l'inauguration des deux cyprès que ce personnage célèbre plaça, l'un devant le palais du roi Gustasp, à Bactres, l'autre, devant le pyrée de Kischmer, dans le Khorassan. Ce qui me porte à présenter cette conjecture, c'est le nom de \emph{dêva-darou}, l'arbre de dieu ou l'arbre divin, qu'a retenu le cyprès pyramidal chez les Kachemîriens. Dans leurs poèmes épiques populaires, il est fréquemment désigné sous cette dénomination. Il y sert de terme de comparaison à peu près comme dans le \emph{Schah-nameh} et dans les autres poèmes modernes de l'Orient ; et nous verrons un peu plus loin que ce ne peut être un emprunt fait aux poètes indiens. Remarquons que \emph{dêva-darou} est identique avec le \emph{div-dar} et le \emph{div-darou} des lexiques persans ou arabes. D'un autre côté, les ornements que, dans les châles de Kachemîr, nous appelons improprement \emph{palmes}, doivent, selon toute probabilité, leur origine à l'imitation du cyprès pyramidal. Cette opinion semble trouver une confirmation dans la comparaison que chacun peut faire de la forme habituelle de ces palmes avec la forme de convention qu'affectent les cyprès sculptés sur les murs des palais assyriens de Khorsabad et de Nimroud, et particulièrement sur les rampes du grand escalier à Persépolis. Or cet arbre est originaire des contrées occidentales de l'Asie, et non de la province de Kachemîr. Si j'ajoute que la fabrication des châles, commune à ce pays et à la Perse, remonte à une époque fort ancienne, de même que le tissage des brocarts d'or ou d'argent et des étoffes de soie, où les Persans se plaisent à prodiguer les palmes, c'est-à-dire les cyprès, nous serons conduits à conjecturer aussi qu'en Perse, comme dans le Kachemîr, ce genre d'ornement a dû primitivement être appliqué aux étoffes riches et précieuses dont on avait, dans tout l'Orient, l'usage de parer les statues et les temples des dieux. Ces étoffes, ainsi que celles dont on se servait pour le costume des prêtres, des rois et des dignitaires de la cour, étaient souvent chargées de symboles religieux. Les sculptures des palais assyriens découverts à Nimroud par M. Layard ne laissent aucun doute sur ce point.

Quant aux traces d'un culte religieux qui, dans les provinces appelées maintenant l'Afghanistân proprement dit, le Kaboul et le Kandahar, aurait compté le cyprès pyramidal au nombre de ses symboles ou de ses emblèmes, nous n'en découvrons aucune. Mais on sait combien sont rares les monuments figurés antiques et les traditions qui se rapportent à ces contrées ; et il y a tout lieu de croire que le culte du cyprès ne fut pas importé par Zoroastre à Bactres et à Kischmer, ni adopté par le roi Gustasp, sans s'étendre dans les pays situés au midi de la Bactriane, de la Margiane et de l'Hyrcanie.

Au-delà de l'Indus et de ses affluents, dans l'Inde anglaise, le cyprès pyramidal ne se montre indigène nulle part. Mais le lieutenant-colonel Fitz-Clarence,\footnote{\emph{Journal of a route across India} (Lond. 1819, in-4°), p. 170.} traversant le plateau élevé du Dékan septentrional, l'a trouvé cultivé dans les jardins d'Aurengabad, au midi du fleuve Tapti. Ce fait nous autorise-t-il à penser que, dès les temps anciens, la culture du cyprès pyramidal n'était pas restée étrangère au sol ni aux usages religieux de l'Inde ? Je n'ose me prononcer affirmativement. M. Carl Ritter se déclare pour la négative\footnote{\emph{Die Erdkunde}, 6., p. 334.} ; et, après avoir constaté que des plantations de cyprès se montrent aux environs d'Aurengabad, il fait remarquer que cet arbre manque absolument dans le Bengale et l'Hindoustan. Il en conclut que le cyprès a dû être importé dans le Dékan par les musulmans, qui, observe-t-il avec raison, aiment beaucoup à le placer sur leurs tombeaux. Cette importation lui paraît remonter au temps d'Ibn-Batoutah, c'est-à-dire vers l'an 1340 de notre ère. Toutefois on trouve, à diverses reprises, m'assure-t-on, une mention expresse du cyprès dans les plus anciens livres sacrés des Indiens, les Vêdas. Il y reçoit, de même que dans les poèmes kachemîriens, le nom de \emph{dêvadarou}, l'arbre de Dieu.\footnote{Les Hindous donnent aussi le nom de \emph{dêva-darou}, arbre de Dieu, à une espèce de pin qui croît dans les forêts de l'Himalaya.} De plus, il y est présenté comme l'objet d'un culte particulier auquel s'associe celui de plusieurs personnages mythologiques, très-analogues aux Faunes, aux Dryades et aux Hamadryades des écrivains occidentaux de l'antiquité ; circonstance qui a conduit à remarquer que ces deux derniers noms nous offrent, dans leur composition, la syllabe \emph{dry},\footnote{« \textgreek{Δρῦς. Πᾶν ξύλον, καὶ δένδρον}, » dit Hésychius, \emph{sub voc.} ed. Alberti. Pline (\emph{Hist. nat.} 16., 95., 1) pense que le nom des \emph{Druides} peut venir de ce mot grec \textgreek{δρῦς}.} identique avec \emph{drou} et \emph{darou}, qui, en sanscrit, comme dans les idiomes de la Perse, signifient \emph{arbre}. Mais, en même temps, il faut observer que les monuments figurés qui nous restent de l'Inde ancienne ne présentent aucun exemple de l'attribution du cyprès à une des nombreuses divinités du panthéon indien, ni de son emploi comme emblème funéraire. Il faut observer aussi que, dans tout l'Hindoustan, on ne voit cet arbre planté auprès d'aucune sépulture, et que les poètes modernes de cette vaste contrée, au lieu de comparer au cyprès, comme le font les poètes arabes ou persans, le port et la taille de leurs héros et de leurs héroïnes, les comparent à d'autres espèces d'arbres qui sont indigènes sur le sol de l'Inde. C'est parmi celles qui portent sur deux pieds séparés leurs fleurs mâles et leurs fleurs femelles, que les Indiens choisissent, je l'ai dit plus haut, les deux arbres qui, selon un très-ancien usage, servent à célébrer d'une manière symbolique les noces des dieux.\footnote{Ci-dessus, p. 7. --- Le culte des arbres, en général, paraît avoir été fort répandu dans l'Inde. Anquetil du Perron (\emph{Zend-Avesta}, t. 1., 1\textsuperscript{re} part. p. Ixxxiij-Ixxxvij) a tiré de l'\emph{Ezour-Vêdam} plusieurs passages qui mettent ce fait hors de toute contestation. Déjà les savants auteurs des \emph{Lettres édifiantes} (t. 12., p. 429) avaient recueilli une tradition dont un de ces passages n'est qu'une variante. Dans d'autres traditions indiennes, il est souvent question de l'\emph{arbre du désir des dieux} et de l'\emph{arbre qui remplit tous les désirs}, qualification que les Hindous donnent encore maintenant à leurs divinités et, par similitude, aux grands personnages historiques. (Voy. à ce sujet les observations intéressantes de M. A. Troyer, \emph{Râdjataranginî, Hist. des rois du Kachemîr}, trad. franç. t. 2., notes du livre 1., p. 165.)} J'ajoute ici que les livres religieux de ce peuple qui nous font connaître cet usage, mentionnent aussi la coutume où étaient les dévots d'adopter pour fils un arbre, et de lui adresser des invocations en l'appelant « mon fils.\footnote{Voici une de ces invocations : « J'invoque comme témoins le feu dans lequel des sacrifices ont été faits par des Brahmanes, d'après la manière prescrite, et le soleil qui est au ciel ; tous deux voient que je te nomme, ô arbre, pour remplir envers moi, à jamais, les devoirs d'un fils. » Je dois la connaissance de cette invocation au savant indianiste nommé dans la note précédente. Déférant, avec son obligeance habituelle, au désir que je lui avais exprimé, il a bien voulu la traduire en français et me la communiquer. Elle est tirée du \emph{Bhavisch yottara Pourana}, que l'on sait être un supplément ancien au \emph{Bhavischya Pourana}, l'un des dix-huit \emph{pouranas} composés par les commentateurs des Vêdas. Ce supplément contient l'exposition détaillée des observances et des devoirs religieux imposés aux Indiens, à diverses époques de leur histoire. La formule qu'en a extraite M. le major A. Troyer justifie, on le voit, Abraham Roger d'avoir dit, il y a deux siècles révolus, que les Hindous « semblent avoir conservé encore des restes d'un ancien magisme dans le respect qu'ils portent à un feu qu'ils appellent \emph{homan}, et qu'ils regardent comme sacré, et à un certain arbre nommé \emph{casta}, dont ils n'oseraient cueillir une feuille, et en l'honneur duquel ils bâtissent de petits temples. » Cette observation, si curieuse par sa date et par son exactitude, se trouve (p. 189 de la trad.) dans un ouvrage qu'en 1651 l'auteur publia à Leyde (in-4°), sous le titre un peu singulier de \emph{De Open-Deure tot het verborgen Heydendom}, c'est-à-dire, \emph{La Porte ouverte pour parvenir à la connaissance du Paganisme caché}. Ce titre est ainsi traduit dans la version française que Thomas La Grue donna du livre d'Abraham Roger, en 1670 (Amsterd. in-4°).} » Toutefois nous ne voyons pas que cet arbre fût de l'espèce du cyprès.

Il n'en est pas de même des arbres dont nous parlent certaines traditions qui de l'Orient étaient passées en Occident, et qui, à l'époque de la décadence, avaient cours encore dans l'empire romain.\footnote{Le culte des arbres et les idées superstitieuses qu'il avait fait naître se perpétuèrent d'âge en âge dans l'Orient, comme dans l'Occident ; l'islamisme, non plus que le christianisme, ne parvint jamais à en faire disparaître les dernières traces. L'empereur Julien, qui, loin de répudier les superstitions du paganisme défaillant, s'efforçait de leur rendre la vie, croyait aux songes et au rôle qu'y jouent les arbres en particulier. Dans une lettre que, pendant ses démêlés avec l'empereur Constance, il écrivait à Oribase (\emph{Epistol.} 17.), il raconte à ce médecin qu'il a vu en songe deux arbres : des racines de l'un, qui était vieux et couché sur le sol, il en sortait un autre, jeune, vigoureux et encore debout. Ce songe avait jeté du trouble dans l'esprit de Julien ; mais un génie lui apparut, et le rassura un peu en lui disant que le jeune arbre resterait intact et recevrait même des forces nouvelles.} Si ces traditions ne nous ont été conservées que dans des ouvrages où elles sont accompagnées de circonstances plus ou moins merveilleuses, fabuleuses, incroyables, on ne peut cependant, à mon avis, leur dénier toute authenticité, ni méconnaître que, quant au fond, elles ont une origine ancienne. Ainsi, par exemple, dans l'histoire fabuleuse d'Alexandre le Grand, nous voyons le conquérant macédonien, à son arrivée dans l'Inde, consulter l'\emph{arbre du soleil} et l'\emph{arbre de la lune}, qui lui répondent en indien et en grec,\footnote{C'est sans doute à l'oracle rendu par ces deux arbres que fait allusion la carte de Peutinger, par ces mots tracés à l'extrémité nord-est de l'Asie : \emph{Hic Alexander responsum accepit}. Nous lisons dans la \emph{Vie d'Apollonius de Tyane}, par Philostrate (6., 10.), que, sur l'ordre de Thespésion, chef des gymnosophistes, un ormeau salua Apollonius en lui adressant la parole d'une manière distincte et d'une voix semblable à celle d'une femme.} et qui, ajoute Julius Valérius,\footnote{\emph{Res gest. Alexandr.} 3., 38-42.} \emph{ressemblaient à des cyprès} de l'espèce appelée, selon lui, \emph{mirobolani}. Dans la lettre \emph{De mirabilibus Indiæ}, attribuée au même conquérant, il est question de l'\emph{arbre du soleil} et de ses fruits\footnote{Cf. Silv. Girald. \emph{Topograph. hibern.} pars 2., cap. 4. Petri Cosmetoris \emph{Scholastic. Histor.} Argentor. 1515, in-4°. --- Le célèbre voyageur Marco Polo, visitant l'Asie vers la fin du 13\textsuperscript{e} siècle, rapporte, au chapitre 40. (27.) de sa relation, que, dans la province de \emph{Tonocaïn} ou \emph{Timochaïn}, on trouve l'\emph{arbre seul}, appelé par les chrétiens l'\emph{arbre sèche} (voy. \emph{Recueil de voyag. et de mém.} publié par la Soc. franç. de Géogr. Paris, 1824, in-4°, tom. 1., p. 37, 38, et p. 326). Les manuscrits donnent les curieuses variantes : \emph{arbore del Sole, arbon Solis}. D'autre part, les voyageurs et les romanciers des 13\textsuperscript{e} et 14\textsuperscript{e} siècles emploient les expressions \emph{sec-arbre}, \emph{sech-arbre} ou \emph{arbre-sec}, pour désigner tantôt un arbre de la vallée de Mambré, aux environs de la mer Morte (Mandeville, Relation de ses voyages, mss. français de la Biblioth. impériale, n° 8392, fol. 157 v°), tantôt et plus généralement les terres mêmes situées à l'extrémité de l'Asie orientale. Voy. \emph{Roman du comte de Poitiers} (13\textsuperscript{e} siècle), publié par M. Francisque Michel (Paris, 1831, in-8°), p. 54, v. 1286 ; p. 68, v. 1636 ; et la \emph{Note supplémentaire} à ce roman, placée à la suite du \emph{Roman de Mahomet et Livre de la loi au Sarrazin} (14\textsuperscript{e} siècle), par MM. Reinaud et Fr. Michel. Les savants éditeurs n'ont pas omis de faire remarquer que le passage de Mandeville sur l'\emph{arbre-sec} manque dans l'édition de la relation de ce voyageur qui fait partie du recueil de Bergeron, édition d'ailleurs fautive et incomplète. Ce passage se trouve dans les manuscrits, dans les éditions en caractères gothiques et dans la traduction allemande.} ; les arbres sacrés des Indiens y sont également assimilés à des cyprès.\footnote{« In medio autem luci sacratæ arbores, simillimæ cupressis frondium genere, pedum altæ centenorum, erant, quas \emph{Betrionas} Indi appellant. »} Or peut-on se défendre de rapprocher de ces traditions les monuments figurés asiatiques où nous avons reconnu le cyprès du soleil et le cyprès de la lune, et notamment les médailles impériales de Perga, qui nous ont montré l'\emph{arbre du soleil} et l'\emph{arbre de la lune} représentés par deux cyprès pyramidaux surmontés, l'un, de l'astérisque du soleil, l'autre, du croissant de la lune ? Peut-on aussi se défendre de songer aux médailles occidentales et aux bas-reliefs mithriaques romains dont j'ai déjà dit quelque chose, et où nous allons tout à l'heure retrouver le cyprès du soleil et le cyprès de la lune ? Puis-je enfin m'abstenir de rappeler ici que de l'Orient les traditions relatives à l'\emph{arbre du soleil} et à l'\emph{arbre de la lune} passèrent même dans les sciences et la littérature de l'Europe chrétienne ? Sur quelques cartes géographiques, qui remontent au moyen âge,\footnote{Je crois faire une chose agréable à l'Académie en plaçant ici sous ses yeux une note qu'à ma prière un de mes meilleurs amis, M. le vicomte de Santarem, qui est à la fois un habile homme d'état et un des membres les plus érudits de l'Académie royale de Lisbonne, a eu la complaisance de me remettre. Elle prouvera, s'il en était besoin, combien ce savant est versé dans la connaissance particulière de tous les documents qui appartiennent à l'histoire de la géographie durant le moyen âge :\\\hspace*{5mm}« Au 12\textsuperscript{e} siècle, on remarque, pour la première fois, dans une mappemonde tirée de l'ouvrage manuscrit de Lambertus, intitulé \emph{Floridus}, qui se conserve à la bibliothèque de l'université de Gand et à celle de Paris, la mention de l'\emph{arbre du Soleil} et de l'\emph{arbre de la Lune}. Le cosmographe place ces arbres dans l'\emph{India ultima}, c'est-à-dire à l'orient de la Perse. » ( « Voyez ce monument dans mon Atlas, et l'analyse dans le tome 2., p. 189, de mon \emph{Histoire de la cosmographie et de la cartographie pendant le moyen âge}, etc. » )\\\hspace*{5mm}« Dans la mappemonde du Polychronicon de Ranulphus Hydgen, ( « Cette mappemonde se trouve reproduite dans mon Atlas. Voyez l'analyse dans le tome 2., p. 28, de l'ouvrage cilé. » ) dressée au commencement du 14\textsuperscript{e} siècle (ms. du \emph{British Museum}), on remarque une légende placée dans l'Inde, qui dit : »\\\hspace*{10mm}Hic Alexander petebat responsum ab arboribus.\\\hspace*{5mm}« La même légende se rencontre dans la mappemonde du musée du cardinal Borgia, dressée au commencement du 15\textsuperscript{e} siècle. ( « Voy. ce monument dans mon Atlas, et l'analyse dans le tome 3., p. 28, de mon ouvrage. » ) Dans cette carte, la légende est placée du côté de la Perse. La voici : »\\\hspace*{10mm}Arbores conserti (\emph{sic}) quibus locutus est Alexander. C'est-à-dire :\\\hspace*{5mm}« La forêt d'arbres auxquels Alexandre s'adressa (pour les consulter). »\\\hspace*{5mm}« Enfin, dans la mappemonde tirée du livre très-rare intitulé : \emph{Rudimentum novitiorum}, monument géographique du 14\textsuperscript{e} siècle, copié d'autres plus anciens, on remarque, au midi de la Perse, un arbre supportant les figures du \emph{Soleil} et de la \emph{Lune}, avec cette légende :\\\hspace*{10mm}Arbor Solis et Lune. » ( « Voyez t. 3., p. 239, de mon ouvrage. » )\\\hspace*{5mm}« Le Soleil se trouve placé à droite, et la Lune à gauche. »\\\hspace*{5mm}« D'après le rapprochement que j'ai fait de ces légendes avec la lettre d'Alexandre à sa mère, extraite du Pseudo-Callisthène, j'ai pensé que ces légendes, ainsi que les figures ou emblèmes qu'elles accompagnent, doivent avoir leur origine dans les traditions religieuses de l'ancienne Asie. »\\\hspace*{5mm}« En rapprochant la date de la première croisade de Pierre l'Hermite, qui eut lieu de 1093 à 1094, et la date de la composition de l'ouvrage de Lambertus (1120) où se trouve, pour la première fois, la mention de l'\emph{arbre du soleil} et de l'\emph{arbre de la lune}, il me semble que les cosmographes occidentales ont dû recevoir cette tradition de l'Orient à l'époque des premières croisades. »\\\hspace*{10mm}« Paris, le 22 mars 1852. »\\\hspace*{15mm}« Vte DE SANTAREM. »} ne voyons-nous pas l'Inde ou l'Asie orientale caractérisée par les mots \emph{Arbor Solis et Lune} (\emph{sic}), ou par deux arbres peints ou dessinés, et surmontés des images du Soleil et de la Lune ? Un romancier du 14\textsuperscript{e} siècle\footnote{L'auteur inconnu du roman italien intitulé : \emph{Guarino} ou \emph{Guerino il Meschino}.} n'est-il pas plus explicite encore, lorsque, parlant de l'arbre du soleil et de l'arbre de la lune, il les place sur le mont \emph{Tigrisonte} et les déclare semblables à des cyprès ? Enfin, dans l'ancien poème italien intitulé : \emph{Alessandro Magno} ou \emph{Alessandreide}, et attribué, ainsi que \emph{Il Trojano}, à Jacopo di Carlo, prêtre florentin, les deux arbres fatidiques consultés dans l'Inde par Alexandre le Grand ne sont-ils pas dépeints en ces termes\footnote{Chant 10. ; édit. de Venise, 1627, in-12, fort incorrecte.} ?

\begin{quotation}

L'arbor del Sole le sue foglie hauia

Rosse si come lo oro lustrante\footnote{\emph{Si come loro e rilustrante}, dans une autre édition non moins incorrecte.} ;

Quello della luna bianche le tenia

Si come argento chiare, candidante.\footnote{\emph{Si come argento chiare e candidante}, ibid.}
\end{quotation}
\paragraph{}
Ces quatre vers, on le voit, peuvent de plus servir à confirmer l'opinion où je suis qu'en Asie, soit à l'entrée des temples, soit dans les sculptures et les peintures religieuses, le cyprès du soleil devait être d'or ou doré, et le cyprès de la lune, d'argent ou argenté.

Tout n'est pas dit, sans doute, sur les traces qui peuvent rester du culte du cyprès dans l'Inde. Je reprendrai ce sujet dans le mémoire suivant, où je traite de l'emploi de cet arbre comme emblème funéraire ; mais j'aurai bien peu de chose à ajouter ; et tant que la partie des \emph{Vêdas} où il est question du cyprès pyramidal n'aura pas été publiée, les observations que j'ai pu recueillir sur le rôle qu'il joue dans les anciennes traditions et dans les usages des Indiens resteront très-incomplètes et fort peu satisfaisantes.

Je ne serais pas davantage en mesure de parler avec quelque certitude des idées que les Chinois\footnote{On sait, depuis longtemps, que le culte des arbres est fort répandu dans le royaume de Tonquin, limitrophe de la Chine, et dans le royaume de Bao, tributaire du Tonquin, dont jadis il était une des seize provinces. (Voyez la relation du P. Philippe de Marini, intitulée : \emph{Delle missioni de' Padri della compania di Giesu nella provincia del Giappone, e particolarmente di quella di Tunkino} (Rom. 1663, in-4°, lib. 2., cap. 11., p. 250.)} et les Japonais attachaient anciennement ou attachent maintenant encore au cyprès pyramidal, si notre confrère M. Stanislas Julien, dont l'obligeance est aussi inépuisable que le savoir, n'avait eu, à ma prière, la complaisance de rechercher dans les livres chinois et de traduire en français les passages ou les articles qui se rapportent à l'arbre appelé \emph{pe}. C'est le nom qu'en Chine et au Japon on donne au cyprès. Voici les extraits que le savant sinologue a bien voulu me communiquer :

\begin{center}
« \emph{Pe} \textchinese{柏}, cyprès (\emph{Pen-tsa'o-kang-mo}, liv. 34.). »
\end{center}

\paragraph{}
« Li-chi-tchin (le compilateur du \emph{Pen-tsao}) s'exprime ainsi d'après le dictionnaire \emph{Lou-chou-tsing-wen}, de Weï-tseu-thsaï : »

« Tous les arbres se tournent vers le midi (côté du principe \emph{Yang}), mais l'arbre \emph{pe}, seul, se tourne vers le couchant ; or, c'est un arbre placé sous l'influence du principe \emph{In}. Le nom de cet arbre se compose du signe générique \emph{mou} \textchinese{木} (arbre) et du mot \emph{pe} \textchinese{白} (blanc), parce que le \emph{blanc} est la couleur affectée à l'occident. »

« On lit dans le dictionnaire \emph{Pi-ya} de Lo-tien : « Le cyprès montre l'occident, comme l'aiguille aimantée montre le nord. »

« \emph{Ibidem.} Keou-tsong-pe dit : « Quand je remplissais une charge dans la province du Chen-si, je montai sur les hautes montagnes et je vis des milliers de cyprès qui tous montraient le couchant. Or cet arbre a un bois extrêmement dur, et il ne craint ni la gelée, ni la neige. »

« \emph{Ibidem.} T'ao-hong-king dit : « En tous lieux on voit des \emph{pe} (cyprès) ; on préfère ceux du mont Thaï-chan ; mais (pour l'usage de la médecine) il faut éviter de prendre (cueillir) les feuilles de ceux qui sont plantés autour des tombeaux. Les meilleures sont celles qu'on recueille en été et en automne. »

« L'auteur du \emph{Pen-ts'ao} donne beaucoup de détails sur le \emph{pe} ; mais ils se rapportent aux propriétés médicinales des feuilles de cet arbre, de ses graines, etc. »

\begin{center}
« Extrait du grand dictionnaire \emph{Tching-tseu-t'ong}. »
\end{center}

\paragraph{}
« On lit dans l'ouvrage \emph{Ou-king-t'ong-i} : « Les princes feudataires (reguli) faisaient planter des cyprès (\emph{pe}) autour de leurs tombeaux. »

« \emph{Ibid.} On lit dans les \emph{Annales des Han}, biographie de Tong-fang-sou : « Le cyprès est le palais (\emph{sic}) des démons. Les démons et les esprits, dit le commentateur Yen-kou, recherchent l'obscurité et le mystère ; c'est pourquoi ils prennent les pins (\emph{song}) et les cyprès pour palais (c'est-à-dire, pour asile).\footnote{M. Stanislas Julien ne m'a pas laissé ignorer qu'ayant consulté, à l'article \emph{Cyprès}, beaucoup d'autres dictionnaires chinois, entre autres le \emph{Pin-tseu-tsien} (qui a servi de base à celui du P. Basile) et le \emph{King-tsie-tsouan-kou} (en soixante volumes), il n'y a rien trouvé qui méritât d'être cité.} »

« Dans l'Encyclopédie botanique (\emph{Kouang-kiun-fang-pou}, l. 71., fol. 16 r°), il est dit, d'après le mémoire intitulé \emph{Hoa-youen-ki}, que « les feuilles de cyprès sont la médecine de l'immortalité (\emph{Tch'ang-sing-yo}). » Il faut, dit-on, faire sécher les feuilles de cyprès, les réduire en poudre, prendre longtemps de celte poudre, et s'abstenir de manger de toute sorte de chair d'animaux (\emph{sic}). »

Plusieurs particularités sont à remarquer dans ces passages importants, dont je ne sache pas que personne ait encore fait usage. Et d'abord, l'affectation que mettent les auteurs chinois à dire que le cyprès se tourne constamment vers le couchant ne soulève-t-elle pas deux questions intéressantes ? Faut-il croire que, par occident, ces auteurs veulent désigner l'empire des morts ? Ou bien devons-nous admettre que, par occident, ils entendent assigner pour patrie à cet arbre l'Asie occidentale, d'où, en effet, il est originaire, et où il jouissait du privilège de représenter symboliquement les dieux créateurs, le soleil, la lune et les planètes ? Il ne m'appartient pas de prononcer ; mais si les Chinois nous disent que le cyprès est placé sous l'influence du principe \emph{In}, qui, chez eux, équivaut au principe femelle ; et s'ils ajoutent que le nom de cet arbre se compose du signe générique \emph{arbre} et du mot qui signifie \emph{blanc}, n'est-il pas très-probable qu'en recevant de l'Asie occidentale le cyprès, leurs ancêtres avaient appris que là cet arbre était particulièrement consacré à une divinité femelle et lunaire ? Et la qualification d'\emph{arbre blanc} ne nous reporte-t-elle pas au \emph{cyprès de la lune}, à ce cyprès qui, selon mes observations, devait être d'argent ou argenté, et que le poète italien cité plus haut\footnote{Ci-dessus, p. 159.} caractérise en disant que ses feuilles sont \emph{blanches} :

\begin{quotation}
Si come argento chiare, candidante ?
\end{quotation}

\paragraph{}
Bien des siècles avant ce poète, une inscription grecque, trouvée dans le royaume de Naples, applique également au cyprès l'épithète de \emph{blanc}, \textgreek{λευκή}, et le consacre à Hadès ou Pluton. Ces diverses coïncidences, sur lesquelles je reviendrai plus loin, ne nous autorisent-elles pas à penser qu'il faut chercher ailleurs que dans la coutume chinoise d'affecter à l'occident la couleur blanche et de porter les deuils en blanc, l'origine de la dénomination \emph{pe, blanc}, attribuée, en Chine, au cyprès pyramidal ? Enfin, si nous lisons dans les Annales des Han que cet arbre est le palais des démons et des esprits, ne sommes-nous pas ramenés aux traditions qui nous ont révélé, chez les peuples de l'Asie occidentale et de l'Arabie, l'usage d'ériger des palais ou des chapelles sur les cyprès consacrés aux dieux\footnote{Ci-dessus, p. 128, 129 et 131.} ?

J'examinerai plus loin\footnote{Voy. mon second mémoire.} les faits qui, dans les livres des Chinois, se rapportent au cyprès pyramidal, considéré comme symbole de l'immortalité et comme emblème funéraire. Ces faits concourent, avec les remarques qu'on vient de lire, à mettre en évidence les communications qui jadis existèrent entre la Chine et l'Asie antérieure ; ils semblent nous permettre de supposer, quant au cyprès, que les idées des Chinois furent communes aux habitants du Thibet et du Japon. J'ignore toutefois si l'on possède à cet égard des renseignements directs, et si la culture du cyprès pyramidal s'est introduite ou non chez ces deux derniers peuples.

Les relations des voyageurs s'accordent à montrer qu'en Chine, comme dans l'Inde, le cyprès ne croît pas spontanément, mais qu'il y a été importé. M. Timkowski,\footnote{\emph{Voyage en Chine}, t. 2., p. 55, 80 ; édit. de Klaproth.} après l'avoir cherché en vain dans les forêts d'arbres verts qu'on rencontre aux environs de Pékin, l'a trouvé très-répandu dans les provinces plus septentrionales du céleste empire. Là il est cultivé par la main de l'homme, et ordinairement planté, avec quelques espèces de junipérus et de thuya, autour des temples et des sépultures. De son côté, un voyageur anglais de date beaucoup plus récente, M. Robert Fortune, ne cite point le cyprès pyramidal au nombre des arbres verts qui sont indigènes dans les diverses parties de la Chine qu'il a parcourues ; mais il a découvert,\footnote{\emph{A Journey to the tea countr. of China} (Lond. 1852, in-8°, fig.), p. 61-64 ; pl. à la p. 63.} dans un des districts où l'on cultive le thé, non loin de la ville appelée \emph{Yen-tchéou-fou}, une belle espèce de cyprès, dont les rameaux retombent vers le sol ; celle-ci croît spontanément, et il la nomme \emph{funereal cypress, cupressus funebris}. Quelques mois plus tard, il la retrouva, parmi des pins, dans un autre district.\footnote{\emph{Ibid.} p. 70.} Le dessin placé au bas de la page 239 de sa relation nous prouve que parfois cette espèce remplace autour des tombeaux le cyprès pyramidal, et que, comme ce dernier, elle y est accompagnée de quelques autres arbres verts. Ce que M. Fortune paraît avoir ignoré, c'est que, dans l'Encyclopédie japonaise, parmi les \emph{arbres odorants}, on trouve décrit et figuré, à l'article \emph{Pe}, le cyprès qu'il appelle \emph{cupressus funebris}. Bien que le dessin en soit médiocrement exécuté, comme la plupart de ceux qui, dans cette Encyclopédie, représentent d'autres végétaux, il m'a été facile d'y reconnaître ce cyprès à l'aide de la relation du voyageur anglais.

\subsubsection{Égypte et littoral de Carthage}
\paragraph{}
Le cyprès pyramidal ne croît pas non plus spontanément sur le sol de l'Afrique, où, comme dans l'Inde, le Thibet et la Chine, rien n'indique même que la main de l'homme ait jamais été employée à planter et cultiver des forêts de cet arbre. C'est de la Syrie que, dans les temps anciens, l'Égypte tirait tout le bois de cyprès dont elle avait besoin pour ses constructions navales.\footnote{Il est étonnant que Vitruve (2., 9.), contrairement au témoignage de toute l'antiquité, ait pu comprendre l'Afrique au nombre des contrées où le cyprès croît spontanément : « Nascuntur autem hæ arbores (cupressi), dit-il, maxime Cretæ et Africæ et nonnullis Syriæ regionibus. » Si l'on trouve aujourd'hui des pieds de cet arbre en plusieurs lieux de l'Afrique septentrionale, ils y ont été plantés de la main des musulmans.} Lors donc que nous trouvons des cyprès représentés sur les anciens édifices religieux ou sur les monuments funéraires que nous a légués l'antiquité égyptienne, nous pouvons hardiment affirmer que les Égyptiens avaient reçu de l'Asie occidentale le culte de cet arbre symbolique avec le culte de quelque divinité génératrice, ou de quelque divinité infernale. Parmi les sculptures égyptiennes que le temps a respectées, je suis en mesure de citer ou de produire plusieurs bas-reliefs et deux stèles dont le témoignage peut être invoqué, avec toute confiance, à l'appui d'une telle assertion.

Le premier monument sur lequel j'appellerai l'attention de l'Académie est un bas-relief qui décore, à l'est, une des faces du grand temple de Dendérah, de ce temple que l'on sait avoir été élevé en l'honneur d'Hâthôr, divinité originaire de l'Asie occidentale, et identique avec la Vénus assyrienne ou babylonienne. Une des significations symboliques que j'attribue au cyprès pyramidal se révèle ici à nos yeux d'une manière non moins explicite que sur les monuments figurés asiatiques qui ont été cités dans les trois premiers paragraphes de ce mémoire. Le bas-relief dont j'entends parler, et dont un dessin est gravé dans la \emph{Description de l'Égypte},\footnote{Antiquités, t. 4., pl. 26., fig. 1.} représente, en effet, un prêtre debout, accomplissant un acte d'adoration devant un dieu mâle et ithyphallique, et portant de la main droite un jeune cyprès pyramidal, qu'il élève vers ce dieu. C'est ainsi qu'un cylindre asiatique,\footnote{Il est figuré sous le n° 11 de la planche 36. (atlas) des \emph{Rech. sur Mithra}.} appartenant à M. Robert Goff, nous montre un arbre de cette espèce, porté, de la même manière, par une prêtresse de Mylitta ou Astarté. Sur ce cylindre, dont j'ai déjà dit quelques mots, ce n'est point une offrande que fait la prêtresse. Debout devant deux mystes debout aussi et armés, elle leur présente un jeune cyprès comme un symbole de vie, c'est-à-dire comme le symbole de cette nouvelle vie où les pas progressifs des initiés sont marqués par les grades qu'ils se rendent dignes d'obtenir.\footnote{Le jeune cyprès que porte ici la prêtresse était sans doute d'or ou doré. Cette supposition, justifiée, ce me semble, par les observations présentées plus haut sur le cyprès du soleil ou de la lumière, rend complète l'allusion à l'initiation aux mystères, qui se découvre sans peine dans le célèbre récit de Virgile, où la Sibylle ordonne au pieux Énée d'avoir à la main, pour descendre aux enfers, un rameau d'or, détaché de l'arbre sacré.} Une croix ansée, autre symbole de nouvelle vie, est gravée, dans le champ du cylindre, entre la prêtresse et les deux mystes ; elle semble confirmer l'interprétation que je donne de cette scène, et nous rappelle que la déesse Hâthôr, identique, nous allons le voir, avec la Vénus qui, chez les Assyriens et les Phéniciens, présidait aux initiations, est souvent représentée tenant à la main une croix ansée.\footnote{Champollion le jeune, \emph{Panth. égypt.} pl. 18. --- Voy. mes \emph{Rech. sur le culte de Vénus}, pl. 14 D, n°\textsuperscript{s} 5 et 5a ; et mes \emph{Observat. sur la croix ansée} (\emph{Mémoires de l'Acad. des inscr.} t. 17., 1\textsuperscript{re} part. p. 374).} Tout à l'heure nous retrouverons un jeune cyprès pyramidal à la main d'un personnage qui, à Edfou, dans deux tableaux des douze mois de l'année, représente, soit le dieu générateur qui préside au mois de \emph{tôbi}, soit le mois de \emph{tôbi} lui-même, personnifié. Par ces rapprochements instructifs, on peut remonter sans peine à l'origine des dendrophories, et constater une certaine communauté d'usages religieux entre les Assyriens, les Phéniciens, les Phrygiens et les Égyptiens.

Malgré la forme caractéristique du cyprès et ses dimensions, sur le bas-relief cité du grand temple de Dendérah, MM. Jollois et Devilliers, à qui l'on doit la description de cet édifice, indiquent dans le texte, au lieu d'un jeune cyprès pyramidal, « une espèce de pomme de pin. » Un de nos savants confrères, M. Jomard, n'hésite pas à penser avec moi qu'il y a inadvertance dans une pareille désignation, et que le dessin qui accompagne la description du bas-relief ne laisse aucun doute sur l'intention qu'avait eue le sculpteur égyptien de représenter un jeune cyprès pyramidal. Or le dieu qui reçoit l'offrande de cet arbre est celui que Champollion le jeune appelle tantôt \emph{Ammon générateur} ou \emph{Mendès},\footnote{\emph{Panthéon égypt.} texte pour la pl. 4.} tantôt le grand dieu générateur, \emph{Ammon-Horus}.\footnote{\emph{Mémoires de l'Acad. des inscriptions et belles-lettres}, nouv. série, t. 15., 1\textsuperscript{re} partie, p. 112.} Il l'assimile à Pan ou à Priape, dont nous trouvons, chez les Grecs, le culte associé à celui de Vénus-Aphrodite, de même qu'à Dendérah le culte d'Ammon générateur se montre associé à celui de Vénus-Hâthôr.

Cette dernière association mystique nous est attestée, avec plus d'évidence encore, par deux stèles égyptiennes, qui mettent le fait hors de toute contestation. Champollion le jeune ne les a pas connues. L'une a été dessinée au Musée britannique et publiée par M. E. Prisse.\footnote{\emph{Revue archéologique}, troisième année, numéro du 15 février 1847 ; \emph{Monuments égyptiens}, pl. 37. --- La stèle que j'indique ici est reproduite dans mes \emph{Rech. sur Vénus} (atlas, pl. 14 F), d'après une réduction du dessin original de M. Prisse. Elle a depuis été répétée, en divers pays, dans plusieurs ouvrages ou dissertations archéologiques. M. Prisse (\emph{Monum. Égypt.} p. 7) cite deux antres stèles du même genre, mais plus petites, qui se conservent également au Musée britannique.} Elle est décrite, par M. Samuel Birch, dans le Catalogue des monuments égyptiens de ce musée,\footnote{\emph{Egyptian Room}, n° 308.} et provient du sanctuaire d'un temple situé en Nubie, au lieu appelé \emph{Abou-Sembil}. Par son style, elle appartient à la série des monuments qui remontent au temps où régnaient, en Égypte, les premiers rois de la 19\textsuperscript{e} dynastie. Deux scènes ou tableaux occupent la face antérieure de cette stèle ; elles y sont superposées l'une à l'autre. Dans la première, c'est-à-dire dans la partie supérieure du monument, on voit une déesse qu'au premier abord on pourrait croire entièrement nue, mais dont le corps, comme celui d'Isis, est, en réalité, couvert d'un vêtement long, très-collant et très-transparent. Elle est vue de face,\footnote{Cette disposition est rare sur les monuments égyptiens.} debout et les pieds posés sur le dos d'une lionne ou d'un lion qui, tourné de gauche à droite, lui sert et de monture et de piédestal. Dans la main droite, elle tient une tige de lotus ; cette tige, en s'enroulant autour de la main, forme un cercle\footnote{Ce cercle, que nous retrouverons sur la seconde stèle (pl. 11 ci-jointe), ne nous rappelle-t-il pas celui qui est une partie intégrante du sceptre de la Vénus assyrienne de Yazili-Kaïa, également placée debout sur le dos d'un lion ou d'une lionne ? (Voy. \emph{Recherches sur Vénus}, atlas, pl. 2.)} et porte une fleur épanouie, accompagnée de deux boutons, allusion manifeste à l'idée de la génération, au nombre trois, qui était consacré à Vénus,\footnote{Voy. \emph{Recherches sur Vénus}, p. 203, 204. --- La Vénus représentée sur la seconde stèle égyptienne (pl. 11 ci-jointe) porte à la main une seule fleur de lotus ; mais son bras gauche est orné d'un triple bracelet qui me semble reproduire l'allusion directe que fait au nombre trois le bouquet de trois fleurs de lotus placé dans la main de la déesse sur la première stèle (\emph{Recherches sur Vénus}, atlas, pl. 2.).} et à la triade dont elle faisait partie chez les Assyriens et les Phéniciens, comme chez les Égyptiens.\footnote{Vénus, en Orient et en Occident, était considérée comme une divinité triple, c'est-à-dire comme \emph{reine du ciel, reine des vivants} ou \emph{de la terre}, et \emph{reine des morts} ou \emph{des enfers} (voy. \emph{Ibid.} p. 40-44, 47-51, 71, 72, 74, 75, 97-100, 226, 227). C'est pourquoi le nombre trois lui était consacré, et pourquoi aussi on lui donnait deux parèdres ou assesseurs, qui formaient avec elle une triade particulière.} Ici, dans sa légende hiéroglyphique, elle reçoit un nom et une qualification que M. Prisse interprète ainsi : « \emph{Koun}, reine absolue du ciel. » M. le vicomte Emmanuel de Rougé, comme on le verra plus loin, dans une lettre pleine d'intérêt et de savoir, qu'il m'a fait l'honneur de m'écrire, n'admet cette lecture qu'avec quelque doute. Il hésite à lire \emph{Ken}\footnote{M. S. Birch lit \emph{Ken}.} plutôt que \emph{Kesch} le nom de la Dame du ciel\footnote{On voit comment cette qualification de \emph{Dame du ciel} nous ramène à \emph{Méléket-aschschamaïm} (Reine des cieux), à \textgreek{Ἥρα}, à \textgreek{Δέσποινα} (\emph{Domina}) et à \emph{Regina cœli}, noms ou qualifications qu'Astarté, Junon et Vénus, divinités identiques, recevaient chez les Hébreux, les Grecs et les Latins.} ; mais, à l'exemple de M. Prisse et de M. Birch, il assimile cette déesse à la Vénus assyrienne, nommée, sur d'autres monuments égyptiens, \emph{Hâthôr} ou \emph{Athyr}. Je n'avais pu me défendre d'établir une semblable identité au moment même où je connus la stèle du Musée britannique par le dessin que M. Prisse voulut bien m'en communiquer avant de le publier. En présence de ce dessin, ma mémoire me rappela et la Vénus assyrienne de Yazili-Kaïa, également montée debout sur une lionne ou un lion,\footnote{\emph{Rech. sur Vénus}, p. 126-130 ; atlas, pl. 2.} et une multitude de figures d'autres déesses, de dieux, de prêtresses et de prêtres que des monuments asiatiques\footnote{Parmi les nombreux cylindres que je pourrais citer, il faut distinguer surtout celui qui est reproduit sous le n° 12 de la pl. 4 de l'atlas de mes \emph{Rech. sur Vénus}. On y voit, debout sur une lionne ou une panthère, une prêtresse de Mylitta, coiffée, vêtue et armée comme devaient l'être certaines images qui, chez les Assyriens, représentaient la déesse elle-même. Une pierre gravée, de travail romain, que j'ai publiée dans le même ouvrage (pl. 19., n° 8), nous montre une Minerve panthée, également debout sur un lion, et accompagnée du symbole du taureau, comme la Vénus assyrienne de Yazili-Kaïa. J'ai publié aussi (\emph{ibid.} pl. 17., n°\textsuperscript{s} 9 et 9a, et n° 10) les dessins d'une amulette et d'un scarabée de terre émaillée, qui proviennent d'Égypte, et qui représentent chacun une divinité mâle, placée debout sur un lion. Ici l'imitation d'un type asiatique se décèle non moins évidemment que sur l'intaille romaine.} représentent placées aussi debout sur des lions, sur des taureaux ou sur d'autres animaux symboliques. Je ne pouvais avoir oublié que tel ne fut pas l'usage suivi par les sculpteurs ou les peintres d'Égypte, lorsqu'ils eurent à reproduire les images des dieux ou des personnages réputés d'origine égyptienne.\footnote{Cette remarque n'a point échappé à M. Chr. Walz ; il fait très-judicieusement observer (\emph{De Nemes. Græcor.} p. 9) que tout, dans la figure qui représente Vénus-Hâthôr sur la stèle du Musée britannique, est étranger à l'art égyptien.} A la gauche de la déesse de notre stèle est sculptée debout une divinité mâle, dont la barbe est taillée en pointe, comme celle des dieux assyriens ou phéniciens, particularité caractéristique, qui ne se montre jamais dans les images des dieux de l'Égypte. Le nom de cette divinité mâle est écrit en caractères hiéroglyphiques, et se lit, selon M. Prisse, \emph{Ranpo} ou \emph{Renpho} ; selon M. Birch et M. de Rougé, \emph{Renpou} ( ? \emph{Réseph}). Ce nom nous ramène dans l'Asie occidentale, tout comme la forme de la barbe du dieu qui le porte. A la droite de la déesse, on remarque un second dieu ; celui-ci est ithyphallique ; le premier éditeur de la stèle le nomme \emph{Amon-Ra, le Fécondateur}. Ce dieu est l'\emph{Ammon-Horus} de Champollion, qui, en le surnommant ainsi, ne connaissait pas l'étroite alliance que la stèle d'\emph{Abou-Sembil} établit entre \emph{Amon-Ra} et Vénus-Hâthôr. Nous allons retrouver ces deux divinités dans les mêmes rapports sur une seconde stèle, de plus ancien style, où M. de Rougé lit le nom du dieu ithyphallique : \emph{Min-Amon-Ra}. Ici, à ses côtés, est placé un emblème très-expressif de génération : une tige fleurie de lotus, plantée entre deux cyprès,\footnote{Cet emblème est reproduit sous le n° 2 de ma planche 10. Comme la triple fleur de lotus placée à la main de la déesse, sur la stèle du Musée britannique, il rappelle le nombre trois, qui était consacré à Vénus (\emph{Recherch. sur Vénus}, p. 203, 204), et la triade divine dont elle fait partie. On remarquera de plus que la fleur de lotus dépasse de beaucoup, en hauteur, les deux cyprès, de même que, sur la face antérieure du trône votif de la Déesse de Syrie (pl. 5., n° 1), le cyprès qui représente cette divinité domine le cyprès du soleil et le cyprès de la lune.} qu'on pourrait prendre pour les symboles asiatiques du soleil et de la lune. Leur sommet s'incline vers la déesse en signe d'adoration ; et c'est ainsi que nous verrons, dans le mémoire suivant, deux cyprès courber leur tête vers un autel et une urne funéraire, sur deux lampes romaines,\footnote{Pl. 19., n°\textsuperscript{s} 1 et 2.} et vers la croix du Christ, sur des tombeaux du moyen âge. Nous retrouvons deux cyprès dans la seconde scène sculptée sur la stèle que je décris ; elle en occupe la partie inférieure. Le sommet des deux arbres n'est pas incliné ; et, au lieu d'être plantés auprès d'une fleur de lotus, ils sont placés entre les trois pieds ou supports d'une table ou d'un autel\footnote{Voy. pl. 10., n° 1.} couvert d'offrandes et posé devant une divinité femelle, assise, coiffée du diadème d'Osiris qu'on appelait \emph{atew}, et armée d'une lance, comme Astarté sur les médailles asiatiques, comme Aphrodite à Lacédémone, en souvenir des fonctions belliqueuses et de l'hermaphroditisme attribués à un type primitif.\footnote{De même Mithra, chez les Perses, d'après ce type primitif, était une divinité androgyne, armée non-seulement d'une lance, mais de plusieurs autres armes qui sont détaillées et décrites avec le plus grand soin dans le \emph{Zend-Avesta} (t. 2., p. 211, 222, 223, et surtout p. 229 et 230).} Notre déesse armée est, de même que la déesse \emph{Ken} ou \emph{Kesch}, une autre forme de la Vénus assyrienne ; mais son culte, dans ce second tableau, n'est pas associé, comme dans le premier, au culte d'\emph{Amon-Ra} et de \emph{Renpou}. Son nom se lit \emph{Anta}\footnote{Selon la lecture de M. de Rougé et de M. Prisse. Ce dernier pense qu'on peut aussi lire \emph{Tanata} (avec le \emph{T}, article féminin), et considérer ce nom comme l'origine primordiale du mot grec \textgreek{θάνατος}, \emph{la mort}. Mais si cette lecture était admise, \emph{Tanata}, ce me semble, devrait plutôt nous rappeler la déesse \emph{Tanath} ou \emph{Taneth} des inscriptions puniques découvertes dans l'Afrique septentrionale, et la déesse \emph{Tanaïtis} ou \emph{Anaïtis} des auteurs occidentaux, l'une et l'autre identique avec la Vénus-\emph{Anahid} des Assyriens, qui, sans nul doute, était une divinité armée comme l'Astarté des Phéniciens, comme le Mithra des Perses, comme l'Aphrodite de Lacédémone, comme la Minerve des Grecs et des Romains, avec qui elle se confondait primitivement. Sur le nom d'\emph{Anta} et sur plusieurs autres particularités propres à l'unique stèle où il est écrit, on lira avec beaucoup d'intérêt les remarques contenues dans la lettre de M. de Rougé que je transcris plus loin.} ou \emph{Anata}\footnote{Selon la lecture de M. Birch (\emph{loc. cit.}). Ce savant, dont l'autorité en ces matières est fort grande, remarque, dans ses \emph{Observations on two Egyptian cartouches, recently found at Nimroud} (\emph{Transactions of the royal Society of litter.} 2d series, vol. 3., part 1, p. 161, 162), que c'est à partir des 18\textsuperscript{e} et 19\textsuperscript{e} dynasties que, sur les monuments égyptiens, on commence à trouver les divinités assyriennes Baal et Astarté, le dieu phénicien Renpou ou Réseph, la déesse mohabite Ken, et la déesse arménienne et assyrienne Anata ou Anaïtis. --- M. le Dr. R. Lepsius, dans un savant mémoire, intitulé : \emph{Über den erst. ægypt. Götterkreis} (p. 30 du tirage à part), place la déesse \emph{Hathur} (\textgreek{Ἅθωρ, Ἀφροδίτη}) au septième ou au huitième rang des dieux de Memphis, c'est-à-dire la dernière, et au huitième rang des dieux de Thèbes, c'est-à-dire l'avant-dernière. Ici, il l'associe à \emph{Hur} (Horus).} dans la légende hiéroglyphique gravée auprès d'elle ; il y est accompagné des qualifications de \emph{Dame du ciel} et de \emph{Régente des dieux}. Un acte d'adoration, avec offrande, s'accomplit devant l'image de la déesse par trois personnages : un chef de famille dont le nom est en partie effacé, sa sœur et son fils. Celui-ci porte, de la main droite, une tige fleurie de lotus, ce qui nous ramène à la fleur de lotus placée dans la main de la déesse \emph{Ken}, dans le premier tableau, et à la fleur de lotus qui fait partie de l'emblème sculpté auprès d'elle, derrière le dieu ithyphallique. De la main gauche, le jeune homme paraît tenir une colombe, offrande dont la Vénus asiatique revendique l'usage primitif.

La seconde stèle que j'ai annoncée appartient au musée du Louvre. Elle représente, comme la première, une divinité femelle, vue de face, debout sur un lion, et placée entre deux assesseurs ou parèdres divins, \emph{Amon-Ra} ithyphallique et \emph{Renpou}.\footnote{Pl. 11. --- Le corps de la déesse, le disque placé entre les deux cornes de vache implantées sur sa tête, et la crinière du lion qui lui sert de monture, sont peints en rouge pâle. Cette particularité n'est pas sans intérêt. Le cuivre fut consacré à Vénus, parce qu'il est rouge ; et il ne faut pas oublier que cette couleur était, en général, celle des dieux générateurs asiatiques, et, en particulier, celle de Mithra et de Bacchus. Dans le \emph{Zend-Avesta}, le feu créateur est appelé \emph{feu rouge, fils d'Ormuzd} ; et l'arme de Mithra, la harpé primitive, est l'\emph{oreille de cuivre rouge}. Sur les monuments mithriaques d'époque romaine, la tiare de ce dieu est parfois peinte en rouge, comme nous le montre un fragment de bas-relief qui se conserve au musée Chiaramonti. (\emph{Rech. sur Mithra}, atlas, pl. 78., n° 1.)} Ici nous retrouvons, à côté du premier de ces dieux, une tige fleurie de lotus, plantée sur un autel, entre deux cyprès de l'espèce pyramidale. La déesse, dans sa légende hiéroglyphique, porte le nom d'\emph{Atesch} ou ( ? ) \emph{Satesch}, mais ne s'identifie pas moins que \emph{Ken} et \emph{Anta} ou \emph{Anata} avec Vénus-Hâthôr, et par conséquent avec Vénus-Mylitta et Vénus-Astarté. Elle est nue ; toutefois une ceinture est attachée au-dessus de ses hanches. À mes yeux, cette ceinture est une des particularités qui ne contribuent pas peu à nous révéler l'origine asiatique de la déesse ; car personne n'ignore que Mylitta, Mithra et leurs sectateurs portaient une ceinture nommée \emph{kosti} dans le \emph{Zend-Avesta}. C'est le \textgreek{κεστός} des Grecs et le \emph{cestus} des Latins, qu'avait seule le droit de ceindre Vénus-Uranie,\footnote{\emph{De Dea syria}, § 32.} identique avec Mylitta. D'autre part, la tête de la déesse est ici surmontée du disque de la lune placé entre deux grandes cornes de vache.\footnote{Cette coiffure avait donné lieu à Champollion (\emph{Mus. Charles 10.}, p. 24, n° 433) de présumer que la stèle représente la déesse \emph{Ioh} ou la \emph{Lune}.} Une telle coiffure nous rappelle tout à la fois les monuments asiatiques qui nous montrent Astarté tantôt sous la forme d'une vache allaitant son veau,\footnote{Voyez ci-dessus, p. 59 et 60, note 4.} tantôt sous celle d'une femme à tête de taureau ou de vache\footnote{\emph{Recherches sur Vénus}, p. 135, 136 ; atlas, pl. 3., n° 1.} ; et les images égyptiennes qui représentent Hâthôr sous les traits d'une femme à tête de vache, dont le front est orné d'un disque posé entre les deux cornes,\footnote{Champollion le jeune, \emph{Panth. égypt.} pl. 18. --- La figure que j'indique ici est reproduite sous les n°\textsuperscript{s} 5 et 5a de la pl. 14 D de l'atlas de mes \emph{Rech. sur Vénus}.} ou sous la forme d'une femme à tête humaine, enveloppée dans la peau d'une vache dont la tête, pareillement ornée d'un disque, vient s'ajuster sur la tête de la déesse, de manière à former une espèce de diadème.\footnote{\emph{Recherch. sur Vénus}, atlas, pl. 14 E.} Ainsi, remarquons-le bien, Vénus-Atesch,\footnote{M. Prisse (\emph{loc. cit.}) lit ce nom \emph{Atsch} ou \emph{Atosch}. Remarquons qu'une princesse de Perse s'appelait \emph{Atossa}.} sur la stèle du Louvre, se présente à nos regards comme une divinité solaire et lunaire à la fois : solaire par le lion qui lui sert de monture ou de piédestal, lunaire par le disque placé entre ses cornes de vache. Ces deux titres réunis nous avertissent que primitivement elle était androgyne. C'est aussi en souvenir de ce dernier caractère que le grand bas-relief de Yazili-Kaïa\footnote{\emph{Rech. sur Vénus}, atlas, pl. 2. M. Texier, \emph{Descr. de l'Asie Min.} 1\textsuperscript{re} part. pl. 78.} représente la Vénus assyrienne ou phénicienne debout sur le lion solaire et accompagnée du taureau lunaire ; elle porte de plus un sceptre surmonté du croissant de la lune. Sur la tablette assyrienne de feu le baron Roger,\footnote{\emph{Rech. sur Vénus}, p. 130-132 ; atlas, pl. 17., n°\textsuperscript{s} 1, 1a, 1b.} Mylitta, debout sur un taureau, se montre avec une tête de lion, au lieu d'une tête humaine.\footnote{On voit comment, par des procédés variés, l'art hiératique, sans s'écarter des bases fondamentales du langage symbolique écrit, parlé ou figuré, parvenait à exprimer une même idée. Aux trois exemples que je viens d'emprunter à la série des monuments figurés de Vénus, ne faut-il pas ajouter ceux que nous fournissent la stèle citée du Musée britannique et un cône asiatique de mon ancienne collection ? L'hermaphroditisme de la déesse et son caractère à la fois solaire et lunaire, ne sont-ils pas exprimés, sur la stèle (\emph{Rech. sur Vénus}, pl. 14 F), par le lion, par le symbole du principe humide placé dans une des mains de \emph{Ken}, et par le serpent mâle et le serpent femelle réunis dans l'autre main ; sur le cône (\emph{ibid.} pl. 1., n° 1), par deux serpents surmontés, l'un, des rayons du soleil, l'autre, du croissant de la lune, et placés dans les mains d'une divinité moitié homme, moitié femme ?}

La petite stèle qui me permet de faire ces divers rapprochements remonte à une époque plus ancienne que la stèle citée du Musée britannique. M. de Rougé\footnote{\emph{Notice des monum. égyptiens du Louvre}, p. 87, n° C, 86 ; 2\textsuperscript{e} édit. (Paris, 1852).} croit y reconnaître le style du siècle de Toutmès 3, ou plutôt du dernier règne des rois de la 18\textsuperscript{e} dynastie. Elle est de pierre calcaire, taillée en forme de porte et sculptée sur ses deux faces et sur les tranches ou épaisseurs. Sa provenance n'est pas connue. Le dessin que j'en donne ici,\footnote{Pl. 11.} et qui la reproduit avec les dimensions de l'original, est le premier, si je ne me trompe, qu'on en ait publié jusqu'à ce jour. Il a été exécuté sous les yeux de M. de Rougé, qui déjà avait décrit ce monument, mais d'une manière succincte, dans le Catalogue du musée confié à ses soins éclairés.\footnote{\emph{Notice}, etc. \emph{loc. cit.}} A ma prière, il a bien voulu, dans la lettre qu'on va lire, décrire plus amplement cette précieuse stèle et me communiquer les savantes observations qu'elle lui a donné lieu de faire :

\begin{center}
« Monsieur, »
\end{center}

\paragraph{}
« J'ai réuni, comme vous me l'aviez demandé, mes notes sur la déesse représentée debout sur un lion dans la stèle (A, 433\footnote{Ancien numéro ; elle est décrite dans ma notice sous le n° C, 86.}) du musée du Louvre ; en voici le résumé : »

« Cette déesse, que je crois d'origine asiatique et seulement implantée en Égypte, à la suite des expéditions de la 18\textsuperscript{e} dynastie, nous est connue par trois monuments : 1° la stèle précitée, du Louvre ; 2° la stèle du \emph{British Museum}, publiée par M. Prisse dans son \emph{Choix de monuments}, pl. 37., et par vous, Monsieur, d'après lui, dans vos \emph{Recherches sur Vénus} (atlas, pl. 2.) ; 3° une stèle du musée de Turin, portant le n° 89. »

« La stèle du Louvre est bien conservée ; une partie des couleurs qui rehaussaient la sculpture est encore visible ; la couleur de chair, \emph{rose d'un ton clair}, qui couvrait le corps de la déesse est surtout remarquable. »

« Au centre de la scène, on voit une divinité féminine représentée de face, entièrement nue et debout sur un \emph{lion passant}, c'est-à-dire dessiné de profil et dans l'attitude d'une marche tranquille. Une légère ceinture était peinte au-dessus du pubis ; la couleur n'a pas tenu. La coiffure est caractéristique : les cheveux, rassemblés en deux grosses masses, se recourbent à droite et à gauche du cou en deux grosses boucles, qui ressemblent à des volutes retournées. Cette coiffure appartient, dans le panthéon égyptien, à la déesse Hâthôr. Il en est de même du disque soutenu par deux cornes en forme de croissant, qui complète la coiffure de la déesse. Elle tient de la main droite un bouquet composé d'une fleur de lotus et de deux boutons ; sa main gauche soutient un serpent. Auprès de la tête on lit la légende : »

\begin{center}
$\hieroAAAG\:\hieroAAAH\:\hieroAAAI\:\:\:\:\:\:\:\:\:\:\hieroAAAJ\:\:\:\:\:\:\:\:\:\:\hieroAAAK\:\hieroAAAL\:\:\:\:\:\:\:\:\:\:\hieroAAAM\:\hieroAAAM\:\hieroAAAM\:\:\:\:\:\:\:\:\:\:\hieroAAAN$
\end{center}

\begin{center}
« Sa TeSch NeV-t Pe HeNT NeTeR-ou NiV. »
\end{center}

\paragraph{}
« Atesch ou Satesch ( ? ), maîtresse du ciel, régente de tous les dieux. »

« Le nom de la déesse est très-important ; c'est pourquoi j'ajouterai ici que j'ai trouvé plusieurs variantes qui me paraissent éclaircir la valeur du premier caractère $\hieroAAAG$, jusqu'ici douteux. Il y a, par exemple, dans le Rituel funéraire, une série de personnages qui se retrouve deux fois (chapitres 144 et 147) ; le second porte le nom de \emph{Sati-ha}. Il est écrit dans le premier endroit $\hieroAAAO\:\hieroAAAG\:\hieroAAAP\:\hieroAAAQ\:\hieroAAAR$, et dans le second $\hieroAAAO\:\hieroAAAS\:\hieroAAAR$ ; il en résulte que, suivant toute apparence, le signe $\hieroAAAG$ est syllabique et se prononçait \emph{sat}. Les signes syllabiques sont, à volonté, escortés de leurs compléments phonétiques, comme ici $\hieroAAAG$ de \emph{s} $\hieroAAAO$ et \emph{t} $\hieroAAAT$ ; cependant, comme on ajoute quelquefois aux radicaux une \emph{s} préfixe (causative), il serait possible que la valeur de $\hieroAAAG$ fût seulement \emph{at}.\footnote{J'ai trouvé, depuis que ceci est écrit, des variantes qui semblent devoir faire préférer la lecture \emph{at, Atesch}.} Je reviendrai sur le nom \emph{Satesch} quand j'aurai décrit toute la stèle. »

« A droite de la déesse, figure Ammon dans sa forme ithyphallique, accompagné des attributs ordinaires de cette forme. Sa légende n'est pas tout à fait celle qu'on trouve habituellement : »

\begin{center}
$\hieroAAAU\:\hieroAAAV\:\:\:\:\:\:\:\hieroAAAW\:\hieroAAAX\:\hieroAAAY\:\:\:\:\:\:\:\:\hieroAAAZ\:\:\:\:\:\:\:\:\hieroAABA\:\:\:\:\:\:\:\:\hieroAABB\:\hieroAABC\:\hieroAABD\:\:\:\:\:\:\:\:\hieroAABE\:\:\:\:\:\:\:\:\hieroAAAM\:\:\:\:\:\:\:\:\hieroAABF\:\hieroAABG$
\end{center}

\begin{center}
« MiN\footnote{La véritable prononciation du nom spécial de la forme ithyphallique d'Ammon est encore l'objet de nombreuses contestations.} ? AMeN-RA Nåå PeHPeH HeR HēSe-w OUeR-t NeTeR Nåå. »
\end{center}

\paragraph{}
« Min-Amon-ra, le plus grand des vaillants, le supérieur de sa grande demeure, dieu grand. »

« Remarquez, je vous prie, cette qualification : \emph{le plus grand des vaillants}, ou peut-être mot à mot : \emph{le grand de la vaillance} ; elle établit une analogie entre ce dieu et le second parèdre de la déesse \emph{Satesch}. »

« Celui-ci est représenté debout, de profil ; au premier abord, on remarque sa coiffure, composée d'un diadème sur lequel se détache une tête de gazelle, et sa barbe, exactement semblable à celle que les artistes égyptiens donnent partout aux Asiatiques. Aucune divinité égyptienne n'en porte une semblable. Sa main droite tient une lance, et sa main gauche, la croix ansée $\hieroAABH$, signe de la vie divine. Sa légende le nomme \emph{Renpou} : »

\begin{center}
$\hieroAABI\:\hieroAABJ\:\:\:\:\:\:\:\hieroAAAM\:\:\:\:\:\:\:\hieroAABF\:\hieroAABG\:\:\:\:\:\:\:\hieroAABK\:\:\:\:\:\:\:\hieroAABL\:\:\:\:\:\:\:\hieroAAAK\:\hieroAABM\:\hieroAAAK\:\:\:\:\:\:\:\hieroAABN\:\hieroAABO\:\:\:\:\:\:\:\hieroAABP\:\:\:\:\:\:\:\hieroAABK\:\:\:\:\:\:\:\hieroAABQ\:\hieroAABR\:\hieroAABS\:\:\:\:\:\:\:\hieroAABT\:\hieroAABU\:\:\:\:\:\:\:\hieroAABV\:\:\:\:\:\:\:\hieroAABW$
\end{center}

\begin{center}
« ReN POU NeTeR Nåå NeV eR NeHeH HiK Té TeN NeV PeHPeH eM CheN-nou PA-ou NeTeR-ou.\footnote{Je ne puis entrer ici dans des détails philologiques ; je me bornerai à remarquer, 1° que dans le mot \emph{peh-ti} ou \emph{pehpeh}, $\hieroAACU$ sert de phonétique (\textcoptic{ⲡⲁϩⲟⲩ} « pars posterior » ) et $\hieroAACV\:\hieroAACV$ de déterminatif ; 2\textsuperscript{e} que $\hieroAABV\:\hieroAABW$ doit se lire \emph{paou} et désigne un ensemble de dieux, ce que j'expliquerai ailleurs.} »
\end{center}

\paragraph{}
« Renpou, dieu grand, seigneur des siècles, roi éternel, seigneur de la vaillance parmi les dieux (ou dans le séjour des dieux). »

« Soit qu'on ait considéré ici ce dieu comme accessoire, soit qu'on l'ait identifié avec Ammon, il est à remarquer qu'il n'est pas rappelé dans les actes d'adoration qui remplissent le reste de la stèle ; ceux-ci s'adressent uniquement à \emph{Satesch} et à Amon-ra. »

« Le bas de la stèle, les deux tranches et toute la partie postérieure sont couverts par des actes d'adoration que prononcent un magistrat nommé \emph{Houi}, $\hieroAABX\:\hieroAAAW\:\hieroAAAW\:\hieroAABY$, et toute sa famille. Ces formules contiennent quelques particularités dignes de remarque, et d'abord plusieurs titres y sont donnés à la déesse ; entre ceux déjà signalés de \emph{dame du ciel} et de \emph{régente des dieux}, on y lit : 1° la qualification de $\hieroAABZ\:\hieroAACA$ « régente des deux mondes, » ce qui s'applique ordinairement à l'Égypte, mais peut aussi s'entendre du monde en général ; 2° $\hieroAAAM\:\hieroAACB\:\hieroAACC\:\hieroAACD$, « déesse nourrissant le monde. » Cette traduction, donnée par Champollion, me laisse des doutes ; c'est d'ailleurs un titre qui appartient ordinairement à Amon-ra, comme dieu suprême. 3° $\hieroAACE\:\hieroAAAW\:\hieroAACF\:\hieroAACG\:\hieroAAAY$ \emph{Scherå-t our Rå}, « fille aînée du Soleil ; » ce qui est un nouveau trait de ressemblance avec Hâthôr. 4° $\hieroAACH\:\hieroAACI\:\hieroAACG\:\hieroAAAY\:\hieroAACJ\:\hieroAACK\:\hieroAACL\:\hieroAACM\:\hieroAACJ\:\hieroAAAV$, \emph{meri-t en Rå, out'å en Toum}, « l'aimée de Ra, l'œil de Toum ; » cette qualification se rapporte, comme vous le savez, aux deux principales formes de l'astre, considéré comme soleil levant et comme soleil couchant. La personne qui donne ce dernier titre à Satesch indique de plus le caractère infernal de la déesse, en lui demandant \emph{de lui faire voir ses beautés dans la région inférieure} : »

\begin{center}
$\hieroAACN\:\:\:\:\:\hieroAACO\:\hieroAACP\:\hieroAACP\:\:\:\:\:\:\hieroAABY\:\:\:\:\:\:\hieroAACQ\:\hieroAACQ\:\hieroAACQ\:\hieroAACR\:\:\:\:\:\:\hieroAACS\:\:\:\:\:\:\hieroAACT$
\end{center}

\begin{center}
« TÅ-t MÅÅ-å NoWRe-ou-t åM NeTeR KeR. »
\end{center}

\paragraph{}
Mot à mot : « Da videam decora tua in orco. »

« Ce caractère funéraire est encore plus tranché dans l'invocation du chef de la famille, $\hieroAABX\:\hieroAAAW\:\hieroAAAW\:\hieroAABY$, \emph{Houi} ; voici ce qu'il demande à Satesch : »

\begin{center}
$\hieroAACW\:\:\:\:\:\:\hieroAACX\:\:\:\:\:\:\hieroAACY\:\:\:\hieroAACQ\:\hieroAACZ\:\:\:\:\:\:\hieroAAAR\:\hieroAADA\:\:\:\:\:\:\hieroAAAW\:\hieroAADB$\footnote{Ce mot curieux, déterminé par le vieillard courbé sur son bâton, $\hieroAADB$, a été souvent confondu avec \emph{our} « grand, » $\hieroAADN$, où l'homme marche droit, tenant le bâton de commandement. La phonétique, écrit complétement, est $\hieroAACP\:\hieroAAAW\:\hieroAACJ$ ååou. (Conf. \textcoptic{ⲟⲩⲱ}, « finire, » \textcoptic{ⲉⲩⲁⲟⲩⲱ}, « deficientes. » )}
\end{center}

\begin{center}
« Tå-s KRaS NoWRe-t Hå CheT Ååou. »
\end{center}

\paragraph{}
« Qu'elle donne une sépulture bonne après la vieillesse. »

« Tel est, Monsieur, l'ensemble des qualifications de la déesse ; mais je veux aussi vous faire remarquer celles que prennent les femmes qui l'invoquent sur ce monument. La première se nomme $\hieroAADC\:\hieroAADD\:\hieroAACP\:\hieroAADE\:\hieroAADF\:\hieroAADG$, \emph{Tacharou}. Ce nom, qui signifie littéralement « la Syrienne, » n'est pas rare sous les 18\textsuperscript{e} et 19\textsuperscript{e} dynasties ; il indique des tendances qui ne peuvent vous échapper. Cette femme prend le titre de : \emph{odiste de sa maîtresse}, $\hieroAADH\:\hieroAAAW\:\hieroAAAW\:\hieroAADI\:\hieroAADJ$ (\emph{hosi en hent-s}) ; cette indication me porte à penser que Satesch avait, à l'époque de notre stèle, obtenu un culte public dans quelque partie de l'Égypte.\footnote{Probablement à Thèbes, puisque le dieu Amon-ra est qualifié sur ce monument : \emph{supérieur de sa grande demeure}.} Cette conjecture se confirme par la qualification que prennent les autres femmes. Elles se disent toutes $\hieroAADK\:\hieroAADL$, \emph{VeK-t-s} « sa servante. » Le fils de Houi se qualifie seul $\hieroAADK\:\hieroAADM\:\hieroAAAW\:\hieroAAAX$, \emph{VeK eN AMouN} « serviteur d'Amon. » »

« Je vous ai dit, Monsieur, que je regardais cette stèle comme appartenant à la 18\textsuperscript{e} dynastie ; les noms propres, le style et le tracé des hiéroglyphes me rappellent l'époque de Thoutmès 3. Il faut remarquer néanmoins que le nom d'Amon n'a pas subi le martelage qui défigure ordinairement les stèles de cette époque ; la petitesse du monument aurait pu facilement le soustraire à cet outrage officiel. J'inclinerais néanmoins, à cause de cette circonstance, à le placer tout à la fin de la 18\textsuperscript{e} dynastie. »

« La stèle du \emph{British Museum} me semblerait un peu plus récente, quoique son style se rapproche beaucoup du même type. Ici la déesse a perdu sa nudité primitive ; elle est enveloppée, comme Isis ou Hâthôr, d'un vêtement long et collant ; sa chevelure est toujours enveloppée de la même façon ; mais elle n'a plus son disque sur la tête. Le lion, à l'encolure fortement caractérisée, du monument du Louvre, est devenu une lionne. La déesse, quoique bien reconnaissable, a changé de nom ; sa légende se lit : $\hieroAADO\:\hieroAADP\:\hieroAAAI\:\hieroAAAJ$, \emph{Ken} ( ? ), « dame du ciel. » C'est ce nom que M. Prisse a rapproché des mots coptes \textcoptic{ⲕⲉⲛ}, \textcoptic{ⲕⲟⲩⲛ} « sinus, pudenda. » (Il faudrait le retrouver une seconde fois pour bien établir la lecture ; la forme un peu large du second signe $\hieroAADQ$ laisse des doutes entre $\hieroAADO$, \emph{Ken}, et $\hieroAADR$, \emph{Kesch.}) Les dieux parèdres, \emph{Amon} ithyphallique et \emph{Renpou}, ont les mêmes caractères que sur la stèle du Louvre ; il faut seulement observer qu'ici la déesse présente à \emph{Renpou} deux serpents au lieu d'un. Ce monument est malheureusement moins riche en légendes que celui du Louvre, mais il est complété par une seconde scène, extrêmement curieuse. Une déesse, coiffée du diadème ordinaire d'Osiris, nommé \emph{atew}, est assise, vêtue à l'égyptienne ; elle tient de la main gauche une lance, comme \emph{Renpou}, et de la main droite la hache de combat. Cette belliqueuse divinité porte le nom d'\emph{Anta}, $\hieroAADS\:\hieroAADT\:\hieroAAAW\:\hieroAAAI$, auquel les analogues asiatiques ne manquent pas. Ses qualifications $\hieroAAAC\:\hieroAADU\:\hieroAAAK\:\hieroAABZ\:\hieroAAAM\:\hieroAAAM\:\hieroAAAM$, « dame du ciel et régente des dieux, » l'identifient avec la divinité que nous venons d'étudier. Les personnages qui sont devant elle lui adressent leur hommage dans des termes qui ne nous apprennent rien de nouveau. »

« Dans la stèle de Turin, la déesse est nue, le pubis noir ; mais le nom et la coiffure sont comme sur la stèle de Paris ; elle est accompagnée des deux mêmes dieux. Ce monument est de la 19\textsuperscript{e} dynastie et d'origine thébaine.\footnote{M. Prisse cite encore au \emph{British Museum} une stèle très-effacée et un fragment où se trouve la même déesse. Ces monuments n'ont pas été publiés. Le fragment est précieux, en ce qu'il porte au revers la figure de Ramsès 2, ce qui enlève toute incertitude sur son époque.} »

« Si le nom douteux de \emph{Ken} ne nous fournit que peu de renseignements sur la patrie primitive de notre déesse, il n'en est pas de même heureusement pour les noms d'\emph{Anta} et de \emph{Satesch} ou \emph{Atesch}. Elle devient ici la déesse éponyme de deux villes qui figurent parmi les conquêtes de la 18\textsuperscript{e} et de la 19\textsuperscript{e} dynastie, et qui nous conduisent directement en Asie. Vous savez, Monsieur, tout ce qui reste à faire pour que nous puissions être guidés un peu plus nettement dans la géographie de l'Asie antique. C'est pourquoi, sans vous proposer une identification positive pour la ville nommée \emph{Atesch} ou \emph{Satesch}, $\hieroAAAG\:\hieroAAAH\:\hieroAADV$, qui est citée bien souvent dans les papyrus et les monuments, je me contenterai de vous indiquer ici les principaux renseignements qui peuvent aider à chercher sa position : 1° sur la grande inscription des campagnes de Toutmès 3, transportée de Karnak au Louvre, on lit (lig. 7) que ce roi, se dirigeant vers le pays des \emph{Rotennou}, s'approcha de \emph{Satesch}, la pilla et la rasa ; 2° Séti 1\textsuperscript{er}, dans son expédition d'Asie, la désigne comme appartenant au pays d'\emph{Amara} (Rosell. \emph{M. R.} pl. 53.), que M. Birch rapproche des Amorrhéens. Ce savant a fait remarquer aussi que notre ville était en rapport fréquent avec les Chétas. Elle faisait probablement partie de leur confédération,\footnote{Je crois que c'est cette ville qui est désignée dans les papyrus historiques sous le nom métaphorique de \emph{épine dorsale des Chétas}.} où il faut, sans aucun doute, reconnaître les \emph{fils de Chet} ou \emph{Chittim} de la Bible, qui nomme même tous les rois de la Palestine \texthebrew{מַלְכֵי הַחִתִּים} « les rois des Chittis » (2. \emph{Rois}, 7). D'un autre côté, Satesch est très-fréquemment mentionnée, ce qui doit faire conjecturer qu'elle se trouvait sur la route ordinaire des expéditions dirigées contre la Mésopotamie. On la trouve aussi figurée dans une sorte de plan ou vue perspective de la marche de l'armée égyptienne, qui a été publié par Rosellini (\emph{M. R.} pl. 91.). »

« Le nom d'\emph{Anta} se retrouve tout aussi clairement dans la ville nommée Båitå-åntå, c'est-à-dire « la demeure d'Anta. » Cette ville figure dans les conquêtes du grand Ramsès et de Séti 1\textsuperscript{er}, son père. La composition du mot nous conduirait directement en Palestine. Les planches, assez fautives, de Burton (\emph{Excerpta}, 16.) permettent cependant de reconnaître qu'on a conservé, dans le nom de cette ville, le déterminatif \emph{dieu}, $\hieroAAAI$, au nom d'\emph{Anta}, $\hieroAADW\:\hieroAADX$ et le déterminatif $\hieroAADY$, \emph{demeure}, au mot båïtå, \texthebrew{בית} ; de sorte $\hieroAAAW\:\hieroAAAW\:\hieroAADT\:\hieroAAAW$ que le sens, \emph{demeure de la déesse Anta}, qui ressort du nom $\hieroAADZ$ de cette ville, était bien compris des Égyptiens eux-$\hieroAADT\:\hieroAAAW\:\hieroAAAI$ mêmes. La Bible nous présente le nom parfaitement identique \emph{Beit-anat}, \texthebrew{בֵית עֲנָת}, dans le territoire de Nephthali (\emph{Jos.} 19., 38). Ce nom, comme celui de \emph{Bet-schems}, a pu être appliqué à plusieurs villes, puisqu'il est constant qu'\emph{Anat} ou \emph{Anta} est le nom d'une grande divinité du pays ; mais l'indication du livre de Josué n'en devra pas moins être mise en première ligne, quand on voudra déterminer de plus près la position de nos deux villes. »

« Enfin, Monsieur, je retrouve la déesse \emph{Anta}, d'une manière assez inattendue, dans le nom d'une princesse à qui le grand Ramsès accorda un rang élevé, et qui paraît avoir rempli, pendant un certain temps, les fonctions de régente. Elle a été signalée par Champollion, à Ipsamboul, sous le nom de \emph{Batianti}. Ce nom propre se lit exactement comme celui de la ville que je viens d'étudier, \emph{Baïta-Anta} : $\hieroAAEA$. Seulement le mot \emph{baït} ne porte plus le déterminatif \emph{habitation}, $\hieroAADY$. Aussi je ne saurais dire si la princesse avait reçu le nom de la ville par souvenir de la conquête, ou si le nom divin \emph{Anta} avait été combiné avec quelque autre élément sémitique, tel que \texthebrew{בַּת}, « fille.\footnote{On trouve une variante écrite \emph{benta-anta}, qui peut se rapporter à la prononciation arabe \emph{bent} pour le mot \emph{fille}.} » \emph{Bat-anat} serait un nom féminin des plus réguliers. Quoi qu'il en soit, ce nom est une médaille de plus qui nous atteste l'influence des idées religieuses de l'Asie à la cour de Séthos 1\textsuperscript{er} et de Ramsès 2, et vous savez déjà, Monsieur, combien de faits curieux se rattachent à cette influence et y trouvent leur explication. »

« J'ai l'honneur de vous assurer de mes sentiments les plus dévoués. »

\begin{center}
« Vicomte EMM. DE ROUGÉ. »

« Paris, le 6 juin 1852. »
\end{center}

\paragraph{}
Les remarques qu'on vient de lire et celles dont je les avais fait précéder servent à confirmer, par le témoignage authentique de trois stèles égyptiennes, d'ancien style, l'identité de la Vénus des Égyptiens avec la Vénus des Assyriens et des Phéniciens. De plus, ces deux précieux monuments nous montrent que, chez les Égyptiens, dès une époque reculée, le symbole du cyprès pyramidal, arbre vert qui, je le répète, est originaire de l'Asie occidentale et non de l'Égypte, se trouve intimement lié au culte de Vénus-Hâthôr, représentée sous deux formes et avec trois noms différents, et au culte de deux divinités mâles, qui servent d'assesseurs ou de parèdres à la déesse, et dont l'un paraît être aussi d'origine asiatique ; l'autre est ithyphallique, comme Hermès ou Mercure, dont nous voyons, en Orient et en Occident, les images accompagnées parfois du même arbre symbolique.\footnote{Ci-dessus, p. 25-27, 108-111.}

Sur un des bas-reliefs qui, à Thèbes, décorent le plafond d'une des salles du palais de Rhamsès 2,\footnote{Ce bas-relief, réduit d'après un dessin rapporté de Thèbes par Champollion le jeune, a été publié par M. Biot, en tête des planches qui accompagnent son beau \emph{Mémoire sur l'année vague des Égyptiens}, inséré dans le tome 13 des \emph{Mémoires de l'Acad. roy. des sciences}, p. 553-707.} nous retrouvons deux cyprès pyramidaux. Ce bas-relief est un tableau des douze mois de l'année personnifiés, et là, les deux arbres sont placés au-dessus d'un autel ou d'un édicule, tout auprès d'un personnage ithyphallique,\footnote{Voyez ce groupe dans la réduction des dessins de Champollion, publiée, par M. Biot, sous le n° 3 de la planche du mémoire cité dans la note précédente. --- Le n° 3 de la planche 10, que je joins ici, reproduit seulement les deux cyprès et l'autel placés derrière le dieu ithyphallique.} qui représente le mois appelé \emph{tôbi}, ou le dieu générateur qui préside à ce mois et que Champollion le jeune désigne sous le nom d'\emph{Ammon-Horus}.\footnote{Mémoire cité, p. 111, 112.} Or c'est dans le mois de \emph{tôbi} que tombe, assure-t-on,\footnote{M. Biot, mémoire cité, p. 622, 623, 632.} l'équinoxe vernal ; ou, comme le dit le même savant,\footnote{Champollion le jeune, mémoire cité, p. 112.} \emph{tôbi} est le premier mois des récoltes.

A Edfou, dans un temple couvert de sculptures, Champollion dessina lui-même\footnote{Voyez le mémoire cité de M. Biot, p. 618.} deux tableaux où sont aussi figurés les douze mois personnifiés.\footnote{\emph{Ibid.} pl. 3., fig. 1 et 2.} Ici les idées de génération, de vie et de reproduction qui s'attachent à l'idée du printemps, ne sont point aussi énergiquement exprimées qu'elles le sont dans le tableau sculpté au Rhamesséum de Thèbes. Le personnage qui, à Edfou, représente le mois de \emph{tôbi}, ou le dieu générateur qui préside à ce mois, n'est pas ithyphallique ; mais il porte, de la main droite, un jeune cyprès pyramidal, symbole de vie, et, de la main gauche, une croix ansée, autre symbole de vie. De plus, on remarque un arbre de même espèce parmi les signes hiéroglyphiques dont se compose la légende gravée auprès du mois de \emph{tôbi} ainsi personnifié. Cette légende est absolument la même dans les deux tableaux d'Edfou. On peut s'en convaincre en jetant les yeux sur les n°\textsuperscript{s} 4 et 5 de la planche 10 ci-jointe, qui reproduisent les deux images du mois de \emph{tôbi} d'après les dessins de Champollion, réduits et publiés par M. Biot.\footnote{Mém. cité, pl. 3., fig. 1 et 2. --- J'ai voulu recourir aux dessins originaux de Champollion, qui m'eussent permis de reproduire moins réduites les deux images du mois de \emph{tôbi} tirées des tableaux d'Edfou, et l'emblème tiré du tableau du Rhamesséum de Thèbes. Ces dessins se sont égarés, et, jusqu'à présent, il a été impossible de les retrouver. La perte en serait d'autant plus regrettable, que la représentation de sujets astronomiques ou de calendriers est rare sur les monuments égyptiens, et que, par une négligence ou un oubli dont on a peine à se rendre compte, le tableau du Rhamesséum de Thèbes et les deux tableaux d'Edfou, si importants à étudier, n'ont été publiés ni dans le grand ouvrage de feu Champollion, ni dans celui de feu Rosellini. Espérons que les publications de la Commission prussienne, qui a visité avec tant de soin les monuments antiques de l'Égypte, combleront une si fâcheuse lacune.}

Le premier de ces deux savants académiciens, commentant les tableaux des mois sculptés à Edfou et au Rhamesséum de Thèbes, commet deux erreurs, je le dis à regret. Il veut\footnote{\emph{Mém. de l'Acad. des inscr.} nouvelle série, t. 15., 1\textsuperscript{re} partie, p. 111, 112.} qu'à Edfou le dieu qui représente le mois de \emph{tôbi} tienne dans la main « un \emph{chaton de palmier}. » « Ce personnage, ajoute-t-il, porte le nom de \textcoptic{ϣϥⲧⲃⲁ}, \emph{Schftba}, dans la composition duquel entre, en effet, le mot \textcoptic{ⲃⲁ}, nom du \emph{chaton de palmier} dans les textes hiéroglyphiques, lequel reçoit ici, pour déterminatif, l'image de ce même chaton. Le tableau du Rhamesséum, poursuit Champollion, consacre le mois de \emph{tôbi}, le \emph{premier mois des récoltes}, au grand dieu générateur Ammon-Horus, le Pan égyptien, représenté sous sa forme ordinaire, et ayant aussi à côté de lui \emph{deux grands chatons de palmier mâle}. » On voit, par cette dernière phrase, que l'auteur, bien qu'averti ici par les dimensions de l'objet dont il s'agit, confond, une seconde fois, le cyprès pyramidal avec le chaton du palmier. Il suffit cependant de jeter un coup d'œil sur les planches de la \emph{Description de l'Égypte} (histoire naturelle), ou d'un ouvrage quelconque de botanique, pour se convaincre que la forme et la disposition du régime ou de la grappe de fleurs que portent, soit le palmier dattier, soit le palmier \emph{doum}, n'auraient pas dû exposer le célèbre académicien à la méprise que je suis obligé de relever. D'autre part, les habiles ægyptologues que j'ai consultés, tout en avouant qu'à Edfou la légende hiéroglyphique du mois de \emph{tôbi} est très-difficile à interpréter, reconnaissent unanimement l'impossibilité de trouver un mot qui signifie \emph{chaton de palmier}, dans le groupe de signes hiéroglyphiques dont là, selon Champollion, serait formé le nom \textcoptic{ϣϥⲧⲃⲁ}, \emph{Schftba}. Un savant étranger, M. le Dr. R. Lepsius, dont l'opinion est d'un si grand poids dans ces questions, va plus loin, et n'hésite pas à dire que la dernière partie de ce nom, dont il donne la transcription sous la forme \emph{Schefteb}, offre très-vraisemblablement le nom même du mois de \emph{tôbi}.\footnote{Je transcris ici en entier le passage où cette opinion se trouve énoncée : « Memphitisch \textcoptic{ⲧⲱⲃⲓ}, thebanisch \textcoptic{ⲧⲱⲃⲉ}, heisst der fünfte Monat, griechisch \textgreek{Τυβί} geschrieben. Im Rhamesseum steht ihm der ithyphallische Gott $\hieroAAAU$ vor, in Edfu derselbe, aber mit dem Namen \emph{Schefteb}. Die Vermuthung liegt nicht fern, dass in dem zweiten Teile dieses letzteren Namens, der des Monats enthalten sei. » (\emph{Einleitung zur Chronologie der Ægypter}, p. 138.)}

On me pardonnera peut-être d'avoir autant insisté sur ces deux points, lorsque j'aurai indiqué les conséquences importantes qu'il est possible de déduire de la présence du cyprès pyramidal dans les tableaux des mois à Edfou et au Rhamesséum de Thèbes. Et d'abord remarquons que cette présence ne contribue pas peu à nous révéler l'origine étrangère, et par conséquent asiatique, du calendrier sacré représenté dans ces tableaux. D'une part, le cyprès caractérise, pour ainsi dire, l'Asie occidentale ; d'autre part, le calendrier hiératique de Thèbes et celui d'Edfou divisent l'année en deux sections de six mois chacune,\footnote{Voyez le mémoire cité de M. Biot et les planches qui l'accompagnent.} et non en trois \emph{tétrades}, comme devrait le faire un calendrier d'origine vraiment égyptienne, c'est-à-dire établi d'après les phénomènes climatériques propres à l'Égypte.\footnote{Voyez, à ce sujet, les observations que j'ai présentées dans mes \emph{Recherches sur Vénus}, p. 225, 226.} D'une autre part enfin, c'est dans le mois de \emph{tôbi}, nous dit M. Biot,\footnote{Mémoire cité, p. 622, 623, 632.} que tombe l'\emph{équinoxe vernal}. Or le dieu générateur \emph{Ammon-Horus} ou \emph{Ammon-Ra}, ithyphallique au Rhamesséum de Thèbes et sur les deux stèles citées, préside au mois de \emph{tôbi}, et se montre à nos regards, là comme à Edfou, caractérisé par l'attribut du cyprès, par un arbre qui, dans toute l'Asie occidentale, nous l'avons vu, est un symbole de vie consacré aux dieux créateurs ou générateurs, et dont la graine, remarquons-le bien, se récolte et se sème au printemps.\footnote{M. Percennius Nolanus, cité par Caton, \emph{De R. R.} 151. ; cf. 48.} A Thèbes et à Edfou, le tableau des douze mois de l'année assigne donc à cette saison le caractère qui lui est propre sous le climat régulier de l'Asie occidentale, c'est-à-dire le caractère de la saison qui voit se manifester sur la terre la germination, la feuillaison, la reproduction des êtres, une nouvelle vie enfin.\footnote{C'est dans le même esprit, n'en doutons pas, que, chez les Romains, à l'équinoxe du printemps, et en conformité d'un usage emprunté à la Phrygie, on dressait des pins en l'honneur de la \emph{Mère des dieux} (Pline, \emph{H. N.} 16., 14., 15. Julien, \emph{Orat. in Matrem deor.} p. 168). Cf. le passage cité plus haut de Julius Firmicus (\emph{De error. profan. relig.} 28., p. 103). --- Le pin, arbre toujours vert comme le cyprès, et, comme lui, symbole de vie, était, à ce titre, consacré, non-seulement à la Mère des dieux ou Rhéa, mais aussi à Vénus et à Mithra. Les monuments figurés en font foi.} Mais si ces phénomènes naturels ne s'observent en Égypte qu'après le solstice d'été, il faut nécessairement conclure, pour la seconde fois, que les tableaux des mois, à Thèbes et à Edfou, représentent, non un calendrier d'origine égyptienne et en harmonie avec les anomalies climatériques propres à l'Égypte, mais un calendrier importé\footnote{Voyez \emph{Rech. sur Vénus}, p. 225, 226.} d'un pays à climat régulier dans cette célèbre contrée, avec des institutions hiératiques dont il était impossible de le séparer.\footnote{Il me sera peut-être permis de rappeler ici que, dès l'année 1833, j'avais signalé ces importations. Loin de trouver faveur, mes idées rencontrèrent une opposition passionnée, qui s'est perpétuée bien longtemps. Plus qu'un autre j'ai donc, ce me semble, quelque droit à me réjouir de la presque unanimité avec laquelle on commence enfin à reconnaître l'influence religieuse, philologique et scientifique qu'exerça sur l'Égypte l'Asie occidentale, cette belle contrée qui dut être civilisée de bonne heure, parce qu'elle fut tout à la fois le berceau du genre humain, le lieu où se conservèrent les traditions, et le lieu aussi où s'illustra dans les sciences et les arts la caste sacerdotale des Chaldéens, qui a fait la gloire de la Babylonie et de l'Assyrie. Un jour viendra, j'aime à l'espérer dans l'intérêt d'une science qui est en progrès, un jour, dis-je, viendra où une étude approfondie et comparée des institutions religieuses et civiles, du langage symbolique, des phénomènes climatériques et des productions naturelles de l'Asie occidentale et de l'Égypte, conduira même à reconnaître, tout aussi unanimement, que l'action directe de la première de ces deux contrées sur la seconde remonte bien au-delà des 18\textsuperscript{e} et 19\textsuperscript{e} dynasties égyptiennes. En attendant ce jour désiré, je ne me laisserai pas de répéter que, physiquement et historiquement parlant, le double fait de la population et de la civilisation doit être considéré comme bien plus ancien dans l'Asie occidentale qu'en Égypte. Aussi, dès à présent, peut-on dire, quant à l'Orient, sans même excepter la Phénicie, que l'Égypte a beaucoup reçu et très-peu donné.} Ne suit-il pas de là qu'un tel calendrier dut être apporté chez les Égyptiens, au plus tard, pendant la période qui vit l'Égypte recevoir le culte des divinités assyriennes et phéniciennes que l'on reconnaît sur les monuments érigés par les rois des 18\textsuperscript{e} et 19\textsuperscript{e} dynasties ? Et, parmi ces divinités importées, la Vénus qu'adorait toute l'Asie occidentale n'occupe-t-elle pas le premier rang ? Cette déesse, qui, en Asie et en Europe, donna deux de ses noms au cyprès pyramidal ; cette déesse qui, selon la théologie asiatique, avait sa place marquée à l'équinoxe du printemps et paraissait montée sur le Taureau,\footnote{Porphyre, \emph{De Antr. nymph.} 24., p. 22, 23, ed. Van Goens. \emph{Mém. de l'Acad. des inscript.} t. 15., 2\textsuperscript{e} partie, p. 230-236. \emph{Recherches sur Vénus}, p. 130-132 ; atlas, pl. 17., n°\textsuperscript{s} 1, 1a, 1b. --- Le n° 17 de la pl. 14 G de ce dernier ouvrage reproduit un groupe qui, sur un vase d'argent d'époque romaine, trouvé aux environs de Vienne en Dauphiné, et orné des quatre saisons personnifiées, représente le printemps. Ce groupe nous offre Vénus assise sur le taureau équinoxial et entourée de plusieurs petits Amours ; composition rare, qui, rapprochée d'un passage de Virgile (\emph{Georgic.} 1., 217, 218) bien souvent cité, mais pas toujours compris,\\\hspace*{15mm}Candidus auratis aperit quum cornibus annum\\\hspace*{15mm}Taurus ... ...\\\hspace*{0mm}est un exemple précieux et mémorable de la fidélité et de la persévérance avec lesquelles les artistes, comme les poètes de l'antiquité, nous ont transmis d'âge en âge les traditions hiératiques les plus anciennes. Sur d'autres monuments figurés, qui ne représentent pas les quatre saisons, nous voyons la Vénus asiatique placée debout sur un taureau (\emph{Recherch. sur Vénus}, p. 130-132 ; pl. 17., n°\textsuperscript{s} 1, 1a, 1b), ou assise sur deux taureaux (\emph{ibid.} p. 128, 129 ; pl. 3 B, n° 1). C'est ainsi que, dans le panthéon des Indiens, la déesse Parvatî, qui a tant d'analogie avec Vénus et Junon, se montre tantôt assise sur un taureau (\emph{ibid.} pl. 14 D, n° 7), tantôt chevauchant sur un quadrupède de même espèce (\emph{ibid.} n° 6), tantôt enfin assise sur un trône, à côté de Çiva, et posant un pied sur un lion, pendant que son divin époux pose un pied sur un taureau (\emph{ibid.} n° 4).} symbole du premier mois de cet équinoxe dans l'ancien calendrier dés Chaldéens d'Assyrie ; cette déesse, sur les deux stèles égyptiennes, nous la retrouvons accompagnée tout à la fois, comme sur les monuments asiatiques, du symbole du cyprès et de deux personnages qui forment avec elle une triade divine. Un de ces personnages, dans les calendriers sculptés à Thèbes et à Edfou, se montre comme le protecteur ou comme la personnification même du mois égyptien de \emph{tôbi}, dans lequel tombe l'équinoxe vernal ; et là, caractérisé, ainsi que ce mois et cet équinoxe, par le symbole du cyprès, il est au lieu et place du mois de \emph{tôbi}, c'est-à-dire, théologiquement parlant, il est placé au seuil de l'équinoxe du printemps, de même que la déesse dont il est l'assesseur. Enfin nous verrons, dans le mémoire suivant, Vénus présider aux cérémonies funèbres en Égypte, comme en Asie, comme chez les Grecs ; et son arbre symbolique, le cyprès, tout étranger qu'il est au sol égyptien, ne le verrons-nous pas aussi figurer, en même temps qu'elle, dans les peintures qui, en Égypte, ornent les caisses de momie ? J'avais donc quelque raison de dire, au début de ce mémoire, que parfois, lorsque l'histoire reste muette, un symbole tiré des productions de la nature peut nous révéler le fait des communications d'une nation avec une autre, et nous apprendre que telle ou telle divinité, tel ou tel usage religieux, que l'on croyait propres à un peuple très-anciennement civilisé, sont néanmoins des emprunts faits par lui à un peuple plus ou moins voisin.

Ajoutons que le Rhamesséum de Thèbes et les deux stèles décrites ci-dessus ne sont pas les seuls monuments où, jusqu'à ce jour, on ait trouvé deux cyprès figurés auprès du personnage divin qui tout à la fois est un des assesseurs de Vénus-Hâthôr et le protecteur du premier mois de l'équinoxe vernal. Ce sont aussi deux jeunes cyprès pyramidaux que l'on reconnaît sans hésitation sur un des autels qui, dans une peinture du sanctuaire de Karnak, sont placés à côté de ce dieu.\footnote{Après tous les exemples que j'ai cités de la présence du cyprès sur les monuments figurés égyptiens, on peut s'étonner de ne point trouver cet arbre parmi les végétaux placés, par Horapollon, au nombre des hiéroglyphes dont il donne la signification.} Ici, il est ithyphallique, et Champollion\footnote{\emph{Panth. égypt.} texte pour la pl. 4.} le désigne sous le nom à \emph{Ammon générateur} ou \emph{Mendès}. Entre ces deux arbres, qui sont peints en vert, et qui nous rappellent les deux cyprès plantés à côté de Vénus-Astarté, d'Artémis ou de Mithra, sur les monuments asiatiques, on voit s'élever, comme sur les deux stèles citées, une tige fleurie de lotus ; la fleur est bleue ; elle s'épanouit au sein d'un calice peint en rouge et en jaune.\footnote{On voit comment un art hiératique, assujetti à des règles conventionnelles, a reproduit dans cette fleur, outre l'idée de l'hermaphroditisme et de la génération, quatre des couleurs, le bleu, le rouge, le jaune et le vert, que l'antiquité attribuait aux divinités génératrices, et qui rappellent le ciel, le feu, le soleil, et la vie ou le printemps.} Cet emblème religieux a été gravé, avec la figure du dieu ithyphallique, dans la \emph{Description de l'Égypte},\footnote{\emph{Antiquités}, t. 3., pl. 47., n° 2.} et reproduit par Champollion dans son \emph{Panthéon égyptien}\footnote{Pl. 4.} ; mais il n'a donné lieu, de sa part, à aucune remarque. Ce savant l'a même complétement passé sous silence, et, par conséquent, nous laisse ignorer s'il prenait ici les deux cyprès pour deux grands chatons de palmier, comme il l'a fait plus tard en décrivant le tableau des mois sculpté au Rhamesséum de Thèbes.

D'autres parties de l'Afrique civilisée, où le cyprès pyramidal ne croît pas spontanément, avaient sans doute, comme l'Égypte, reçu de l'Asie occidentale le culte de cet arbre avec celui de quelque divinité génératrice. Nous pouvons même tenir pour certain que ce symbole fut du nombre de ceux que les Phéniciens portèrent, avec le culte d'Astarté, dans les colonies qu'ils établirent sur les côtes septentrionales de l'Afrique, à l'ouest du Delta. Toutefois nous n'en avons qu'une preuve indirecte ; elle se tire des traditions relatives aux rites funèbres qui étaient en usage chez les Carthaginois. Je dois donc différer de la produire jusqu'au moment où je serai amené à considérer le cyprès comme un emblème funéraire.

Ne doutons point que si les révolutions politiques dont Carthage fut le théâtre n'avaient pas détruit ou dispersé les monuments d'un culte apporté de Tyr par les fondateurs de la nouvelle cité, nous aurions, dans les antiquités figurées, comme dans les livres sacrés des Carthaginois, d'aussi fréquents exemples de l'attribution du cyprès à la déesse Astarté que nous en fournissent les antiquités qui, sur le sol de la Phénicie, ont survécu aux bouleversements politiques et physiques. Et ce que je dis ici de Carthage doit s'entendre également de tous les établissements que les Phéniciens avaient formés sur les côtes de l'Afrique septentrionale.

\clearpage
\subsection{Deuxième Section --- Occident}
\subsubsection{Grèce et colonies grecques}
\paragraph{}
Si le culte du cyprès pyramidal fut importé de l'Asie antérieure dans ce vaste continent qui renfermait l'Égypte et le territoire de Carthage, ne devons-nous pas nous attendre à le trouver aussi établi dans l'archipel situé entre les côtes de l'Asie et celles de la Grèce, et dans plusieurs parties du continent européen ? Nous savons que, dès une haute antiquité, le culte des arbres, en général, et des arbres verts, en particulier, était fort répandu dans plusieurs îles de la mer Ægée et de la Méditerranée, dans la Thrace, la Grèce, l'Italie, la Germanie, les Gaules, l'Ibérie, la Lusitanie. C'est sans doute à l'institution d'un tel culte qu'il faut rapporter l'antique usage de consacrer aux dieux des forêts, des bois, et d'y placer leurs temples, leurs statues, leurs autels.\footnote{Cet usage dut naître du besoin de continuer à célébrer les cérémonies religieuses dans des lieux retirés, couverts, mystérieux, comme les bois et les forêts, alors que l'accroissement successif de la population et les progrès de l'architecture, de la sculpture, de la statuaire, obligèrent le sacerdoce à restreindre à la célébration des mystères ou des initiations, l'emploi des grottes et des antres sacrés.} Des troncs d'arbres servirent à façonner les premières images des divinités de la Grèce ; et Pausanias,\footnote{8., 17., 2.} lorsqu'il rappelle cette coutume, place l'ébène et le cyprès en tête des différentes espèces d'arbres employées par les anciens à faire des \emph{xoana}. De bonne heure, en effet, et certainement à une époque antérieure à la fondation en Asie des premières colonies grecques, le culte asiatique du cyprès avait pénétré dans la Grèce comme dans la Thrace ; et si Ovide\footnote{\emph{Metamorph.} 10., 106.} nomme cet arbre parmi ceux qui vinrent se ranger autour d'Orphée,\footnote{Pierius Valerianus, connu aussi sous le nom de Valeriano Bolzani, rapporte, dans ses \emph{Hieroglyphica} (52., 8.), mais sans citer aucune autorité, qu'une statue de bois de cyprès avait été consacrée à Orphée, dans la ville de Libèthre, près du mont Olympe, qui sépare la Thessalie de la Macédoine. Voici en quels termes il parle de cette statue et d'un prodige qu'on lui attribua du temps d'Alexandre le Grand : « Et in Lebethris Orphei simulacrum ex cupresso cultum, quod cum Alexandri temporibus sudorem multum emisisset, omnibus eo prodigio perterrefactis, responsum est ab Aristandro, viro sapiente, bene sperandum, siquidem quod nunc opus Alexander aggrederetur, ob rerum gestarum magnitudinem maximum esset sudorem et laborem poetis ac Musarum cultoribus allaturum. »} attirés par le charme irrésistible des sons que le chantre de Thrace savait tirer de sa lyre, Pausanias,\footnote{2., 15., 2.} de son côté, reconnut dans l'Argolide, au milieu d'un \emph{cupressetum}, les ruines de l'antique temple de Jupiter Néméen qui avait été érigé près du bourg de Némée, à quinze stades environ de l'antre du lion terrassé par Hercule. Enfin, dans l'Arcadie, le mont Érymanthe, également célèbre dans l'antique légende d'Hercule par la victoire que remporta sur un sanglier ce héros d'origine asiatique, était lui-même couvert d'une plantation de cyprès.\footnote{Ovide nous l'apprend dans ces deux vers (\emph{Heroïd. Epist.} 9., v. 87, 88) :\\\hspace*{10mm}Ut Tegeæus aper cupressifero Erymantho\\\hspace*{10mm}Incubet, et vasto pondere lædat humum ?}

En quittant l'Argolide, cette province à laquelle se rattache avec éclat la légende de Persée, autre héros asiatique, le voyageur grec\footnote{2., 24., 6.} nous signale, sur la route d'Argos à Tégée et au milieu des beaux cyprès qui couvraient entièrement le mont Lyconé, le temple d'Artémis \emph{Orthia}.\footnote{Peut-être ce surnom d'\emph{Orthia} était-il en rapport avec la forme \emph{droite} du cyprès, image symbolique de la déesse.} Ce temple, bâti au sommet d'une montagne dont le nom, Lyconé, fait allusion à la lumière, renfermait trois statues de marbre blanc, qui représentaient Latone avec ses deux enfants, Apollon et Artémis.\footnote{La statue de Latone, ainsi placée sur la montagne de la lumière, avec ses deux enfants, qui représentent, l'un le soleil, l'autre la lune, nous rappelle que le nom de Latone emporte avec lui le sens de \emph{nuit} ou d'une \emph{chose cachée}. J'ajoute que le nom de \emph{Damia}, donné à la déesse dans une tradition qui la fait arriver du pays des Hyperboréens à Délos, signifie, en zend et en sanscrit, \emph{ténèbres}. Cette remarque, si je ne me trompe, n'avait pas encore été faite.} L'ordre que suit Pausanias, en énumérant ces trois figures, nous autorise à croire que Latone était placée entre Apollon et Artémis, comme la Vénus assyrienne, mère des dieux, est elle-même placée entre ces deux dernières divinités, ou entre le cyprès du soleil et le cyprès de la lune, sur les monuments figurés asiatiques que j'ai cités dans la première section de ce mémoire. Le récit de notre auteur mérite d'être rapproché des détails que nous trouvons dans un passage de Strabon\footnote{\emph{Geogr.} 14., p. 639, 640.} dont précédemment je n'ai dit que quelques mots. Il s'agit d'un bois sacré, nommé Ortygie, et planté de plusieurs sortes d'arbres, mais surtout de cyprès. Ce bois, dont le nom est un de ceux que portait anciennement l'île de Délos, était situé non loin d'Éphèse et traversé par le Cenchrius, petit fleuve où Latone, disait-on, fit ses ablutions après avoir mis au monde, dans cette contrée, Apollon et Artémis, qui eurent pour nourrice une femme également appelée Ortygie. Aussi Strabon nous apprend-il que, dans un des temples que renfermait le bois sacré, on voyait la statue de Latone portant un sceptre, et accompagnée d'un groupe composé de ses deux enfants et de leur nourrice Ortygie, qui les tenait chacun par la main. Il y a donc lieu de croire que le cyprès fut consacré à Latone, mère des dieux, aussi bien qu'à Vénus et à Rhéa, qualifiées chacune mère des dieux, et, comme Latone, assimilées à la Nuit, divinité essentiellement génératrice. Et si, dans l'Asie occidentale, les monuments figurés ou les traditions nous ont montré le cyprès du soleil et le cyprès de la lune attribués à Vénus, à Rhéa et à la déesse de Syrie ; si, comme j'en ai fourni la preuve, le cyprès fut nominativement consacré à Apollon et à Artémis chez les Grecs asiatiques, ma supposition, quant à Latone, se trouve confirmée par la loi hiératique qui transmettait aux jeunes divinités les symboles et les attributs de leur père ou de leur mère.

La Grèce, par conséquent, doit nous offrir plus d'un exemple de l'attribution du cyprès à Apollon et à Artémis. Et d'abord, notons ici que, selon le témoignage exprès de Pindare,\footnote{\emph{Pythic.} 5., 52.} les lambris du temple d'Apollon à Delphes,\footnote{C'est ainsi que Winckelmann (\emph{Opp.} t. 6., p. 162) entend le passage de Pindare.} ou du moins d'une des \emph{cella} de ce temple,\footnote{Feu M. Avellino (mémoire cité, p. 8, note 3), s'appuyant sur un commentaire du scoliaste de Pindare, croit que les mots \textgreek{κυπαρίσσινον μέλαθρον}, dont se sert le poète, s'appliquent, non à tout le temple, mais seulement à la \emph{cella} où Arcésilaüs consacra son char.} étaient de bois de cyprès. Notons aussi que très-anciennement, sur le mont Parnasse, consacré à Apollon et aux Muses, ses fidèles compagnes, il existait, au milieu d'un \emph{cupressetum}, une ville qui, primitivement appelée \emph{Cyparissia}, fut ensuite nommée \emph{Apolloniade}.

Les médailles impériales de deux villes de l'Argolide, Épidaure et Trézène, fournissent des témoignages non moins explicites. Parmi les premières, je citerai deux pièces longtemps attribuées à une prétendue ville de \emph{Cyparisus} ou \emph{Cyparissus}, dont on ne pouvait, et pour cause, indiquer la position topographique ; elles furent frappées en l'honneur d'Antonin le Pieux. Au revers de la tête de cet empereur, elles nous montrent, l'une et l'autre, une chèvre allaitant un jeune enfant au pied d'un grand cyprès.\footnote{Mionnet, sans faire remarquer que Sestini, dans l'écrit intitulé : \emph{In Catalogi musei Hedervariani partem primam Castigationes} (p. 46, n° 6012), restitue à Épidaure les médailles dont il s'agit, classe, avec doute cependant, parmi celles de l'île de Crète, la médaille impériale décrite par Sestini dans ses \emph{Lettere numismatiche} (\emph{Continuazione}, t. 3., p. 44 et 45), et l'attribue à une ville qu'il désigne sous le nom de \emph{Cyparisus incerti situs}. (Voy. \emph{Suppl.} 4., 316, n° 137.)} Sestini commet l'erreur de désigner l'enfant sous le nom d'Apollon.\footnote{\emph{In Catalog. mus. Hederv. part. 1 Castigationes, loc. cit.}} Feu M. Avellino,\footnote{Mémoire cité, p. 22.} mieux inspiré, prend ici le cyprès pour l'image symbolique de ce dieu, et l'enfant pour Esculape, nourri à l'ombre de l'arbre consacré à son père. Le savant archéologue napolitain n'oublie pas de rappeler que, selon les traditions grecques qui avaient encore cours du temps de Pausanias, Esculape enfant aurait été découvert, dans le pays des Épidauriens, par le chevrier Aresthanas,\footnote{Pausanias, 2., 26., 4.} ou par Autolaüs, fils illégitime d'Arcas.\footnote{\emph{Id.} 8., 25., 6.} Les monnaies impériales d'Épidaure qui ont pour revers un sujet si habilement interprété, sont encore rares. La Bibliothèque impériale de Paris en possède deux depuis peu d'années seulement ; leur état de conservation laisse beaucoup à désirer. Elles proviennent de la collection formée, au musée Hedervar, par le comte de Wiczay, et ne sont pas d'un même flan.\footnote{Une de ces médailles est décrite par l'abbé Caronni dans le Catalogue du musée Hedervar (pars 2., n° 6012).} Je les ai fait dessiner avec soin, et placer sous les n°\textsuperscript{s} 1 et 2 de la planche 21 ci-jointe. Au revers, la première offre, coupées par le haut, les trois lettres \textgreek{ΕΠΙ}, gravées du même côté que le cyprès, et les deux lettres terminales \textgreek{ꞶΝ}, placées à l'exergue. Les lettres \textgreek{ΔΑΥΡΙ}, qui sont ici presque entièrement effacées, se voient nettement au revers du n° 2 ; de sorte qu'en rapprochant les deux médailles, on a la légende entière \textgreek{ΕΠΙΔΑΥΡΙꞶΝ}, « des Épidauriens. » Il existe au cabinet impérial de Vienne un autre exemplaire de la pièce n° 2. Neumann en a donné\footnote{\emph{Num. populor.} t. 2., tab. 7., fig. 7.} un dessin très-inexact, où l'on ne retrouve pas le jeune Esculape allaité par la chèvre. Le dessin qu'a publié Sestini\footnote{\emph{Descript. numor. veter.} tab. 13., fig. 2.} d'un quatrième exemplaire, semblable à celui de Vienne et tiré du musée Bondacca, est un peu moins mauvais. Enfin il en indique un cinquième exemplaire dans le musée Sanclemente.\footnote{\emph{Letter. numism. Continuaz.} t. 3., p. 45.} En 1846, M. Th. Panofka, dans une savante dissertation intitulée : \emph{Asclepios und die Asklepiaden},\footnote{Berlin, in-4°, fig.} nous a fait connaître deux autres médailles impériales qui offrent, au revers, le même sujet que ces cinq pièces : l'une porte l'effigie de Caracalla, et se conserve au médaillier impérial de Vienne\footnote{Pag. 10, pl. 1., n° 1.} ; l'autre, où nous retrouvons l'effigie d'Antonin le Pieux, est déposée dans le médaillier royal de Copenhague.\footnote{Pag. 10 et 11, pl. 1., n° 2.} L'habile académicien de Berlin rapporte, sans hésitation, la première à Épidaure, mais ne sait à quelle ville il convient d'attribuer la seconde ; il indique avec doute Cyparissie de Messénie, ou Cratie de Bithynie. Comme les six autres cependant, celle-ci appartient à Épidaure.

Parmi les monnaies impériales de Trézène qui nous offrent le cyprès, je trouve deux moyens bronzes à citer : le premier représente, au revers de la tête de Septime Sévère, un temple placé sur une colline qui s'élève entre un cerf et un cyprès\footnote{Sestini, \emph{Descr. num. veter.} p. 215, n° 3.} ; le second, qui est à l'effigie de Julia Domna, porte le même revers, avec cette seule différence que le cerf y est remplacé par un laurier.\footnote{\emph{Mus. Sanclemente}, num. select. t. 2., p. 299, tab. 25., fig. 235. --- C'est à tort que l'abbé Sanclemente désigne ici sous le nom d'olivier l'arbre placé en regard du cyprès.} Quoique le cerf et le laurier soient, comme le cyprès, des emblèmes consacrés à Vénus tout aussi bien qu'au dieu des Muses, je n'hésite pas à croire, avec M. Avellino,\footnote{Mémoire cité, p. 22-24.} que l'édifice qui, sur chacune de ces deux médailles, orne le sommet d'une colline, ne soit un de ces deux temples qu'au rapport de Pausanias les Trézéniens avaient dédiés à Apollon, surnommé, dans l'un, \emph{Théarius},\footnote{Pausanias, 2., 31., 9.} dans l'autre, \emph{Épibatérias}.\footnote{\emph{Id. ibid.} 32., 2.} À cette occasion, le savant auteur du mémoire sur le mythe de Cyparisse a soin de mentionner l'autel que, selon le voyageur grec,\footnote{\emph{Id. ibid.} 31., 8.} ces mêmes Trézéniens avaient consacré à \emph{Hélios Éleuthérios} ; mais il ne connaissait pas le sens philosophique d'un des noms, \emph{serv-azâd}, qu'en Perse on donne au cyprès, à cet arbre qui fut l'emblème tout à la fois de Mithra, du soleil, de la lune et des cinq autres planètes ; c'est-à-dire l'emblème du dieu psychopompe et des sept cieux où les âmes, rendues à la liberté et à la vie spirituelle, doivent achever de se purifier avant de rentrer dans le séjour des bienheureux. Dès lors, M. Avellino n'a pu faire un rapprochement qui se présente à notre esprit ; il n'a pas remarqué la corrélation qui existe entre le surnom d'\emph{Éleuthérios}, donné par les Trézéniens au Soleil, porte du ciel, et attribué ailleurs à Dionysus, prince des mystères ; les surnoms de \emph{Liber} et de \emph{Libera} ou \emph{Libitina}, appliqués, chez les Romains, l'un à Bacchus, l'autre à Vénus ou à Proserpine, déesses des vivants et des morts ; et l'épithète \emph{azâd},\footnote{Dans mon second mémoire, je donne sur la formation de cette épithète philosophique les explications nécessaires.} \emph{sans vie} ou \emph{libre}, ajoutée au nom de l'arbre qui chez les Perses, comme chez les Assyriens et les Phéniciens, était non-seulement un symbole de vie, de liberté ou de vie spirituelle, mais aussi l'image immortelle de la divinité qui préside à des mystères fondés sur le dogme de la descente et de l'ascension des âmes. C'est ici le lieu de rapporter en entier un passage de Pausanias dont j'ai déjà dit quelque chose\footnote{Ci-dessus, p. 26, 27.} à propos d'un autre récit où le même écrivain s'impose un silence religieux sur le rôle de Mercure et de son bélier\footnote{Une agate-onyx, que M. le duc de Luynes a récemment acquise et qu'il veut bien me communiquer au moment où cette feuille est sous presse, fournit une preuve nouvelle et irrécusable de l'authenticité des traditions qui attribuent au symbole du bélier un rôle important dans la célébration des mystères d'origine asiatique. D'un côté, cette agate-onyx, dont le travail me paraît phénicien ou assyrien, offre en relief une tête de bélier ; de l'autre, elle représente, gravé en creux, un griffon ailé et crêté, dévorant un bouc terrassé. Dans mes \emph{Recherches sur Mithra}, je montrerai comment ce groupe se rattache directement aux mystères de la Vénus assyrienne.} dans la célébration des mystères de la Mère des dieux. Malgré une seconde réticence, nous allons trouver ici la preuve que, chez les Grecs, on associait le culte du cyprès aux cultes réunis d'Apollon, des Grandes Déesses adorées dans les mystères, et du dieu que les Égyptiens, sous le nom de Thoth ou Thaut, et les Grecs, sous le nom d'Hermès, considéraient comme une divinité psychopompe. « En face des champs stényclériques, dit Pausanias,\footnote{4., 33., 5.} il y a un lieu que les anciens appelaient \emph{Œchalia}, et qui se nomme maintenant le bois Carnasius.\footnote{\textgreek{Καρνάσιον ἄλσος}.} On y voit une forêt très-épaisse de cyprès, qui renferme les statues de plusieurs dieux : Apollon Carnéus, Hermès portant un bélier, et Coré la Chaste, fille de Déméter. Près de cette dernière statue coule une fontaine. Dans le Carnasius, on célèbre des mystères en l'honneur des Grandes Déesses ; mais je garderai le silence sur le rite qu'on y suit. Ces mystères m'ont paru tenir le premier rang après ceux d'Éleusis. » Ce passage, rapproché des sculptures de l'autel palmyrénien du Musée capitolin et des observations placées dans la section précédente, prouve jusqu'à l'évidence qu'en Grèce, comme dans l'Asie occidentale, le culte d'Apollon, ou d'un dieu-soleil désigné sous divers noms, se liait intimement à l'institution des mystères.

Un autre lieu, un autre monument, peuvent ajouter une nouvelle preuve à celles que déjà nous avons de l'attribution du cyprès à Apollon chez les Grecs. J'entends parler de la fresque dont un dessin au trait accompagne le mémoire cité de feu M. Avellino, et qui a été fidèlement reproduit sur la planche 12 ci-jointe. Bien que cette belle peinture provienne des ruines de Pompeï,\footnote{Elle fut découverte en 1833 ou 1834.} elle a droit, par la simplicité de sa composition et par l'élévation de son style, à être comptée au nombre des monuments dont les artistes grecs revendiquent l'invention. Elle représente, vu de face, Apollon debout, presque nu, la tête ceinte d'une couronne de laurier et entourée d'un limbe, la main gauche appuyée sur une grande lyre à dix cordes posée à terre, et une branche de laurier dans la main droite. Une chlamyde, dont un bout est jeté sur le bras gauche, est agrafée au-dessous de l'épaule droite, et tombe par derrière jusqu'à mi-jambe. Les pieds sont chaussés du cothurne. A la gauche du dieu, et plongé dans la tristesse, on voit le beau Cyparisse assis, tenant de la main gauche deux longues lances. Une draperie, jetée sur le bras droit, passe sur la cuisse droite et enveloppe le bas des reins, la cuisse et la jambe gauches ; le reste du corps est nu. Du sommet de la tête sort la cime d'un cyprès pyramidal. Cyparisse est donc représenté au moment même où, selon la fable, Apollon, touché de la douleur que cause à ce jeune homme la mort d'une biche ou d'un cerf qu'il aimait et qu'il a tué par imprudence, le métamorphose en cyprès. Aussi le peintre a-t-il placé auprès du siège de Cyparisse un jeune cerf, ou peut-être une biche. Derrière les deux seuls personnages qu'on voie dans le tableau, s'élèvent de belles constructions, qui nous annoncent qu'ici la métamorphose de Cyparisse s'opère devant un temple dédié à Apollon : le trépied, l'arc et le carquois posés sur ces constructions, et le bois de cyprès dont elles sont entourées, ne laissent aucun doute à cet égard. Ce sont, avec la branche de laurier qui pend de la main droite d'Apollon, les mêmes accessoires caractéristiques que nous avons trouvés auprès de ce dieu, sur les médailles coloniales d'Alexandria-Troas\footnote{Ci-dessus, p. 105-107 ; et pl. 13., n°\textsuperscript{s} 1-3.} ; mais, au lieu d'un seul cyprès que ces médailles nous montrent planté à ses côtés, la fresque de Pompeï, seul monument figuré, authentique, qui nous reste de la métamorphose de Cyparisse, laisse distinctement apercevoir un \emph{cupressetum} autour de l'édifice consacré à Apollon.

Les traditions relatives au mythe de Cyparisse varient beaucoup dans les auteurs grecs, comme dans les auteurs latins.

Philostrate\footnote{\emph{In Vita Apollon.} 1., 16.} reconnaît à ce mythe une origine asiatique : en parlant du bois sacré de cyprès et de lauriers planté auprès du temple d'Apollon, dans le faubourg d'Antioche de Syrie, appelé \emph{Daphné}, il a soin de dire que Cyparisse était un jeune Assyrien, qui passait pour avoir été changé en cyprès ; et, ajoute-t-il, la beauté de l'arbre fait foi d'une telle métamorphose.

Une autre tradition, recueillie par Servius,\footnote{\emph{Ad} Virgil. \emph{Æneid.} 3., v. 680.} assigne l'île de Crète pour patrie à Cyparisse, jeune homme aussi chaste que beau, dont Apollon, ou Zéphyre, était éperdument amoureux, et qui, voulant conserver la pureté de ses mœurs, s'enfuit vers le fleuve Oronte, et se réfugia sur le mont Casius, où il fut changé en cyprès. Si, dans cette légende, l'île de Crète est le lieu où Apollon chercha à séduire Cyparisse, n'oublions pas que les écrivains grecs ou latins désignent comme la patrie du cyprès une ville de cette île, Tarrha, célèbre, de même que Cnosse, par le culte qu'on y rendait à Apollon.\footnote{Sur la haute antiquité et l'importance du culte d'Apollon dans l'île de Crète, nommément à Cnosse et à Tarrha, voyez les savantes remarques de K. O. Müller (\emph{Die Dorier}, t. 1., p. 206-209).} Remarquons, d'autre part, que, dans le récit de Servius, le mont Casius et le fleuve Oronte nous ramènent en Syrie, cette partie importante de l'Asie occidentale, qui déjà, par ses monuments figurés et par ses traditions écrites, nous a fourni la preuve que le culte du cyprès y fut très-anciennement établi en l'honneur d'Apollon et du soleil, comme en l'honneur de Vénus et de la lune.\footnote{Ci-dessus, p. 19, 20, 37-41, 51-58, 96-99, 104-107.} Si Zéphyre est substitué à Apollon dans une variante de cette légende recueillie par Servius,\footnote{Servius, \emph{loc. cit.}} nous le trouvons nommé dans un passage où Nonnus semble dire qu'Apollon et Zéphyre s'étaient disputé le cœur de Cyparisse.\footnote{\textgreek{Καὶ Ζέφυρον κλονέεσκε Λάκων νέος · ἀλλὰ [θανόντα\\\hspace*{5mm}Ἡβητὴν Κυπάρισσον ἰδὼν ἐρατεινὸς ἀήτης\\\hspace*{5mm}Εὗρεν Ἀμυκλαίοιο παραιφασίην Ὑακίνθου}.\\\hspace*{15mm}\emph{Dionysiac.} 11., 363-365.} A ce nom de Zéphyre, pouvons-nous ne pas nous rappeler que la Vénus asiatique était assimilée à l'air, cosmologiquement parlant, et que, chez les anciens habitants de l'Arménie, peuple sorti du territoire de l'empire assyrien, cet élément intervenait dans les augures qui se tiraient de la plus ou moins grande agitation du feuillage des cyprès sacrés d'Armavir\footnote{Ci-dessus, p. 64-73.} ?

Ovide, sur l'autorité, sans doute, de quelque mythographe grec, dont les écrits et même le nom ne nous sont point parvenus\footnote{Voyez, à ce sujet, les observations de Gierig, dans la préface de son édition des \emph{Métamorph. d'Ovide}, et celles de M. Avellino (mémoire cité, p. 13-17, 21).} ; Ovide, le plus ancien des écrivains de l'antiquité qui nous ont conservé la fable de Cyparisse et Apollon ; Ovide\footnote{\emph{Metamorph.} 10., 106-142.} place la scène principale dans l'île de Céos ou Céa,\footnote{Aujourd'hui Zéa.} dans une île presque contiguë à l'Attique, cette célèbre contrée où le culte d'Athéné, la légende et les symboles de la déesse, l'assistance qu'elle prête à Hercule dans ses douze travaux ou ses initiations,\footnote{Voy. surtout les peintures d'une série nombreuse de vases grecs, d'ancien style.} la légende de Thésée et bien d'autres faits, nous révèlent une civilisation importée de l'Asie occidentale. Aussi trouvons-nous dans le récit du docte et ingénieux chantre des Métamorphoses, plus d'un trait qui tout à la fois décèle aussi une origine asiatique et s'accorde avec la fresque de Pompeï. D'autre part, des témoignages authentiques nous apprennent que, sous divers noms et dès une époque reculée, Apollon fut adoré dans l'île de Céos,\footnote{Voy. sur l'île de Céos, sur les ruines de Carthéa et sur le culte d'Apollon dans cette île, les manuscrits autographes et inédits d'Anse de Villoison, déposés à la Bibliothèque impériale (le 3\textsuperscript{e} vol. surtout), et les notes écrites, par le même voyageur, à la marge des pages 15, 16 et suivantes du 2\textsuperscript{e} volume du \emph{Voyage de Tournefort au Levant} (éd. de Lyon, 1727 ; trois vol. in-8°). L'exemplaire se conserve aussi à la Bibliothèque impériale. (Cf. Bröndsted, \emph{Voyag. et Recherch. dans la Grèce}, p. 1-76 ; pl. 8., 16-19.)} et qu'il continua de l'être jusqu'aux derniers temps du paganisme. Il y portait principalement les surnoms d'\emph{Aristée} et de \emph{Carthéen}. De nos jours on a découvert, dans les ruines de Carthée, les restes d'un temple dédié à Apollon Carthéen\footnote{Bröndsted, \emph{ibid.} p. 18 et suiv. pl. 8., 16-19.} ; et c'est aux environs mêmes de Carthée, que, selon le poète latin, Cyparisse tua par mégarde un très-beau cerf objet de tous ses soins, et qui était consacré aux nymphes des champs appelés \emph{carthéens, carthæa arva}. Ce cerf, comme le cyprès, doit être pour nous une image symbolique d'Apollon lui-même.\footnote{A l'appui de cette opinion, j'aurais à citer les témoignages que plusieurs savants archéologues, nommément MM. le duc de Luynes, Raoul-Rochette, Rathgeber, Ph. le Bas, Em. Braun, Avellino, ont produits pour prouver que le cerf et la biche furent consacrés à Apollon, comme à Artémis ou Diane. Je me bornerai à faire une mention particulière des monnaies autonomes de Caulonia, qui nous montrent, au droit, un cerf placé sur un autel, à côté de l'image d'Apollon, et, au revers, un cerf regardant tantôt à droite, tantôt à gauche (voy. l'intéressante notice publiée par M. Émile Braun, dans le \emph{Bulletino dell' Instit. arch.} année 1843, p. 94, 95). C'est ainsi que, sur un des bas-reliefs de l'obélisque de Xanthus, l'image symbolique de Vénus, une vache allaitant son veau, est placée en regard de l'image anthropomorphique de la déesse.} Le portrait qu'en trace Ovide ne permet aucun doute sur ce point : il était très-grand\footnote{« Ingens cervus. »} ; il avait des cornes d'or ou dorées,\footnote{« Cornua fulgebant auro. »} un collier de pierres précieuses, une bulle d'argent suspendue sur le front, ce qui,\footnote{Bulla super frontem parvis argentea loris\\\hspace*{5mm}Vincta movebatur ... ...} pour le dire en passant, caractérisait Apollon comme un dieu solaire et lunaire à la fois.\footnote{Voyez mes observations sur l'emploi simultané de l'or et de l'argent dans les images des divinités (\emph{Recherches sur Vénus}, p. 108-116).} Cette description, que j'abrège, ne se rapporte nullement au quadrupède placé près du siége de Cyparisse, dans la fresque de Pompeï. Les petites proportions que lui a données le peintre ne répondent ni à un cerf ordinaire, ni au cerf des champs carthéens, tel que le poète le dépeint. De plus, l'animal, plein de vie, n'est point dans l'attitude où devrait être un cerf blessé à mort. Ne dirait-on pas plutôt une biche\footnote{Les biches qui jouent un rôle dans les fictions mythologiques des Grecs et des Romains, sont souvent représentées avec un bois de cerf.} ou un jeune cerf apprivoisé, qui tourne la tête vers Cyparisse, pour exprimer qu'il l'aime et qu'il attend de lui un regard, une caresse ? Et si c'est un cerf, comme je suis porté à le croire, ne représente-t-il pas ici symboliquement Apollon lui-même, le protecteur céleste de Cyparisse ? Dans les traditions écrites, le quadrupède tué par Cyparisse est tantôt un cerf,\footnote{Ovide, \emph{loc. cit.} Servius, \emph{ad} Virgil. \emph{Æneid.} 3., 64, 680. Lactantius Placidus, \emph{Argum.} Ovid. \emph{Metamorph.} 10., 3. \emph{Scriptores rer. mythic. latini tres}, t. 1., fab. 177., p. 134.} tantôt une biche\footnote{Servius, \emph{ad} Virgil. \emph{Georgic.} 1., 20. \emph{Scriptores rer. mythic. latini tres}, t. 1., fab. 6., p. 2 ; fab. 178., p. 135.} ; mais le trait principal du mythe, qui, dans la fresque de Pompeï, marque la métamorphose du jeune chasseur, ce trait est fidèlement reproduit dans le récit d'Ovide\footnote{Cette conformité avait tellement frappé M. Avellino, que, selon lui (Mémoire cité, p. 15), l'auteur de la fresque l'aurait peinte, l'imagination pleine des vers d'Ovide. Il a soin de dire que les inscriptions latines trouvées à Pompeï prouvent que les poètes de la haute latinité étaient parfaitement connus dans cette colonie romaine, et il renvoie à l'opuscule de M. Wodsword intitulé : \emph{Inscriptiones pompejanæ} (Lond. 1837). Toutefois les différences que je signale entre le récit du poète et l'œuvre du peintre ne me permettent pas de souscrire sans restriction à l'opinion du docte académicien de Naples.} : le peintre fait naître du sommet de la tête de Cyparisse la cime d'un cyprès pyramidal ; le poète, à son tour, après ces mots caractéristiques, \emph{metas imitata cupressus},\footnote{Il n'est pas sans intérêt de faire remarquer ici que d'autres écrivains latins, pour désigner le cyprès, emploient quelquefois, au lieu de \emph{cupressus}, le mot \emph{cyparissus}, identique avec le nom du personnage changé en cyprès par Apollon. Virgile nous en fournit deux exemples, l'un dans les \emph{Géorgiques} (2., v. 84), le second dans l'\emph{Énéide} (3., v. 680).} dépeint la métamorphose dans les vers que voici :

\begin{quotation}
In viridem verti cœperunt membra colorem ;

Et modo, qui nivea pendebant fronte capilli,

Horrida cæsaries fieri, sumptoque rigore

Sidereum gracili spectare cacumine cœlum.
\end{quotation}

\paragraph{}
On chercherait en vain dans les autres poètes ou mythographes de l'antiquité les précieux détails qui accompagnent cette fiction ; et il est difficile de ne pas reconnaître dans le récit d'Ovide, comme dans la fresque de Pompeï, un exemple de l'habitude qu'avaient les Grecs et les Romains de transformer en mythe le symbole. Remarquons de plus qu'ici l'origine asiatique de la légende de Cyparisse ne se trahit pas moins par le nom même du lieu où s'accomplit la métamorphose. Ce nom, \emph{Carthéa}, appartient à plusieurs idiomes de l'Asie occidentale ; il signifie \emph{la ville}, et dut nécessairement être imposé par une colonie phénicienne à la cité qu'elle fonda dans l'île de Céos. Le nom seul de \emph{Carthage} justifierait, au besoin, une pareille assertion.

Lactantius,\footnote{\emph{Loc. cit.}} dans l'analyse succincte qu'il fait de chaque livre des Métamorphoses d'Ovide, n'avait pas à reproduire les détails que je loue dans le récit du poète ; mais, tout en se bornant à rapporter les deux faits principaux, la coupable passion d'Apollon et la métamorphose de Cyparisse, il place aussi le lieu de la scène dans l'île de Céos ; et s'il omet de désigner nominativement la ville de Carthée, en revanche il nous montre qu'il avait puisé ses renseignements ailleurs que dans Ovide, car il nous apprend que Cyparisse avait pour père Amyclée. Cette dernière notion ne se trouve ni dans Ovide, ni dans aucun autre écrivain. Rapprochée des trois vers cités de Nonnus,\footnote{Ci-dessus, p. 201, note 4.} où Cyparisse a pour patrie la Laconie, elle nous ramène dans cette contrée ; et là nous trouvons, non-seulement une très-ancienne ville du nom de \emph{Cyparissia},\footnote{Ci-après, p. 245.} mais une Aphrodite armée, dont le culte dut être, de bonne heure, importé en Grèce par les Phéniciens, adorateurs d'Astarté armée, et une Athéné \emph{Cyparissia},\footnote{Pausanias, 3., 22., 7.} autre divinité armée, qui se confond à la fois avec la déesse \emph{Onka} ou \emph{Onga} de ces mêmes Phéniciens, et avec la Vénus Mylitta de Babylone et de Ninive.

Une autre tradition\footnote{Servius, \emph{ad} Virgil. \emph{Æneid.} 3., 680.} donne pour père à Cyparisse, au lieu d'Amyclée, Télèphe, qu'une biche, disait-on, avait nourri ; et, par une coïncidence digne d'attention, nous voyons, sur un bas-relief de la villa Borghèse,\footnote{Visconti, \emph{Monum. scelti borghesiani}, pl. 33. ; édit. de Milan.} une biche accroupie auprès d'Augé, mère de Télèphe, comme nous trouvons une biche ou un cerf accroupi auprès de Cyparisse sur la fresque de Pompeï.

Enfin quelques écrivains latins substituent le nom de Silvain ou Sylvain à ceux d'Apollon et de Zéphyre dans la légende, de Cyparisse.\footnote{Servius, \emph{ad} Virgil. \emph{Georgic.} 1., 20 ; \emph{ad Eclog.} 10., 26. \emph{Scriptores rer. mythic. latini tres}, t. 1., fab. 6., p. 2 ; fab. 178., p. 135.} Parmi ces écrivains, nous retrouvons Servius, et ici le témoignage qui résulte de sa riche érudition est d'autant plus important qu'il se produit à l'occasion de ce vers où Virgile\footnote{\emph{Georgic.} 1., 20.} nous apprend que l'on représentait Silvain portant à la main un jeune cyprès :

\begin{quotation}
Et teneram ab radice ferens, Silvane, cupressum.
\end{quotation}

\paragraph{}
Cette attribution du cyprès à Silvain est confirmée tout à la fois par la substitution de ce dieu à Apollon, dans la fable de Cyparisse, et par les monuments figurés que citent à ce sujet Heyne\footnote{\emph{Ad} Virgil. \emph{loc. cit.}} et Visconti.\footnote{\emph{Mus. Pio-Clement.} t. 7., explication de la pl. 10. (Cf. le Dr. Frédéric Creuzer, \emph{Symbolik}, t. 3., p. 843 ; 3\textsuperscript{e} édition.)} A ces monuments, selon la judicieuse observation de M. Avellino, il faut ajouter une inscription latine, métrique, composée en l'honneur du dieu Silvain et découverte dans l'Abruzze citérieure, sous le château de Capistrano, diocèse de Valva.\footnote{Copiée par l'abbé Costanzo, cette inscription fut publiée dans les \emph{Papiri diplomatici} (p. 332) de Marini, puis dans les \emph{Nouvelles annales de l'Institut archéologique} (vol. 6., p. 150), ou M. Avellino (mémoire cité, p. 27, note 1) est fondé à dire qu'elle n'aurait pas dû être annoncée comme inédite. Elle a, plus récemment, été reproduite par M. Liberatore, dans ses \emph{Opuscoli} (p. 112).} Le vers cité des Géorgiques de Virgile est introduit dans cette inscription.

L'identité de Silvain et d'Apollon ressort donc à la fois, on le voit, de la substitution du premier au second dans la légende de Cyparisse, et de l'attribution du cyprès à Silvain, comme à Apollon. Le culte de Silvain, divinité fort énigmatique, dont nous ne découvrons en Grèce aucune trace, avait-il été importé de l'Orient chez les Romains\footnote{Voyez, sur le dieu Silvain, les inscriptions latines que M. le Dr. Jean Labus a rapportées dans ses \emph{Marmi antichi bresciani}, et les savantes observations dont il a accompagné ces inscriptions (p. 74-77).} ? ou bien ceux-ci trouvèrent-ils ce culte établi chez les anciens peuples d'Italie, à qui une émigration asiatique l'aurait apporté en même temps que le symbole du cyprès ? Ce sont des questions insolubles, ce me semble, dans l'état actuel de nos connaissances. Bornons-nous à remarquer que le vers cité des Géorgiques nous rappelle un usage originaire de l'Asie occidentale, comme le cyprès ; un usage simultanément constaté par des cylindres asiatiques et par des monuments égyptiens, où nous avons vu un jeune cyprès placé à la main, soit d'un prêtre, soit d'un initié, soit d'une divinité chargée de présider à la vie et au printemps.\footnote{Ci-dessus, p. 64, 165, 166, 183-185.} Mais complétons cette remarque en transcrivant ici le commentaire entier de Servius sur le même vers de Virgile, commentaire où Silvain, à l'exemple des divinités de l'Asie occidentale,\footnote{Voy. \emph{Mém. de l'Acad. des inscriptions}, nouv. série, t. 14., 2\textsuperscript{e} partie, p. 68, 173. \emph{Recherches sur Vénus}, p. 43, 44, 72, 74, 75, 226, 227, 238-240.} se montre avec le double caractère de dieu des vivants ou de la terre et de dieu des morts ou des enfers : « Silvain est le dieu des forêts. Il aima un jeune homme, nommé Cyparisse, qui avait une biche parfaitement apprivoisée. Le dieu, sans le savoir, ayant tué cette biche, Cyparisse en mourut de douleur. Silvain, qui le chérissait,\footnote{Mot à mot : « le dieu amoureux. »} le changea en cyprès, arbre ainsi appelé du nom de Cyparisse ; et, pour se consoler, dit-on, il porte à la main un arbre de cette espèce. Quelques-uns pensent que Silvain est une divinité infernale,\footnote{Mot à mot : « un dieu funèbre. »} et, pour cette raison, ils lui attribuent le cyprès, qui, à cause de la faiblesse de ses racines, convient aux rites funéraires.\footnote{« Silvanus deus est silvarum. Hic amavit puerum Cyparissum nomine, qui habebat mansuetissimam cervam. Hanc quum Silvanus nescius occidisset, puer est extinctus dolore : quem amator deus in cupressum arborem nominis ejus vertit, quam pro solatio portare dicitur. (Hunc Silvanum quidam funebrem deum putant, et ideo cupressum tribuunt ei, quia eadem arbor apta sit funeri ob radicum infirmitatem). » Servius, \emph{ad} Virgil. \emph{Georgic.} 1., 20. --- Ces mêmes renseignements se trouvent, mot pour mot, dans le premier et le second des trois mythographes latins dont les écrits ont été retrouvés à la bibliothèque du Vatican, par les soins de S. Em. Mgr. le cardinal Maï. (Voy. \emph{Script. rer. mythicar. latini tres Romæ nuper reperti}, t. 1., fab. 6., p. 2 ; fab. 178., p. 135 ; ed. Bode.)} »

Que le dieu qui change Cyparisse en cyprès s'appelle Silvain ou Apollon, cette métamorphose ne cache pas moins un sens élevé, que mes savants devanciers ne semblent pas avoir pénétré à travers la fiction érotique inventée par l'imagination sensuelle et dépravée des Grecs. On n'a pas compris que Cyparisse est un initié qui atteint, non sans de pénibles efforts, le but que s'étaient proposé les fondateurs des mystères. Son âme est restée victorieuse dans le combat qu'elle avait à livrer à la matière, source de toutes les passions charnelles et déréglées ; et cette victoire est représentée symboliquement par la mort que Cyparisse donne lui-même au cerf ou à la biche qu'il chérissait. Le cerf, dans certains rites propres à l'institution des mystères chez les Assyriens et les Phéniciens, se substitue au taureau.\footnote{De là, cette double attribution du cerf et du taureau à Apollon et à Artémis ou Diane ; de là, sur des bas-reliefs assyriens et sur des pierres gravées ou des scarabées de travail phénicien, ces groupes composés d'un ou deux lions, d'un ou deux griffons, qui dévorent un cerf au lieu d'un taureau ; de là aussi, des têtes de cerf substituées à des cornes ou à une tête de taureau, comme ornement des tiares des rois parthes.} Dominé par des désirs impétueux, emporté par sa grande vigueur, furieux même au temps des amours, le cerf, animal herbivore comme le taureau, est, comme lui aussi, le symbole du principe humide,\footnote{Voy. \emph{Rech. sur Vénus}, p. 156-162, 191, 192, 220-225.} et, par conséquent, le symbole de la vie matérielle, le symbole des dieux créateurs ou qui dispensent la vie.\footnote{Que si l'on objectait que, dans le récit d'Ovide et sur la fresque de Pompeï, je prends le cerf des champs carthéens pour l'image symbolique d'Apollon, je demanderais si Zeus ou Jupiter, Osiris, Çiva, Dionysus, ne se métamorphosent pas en taureau sous la plume des mythographes et des poètes, ou ne sont pas représentés avec les formes de cet animal dans les monuments de l'art.} Mais il a sur le taureau l'avantage de marquer, tout à la fois par son rut et par la perte annuelle et périodique de son bois, l'époque du printemps, cette saison si impatiemment attendue, où, chaque année, une nouvelle vie se manifeste à la surface de la terre.\footnote{On voit encore une fois ici que les institutions religieuses et le langage symbolique des Chaldéens ou des Assyriens n'avaient rien emprunté aux phénomènes climatériques de l'Égypte, ni aux productions de son sol.} Dans le langage symbolique des mystères, dompter et tuer le cerf ou le taureau, c'est donc la victoire de la vie spirituelle sur la vie charnelle, la victoire de l'âme sur le corps. Ce triomphe appartient à Cyparisse, dont les habitudes pieuses et la pureté de mœurs nous sont attestées par les légendes grecques qu'Ovide et Servius avaient eues sous les yeux. Sorti victorieux des épreuves et des combats institués, en Orient, dans les initiations, il meurt, c'est-à-dire il achève, par la mort de son corps, de rendre à son âme la liberté qu'elle avait perdue. Comme Hercule, après que sa dépouille mortelle eut été consumée sur le bûcher mystique, il reçoit, avec le don de l'immortalité, les honneurs insignes de l'apothéose. Il les reçoit d'un dieu solaire, Apollon ou Silvain, qui, comme Mithra, modèle de perfectibilité proposé aux initiés, préside à la vie future et à la célébration des mystères. La métamorphose de Cyparisse en cyprès, symbole d'Apollon et de Silvain, symbole de la vie éternelle ou de l'immortalité, est donc une véritable apothéose.\footnote{M. Émeric David et M. Avellino reconnaissent bien que le cyprès est un symbole solaire, et le cerf ou la biche, un symbole du principe humide ; mais ils ne tirent pas de ces deux faits les conséquences qu'il convient d'en déduire, pour les appliquer à l'interprétation du mythe de Cyparisse. L'académicien napolitain n'entre même pas dans l'examen du sens psychologique de ce mythe. L'académicien français, j'en fais à regret l'observation, se trompe lorsqu'il dit (\emph{Jupiter}, t. 2., p. 436, 437) : « Ce n'est point Cyparisse qui fut changé en cyprès ; c'est, au contraire, l'amour d'Apollon pour le cyprès qui donna naissance à la fable de Cyparisse ... »} Je ne sais si je m'abuse, mais il me semble que la fresque de Pompeï, monument où se révèle un sentiment profondément religieux, confirme le sens élevé que je prête au mythe de Cyparisse.

L'attribution du cyprès à Apollon, chez les Grecs d'Europe, de même que chez les Grecs asiatiques, est, on le voit, pleinement justifiée par le témoignage indirect de la légende de Cyparisse où nous lisons le nom de Silvain, comme par les témoignages directs que j'ai rapportés plus haut. Je ne puis produire un nombre aussi imposant de preuves en faveur d'une attribution semblable à Artémis, la Diane grecque, sœur et compagne inséparable d'Apollon. Mais le fait n'est pas moins certain ; il pourrait même être affirmé par voie d'analogie ou de consanguinité, à défaut de preuves.

Déjà nous avons appris de Pausanias que, sur la route d'Argos à Tégée, au milieu du \emph{cupressetum} planté sur le mont Lyconé, on avait érigé, en l'honneur d'Artémis \emph{Orthia}, un temple qui renfermait les statues de Latone, d'Apollon et d'Artémis.\footnote{Ci-dessus, p. 192, 193.}

Dans la Phocide, à Œanthéa, où Aphrodite avait un temple, le même voyageur\footnote{Pausanias, 10., 38., 5.} vit, un peu au-dessus de la ville, un bois sacré de cyprès et de pins, et, dans ce bois, un temple et une statue d'Artémis.

Xénophon, exilé de sa patrie, se retire à Scillonte,\footnote{Ou \emph{Scillunte.} Ce fut là qu'il écrivit sa célèbre histoire de la retraite des dix mille.} ville bâtie par les Lacédémoniens aux environs d'Olympie. Par l'ordre même des dieux, il y achète, pour le consacrer à Artémis, un territoire qu'arrosait un petit fleuve nommé \emph{Sélinus}, comme celui qui coulait près du temple de la déesse, à Éphèse. Là, avec les dîmes des richesses rapportées de son expédition en Asie, Xénophon érige un temple, une statue et un autel à Artémis. « Le temple, nous dit-il lui-même, ressemble en petit à celui d'Éphèse ; mais à Éphèse, la statue de la déesse est d'or ; ici, elle est de cyprès.\footnote{\textgreek{Ὁ δὲ ναὸς ὡς μικρὸς μεγάλῳ τῷ ἐν Ἐφέσῳ εἴκασται, καὶ τὸ ξόανον ἔοικεν ὡς κυπαρίττινον χρυσῷ ὄντι τῷ ἐν Ἐφέσῳ.} (\emph{Cyri expeditio}, 5., 3., 12 ; ed. Lud. Dindorf.) --- Les écrivains latins ne sont d'accord, ni entre eux, ni avec Xénophon, sur la matière qui avait servi à faire la statue de Diane à Éphèse. Pline (\emph{H. N.} 16., 79.) rapporte que, selon la plupart, cette statue était d'ébène, mais que, selon Mucien, le plus récent des écrivains qui l'avaient vue sur les lieux, elle était de bois de vigne, et n'avait jamais été changée, bien que le temple où elle était enfermée eût été rebâti jusqu'à sept fois. Mucien, ajoute Pline, nomme le sculpteur qui avait choisi le bois de vigne pour la statue : il s'appelait Pandémios.} »

Ovide, dans ses Métamorphoses,\footnote{3., fab. 2., 155, 156.} représente comme couverte d'une épaisse forêt de cyprès\footnote{\emph{Acuta densa cupressu.}} la vallée de Gargaphie\footnote{Ou \emph{Gargaphia}.} consacrée à Diane, qui s'y baignait avec les nymphes lorsqu'elle fut surprise par Actéon. La détermination de la position de cette vallée dans la Béotie n'offre point de difficulté. M. Panofka\footnote{\emph{Antiques du cabinet Pourtalès}, p. 53.} me semble fondé à la placer au pied même du mont Cithéron. Décrivant un bas-relief de sarcophage,\footnote{Ce sarcophage, découvert en 1738, dans une vigne appartenant à l'illustre famille des Borghèse, est passé de la villa Borghèse, autrefois villa Pinciana, au musée du Louvre. Il est de marbre de Luni, et a bien souvent été décrit ou figuré (voyez notamment : \emph{Villa Pinciana}, st. 7., n°\textsuperscript{s} 16 et 17. Millin, \emph{Monum. inéd.} t. 1., p. 40 ; \emph{Galerie mytholog.} 100., 101., 405, 406, 407. Bouillon, \emph{Musée des antiq.} t. 3., bas-reliefs, pl. 17. Clarac, \emph{Mus. de sculpt. anc. et mod.} t. 2., 1\textsuperscript{re} part. n° 315, p. 321-330 ; pl. 113 et 114). Je reparlerai de ce précieux monument dans mon second mémoire. Feu M. Avellino, le trouvant décrit dans l'ouvrage de M. Panofka sur le cabinet de M. le comte de Pourtalès-Gorgier, a commis l'erreur de le désigner comme appartenant à ce riche cabinet. Le savant académicien de Berlin le mentionne à l'occasion d'un vase peint grec, qui déjà avait été publié par feu le comte Alexandre de Laborde (\emph{Collection du comte de Lamberg}, p. 37, vignette), et qu'il reproduit (ouvrage cité, p. 53-57, pl. 21.) d'après l'original que possède M. de Pourtalès.} où l'on voit Actéon attaqué par quatre de ses propres chiens, au moment où la déesse offensée le métamorphose en cerf, il dit : « Des cyprès et des pins contribuent à désigner le mont Cithéron, sur lequel ces deux espèces d'arbres croissaient en abondance. » Toutefois cette dernière assertion n'est pas justifiée par les deux passages, l'un d'Ovide,\footnote{\emph{Metamorph.} 3., v. 155.} l'autre de Philostrate,\footnote{\emph{Imagin.} 1., 14.} auxquels il renvoie le lecteur. Si le poète latin et l'écrivain grec ne nous apprennent rien sur l'emplacement précis de la vallée et de la fontaine de Gargaphie, le second fait mention d'Actéon à propos d'une fontaine ou d'une source qui doit être celle de Gargaphie, située près du mont Cithéron, bien qu'il ne la désigne par aucune dénomination.

En diverses parties de la Grèce, d'anciens usages, d'anciens monuments de l'art, d'anciens noms de lieux ou de cités vont nous révéler quelle importance, quelle extension avaient acquises, chez les Hellènes, le culte et la culture du cyprès pyramidal, ou plutôt le culte et les rites des divinités dont cet arbre vivace et toujours vert était le symbole ou l'image idéographique. Dans cette revue, quelquefois fastidieuse, nous recueillerons, du moins, des témoignages qui attestent qu'en Grèce le cyprès ne fut pas seulement consacré à Apollon et à Artémis, mais, pour ainsi dire, à toutes les divinités créatrices, aux divinités célestes, comme aux divinités terrestres et infernales. Nous trouverons enfin, dans les récits des auteurs anciens, ou dans la composition des monuments d'antiquité figurée, quelques particularités importantes qui ne nous permettront pas de douter que le culte du cyprès ne fût passé de l'Asie antérieure chez les Grecs avec le culte des dieux que révéraient les peuples d'Orient les plus anciennement civilisés.

C'est par l'intermédiaire des îles situées entre le continent asiatique et la Grèce que se fit cette double importation. Les documents écrits, de même que les monuments de l'art, mettent ce fait en évidence. Déjà nous venons de trouver le mythe de Cyparisse transporté de Syrie, où il avait pris naissance, dans l'île de Céos, auprès d'une cité appelée d'un nom qui, dans un des idiomes de l'Asie occidentale, signifie ville. Déjà aussi j'ai cité, quant à l'île de Cypre, des traditions et deux coupes d'argent doré, ciselées, ou gravées, qui prouvent que, dès une haute antiquité, le culte du cyprès y avait été introduit. À défaut de ces preuves, le nom de \emph{Cypre},\footnote{Cette île, avant d'avoir reçu le nom de la déesse même, Vénus-Cypris, dont le culte y fut introduit par les Phéniciens, avait successivement porté plusieurs autres noms. Le témoignage de Pline l'Ancien (\emph{Hist. nat.} 5., 35., 1) est formel à cet égard.} rapproché, comme je l'ai fait, du nom de la déesse \emph{Cypris}, du nom de l'arbre qui nous occupe, et du nom du métal consacré à cette divinité, suffirait pour nous donner lieu de croire que les cyprès, dont la cime pyramidale s'élance au-dessus de tous les autres arbres dans les forêts ou les bois qui, sur le sol de l'île, attirent les regards du voyageur moderne, sont autant de rejetons des plants de cyprès que les Phéniciens durent autrefois apporter d'Ascalon en Cypre avec le culte de la Vénus Assyrienne. L'importation de ce culte est attestée, je le répète, par le témoignage formel d'Hérodote\footnote{1., 105.} et de Pausanias.\footnote{1., 14., 6.} Et nous ne pouvons trouver la Vénus assyrienne représentée sous l'emblème d'un cyprès, au revers des médailles d'une autre île phénicienne, Aradus,\footnote{Ci-dessus, p. 83-85.} et son culte introduit avec celui du cyprès dans la Phénicie continentale, en Syrie, en Arménie, dans l'Asie Mineure, en Arabie et en Égypte, sans admettre que les mêmes faits s'étaient produits lors de la colonisation de l'île de Cypre par les Phéniciens. Au témoignage formel des deux coupes citées, nous aurions à ajouter plus d'une preuve matérielle, si la numismatique de cette île, trop souvent conquise et dévastée depuis cette époque, était plus complète qu'elle ne l'est dans nos collections d'antiquités.

Au-dessus de l'île de Cypre, nous trouvons celle de Rhodes, où les médailles et les traditions s'accordent à montrer que les croyances religieuses du continent asiatique avaient jeté de profondes racines. Le culte d'Aphrodite et d'Éros y tenait le premier rang, et là, sans doute, le symbole du cyprès marquait, comme la rose, l'origine phénicienne ou assyrienne de ce culte. Toutefois les historiens et les poètes ne me laissent pas autre chose à dire sur la question qui nous intéresse, si ce n'est que l'île de Rhodes était renommée pour ses bois de cyprès.\footnote{Théophraste, \emph{Histor. plantar.} 4., 6.}

Samos, cette île dont les traditions et les médailles autonomes conservaient, sous la domination successive des Grecs et des Romains, l'empreinte d'une civilisation asiatique ; cette île où Héra, par la double attribution du lion et du taureau,\footnote{Voy. \emph{Mémoires de l'Acad. des inscr.} t. 15., 2\textsuperscript{e} partie, p. 93, 94 ; \emph{Recherch. sur Vénus}, pl. 3 B, n°\textsuperscript{s} 9, 11 et 13, pl. 18., n° 7 ; et rapprochez les médailles autonomes de Cnide (Carie), \emph{Rech. sur Vénus}, pl. 3 B, n°\textsuperscript{s} 12 et 14-18. Sestini, \emph{Descr. dell. med. ant. gr. del Mus. Hederv.} 2., 223, n° 9.} se confondait, comme en Syrie, avec la Vénus assyrienne, Samos me donne lieu de remarquer que, selon le témoignage formel de Pline,\footnote{\emph{Hist. nat.} 5., 37., 1.} elle s'était primitivement appelée \emph{Cyparissia}. Je n'ignore point que cette dénomination a été jugée \emph{fabuleuse} par un habile archéologue de Berlin,\footnote{M. Th. Panofka, \emph{Res Samiorum}, p. 8, 9.} qui a fait de l'histoire des Samiens l'objet d'une étude particulière. Mais son opinion n'a pas obtenu l'assentiment de feu M. Avellino\footnote{Mémoire cité, p. 11.} ; et il me sera peut-être permis d'ajouter que, selon toute probabilité, M. Panofka eût modifié son sentiment, et M. Avellino insisté davantage sur l'argument que lui fournissait le passage de Pline, si, remontant l'un et l'autre à l'origine du nom de \emph{Cyparissia}, ils avaient tenu compte des faits qui se rattachent directement ou indirectement à l'histoire du cyprès. Ils eussent, par exemple, remarqué que cet arbre avait nécessairement dû être attribué à Héra et à Junon, comme nous avons vu, et comme nous verrons plus loin, qu'il le fut à Zeus et à Jupiter. Ils en avaient même une preuve directe et formelle dans un passage où Tite-Live\footnote{27., 37. --- Cf. Obsequens, \emph{De prodig.} 106.} rapporte qu'à Rome on conservait, dans un temple érigé sur le mont Aventin, deux statues de \emph{Junon Reine} (\emph{Juno Regina}) d'origine étrusque et faites avec du bois de cyprès. Ils eussent remarqué, de plus, que l'île de Samos s'était aussi appelée \emph{Parthénia},\footnote{Varron cité par Lactance, \emph{De fals. relig.} 1., 17. Virgile, \emph{Æneid.} 1., 15, 16. Étienne de Byzance, \emph{voc.} \textgreek{Σάμος}.} et que si Héra, qui passait pour y être née et y avoir épousé son frère Zeus, reçut de cette île le surnom de \emph{Samia}, elle avait pu, à des époques antérieures, porter ceux de \emph{Parthénos} (vierge) et de \emph{Cyparissia}. D'une part, ces deux derniers surnoms expliqueraient les noms de \emph{Parthénia} et de \emph{Cyparissia} qui furent donnés à l'île natale de la déesse\footnote{Ovide (\emph{Metamorph.} 8., v. 220) désigne l'île de Samos sous le nom d'\emph{Insula junonia}.} ; d'autre part, ils confirmeraient l'identité d'Héra avec Athéné et Vénus-Uranie, toutes deux réputées \emph{vierges}, et surnommées, l'une \emph{Cyparissia},\footnote{Ci-dessus, p. 205 ; ci-après, p. 243.} l'autre \emph{Cypris}. Enfin, les deux savants académiciens eussent remarqué que l'attribution commune du cyprès à chacune de ces trois divinités est une conséquence de leur identité primitive, et une confirmation du passage controversé de Pline. N'allons-nous pas tout à l'heure trouver, en Étrurie et en Picentine, Héra ou Junon adorée sous le nom de \emph{Cypra} ou \emph{Cupra}, et ce nom donné même à plusieurs villes de la Picentine qui rendaient à cette déesse un culte particulier ?

L'île de Crète, plus voisine du continent grec que l'île de Samos, et célèbre par plusieurs mythes dont l'origine assyrienne ou phénicienne ne saurait être contestée et sera de ma part, dans un autre travail, l'objet d'un examen approfondi ; l'île de Crète, dis-je, est le lieu auquel se rattache une des plus anciennes légendes de la mythologie grecque. Diodore de Sicile,\footnote{5., 66., 1 ; ed. C. Müllero.} dans un passage dont j'ai déjà cité quelque chose, rapporte, en effet, une tradition qui avait cours chez les Crétois et qui confirme mon observation. Ces insulaires racontaient que les Titans avaient habité le territoire de Cnosse. On voit encore sur ce territoire, ajoute l'historien sicilien, les fondations et la cour de la maison où vécut Rhéa, ainsi qu'un bois de cyprès, d'une haute antiquité.\footnote{... \textgreek{καὶ κυπαρίττων ἄλσος ἐκ παλαιοῦ χρόνου ἀνειμένον}. --- Cf. les vers de Virgile cités plus haut, p. 80, 81.} Or nous avons déjà recueilli, dans les autres textes et dans les monuments figurés que j'ai produits, plus d'une preuve de l'attribution du cyprès à la Déesse de Syrie et à Rhéa, divinités primitivement identiques. D'autre part, Zeus, qu'on disait avoir été élevé dans l'île de Crète, était adoré en Syrie et dans l'Asie Mineure sous l'emblème d'un cyprès ; en Occident, on lui érigeait des statues faites avec le bois de cet arbre. Selon Pythagore, le sceptre de ce dieu est aussi de cyprès.\footnote{Jamblique, \emph{in Vita Pythag.} 28. Hermippe, \emph{apud} Diogen. Laërt. 8., 10. --- M. Boeckh (\emph{ad} Pindar. \emph{Fragm.} p. 631) et M. Lobeck (\emph{Aglaoph.} p. 896) adoptent l'opinion d'Aldobrandini, qui, dans ses notes sur Diogène de Laërte, veut qu'on ait donné à Jupiter un sceptre de bois de cyprès, par la seule raison que ce dieu était né dans l'île de Crète, patrie de cet arbre. Mais l'attribution du cyprès aux divinités asiatiques que j'ai désignées dans la première section de ce mémoire, prouve qu'il faut chercher ailleurs que dans l'île de Crète l'origine et le motif de l'usage dont il s'agit. Bochart (\emph{Geograph. sacr.} 1., 4.) est bien plus dans le vrai, ce me semble, lorsqu'il dit qu'aux mains de Jupiter un sceptre de bois de cyprès était le symbole de la durée éternelle de son empire. Il faut louer Meursius d'avoir approuvé cette interprétation. Ils auraient pu, l'un et l'autre, ajouter que ce sceptre de cyprès marquait aussi que Jupiter est le souverain maître de la vie.} C'est une modification grecque de l'usage oriental de placer un jeune cyprès à la main des divinités, ou des prêtres qui les représentent. Enfin, d'après les légendes rapportées plus haut, Cyparisse passait tantôt pour être né dans l'île de Crète, tantôt pour être né en Syrie. A ces divers témoignages, il faut ajouter que la culture du cyprès était si répandue en Crète\footnote{On sait aussi, par le témoignage d'Athénée (\emph{Deipnosoph.} 5., 10.), que la ville de Milet, colonie primitivement fondée par les Crétois, sur les côtes de la Carie, fournissait au commerce du bois de cyprès, qui jouissait d'une grande réputation.} et remontait à une époque si reculée, que les auteurs grecs et les auteurs latins s'accordent à considérer cette île comme la patrie du cyprès.\footnote{Théophraste, \emph{Hist. plantar.} 4., 1., 6. Virgile, \emph{Georgic.} 2., 84. Vitruve, 2., 9. Pline, \emph{Hist. nat.} 16., 33., 60. Athénée, \emph{Deipnosoph.} 1., p. 27 \emph{f.} Plutarque, \emph{Sympos.} 1., problem. 2, n° 5. Philostrate, \emph{Imagin.} 1., 9. Claudien, \emph{De rapt. Proserpin.} 3., 370-372. Cf. Eurip. \emph{ap.} Porphyr. \emph{De abstin.} 4., 19 ; ed. Rhoer. Bod. \emph{ad} Theophrast. \emph{loc. cit.} Meursius, \emph{in Creta}, p. 105. Spanhem. \emph{ad} Callimach. p. 159, 160. --- M. Avellino (mémoire cité, p. 4, note 3) fait remarquer que Saumaise a eu tort de vouloir lire, dans le passage cité de Pline, \emph{in Tarento}, au lieu de \emph{in Creta}, et qu'il a été justement repris à ce sujet par le P. Hardouin. Mais il omet de dire que les éditeurs postérieurs de Pline ont conservé le passage où ce naturaliste dit : « La Crète est la patrie du cyprès, que pourtant Caton qualifie de tarentin, sans doute parce que Tarente fut la première ville où on l'importa. » Les voyageurs modernes s'accordent à dire, avec Théophraste et Pline, que le cyprès est très-répandu dans certaines parties de l'île de Crète (Candie), et qu'il y croît même, au milieu des neiges, sur les plus hautes montagnes (voy. Tournefort, \emph{Voyage dans le Levant}, t. 1., p. 32. Sieber, \emph{Reise nach der Insel Kreta} (Leipzig, 1833), t. 2., p. 86, 87. Cf. M. Carl Ritter, \emph{Die Erdkunde}, t. 11., p. 581, 582). Les montagnes désignées, à ce sujet, par les écrivains anciens, sont les \emph{Monts Idéens} et les \emph{Monts blancs} (\textgreek{λευκοί}), ces derniers probablement ainsi nommés à cause des neiges perpétuelles qui couvraient leurs sommets. Les uns et les autres nous rappellent l'épithète de \textgreek{κυπαρισσόκομος}, qui nous est connue par les anciens scoliastes d'Homère (\emph{ad Iliad.} Δ, 131 ; 2., 797, etc.). Voy. \emph{Thesaur. ling. græc. sub voce} ; édition de MM. Hase et Dindorf frères.} On précise davantage : on désigne nominativement la ville crétoise de Tarrha,\footnote{Théophraste, \emph{loc. cit.} --- Cet auteur avance même qu'à Tarrha le cyprès recevait l'épithète de \textgreek{κουριζομένη}, parce que, de quelque manière qu'il eût été coupé, il repoussait du tronc et même des racines. Ceci n'est nullement justifié par l'expérience, pas plus pour le cyprès que pour les autres arbres verts ; ce qui n'a pas empêché Pline (\emph{ubi supra}) de dire, de son côté, que, dans l'île d'Énarie, le cyprès repousse après avoir été coupé.} dont les habitants honoraient d'un culte particulier Apollon,\footnote{K. O. Müller, \emph{Die Dorier}, t. 1., p. 207.} à qui les monuments figurés des peuples de l'Orient et des peuples d'Europe attribuent le cyprès, en même temps que les traditions écrites font jouer à ce dieu et à son symbole un si grand rôle dans la légende de Cyparisse.

Bien que la Sicile ne soit pas située sur les mers que nous parcourons en ce moment, je demande à l'Académie la permission d'appeler son attention sur une idylle de Théocrite qui nous a conservé le souvenir de l'importation du cyprès dans cette île célèbre. Ce charmant poète, décrivant la grotte où Polyphème supplie Galathée de venir partager son ardeur amoureuse, dit expressément que cette grotte était ombragée par des lauriers et par des cyprès élancés.\footnote{\textgreek{Ῥαδιναὶ κυπάρισσοι}. (\emph{Idyll.} 11., v. 45.)} Mais si le personnage de Polyphème, comme chaque cyclope, comme la légende d'Ulysse et Polyphème, introduite par Homère dans l'Odyssée, appartient à des traditions dont l'origine asiatique ne saurait être méconnue ; si Théocrite, pas plus qu'Homère décrivant la grotte de Calypso,\footnote{\emph{Odyss.} 5., v. 64.} n'oublie ici de faire mention des cyprès, n'est-il point singulier que cet arbre, qui joue un si grand rôle dans les antiquités figurées des Phéniciens et des Syriens, ne se soit pas encore rencontré sur les monuments de la Sicile ? N'y a-t-il pas lieu aussi d'être surpris de ne trouver que dans un seul auteur la preuve de l'importation du cyprès dans une île successivement occupée par les Phéniciens et les Grecs ? Ces deux espèces de problèmes semblent pouvoir se résoudre en admettant que le palmier, qui, dans l'Asie occidentale, partageait avec le cyprès les honneurs d'un culte symbolique, se substitua complètement à ce dernier arbre.

Si de l'archipel grec, de l'île d'Ogygie et de la Sicile, revenant sur le continent grec, nous abordons en Béotie et parvenons à ces lieux où jadis les Phéniciens, conduits par Cadmus, bâtirent une ville et des temples, qu'ils consacrèrent à la Vénus assyrienne, nous y découvrons sans peine des traces certaines de l'importation du culte du cyprès. Déjà, dans la Phocide, province voisine de la Béotie, nous avons constaté qu'à Delphes on avait revêtu de lambris de cyprès le temple d'Apollon, ou, du moins, une des \emph{cella} de cet édifice ; et que, sur le mont Parnasse, il avait existé, au milieu d'un \emph{cupressetum}, une ancienne ville, primitivement appelée \emph{Cyparissia}.\footnote{Ci-dessus, p. 194.} Je ne m'arrêterai pas à rechercher si du bois de cyprès, qui avait servi à construire les galères des Phéniciens, fut, comme on l'assure, employé à fabriquer les trois \emph{xoanon} qu'Harmonie consacra à Vénus.\footnote{Pausanias, 9., 16., 2.} Mais je dirai, avec toute certitude, qu'à Tyr le cyprès et le cèdre entraient principalement dans la construction des vaisseaux, et que ces deux espèces de bois provenaient des forêts du mont Sénir,\footnote{Plus tard on l'appela le mont Hermon.} dans le Liban. C'est Ézéchiel\footnote{27., 5.} qui nous donne ce renseignement.\footnote{Cet usage s'introduisit en Occident. Du moins voyons-nous Platon (\emph{De legib.} 4.) le recommander particulièrement aux constructeurs de navires.} Théophane,\footnote{\emph{In Vita Anastasii} (Baron. \emph{Annal.}, ad A. 714). Théophane dit formellement que les Sarrasins d'Alexandrie allaient couper en Phénicie et dans l'île de Cypre les bois de cyprès dont ils avaient besoin pour leur flotte.} à son tour, nous apprend que, de son temps, en Orient, on continuait à construire les navires avec du bois de cyprès. Le même usage, à l'époque d'Alexandre le Grand, se pratiquait dans les ports du golfe Persique ; car c'est à l'embouchure du Tigre que ce prince, voulant conquérir l'Arabie, fit réparer sa flotte avec du bois de cette espèce, tiré de la Babylonie, où les forêts de cyprès abondaient, tandis que le pays était dépourvu d'autres arbres propres aux constructions navales.\footnote{Aristobule, cité par Arrien, 7., 19., 4 ; ed. Dübner. --- Strabon (\emph{Geogr.} 16., p. 741) dit même qu'Alexandre fit construire une partie de sa flotte dans la Babylonie avec des cyprès tirés des forêts et des \emph{paradis}, ce qui, pour le dire en passant, montre que les plantations des parcs ou des jardins royaux ne furent pas épargnées par le vainqueur.}

Dans l'Arcadie, cette province si riche en souvenirs asiatiques, nous voyons Hercule vaincre le terrible sanglier d'Érymanthe, montagne qui, de même que le Cyllène, reçoit d'Ovide\footnote{\emph{Epist.} 9., 87.} l'épithète de \emph{cupressifère}. Un tel combat, dans la légende de ce héros, nous avertit suffisamment que les flancs de l'Érymanthe recelaient une grotte consacrée à la célébration de mystères originaires de l'Asie occidentale. Dès lors, si Ovide nous montre cette montagne plantée de cyprès,\footnote{Horace pensait sans doute aux cyprès du mont Érymanthe lorsqu'il exprime cette opposition :\\\hspace*{10mm}Nigris aut Erymanthi\\\hspace*{10mm}Silvis, aut viridis Cragi.\\\hspace*{15mm}(1., \emph{Od.} 21., 7 et 8.)\\\hspace*{5mm}M. Döring, dans ses savants commentaires sur les poésies d'Horace, ne paraît pas s'être rappelé qu'Ovide donne au mont Érymanthe l'épithète de \emph{cupressifère}.} il ne faut oublier ni les cylindres asiatiques, ni les médailles lydiennes, où nous avons remarqué un cyprès placé, comme image symbolique de la Vénus assyrienne, au milieu de scènes d'initiation.\footnote{Voy. pl. 9., n°\textsuperscript{s} 1, 2, 4 ; pl. 4., n° 3.} Nous retrouverons plus loin des arbres de cette espèce sur des coupes d'origine phénicienne, qui, comme les deux coupes citées de l'île de Cypre,\footnote{Ci-dessus, p. 95.} représentent également des initiations aux mystères de la même déesse.

Sans quitter l'Arcadie, nous pouvons recueillir un témoignage non moins direct de l'alliance du culte du cyprès avec le culte de Vénus. Un temple et une statue avaient été érigés en l'honneur de la déesse au milieu d'un bois sacré de cyprès, planté vers le confluent du Lymax avec la Néda ; et là, bien que Vénus fût adorée sous un nom dont nous ne connaissons pas l'équivalent dans sa légende orientale, elle était représentée sous une des formes qui, empruntées à l'iconographie religieuse des Assyriens et des Phéniciens, servaient à exprimer l'antique croyance, que les premiers êtres vivants naquirent au sein des eaux ou de la mer. « A environ douze stades au-dessus de Phigalie, dit Pausanias,\footnote{8., 41., 4.} il y a des bains chauds, et le Lymax se jette dans la Néda, à peu de distance de ces bains. On voit, vers leur confluent, un temple d'Eurynome, qui, depuis longtemps, est en grande vénération. Situé dans un endroit très-escarpé et d'un abord très-difficile, il est entouré de beaucoup de cyprès, très-serrés les uns contre les autres. Le peuple, à Phigalie, croit qu'Eurynome est un surnom d'Artémis ; mais ceux d'entre eux qui connaissent les écrivains anciens disent qu'Eurynome était fille de l'Océan, et qu'Homère en fait mention dans l'Iliade, lorsqu'il dit qu'Héphæstus fut recueilli par Eurynome et Thétis. Le temple d'Eurynome ne s'ouvre qu'un seul jour dans l'année ; il reste fermé tout le reste du temps. Le jour qu'on l'ouvre, on y offre des sacrifices tant au nom de la ville qu'au nom des particuliers. Comme je ne m'y suis pas trouvé dans le temps de la fête, je n'ai pu voir la statue d'Eurynome ; mais j'ai appris des Phigaliens que ce \emph{xoanon}\footnote{Statue de bois. Voy. plus loin la note 3 de la page 224.} est lié avec des chaînes d'or ; qu'il représente une femme jusqu'aux cuisses, et que le reste du corps a la forme d'un poisson ; ce qui indique, d'une manière bien évidente, une fille de l'Océan, habitant le fond de la mer avec Thétis. Mais on ne voit pas, ajoute le voyageur grec, comment, et d'après quelle tradition vraisemblable, on aurait pu attribuer cette forme à Artémis. » Pausanias, il faut bien le dire, ne nous donne pas ici une haute idée de son savoir, lorsqu'il repousse une tradition fondée tout à la fois sur l'identité primitive de Vénus et de Diane chez plusieurs peuples de l'Asie occidentale, et sur les rapports que la théologie assyrienne avait établis entre l'eau ou la mer et ces deux divinités.\footnote{Ce sont précisément les rapports d'Artémis avec le principe humide que K. O. Müller (\emph{Die Dorier}, t. 1., p. 376) prend en considération pour justifier les Phigaliens d'identifier cette déesse avec Eurynome.} Il ne paraît même pas avoir soupçonné que l'Eurynome des Phigaliens, moitié femme, moitié poisson, est identique avec Dercéto ou Atergatis, cette Vénus marine que l'Orient représentait sous une forme semblable, en même temps que Dagon, son époux, était figuré sous les traits d'un homme barbu, à corps et queue de poisson.\footnote{\emph{Recherches sur Vénus}, pl. 19., n° 7 ; pl. 22., n°\textsuperscript{s} 1 et 1a, 9 et 9a, pl. 24., n°\textsuperscript{s} 15, 16, 19 et 20.} Nous avons pour témoin de l'identité non indiquée par Pausanias une image authentique de Dercéto, sculptée sur un des bas-reliefs découverts à Yazili-Kaïa par M. Ch. Texier.\footnote{\emph{Descript. de l'Asie Mineure}, etc., 1\textsuperscript{re} partie, t. 1., pl. 79., n° 1. Voy. mes \emph{Recherches sur Vénus}, pl. 22., n° 2.} Ici, la déesse, vue de face, et représentée avec un corps et une queue de poisson, a la tête surmontée d'un grand \emph{mihr}.\footnote{Altération du type primitif d'une colombe à ailes éployées, emblème de Mylitta et de Mithra.} Le centre et le sommet de cet emblème sont ornés chacun d'un astérisque. Superposés l'un à l'autre, les deux astérisques représentent, sans aucun doute, la planète Vénus sous ses deux aspects du soir et du matin. Aux épaules de la déesse sont attachées deux ailes, qui, en se recourbant, simulent, avec une intention marquée, le croissant de la lune. Je rapprocherai de cette figure une autre image, non moins curieuse, non moins instructive, qui représente Mylitta, la Vénus assyrienne, sortant du sein de la mer ou des eaux. Celle-ci est brodée ou peinte sur la stole d'un roi d'Assyrie revêtu d'un costume solennel, que, dans une autre occasion, j'ai comparé à la \emph{stole olympiaque} décrite par Apulée.\footnote{\emph{Opp.} t. 1., \emph{Metam.} 11., p. 804, 805 ; ed. Oudendorp.} Ce roi, entouré de sa cour, est assis sur son trône, au milieu d'un beau bas-relief, qui provient des ruines de Nimroud, et qui n'est pas une des moins précieuses conquêtes dues aux premières fouilles de M. A. H. Layard. La déesse\footnote{Voy. Layard's \emph{Monuments of Nineveh}, pl. 9. Voy. aussi mes \emph{Recherches sur Vénus}, pl. 19 B, n° 6.} n'a point un corps de poisson ; et sa tête, au lieu d'être surmontée du \emph{mihr}, est coiffée d'une tiare conique, ornée de deux cornes de vache. Mais son corps et ses pieds sont enveloppés d'un grand méandre, symbole qui des monuments assyriens, où il représente les eaux, est passé, avec la même signification, dans les monuments des Phéniciens et des Grecs ; et ce méandre est disposé de manière à nous faire comprendre que la déesse sort du sein des eaux ou de la mer. Nous avons donc sous les yeux une image de la divinité féminine que Bérose, cité par Alexandre Polyhistor,\footnote{\emph{Apud} Euseb. \emph{Chronic.} 1., 4 et 5.} nomme \emph{Omoroca} ou \emph{Omorca}, \emph{Marghaia} ou \emph{Marcaja}, et même \emph{Thagattha} ou \emph{Thaladta},\footnote{Il est curieux de retrouver cette forme du nom de Vénus dans le nom d'une courtisane qui, selon Athénée (\emph{Deipnosoph.} 13., p. 567 \emph{c}), s'appelait \emph{Thalatta}. C'était aussi le titre d'une comédie grecque, qui ne nous est point parvenue (\emph{ibid.}).} c'est-à-dire \emph{la mer}\footnote{Voy. mes \emph{Recherches sur Vénus}, p. 41, 42.} ; cette déesse, n'en doutons pas, fut le type de l'Aphrodite Anadyomène des mythographes, des peintres et des sculpteurs grecs. Fille de l'Océan comme Aphrodite, comme Eurynome, et assimilée, comme elles aussi, à Artémis, elle tient de la main gauche une coupe qui a la forme d'un croissant, et qui ainsi est, à double titre, le symbole du principe humide. La même idée est exprimée sur le grand bas-relief d'Yazili-Kaïa,\footnote{M. Ch. Texier, ouvr. cité, pl. 78. Voy. mes \emph{Recherches sur Vénus}, pl. 2.} où la Vénus asiatique tient, dans la main gauche, un sceptre court, orné d'une fleur de lotus, et, dans la main droite, un long sceptre surmonté du croissant de la lune. Après cette remarque, je reviens au méandre de la Vénus Anadyomène des Assyriens, pour dire qu'à Nimroud ce symbole affecte une forme qui pourrait aisément le faire confondre avec une chaîne, comme déjà plus d'une fois j'en ai été témoin. Dès lors nous sommes amenés à conjecturer que les chaînes qui, selon le récit fait à Pausanias par les Phigaliens, liaient la statue d'Eurynome, déesse-poisson, étaient en réalité un méandre\footnote{Quelquefois, sur les cônes et les cylindres assyriens ou phéniciens, l'eau est représentée non par un méandre à enroulements, mais par plusieurs filets réunis parallèlement et plus ou moins ondulés. On en voit quelques exemples dans mes \emph{Recherches sur Mithra} (Atlas, pl. 31. n°\textsuperscript{s} 4, 5, 7 ; pl. 32., n°\textsuperscript{s} 1 et 2). Je reproduis ici, sous les n°\textsuperscript{s} 5 et 6 de la planche 8., les dessins des sujets gravés en creux sous la base de deux cônes qui, chacun, représentent précisément deux figures barbues, à corps et queue de poisson, analogues, quant à leur forme, aux images citées d'Eurynome et de Dercéto, et, comme elles, enveloppées de filets d'eau qu'on pourrait, au premier abord, prendre pour des liens.} semblable à celui dont les ondulations enveloppent le corps de la Vénus Anadyomène de Nimroud.\footnote{On objectera peut-être que, selon les Phigaliens, le \emph{xoanon} d'Eurynome était lié avec des chaînes d'or. Mais il ne faut pas oublier qu'une des croyances les plus répandues dans l'antiquité, était que le soleil, à son lever, sort du sein de l'Océan et qu'il s'y replonge au moment du coucher. Dès lors, il n'est pas étonnant qu'on eût employé, pour un méandre destiné à représenter l'Océan ou la mer, le métal consacré au Soleil et à Vénus, qualifiée tantôt fille de l'Océan, tantôt fille du Soleil.} Dès lors aussi ne convient-il pas de remarquer que, si divers traits de conformité se révèlent entre l'image de l'Eurynome des Phigaliens et les images citées de la Vénus assyrienne ou phénicienne, une preuve de l'identité de ces deux divinités ressort également du soin qu'avaient eu les ancêtres des Phigaliens d'entourer le temple et le \emph{xoanon}\footnote{Remarquons que, selon Diogène de Laërte (\emph{in Præfat.}), les mages passaient pour être les inventeurs des statues de bois appelées \emph{xoana} par les Grecs.} d'Eurynome d'un bois de cyprès, arbre particulièrement consacré à Vénus par les Assyriens et les Phéniciens ? Tout à l'heure nous verrons en Illyrie, dans une colonie de Corinthe, une statue de Vénus placée au milieu d'un temple orné de sept cyprès.

Si nous suivons Pausanias dans l'antique cité que je viens de nommer, et si nous n'avons pas perdu le souvenir des monuments figurés que nous ont légués les Corinthiens, nous ne pouvons méconnaître l'origine asiatique de plusieurs croyances religieuses et de certains usages privés de ce peuple célèbre. Déjà nous avons trouvé dans un temple de l'acropole de Corinthe, comme dans les sanctuaires des Syriens, le culte d'Aphrodite associé au culte d'Éros et à celui d'Hélios. Non loin de la ville, le voyageur grec\footnote{2., 2., 4.} nous montre, réunis dans un bois sacré de cyprès, appelé le \emph{Cranéon}, le temple d'Aphrodite surnommée \emph{Melænis} ou la Noire, le temple de Bellérophon et le tombeau de la célèbre Laïs. Ici le surnom d'Aphrodite nous rappelle que la théologie orientale confondait dans une même divinité les idées de nuit et de génération. Cette observation n'ajoute pas peu de prix, ce me semble, à un témoignage qui nous apprend que, chez les Grecs, on avait planté auprès de Vénus la Noire l'arbre pyramidal et toujours vert, qui, en Asie, était un symbole de vie, consacré aux divinités génératrices.

Sans sortir de la Corinthie, nous allons trouver, sur les pas de Pausanias, un exemple de l'attribution de cet arbre à une antique déesse qui, adorée par les Grecs sous deux noms qu'au premier aperçu on ne serait pas tenté de rapporter à Vénus, se confond néanmoins avec cette divinité, tout comme l'Eurynome des Phigaliens. « Dans l'acropole de Phlionte,\footnote{Ou Phliunte, à quarante stades environ de Titané, dans la Corinthie.} dit notre guide,\footnote{Pausanias, 2., 13., 3.} vous verrez d'abord un bois de cyprès, puis un temple qui, depuis les siècles les plus reculés, est l'objet d'une grande vénération. La déesse qu'on y adore portait anciennement, dans le pays, le nom de \emph{Ganyméda} ; plus tard, elle prit celui d'\emph{Hébé} ; Homère la nomme ainsi dans le récit du combat singulier entre Pâris\footnote{\emph{Alexandre}, dans le texte grec.} et Ménélas, où il dit qu'elle versait à boire aux dieux, et dans le récit de la descente d'Ulysse aux enfers, où il nous apprend qu'elle était l'épouse d'Héraclès (Hercule). Olen, dans son hymne à Héra (Junon), dit que cette déesse avait été élevée par les Saisons, et que ses enfants étaient Arès (Mars) et Hébé. » Plus loin, Pausanias nous fait remarquer que dans le temple de l'acropole de Phlionte, aucune statue n'est exposée à la vue du public ni gardée en un lieu secret. Il ajoute que, pour agir ainsi, les Phliasiens se fondent sur une tradition sacrée. Je n'insisterai pas ici sur les diverses circonstances qui, dans ce récit, me donnent lieu de supposer que la déesse Ganyméda était identique avec la Vénus assyrienne. Je me bornerai simplement à faire remarquer que le cyprès lui était consacré, et que son premier nom, comme le passage d'Homère où elle est appelée Hébé, nous reporte dans l'Asie occidentale. A l'appui de cette dernière observation je citerai une pierre gravée, phénicienne, qui prouve que l'idée de représenter Hébé versant le nectar dans la coupe de Zeus ou Jupiter fut un emprunt fait par les Grecs aux peuples de l'Asie occidentale. Sur cette intaille, Baltis ou Beltis est placée, dans la même attitude, devant Baal assis sur son trône.\footnote{Voy. mes \emph{Recherch. sur Vénus}, atlas, pl. 14 B, n° 1.} Une autre intaille phénicienne\footnote{\emph{Ibid. ibid.} n° 3.} nous montre la déesse assise sur les genoux de Baal et recevant de la main droite de ce dieu une fleur de lotus, en échange de la coupe qu'elle lui a présentée et qu'il tient de la main gauche.

D'autre part, des médailles authentiques peuvent apporter dans la question de l'importation en Grèce du culte du cyprès associé à celui de Vénus, un témoignage direct et décisif, bien que personne, à ma connaissance, n'ait encore invoqué ce témoignage pour éclaircir l'histoire du cyprès ou la légende de la déesse. Le type de ces médailles a de plus, à mes yeux, l'avantage de montrer comment, dans le cadre si resserré du revers d'une monnaie, l'antiquité savait, à l'aide de symboles empruntés aux formes géométriques, aux nombres et au règne végétal, exprimer simultanément plusieurs idées théologiques et cosmologiques de l'ordre le plus élevé. Les pièces dont j'entends parler appartiennent à la numismatique de la ville d'Apollonia, en Illyrie, colonie de Corinthe, et, par conséquent, fille d'une métropole et d'une province où nous venons de trouver plus d'une trace du culte de Vénus, du culte du cyprès, et de l'influence exercée par l'Asie occidentale sur les croyances et les usages religieux de la Grèce. Ce sont de grands bronzes, à l'effigie de Septime Sévère, de Julia Domna et de Géta, qui jamais n'ont été décrits ou figurés avec exactitude.\footnote{Voy. plus loin, p. 228, note 1.} Leur revers est uniforme ; il nous fait connaître un édifice religieux dont aucune description, aucun autre dessin ne nous est parvenu, et dont les ruines sont encore à découvrir. Cet édifice, certainement érigé dans Apollonia, d'après un modèle apporté de Corinthe, est un temple tétrastyle, orné de sept cyprès, et où l'on voit, assise sur un trône, une divinité féminine, qui n'a jusqu'à ce jour reçu aucune dénomination mythologique dans les recueils de numismatique ou d'archéologie. Cette circonstance, la construction de l'édifice, la statue de la déesse, les accessoires, tout concourt à recommander un tel type à notre attention. Je place sous les yeux de la Compagnie\footnote{Pl. 8., n° 4.} le dessin d'un grand bronze à l'effigie de Julia Domna,\footnote{L'original appartient au médaillier de la Bibliothèque impériale, où j'ai pu aussi examiner les grands bronzes frappés à Apollonia en l'honneur de Septime Sévère et de Géta, que j'ai cités plus haut.} où ce type est reproduit dans un état de conservation qui ne laisse aucune incertitude sur les détails que nous avons à examiner.\footnote{Le docte Eckhel ne fait aucune mention de ce type. Il n'est pas reconnaissable dans les descriptions ou les dessins qui ont été publiés des trois grands bronzes où il se reproduit uniformément. Dans le \emph{Musée Arigoni} (t. 1., tab. 8., n. 119), le texte, d'accord avec un dessin très-inexact, décrit ainsi le revers du grand bronze de Julia Domna : « Temple tétrastyle, dans lequel est Jupiter assis ; de chaque côté, un arbre. » Plus loin, dans le même ouvrage (t. 4., tab. 10., n. 51), le revers du grand bronze de Septime Sévère est décrit en ces termes : « Temple tétrastyle, dans lequel est Jupiter assis, tenant de la main droite une patère, et de la gauche une haste ; autour du temple, des arbres. » Mionnet, malgré son exactitude habituelle, ne commet pas moins une erreur tout aussi grave. Oubliant que, dans le deuxième volume de sa \emph{Description de médailles} (p. 34, n° 67), il avait décrit ainsi le revers du grand bronze de Julia Domna : « Femme assise à gauche, dans un temple tétrastyle, le chapiteau orné de cyprès ; de chaque côté, un cyprès, » l'habile numismate ne s'aperçoit pas que ce type est exactement reproduit au revers d'un grand bronze frappé, en l'honneur de Géta, dans la même ville d'Apollonia. Il décrit succinctement ce dernier revers, en disant (\emph{Suppl.} 3., 327, n° 105) : « Jupiter assis dans un temple tétrastyle ; de chaque côté, un cyprès. »}

Le fronton du temple est triangulaire, et l'entablement supporté par quatre colonnes ; ce qui, à mon avis, dut originairement exprimer que la divinité adorée dans un temple de cette forme était revêtue d'un triple caractère et présidait aux quatre éléments, qui sont les agents de la création et de la reproduction, et, pour ainsi dire, les quatre soutiens du monde créé. Telle fut Vénus en Orient et en Occident. Reine, tout à la fois, du ciel, de la terre et des enfers,\footnote{\emph{Recherch. sur Vénus}, p. 43, 44, 72, 74, 75, 226, 227, 238-240. Mithra était revêtu de ce même triple caractère (\emph{Mém. de l'Acad. des inscr.} nouv. série, t. 14., 2\textsuperscript{e} part. p. 68, 173).} et divinité génératrice, elle avait, parmi ses attributs, les quatre éléments ; les nombres trois, quatre et sept lui étaient consacrés ; et si nous trouvons ici un globe sculpté sur le fronton, au centre du tympan, il doit nous rappeler que fréquemment, à la même place, on voit un globe semblable sur les médailles asiatiques dont le revers représente le temple d'Astarté.\footnote{Voy. le n° 2 de la pl. 8 ci-jointe. Voyez aussi mes \emph{Recherches sur Vénus}, atlas, pl. 1., n° 9 ; pl. 4., n° 5 ; pl. 15., n° 7 ; pl. 35., n°\textsuperscript{s} 1, 2, 3 et 11.} Dans l'un et l'autre cas, nous ne pouvons oublier qu'à Aphaques, dans le Liban, le jour de la fête de la déesse, on faisait paraître dans les airs un globe de feu.\footnote{Zosime, \emph{Histor.} 1., 58., 4 et 5 ; ed. Reitemeier. --- Sozomène (\emph{Histor. ecclesiast.} 2., 5.), au lieu d'un globe de feu, parle d'un feu semblable à une étoile.} Deux grands cyprès pyramidaux s'élèvent, l'un à droite, l'autre à gauche du temple ; ils représentent symboliquement le soleil et la lune, et nous ramènent aux deux cyprès qui, sur la face antérieure du trône votif de la déesse de Syrie,\footnote{Ci-dessus, p. 51-58, et pl. 5., n° 1.} tiennent lieu de ces deux astres, et aux deux autres cyprès qui, sur les médailles impériales de Perga,\footnote{Ci-dessus, p. 55, 56, et pl. 14., n°\textsuperscript{s} 1 et 2.} sont surmontés, l'un de l'image du soleil, l'autre du croissant de la lune. Mais, de plus, au revers du grand bronze frappé à Apollonia, cinq petits cyprès s'élèvent derrière le sommet et les deux côtés inclinés du fronton du temple. Là, comme sur un monument étrusque\footnote{Voy. pl. 13., n° 5.} et sur deux bas-reliefs mithriaques dont je parlerai plus loin, ils sont les emblèmes de cinq des sept planètes que comptaient les anciens, en y comprenant le soleil et la lune. Très-fréquemment les cylindres et les cônes asiatiques que je rapporte aux mystères de la Vénus assyrienne ou aux mystères de Mithra, représentent aussi les sept planètes, mais sous la forme de sept petits globes, ou de cinq petits globes placés auprès de l'astérisque du soleil et du croissant de la lune. Sur les monuments mithriaques d'époque romaine, les sept planètes sont figurées, tantôt par sept cyprès,\footnote{Voy. mes \emph{Recherch. sur Mithra}, atlas, pl. 92.} tantôt par sept autels allumés ou non,\footnote{\emph{Ibid.} pl. 78., n° 1 ; pl. 79., n° 2 ; pl. 96., n°\textsuperscript{s} 1 et 2.} tantôt par cinq astérisques accompagnés du soleil et de la lune,\footnote{Voy. mes \emph{Recherch. sur Mithra}, atlas, pl. 80., n° 1.} tantôt enfin par sept astérisques.\footnote{\emph{Ibid. ibid.} n° 2.} Dans ces deux derniers cas, les sept planètes sont sculptées sur le manteau (\emph{candys}) même de Mithra. C'est donc au milieu d'un temple orné des sept planètes symboliquement représentées, qu'était placée l'image de la déesse adorée à Apollonia. Le long sceptre qu'elle porte de la main gauche, la patère que probablement elle tenait de la main droite, mais qui est cachée par une des quatre colonnes, l'arrangement des cheveux, l'agencement de la draperie, qui s'enroule au bas des reins, de manière à laisser nu le haut du corps, toutes ces particularités, ajoutées au témoignage qui résulte du globe et des sept cyprès, nous autorisent à reconnaître ici Vénus représentée avec son triple caractère de reine du ciel, reine de la terre et reine des enfers, et avec son caractère aussi de divinité \emph{génétrice} ou génératrice. Rapprochée du type des médailles autonomes de Nagidus de Cilicie, dont j'ai parlé plus haut,\footnote{Ci-dessus, p. 29-31.} l'image qui est sous nos yeux justifie pleinement la dénomination que je propose de lui appliquer. Bien plus, dans sa pose, dans son ensemble et dans quelques-uns de ses détails caractéristiques, elle nous offre la preuve que, comme la Vénus de Nagidus figurée sur la médaille n° 9 de la planche 8, la Vénus d'Apollonia est due à l'imitation d'un modèle asiatique, modifié et embelli par le goût et l'imagination des Grecs.\footnote{On retrouve l'imitation du même modèle, avec la légende VENUS FELIX, au revers d'un joli quinaire d'argent, frappé en l'honneur de Fausta, épouse de Constantin le Grand. Il est reproduit sous le n° 4 de la pl. 26 de mes \emph{Recherches sur Vénus}, d'après un dessin dont je suis redevable à l'obligeance et au talent de M. E. Carlier. Cf. le bel aureus de la Bibliothèque impériale que j'ai publiée (\emph{Ib.} pl. 21 A, n° 11). On y voit, au revers de la tête d'Hadrien, Vénus assise sur son trône et entourée de la légende VENERIS FELICIS.} Si des médailles frappées en Illyrie sous la domination romaine, nous remontons aux monnaies autonomes d'Apollonia, de Dyrrachium ou Dyrrhachium, de Corcyre, et à celles que Mionnet\footnote{\emph{Descript. de méd.} 2., 44, n° 165 ; \emph{Suppl.} 3., 353, n° 315 ; pl. 12., fig. 8.} attribue à Euchélies ou Enchéli d'Illyrie,\footnote{M. l'abbé Cavédoni (\emph{Bulletino del Inst. arch.} n° 12, déc. 1841, p. 190) n'admet pas cette attribution. Il penche à classer les pièces dont il s'agit parmi les monnaies archaïques de Corcyre ou d'Apollonia, ou même d'un Dyrrhachium d'Épire, dont je ne trouve aucune autre indication.} nous voyons même Corinthe répandre au loin, sans les avoir modifiés, d'autres types quelle avait directement reçus de l'Asie occidentale. Citons surtout la vache allaitant son veau, groupe qui, en Orient, représentait symboliquement Vénus et l'Amour, et qui se fait remarquer, avec divers accessoires, sur les monnaies autonomes dont j'entends parler.\footnote{Voy. \emph{Mém. de l'Acad. des inscrip.} nouvelle série, t. 15., 2\textsuperscript{e} partie, p. 267, 268. \emph{Recherches sur Vénus}, p. 165-168 ; atlas, pl. 14 G, n° 9-13, 18 ; pl. 14 H, n°\textsuperscript{s} 2-6. Cf. \emph{ibid.} n° 9.}

Non loin de Corinthe, à Sicyone, furent frappées, à l'effigie de Plautille, épouse de Caracalla, deux monnaies de bronze, l'une de moyen module,\footnote{Mionnet, \emph{Suppl.} 4., 172, n° 1141.} l'autre de petit module,\footnote{\emph{Id. ibid.} n° 1142.} qui nous offrent chacune un nouvel exemple de l'attribution du cyprès à une divinité adorée dans une cité grecque. Je joins ici\footnote{Pl. 7., n° 3. C'est le n° 1141 de Mionnet.} un dessin de la première de ces deux pièces, d'après l'exemplaire bien conservé qu'en possède la Bibliothèque impériale.\footnote{Un second exemplaire se trouvait autrefois dans le médaillier de feu M. Allier de Hauteroche.} Sur cette monnaie, comme sur l'autre,\footnote{N° 1142 de Mionnet, décrit par ce numismate d'après l'exemplaire qui appartenait à feu M. J. Millingen.} on voit, au revers, un petit temple tétrastyle, sans statue, érigé au sommet d'un rocher placé entre deux statues terminées en gaine et représentant chacune un même dieu barbu. À côté de l'une et de l'autre de ces deux statues, s'élève un beau cyprès pyramidal. Une grande incertitude règne sur la divinité dont elles nous offrent l'image et à qui avait été dédié le petit temple. Mionnet se contente de les désigner ici par le mot \emph{termes} ; mais en décrivant la seconde monnaie,\footnote{N° 1142.} il leur applique la dénomination d'\emph{hermès}. M. l'abbé Cavédoni\footnote{\emph{Spicileg. numism.} p. 92, 93.} se croit autorisé à reconnaître sur la première, comme sur la seconde, le temple d'Esculape dont Pausanias\footnote{2., 10., 2 et 3.} fait mention dans sa description de Sicyone. Malgré la déférence habituelle que j'ai pour les décisions d'un juge aussi compétent, je ne puis m'empêcher de placer ici quelques remarques qui montreront que son sentiment n'est pas à l'abri de toute contestation. Et d'abord constatons qu'à Sicyone le temple dont parle Pausanias n'était pas exclusivement consacré à Esculape. On y voyait, dit-il, outre la statue de ce dieu, celles d'\emph{Hypnos} (le Sommeil), d'Apollon Carneïus, de Pan, d'Artémis, etc. Il n'en était pas de même du temple érigé en l'honneur d'Aphrodite par les Sicyoniens\footnote{Pausanias, 2., 10., 4.} : la déesse y régnait sans partage\footnote{Une statue d'Antiope avait seule obtenu la faveur d'être placée dans ce temple avec la statue d'Aphrodite.} ; sa statue d'or et d'ivoire, ouvrage célèbre de Canachus, la représentait assise ; telle était aussi la pose de la Vénus de Nagidus et celle de la Vénus d'Apollonia adorée dans un temple tétrastyle, flanqué de deux cyprès et surmonté de cinq autres arbres de cette même espèce. Ajoutons que les monnaies autonomes de Sicyone ne laissent apercevoir aucune trace du culte d'Esculape, et que, parmi les médailles impériales de la même ville, deux seulement nous rappellent ce culte.\footnote{L'une, au revers de la tête de Géta (Mionnet, \emph{Descr. de méd.} 2., 201, n° 382), représente Hygie ; l'autre, au revers de la tête de Caracalla (Vaillant, \emph{Numism. græc.} p. 110), porte l'image d'Esculape.} Au contraire, une multitude de monnaies autonomes ou impériales de Sicyone attestent, par les types les plus expressifs, que le culte d'Aphrodite, d'Hermès\footnote{Le revers d'une médaille de Caracalla, frappée à Sicyone (\emph{Mus. Arig.} 1., 9, n° 129), offre pour type une statue d'Hermès terminée en gaine.} et d'Éros, ou les cultes réunis d'Aphrodite et d'Apollon, étaient la religion dominante chez les Sicyoniens, comme chez les Corinthiens. Si, d'autre part, nous nous rappelons qu'Hermès ou Mercure, tout comme Esculape et Jupiter, fut représenté par des statues terminées en gaine ; et si nous remarquons qu'ici les deux statues ainsi façonnées ne sont point accompagnées du symbole qui habituellement sert à caractériser Esculape, un serpent enroulé ou non autour d'un bâton, il ne restera guère plus de raison de rapporter à ce dieu plutôt qu'à Hermès ou à Aphrodite le petit temple qui se voit sur notre médaille. Car les témoignages ne nous ont pas fait faute pour établir que l'antiquité attribuait le symbole du cyprès à Hermès et à Zeus ou Jupiter, aussi bien qu'à Esculape. Les deux arbres de cette espèce que nous trouvons ici plantés auprès des deux statues en gaine ne peuvent donc résoudre une question qui me paraît devoir rester longtemps indécise. Toutefois je n'ai encore produit, quant à l'attribution du cyprès à Esculape, que le témoignage indirect qui résulte d'un revers gravé sur quelques médailles impériales d'Épidaure, dont l'attribution a même été le sujet d'une longue controverse entre les numismates.\footnote{Voyez ci-dessus, p. 194-196.} Il convient à tous égards que nous interrogions ici les autres monuments de la Grèce. Je n'en trouve qu'un à citer. C'est un antique temple d'Esculape, dont l'enceinte était intérieurement ornée d'une plantation de vieux cyprès. Situé à Titané, non loin de Sicyone, ce temple, selon Pausanias,\footnote{2., 2., 6.} renfermait une statue d'Hygie, et avait, disait-on, été fondé par Alexanor, fils de Machaon et petit-fils d'Esculape lui-même.

Parmi les personnages mythologiques dont la Grèce avait reçu la légende avec les divinités, les systèmes religieux, les sciences et les arts importés de l'Asie occidentale en Europe, Cyparisse n'est pas le seul que les traditions grecques nous montrent métamorphosé en cyprès. Personne n'ignore qu'Étéocle passait pour avoir, le premier, institué en Béotie des sacrifices en l'honneur de Charis,\footnote{Théocrite, 16., 104. Scoliaste de Pindare, \emph{ad Olymp.} 14., 1. Pausanias, 9., 35., 1. Cf. K. O. Müller. \emph{Orchomen.} p. 134, 177.} l'une des trois Grâces, c'est-à-dire en l'honneur d'Aphrodite elle-même, puisque, dans l'Iliade,\footnote{18., v. 382. Cf. Pausan. 9., 35., 4.} Homère donne le nom de Charis à l'épouse d'Héphæstus ou Vulcain. Personne n'ignore non plus que la Béotie est précisément la province où les Phéniciens, conduits par Cadmus, avaient importé le culte de la Vénus assyrienne, et très-probablement, avec ce culte, le symbole du cyprès pyramidal. Or il est dit, dans un passage curieux des \emph{Géoponiques},\footnote{11., 4.} que les filles d'Étéocle, pendant qu'elles dansaient en l'honneur des Déesses, c'est-à-dire Déméter et Cora, tombèrent dans un puits, et furent changées en cyprès, par la miséricorde de Ghé, la déesse de la Terre. Ces filles d'Étéocle, appelées ici, pour cette raison, les \emph{Cyprès}, \textgreek{κυπάρισσοι}, sont désignées ailleurs\footnote{Voy. Eustathe, Didyme et les Scoliastes publiés par d'Anse de Villoison, \emph{ad} Homer. \emph{Iliad.} 5., 51.} sous le nom de \emph{Charites}, \textgreek{χάριτες}. Par là nous acquérons la preuve qu'on les assimilait aux Grâces ; et dès lors nous sommes autorisés à croire que les compagnes fidèles de la déesse appelée \emph{Cyprès}\footnote{\emph{Bérouth, Brathy, Cypris.} Voy. ci-dessus, p. 88-91.} avaient réellement pu, comme Vénus, être surnommées elles-mêmes \emph{Cyprès}, et, comme elle aussi, être représentées sous la forme symbolique du cyprès ; car Vénus, je le répète, est l'aînée des trois Grâces, de même qu'elle est l'aînée des trois Parques.\footnote{\emph{Recherches sur Vénus}, p. 59, 60, 98.} Les filles d'Étéocle, ou plutôt Vénus et les deux autres Grâces, composaient donc une triade qui, chez les Béotiens, était symboliquement représentée par trois cyprès, comme l'était, chez les Syriens et, selon toute probabilité, chez les Phéniciens, la triade formée de Vénus, du soleil et de la lune. N'avons-nous pas reconnu cette dernière triade dans les trois cyprès qui ornent la face antérieure du trône votif de la Déesse de Syrie, que j'ai décrit plus haut\footnote{Ci-dessus, p. 51-58 ; p. 5., n°\textsuperscript{s} 1 et 2.} ?

À leur tour, les myrtes de Vénus furent parfois appelés les \emph{Charites}, \textgreek{χάριτες} : nous l'apprenons de quelques anciens commentateurs d'Homère.\footnote{Ils sont indiqués dans la note 4 de la page 234.} D'autre part, en même temps que les filles d'Étéocle ou les trois Grâces recevaient le nom de \emph{Cyprès}, \textgreek{κυπάρισσοι}, on appliquait l'épithète de \textgreek{παρθένοι}, \emph{les vierges}, aux grands cyprès qui, à Psophis en Arcadie, entouraient le tombeau d'Alcmæon, fils d'Amphiaraüs.\footnote{Pausanias, 8., 24., 4. --- Selon cet écrivain, ces cyprès étaient d'une telle hauteur qu'ils dérobaient à la vue la montagne ou la colline située auprès de Psophis.} Les rapprochements que je viens de faire semblent attribuer à cette épithète une origine préférable à celle que veulent lui reconnaître les commentateurs de Pausanias, lorsqu'ils disent que les cyprès du tombeau d'Alcmæon étaient appelés \emph{les vierges}, \textgreek{παρθένοι}, parce qu'on ne les coupait jamais. Nous verrons, dans la deuxième partie de ces Recherches, comment les cyprès sculptés sur les monuments funéraires des Étrusques peuvent justifier l'interprétation mythologique que je propose de substituer à cette explication. Remarquons ici qu'Amphiaraüs avait obtenu les honneurs divins, un autel, un temple même et un oracle.\footnote{\emph{Id.} 1., 34., 1-3.} Remarquons aussi que la légende de son fils paraît être intimement liée à celle d'Aphrodite, la déesse qui présidait à l'empire des morts, comme aux initiations, dont le suprême degré était l'apothéose. Alcmæon avait eu en sa possession le fameux collier d'Ériphyle.\footnote{Pausanias, 8., 24., 4.} Il le donna à sa femme Alphésibée, dans cette même ville de Psophis, où l'on voyait un antique temple dédié à Vénus-Érycine et le tombeau d'Amphiaraüs, mort avec les honneurs insignes de l'apothéose.

Deux autres faits, rapportés par Athénée,\footnote{\emph{Deipnosoph.} 5., p. 204, 205, 207 ; ed. Casaubon. --- Au liv. 9., p. 402, Athénée rapporte un fragment de l'\emph{Hippotrophus} de Mnésimaque où il est parlé de lits dont le dais plaqué avec du bois de cyprès, \textgreek{θάλαμοι κυπαρισσόροφοι}.} concourent plus directement à établir que, chez les Grecs, le cyprès était, en particulier, consacré à Aphrodite. Cet auteur, dans la description qu'il fait de deux élégants et somptueux navires, construits, l'un pour Ptolémée Philopator, l'autre pour Hiéron, nous apprend que ces navires contenaient chacun un \emph{Aphrodisium}. Dans le premier, appelé le \emph{Thalamégus}, le petit temple, ou, comme nous dirions, la chapelle d'Aphrodite, ornée d'une statue de la déesse, était placée au-dessus d'une vaste pièce à colonnes, renfermant trente lits. Le fût de chaque colonne avait été fait avec du bois de cyprès tiré de Milet, colonie fondée dans l'Ionie par les Crétois ; les chapiteaux, d'ordre corinthien, étaient composés d'or et d'ivoire ; le plafond, également de bois de cyprès, attirait l'attention par ses ornements sculptés et dorés. Dans le second navire,\footnote{Athénée (ouvrage cité, 5., p. 207) emprunte ici à Moschion les traits principaux de la description de ce second navire.} construit à Syracuse par le célèbre Archimède, les lambris et le plafond de l'Aphrodisium étaient de bois de cyprès. Ce lieu, tout resplendissant de pierres précieuses, et orné de peintures, de statues, de vases de formes diverses, ne renfermait que trois lits.

Hermippe, dans un passage dont j'ai déjà dit quelque chose, s'exprime de manière à nous donner lieu de croire que le cyprès pyramidal ne fut pas seulement attribué aux dieux et aux déesses que j'ai indiqués par leurs noms, mais à toutes les divinités de l'antiquité qui présidaient aux choses du ciel et de la terre, \textgreek{τοῖσι θεοῖσιν}, dit le philosophe grec.\footnote{\emph{Ap.} Athen. \emph{Epitom.} 1., p. 27 ; ed. Dalech.} Servius semble partager l'opinion d'Hermippe, lorsqu'il rapporte que Cyparisse fut changé en cyprès « par la miséricorde des dieux, » \emph{misericordia deorum}.\footnote{\emph{Ad.} Virgil. \emph{Æneid.} 3., 680.} Aussi peut-on ajouter quelques faits à ceux que déjà j'ai cités dans le cours de ce mémoire, et qui tendent à généraliser l'attribution du cyprès, en nous montrant cet arbre comme le symbole de plusieurs divinités génératrices, mâles ou femelles. C'est ainsi que, selon Théocrite,\footnote{\emph{Epigramm.} 4., dans les \emph{Analecta} de Brunck, t. 1., p. 377.} les statues du dieu Priape devaient être entourées de cyprès, de myrtes et de lauriers. A Byzance, un \emph{xoanon} représentant Triton était de bois de cyprès.\footnote{Athén. \emph{Deipnosoph.} 11., p. 479, 480.} Ce dernier usage s'étendit même jusqu'aux statues destinées à perpétuer les traits et le nom des athlètes vainqueurs aux jeux olympiques\footnote{Pausanias, 6., 18., 7.} ; et si je n'invoque pas ici les témoignages qui établissent que le cyprès était consacré à plusieurs divinités infernales, c'est qu'ils trouveront plus naturellement leur place dans le mémoire suivant.

La Grèce, en empruntant à l'Asie occidentale la coutume d'attribuer ce symbole aux dieux générateurs, lui avait aussi emprunté l'usage de prendre le cyprès pour exprimer poétiquement l'idée d'une beauté humaine égale à la beauté d'un dieu. Déjà nous venons de constater que, dans la Béotie, les Grâces ou les filles d'Étéocle sont changées en cyprès, et prennent le nom même de \emph{Cyprès} (\textgreek{κυπάρισσοι}), qui est un des noms de Vénus.\footnote{De même, dans une autre légende, les Grâces se changent en myrte, et nous rappellent que Vénus était surnommée \emph{Myrthia} et \emph{Myrtæa} (Plutarque, \emph{Quæst. roman.} Opp. t. 7., p. 87 ; ed. Reiske. Pline, \emph{H. N.} 15., 36., 2).} Déjà aussi nous avons vu appliquer à de très-beaux cyprès la qualification de \emph{vierges} (\textgreek{παρθένοι}), qui appartenait à Vénus et à Minerve. Disons maintenant que Théocrite,\footnote{\emph{Idyll.} 18., 30.} voulant peindre la beauté d'Hélène, fille de Léda, compare cette princesse à un cyprès, c'est-à-dire à Vénus, si nous prenons cette comparaison dans son acception symbolique. De son côté, Anacréon souhaite un bel époux à une femme, en lui disant : « Qu'un beau cyprès croisse dans votre jardin.\footnote{\textgreek{Κυπάριττος πεφύκοι σεῦ ἐνὶ κήπῳ} (\emph{Od.} 65., v. 15).} » Enfin Aristénète, dans une de ses lettres,\footnote{Lib. 1., epist. 1., p. 5 ; ed. Boissonade.} compare au cyprès et au palmier le port de Laïs. Bien des siècles après ces divers auteurs, une princesse qui a pris rang parmi les écrivains byzantins, Anne Comnène, croit nous donner, dans son \emph{Alexiade},\footnote{Liv. 3., p. 74.} l'idée la plus avantageuse de la taille et du port de Marie, épouse de Michel Ducas, en comparant aussi cette princesse à un cyprès. Cet usage s'est perpétué d'âge en âge chez les Grecs. Au 12\textsuperscript{e} siècle, dans le roman des \emph{Amours de Rhodante et Dosiclès},\footnote{Liv. 2.} un moine grec, Théodore Prodrome, compare la jeune fille tout à la fois au cyprès et à la vigne qui s'enlace aux arbres. De nos jours, dans les chansons et les autres poésies légères des Grecs, il n'est pas rare de trouver la taille et le port d'une jeune beauté ou de son amant comparés de même au cyprès. Pouqueville, Fauriel, et plus récemment M. le comte de Marcellus,\footnote{\emph{Chants da peuple en Grèce.} Paris, 1851, 2 vol. in-8°. M. de Marcellus, dans le deuxième volume (p. 434, B'), rapporte uv distique populaire qu'il traduit ainsi : « Quand le cyprès a brûlé, sa racine embaume. Dès qu'un cœur est amoureux, tout le monde s'en aperçoit. »} en ont cité plusieurs exemples. Firdousi et tous les poètes arabes ou persans, j'en ai déjà fait la remarque, emploient la même comparaison pour rehausser les mérites des personnages, hommes ou femmes, dont ils célèbrent les actions. On voit ici une preuve remarquable de l'influence exercée par le langage hiératique ou symbolique sur le langage des poètes : en Occident, comme en Orient, le cyprès devint l'emblème de la beauté, parce qu'il était l'emblème des dieux, des déesses, et, en particulier, le symbole de Vénus, de la déesse dont la beauté avait mérité la pomme. Sous la plume éloquente et sévère de nos écrivains sacrés, le cyprès est l'emblème de la divine sagesse,\footnote{\emph{Ecclesiastic.} 24., 27.} ou un terme de comparaison, qui sert à élever au-dessus des autres hommes le pontife du vrai Dieu.\footnote{\emph{Ibid.} 50., 11.}

Au témoignage des monuments de l'art et des traditions écrites qui nous montrent combien fréquemment les Grecs firent du cyprès un symbole religieux non funéraire, viennent se joindre des renseignements topographiques et quelques observations qui vont nous révéler combien aussi furent nombreux, sur le sol antique de la Grèce, les lieux dont la dénomination dérivait ou d'une divinité éponyme, appelée du même nom que le cyprès, ou d'un bois de cyprès consacré par la religion. M. Avellino\footnote{Mémoire cité, p. 11, 12.} n'hésite point à reconnaître que chaque \emph{cupressetum} se liait à un mythe ; mais il admet que le mythe avait pu être postérieur à la plantation ; c'est-à-dire que, selon lui, le \emph{cupressetum} aurait parfois été planté sans un but religieux, et ne serait devenu plus tard l'objet de la vénération des peuples que lorsqu'on l'aurait rattaché à un mythe quelconque. Je ne partage pas, je l'avoue, cette manière de voir ; et, à mon grand regret, je me trouverais ici en complet désaccord avec le savant académicien de Naples, s'il ne terminait son observation par cette conclusion, dont je prends acte : « Il est toujours sûr qu'une idée religieuse se trouve associée au \emph{cupressetum}, et qu'on peut en suivre les traces avec une certitude suffisante.\footnote{« È sempre sicuro che una idea religiosa venne ad esse associata, e di questa possono andarsi determinando con sufficiente certezza le trace. »} »

Malgré la variété que cette judicieuse remarque semblerait nous promettre dans la revue topographique que je vais entreprendre, je demande d'avance pardon à l'Académie de la sécheresse et même de la monotonie des détails où je serai obligé d'entrer. J'ai besoin, en cette occasion, plus encore qu'en toute autre, de solliciter et d'obtenir son indulgence.

Une ville du nom de \emph{Cyparissus} (\textgreek{Κυπάρισσος} ou \textgreek{Κυπαρισσοῦς}) et une autre appelée \emph{Cyparisséeis} (\textgreek{Κυπαρισσήεις}) existaient en Grèce longtemps avant la guerre de Troie. Homère\footnote{\emph{Iliad.} 2., 519, 593. Dicéarque, \textgreek{Ἑλλάδ. ἀναγρ.} p. 17, edit. Rom. Strabon, \emph{Geograph.} 9., p. 423. Stace, \emph{Thebaid.} 7., 344. Pausanias, 10., 36., 3. Étienne de Byzance, \textgreek{υ. Κυπάρισσος.} Eustathe, \emph{ad} Homer. \emph{Iliad.} 2., 519.} les mentionne toutes deux. La première était située dans la Phocide, sur le mont Parnasse, près de Delphes ; suivant les uns, dit le chantre immortel de l'Iliade, elle devait son nom à l'arbre que produit son territoire ; selon d'autres, au héros Cyparisse. Plus tard, elle fut nommée \emph{Apolloniade} (\textgreek{Ἀπολλωνιάς}), au dire d'Eustathe\footnote{\emph{Loc. cit.}} et d'Étienne de Byzance,\footnote{\emph{Loc. cit.}} et \emph{Anticyre}, d'après une tradition recueillie sur les lieux par Pausanias.\footnote{\emph{Loc. cit.}} Ce voyageur ajoute que, dans la liste des Phocéens, Homère a préféré la désigner sous son ancien nom de \emph{Cyparissus}, quoique déjà elle portât celui d'Anticyre, qu'elle avait reçu d'Anticyrée, contemporain d'Hercule. Dicéarque\footnote{\emph{Loc. cit.}} et Strabon\footnote{\emph{Loc. cit.}} ne la mentionnent que sous le nom de \emph{Cyparissus}. Le second de ces deux écrivains la place au-dessous de Lycoréa, et semble la réduire à la simple condition d'un bourg ou d'un village.\footnote{Selon Wheler (\emph{Voyag. de Zanthe à Athènes}, t. 2., p. 355, 356), ce bourg aurait été remplacé par \emph{Arachova} ou \emph{Arachovi}.} Pour concilier entre eux ces divers témoignages, il faut croire que la ville dont je parle s'appela successivement Cyparissus, Anticyre et Apolloniade, et qu'elle avait tiré son premier nom, soit d'un \emph{cupressetum} consacré à Apollon, soit du dieu lui-même, si, à l'exemple de la déesse appelée \emph{Bérouth, Brathy} et \emph{Cypris}, ce dieu avait reçu un nom identique avec celui de l'arbre que les Grecs connurent sous la dénomination de \textgreek{κυπάρισσος} ou \textgreek{κυπάριττος} (\emph{cyparissus}). La place qu'occupait la ville sur le mont Parnasse, consacré à Apollon et aux Muses ; la quantité de cyprès dont elle était entourée ; le soin qu'a eu Pindare\footnote{\emph{Pythic.} 5., 52.} de nous apprendre que les lambris du temple d'Apollon à Delphes étaient de bois de cyprès ; les autres traditions et les monuments figurés qui nous montrent cet arbre au nombre des symboles du même dieu ; le nom de \emph{Cyparisse} imposé au jeune héros changé en cyprès par Apollon ; et enfin la dénomination d'\emph{Apolloniade} substituée à celle de \emph{Cyparissus} : telles sont les considérations qui peuvent justifier mon sentiment. Disons cependant que la divergence d'opinion signalée par Homère n'a pas cessé de se manifester parmi les écrivains grecs postérieurs, qui ont recherché l'origine du nom de la ville de Cyparissus : les uns le font dériver simplement du grand nombre de cyprès qui entouraient cette antique cité ; les autres, de son fondateur, appelé Cyparisse et supposé fils de Minyas et frère d'Orchomène.\footnote{Didyme et Eustathe, \emph{ad} Homer. \emph{Iliad.} 2., 519. Strabon, \emph{ubi supra}. Étienne de Byzance, \emph{ubi supra}.} Mais cette dernière version est en faveur de ma conjecture, car de nombreux exemples analogues nous autoriseraient, dans ce cas, à considérer le nom du fondateur comme un de ceux que portait Apollon lui-même.

La ville appelée \emph{Cyparisséeis} par Homère était située dans les états de Nestor.\footnote{\emph{Iliad.} 2., 600, 601.} M. Avellino\footnote{Mémoire cité, p. 9 ; p. 10, note 5.} croit qu'elle correspond à celle qu'Étienne de Byzance désigne sous le même nom et qu'il place en Messénie. Selon Heyne,\footnote{\emph{Ad} Homer. \emph{Iliad.} 2., 593.} remarque-t-il, la \emph{Cyparisséeis} d'Homère se confond avec la \emph{Cyparisséeis} de Strabon,\footnote{\emph{Geograph.} 8., p. 345, 348, 349.} également située dans les états de Nestor et bâtie sur le territoire de la Triphylie, aux confins de la Messénie.\footnote{Dans ses Recherches géographiques sur les ruines de la Morée, feu M. Puillon Boblaye, officier supérieur d'état-major, ne fait pas une mention particulière de cette Cyparisséeis de Triphylie. Il se borne à dire (\emph{Expéd. scient. de Morée}, sect. des sciences physiq. t. 2., p. 134) : « Les deux distances de la table théodosienne : \emph{Olympia-Samaco} 15. et \emph{Samaco-Cyparissa} 24. sont, la première, fautive, et, la seconde, parfaitement exacte. » --- Au lieu de l'ablatif \emph{Samaco}, les cartes rédigées par les géographes modernes portent \emph{Samicum}.} Cette identité n'est pas mise en doute par l'habile traducteur français de Strabon, M. de Laporte du Theil. Toutefois le savant napolitain, après avoir assimilé, non à la \emph{Cyparisséeis} d'Homère, mais à celle de Messénie,\footnote{Mémoire cité, p. 9.} la \emph{Cyparissia} mentionnée par Diodore de Sicile, dans son 15\textsuperscript{e} livre,\footnote{Chap. 67.} se montre disposé\footnote{Mémoire cité, p. 9, note 9 ; p. 10, note 5.} à identifier celle-ci avec la \emph{Cyparisséeis} de Strabon, bien que K. O. Müller,\footnote{\emph{Das neue Rhein. Mus.} t. 2., p. 171.} observe-t-il, n'ait pas hésité à rapporter à la \emph{Cyparissia} de Messénie le passage de Diodore, comme déjà l'avait fait le docte Eckhel.\footnote{\emph{D. N.} 2., 276.}

Les commentateurs d'Homère, de Strabon et de Pausanias ont tous reconnu que la concordance des témoignages de l'antiquité touchant les villes grecques qui s'appelèrent \emph{Cyparisséeis} ou \emph{Cyparissia}, présente de graves difficultés. Les passages du géographe grec ne sont pas eux-mêmes exempts d'obscurité. Cependant, après les avoir examinés avec attention, il me paraît probable que Strabon distingue deux \emph{Cyparissia}, l'une dans la Phocide,\footnote{\emph{Geograph.} 9., p. 423.} l'autre dans la Messénie,\footnote{\emph{Ibid.} 8., 349, 359.} non loin de Pylos, et une \emph{Cyparisséeis},\footnote{\emph{Ibid.} p. 345, 348, 349.} qui, dit-il, est déserte et se nomme aujourd'hui \emph{Cyparissia} au singulier et au féminin, comme là \emph{Cyparissia} de Messénie.\footnote{Ici le texte de Strabon manque de clarté. Le traducteur français en fait la remarque, et propose, sans cependant garantir la correction, de lire : « L'une et l'autre s'appelaient autrefois \emph{Cyparisséeis} ; mais toutes deux portent aujourd'hui le nom de \emph{Cyparissia}. »} Il place \emph{Cyparisséeis} dans l'ancienne Macistie, province qui faisait partie de la Triphylie et, par conséquent, de l'Élide. À cette occasion, il mentionne un fleuve du nom de \emph{Cyparisséeis} au masculin.\footnote{Anne Comnène, dans son \emph{Alexiade} (liv. 6., p. 173), donne à un lieu situé en Bithynie le nom neutre de \emph{Cyparissium}, \textgreek{Κυπαρίσσιον} ; mais je ne découvre aucune trace de cette localité dans les géographes anciens, ni sur nos meilleures cartes de géographie.} La \emph{Cyparisséeis} de Macistie était située au-delà du petit fleuve appelé \emph{Néda}, et ainsi non loin du lieu où nous avons trouvé, sur les pas de Pausanias, un temple et un \emph{cupressetum} consacrés à la Vénus assyrienne ou phénicienne sous le nom grec d'Eurynome.\footnote{Ci-dessus, p. 220-234.}

Pline\footnote{\emph{H. N.} 4., 7., 1.}, Ptolémée\footnote{3., 16.} et Eustathe\footnote{\emph{Ad} Homer. \emph{Iliad.} 2., p. 297.} mentionnent aussi là \emph{Cyparissia} de Messénie. Pausanias,\footnote{4., 36., 7.} qui la nomme \emph{Cyparissies}, \textgreek{Κυπαρισσίαι}, parle d'une route qui d'Andania conduisait à cette ville, en passant par Électra. Il nous apprend, de plus, qu'auprès de \emph{Cyparissies} on voyait un temple dédié, comme je l'ai déjà dit, à Apollon et à Athéné surnommée \emph{Cyparissia}, circonstance qui semblerait nous révéler un nouveau trait de conformité entre la légende primitive de cette déesse et celle de la Vénus asiatique. \emph{Cyparissies} donna son nom à un port, à une rivière, à un promontoire et à un golfe situés dans son voisinage.\footnote{Strabon, \emph{Geogr.} 8., p. 359. Pline, \emph{loc. cit.} Pomponius Méla, 2., 3., 9.} Ce nom s'étendit même aux îles Plotæ ou Strophades : les anciens les appellent souvent les \emph{îles des Cyparissiens}, \textgreek{Κυπαρισσίων νῆσοι}.

Scylax\footnote{\emph{Geogr. gr. minor.} tome 1. (\emph{Peripl.}), page 16 ; édition Hudson ; (\emph{Peripl.}), § 46, page 262 ; édition Gail. --- Voyez (\emph{ibid.} page 393) les observations de ce dernier éditeur.} parle d'un port de la Messénie, qu'il nomme \emph{Cyparissus}, \textgreek{Κυπάρισσος}. Ce ne peut être pour nous que le port de la ville dont il vient d'être question. On sait que feu M. de Sainte-Croix\footnote{\emph{Mém. de l'Acad. roy. des inscr.} t. 44., p. 373 et suiv.} avait proposé de corriger le passage de ce géographe et de considérer \textgreek{Κυπάρισσος} comme une ville située dans l'intérieur des terres et non comme un port. On sait aussi que les juges compétents n'ont point admis cette opinion. En dernier lieu, feu M. Puillon Boblaye\footnote{Ouvrage cité, t. 2., p. 115.} a exposé les raisons très-plausibles qui le portent à croire que Scylax, dans son Périple, avait indiqué avec précision la position de \emph{Cyparissies} de Messénie, mais que le texte, altéré par les copistes, a été, de nos jours, restitué d'une manière fautive. Selon lui, ce texte devait, dans l'original, présenter le sens que voici : « La première ville de Messénie est Cyparisse avec un port, ville éloignée de sept stades de la mer. » L'exploration des ruines situées au lieu actuellement nommé \emph{Arcadia} paraît avoir démontré à M. Puillon Boblaye qu'en effet la ville antique ne s'étendait pas jusqu'au port.

Un bois sacré de cyprès, dont je ne rencontre aucune mention dans les auteurs anciens, couvrait jadis les flancs de deux montagnes situées non loin des ruines de \emph{Cænopolis} ou \emph{Cœnopolis} et près d'un lieu où l'on arrive en sortant de Messène par la porte de Laconie. Là se trouve maintenant un monastère qui porte plusieurs noms : \emph{le couvent des Cyprès, la Trinité, la Panagie de Vourcano}. En avant de cet asile religieux, on voit un bouquet de cyprès ; c'est tout ce qui reste de l'ancien \emph{cupressetum}.\footnote{\emph{Expédit. scientif. de Morée}, Relation, t. 1., p. 302 ; atlas, p. vj, cul-de-lampe.}

Dans le Péloponnèse, sur le territoire de la province appelée \emph{Éleuthéro-Laconie}\footnote{Strabon, \emph{Geogr.} 8., p. 363.} et au fond d'un port aujourd'hui nommé \emph{Xyli},\footnote{Puillon Boblaye, ouvrage cité, t. 2., p. 97. Cf. Leake's \emph{Travels in the Morea}, vol. 1., ch. 8. --- Selon le traducteur français de Strabon (t. 3., p. 209), les ruines de cette \emph{Cyparissia} de Laconie seraient situées près du bourg appelé \emph{Castel Rampano}. Mionnet (\emph{Suppl.} 4., p. 210), d'après une autorité qu'il ne nous fait pas connaître, applique cette dénomination moderne aux ruines de la \emph{Cyparissia}, de Messénie ; mais Siebélis, dans son édition de Pausanias, M. le major Leake (\emph{ubi supra}, ch. 2., 10., 11.) et feu M. Puillon Boblaye (\emph{loc. cit.} p. 115), dans deux relations d'une date plus récente, s'accordent à placer les ruines de cette \emph{Cyparissia} de Messénie au lieu qui maintenant porte le nom d'\emph{Arcadia}.} existait très-anciennement une autre cité maritime, du nom de \emph{Cyparissia}. Elle était déjà détruite à l'époque où Pausanias visitait la contrée ; il paraît qu'on donnait aux ruines de cette \emph{Cyparissia} le nom de \emph{cité des Achéens Paracyparissiens}, \textgreek{πόλις Ἀχαιῶν τῶν Παρακυπαρισσίων}.\footnote{Pausanias, 3., 22., 7. Voy. K. O. Müller, \emph{Die Dorier}, 3., 2.} L'officier français déjà cité les a retrouvées, et leur reconnaît un caractère d'antiquité que ne lui ont pas présenté les ruines d'Asopus,\footnote{Puillon Boblaye, ouvrage cité, t. 2., p. 97.} autre ville de Laconie, située non loin de la cité des Achéens Paracyparissiens.

C'est dans l'acropole d'Asopus que Pausanias\footnote{\emph{Loc. cit.}} nous a signalé un temple dédié à Athéné \emph{Cyparissia}. C'est aussi en Laconie que nous avons vu deux traditions\footnote{Ci-dessus, p. 201, note 4 ; et p. 205.} transporter le mythe de Cyparisse. Un village situé dans la même province, près du promontoire du Ténare, porte encore maintenant le nom de \emph{Cyparisso}, ou de \emph{Tiparisto}, par corruption.\footnote{C'est là que fut trouvée une inscription grecque, que M. Boeckh, d'après Pouqueville, rapporte dans son \emph{Corpus inscript. græcar.} t. 1., p. 645, n° 1317.} Enfin je trouve, dans les Recherches déjà citées de feu M. Puillon Boblaye,\footnote{Ouvrage cité, t. 2., p. 102.} l'indication d'un joli port, nommé aujourd'hui \emph{Kyparissi},\footnote{J'ignore cependant si ce nom moderne représente le nom antique ; et je remarque que feu M. Puillon Boblaye (\emph{loc. cit.}), tout en se décidant, d'après l'autorité de Ptolémée, et contrairement au sentiment de Niger, à placer l'ancien port, appelé \emph{Cyphanta}, au lieu actuellement nommé \emph{Kyparissi}, avoue qu'il n'a trouvé là aucune trace de ville ancienne. « Le joli port de Kyparissi, dit-il, est entouré de palæokastro du moyen âge, qui ont pu faire disparaître les ruines antiques. »} et situé sur le littoral de l'ancienne Éleuthéro-Laconie. Tout concourt, on le voit, à nous apprendre que, dès une époque fort reculée, le culte du cyprès et de quelques divinités qui comptent cet arbre parmi leurs attributs symboliques était fort répandu chez les Laconiens, en même temps que Lacédémone adorait une Aphrodite armée, dont l'origine asiatique n'est pas douteuse.

Les seuls monuments numismatiques que, jusqu'à ce jour, leur légende ait permis de rapporter sans hésitation à une ville grecque du nom de \emph{Cyparissia}, de \emph{Cyparisséeis} ou de \emph{Cyparissus}, sont des monnaies impériales frappées à l'effigie de Septime-Sévère, de Julia Domna, de Caracalla, de Plautille et de Géta ; elles portent toutes, au revers, le nom grec \textgreek{ΚΥΠΑΡΙϹϹΙΕΩΝ}, gravé autour de divers types, dont aucun ne nous offre le symbole du cyprès. Les numismates\footnote{Eckhel, \emph{D. N.} 2., 276, 277. Mionnet \emph{Descript. de méd.} 2., 212, 213, n°\textsuperscript{s} 31-33, \emph{Suppl.} 4., 210-212, n°\textsuperscript{s} 20-31. M. l'abbé Cavédoni, \emph{Spicileg. numism.} p. 97.} s'accordent à attribuer ces pièces à la \emph{Cyparissia} de Messénie ; mais un grave dissentiment règne parmi eux au sujet d'une monnaie autonome, de grand module, que Pellerin\footnote{\emph{Rec. de méd. de peuples et de villes.} --- Eckhel, qui n'avait pas eu la médaille sous les yeux, la rapporte à \emph{Cyparissus du mont Parnasse, près de Delphes}, en se bornant à citer et à copier Pellerin. (Voy. \emph{D. N.} 2., 193).} rapporte à Cyparissus de Phocide. Elle a pour type, au revers de la tête d'Apollon, un trépied placé dans une couronne, et pour légende un nom, en partie effacé, dont ce savant lit les premières lettres : \textgreek{ΚΥΠΑ} ... Mionnet\footnote{\emph{Descript. de méd.} 2. 96, n° 20.} croit devoir adopter cette lecture et cette attribution, tout en prévenant ses lecteurs que quelques antiquaires rangent la médaille parmi les autonomes de Cyzique de Mysie. Après lui, feu M. du Mersan,\footnote{\emph{Méd. inéd.} p. 41 et suiv.} au lieu de \textgreek{ΚΥΠΑ} ... , ou de \textgreek{ΚΥΖΙ} ... , a proposé de lire : \textgreek{ΚΥΤΕΙ} ... , et de restituer cette pièce a Cytinium, ville de la Doride. Plus récemment, M. Avellino\footnote{Mémoire cité, p. 11.} s'est prononcé contre l'opinion de Pellerin ; mais il n'approuve ni ne combat celle de M. du Mersan, et il s'abstient de nous faire connaître son propre sentiment. Ce n'est pas à moi qu'il appartient de décider la question.

En Occident, \emph{Cypresseta}\footnote{Antonini Aug. \emph{Itiner. Itinerar. a Burdig. Hierusal. usque}, p. 553, ed. Wesseling.} me paraît être la dernière ville qui nous présente uu nom analogue à ceux des diverses cités grecques dont je viens de parler. Elle était située, dans la Gaule narbonnaise, entre \emph{Arausio} et \emph{Avenio}, au lieu actuellement nommé le \emph{Pont de Sorgues}.\footnote{Voy. d'Anville, \emph{Notice sur l'anc. Gaule} (Paris, 1760, in-4°), p. 260, 261.} Mais on ne possède aucun renseignement sur l'époque de sa fondation, ni sur l'origine de son nom ; et, à ma connaissance, nos médailliers ne renferment pas une seule pièce que l'on puisse rapporter à cette ville.

Selon un antique usage, qui se révèle dans un grand nombre de prescriptions thérapeutiques indiquées par les médecins grecs, ceux-ci et, à leur exemple, les médecins latins attribuaient plus d'une vertu médicale au cyprès, de même qu'à quelques autres arbres ou plantes que la religion avait consacrés aux dieux. Ces vertus médicales, vraies ou supposées, étaient d'ordinaire en rapport avec les fonctions ou attributions dévolues à chaque divinité. Un tel usage, n'en doutons pas, avait pris naissance dans l'Asie occidentale ; et les peuples de cette vaste contrée l'avaient introduit dans la Grèce, en y important, avec leurs systèmes religieux, toutes leurs connaissances scientifiques comprises sous la dénomination unique de théologie, la science universelle. Les noms que donnaient les Grecs aux diverses productions des trois règnes de la nature en fourniraient plus d'une preuve ; mais ce n'est pas le lieu de traiter un pareil sujet. Nous n'avons à nous occuper ici que des prescriptions qui se rapportent à l'emploi médical du cyprès. Le P. Hardouin, dans son édition de Pline,\footnote{\emph{H. N.} t. 2., p. 328.} a réuni la plupart des renseignements que fournissent à cet égard Dioscoride, Pline lui-même, Galien et quelques autres écrivains de l'antiquité ; et je demande à l'Académie la permission de me référer aux citations du savant éditeur. Il n'a oublié ni Q. Sérénus Samonicus, ni Marcellus l'Empirique ; mais il m'a laissé le soin de mettre en relief les faits qui confirment mes remarques sur l'origine asiatique de cette thérapeutique et sur sa concordance avec les idées religieuses que l'Occident, comme l'Orient, attachait au symbole du cyprès.

Observons, en premier lieu, que Dioscoride était né dans la Cilicie, et que Pline,\footnote{\emph{Ibid.} 25., 5., 4.} qui avait principalement tiré des ouvrages attribués à Pythagore et à Démocrite, sur les effets des plantes, l'indication des diverses propriétés médicales de cet arbre, n'omet pas de nous dire\footnote{\emph{H. N.} 25., 5., 4.} que ces deux philosophes avaient visité les mages de Perse, d'Arabie, d'Éthiopie et d'Égypte. Nous pouvons, sans craindre de nous tromper, ajouter à ces mages les prêtres chaldéens de la Babylonie. Pline\footnote{\emph{Ibid.} 12., 39., 1.} avait, de plus, trouvé dans les Histoires écrites par l'empereur Claude quelques détails sur l'usage médical que les Parthes faisaient des feuilles et du bois de l'espèce de cyprès appelée \emph{bratus, brathus} ou \emph{brathys},\footnote{Voy. ci-dessus, p. 88-91.} qui croît sur le mont Zagrus, au-delà du Pasitigre, près de la ville de Sittacé.\footnote{Dans l'Élymaïde.}

Les anciens employaient comme médicaments, non-seulement les feuilles, le bois, la racine, les semences et l'huile de cyprès\footnote{\emph{Oleum cyprinum.} Pline, \emph{H. N.} 15., 7., 4.} ; mais ils composaient avec une partie quelconque de cet arbre un onguent\footnote{Pline, \emph{ibid.} 13., 2., 2 et 7.} et une sorte de vin et de vinaigre, désignés chacun par l'épithète de \emph{cyprin}. Ils avaient aussi des pilules cyprines. Galien, dans un long chapitre\footnote{\emph{De simplic. medicam. facultatib.} 7., 63.} sur les cas où il convient de faire usage des diverses préparations thérapeutiques que l'on tire du cyprès, recommande ces pilules, comme les feuilles et les semences, pour la guérison des ulcères, mais il ne s'explique pas sur leur composition.

Les feuilles s'employaient seules, broyées ou infusées ; on en exprimait aussi le suc ; d'autres fois on les pilait avec les semences. Elles étaient réputées très-efficaces pour la réduction des hernies, pour la guérison des maux de vessie, des affections morbides qui attaquent les parties génitales, et des maladies qui sont propres au sexe féminin.\footnote{On attribuait à l'espèce de sabine appelée \emph{cyprès de Crète}, la propriété de délivrer la matrice d'un fœtus mort pendant la gestation (Pline, \emph{Hist. nat.} 24., 61., 1). --- En effet, elle provoque l'avortement.} Broyées et appliquées sur la plaie, ou prises en breuvage, elles passaient pour un excellent remède contre la morsure des serpents venimeux. Dans le cas de piqûre de scorpion, on attribuait la même efficacité à une infusion ou décoction de racine raclée, qui devait être prise intérieurement, et cette boisson était prescrite aussi pour faciliter ou rétablir les menstrues.\footnote{Pline, \emph{H. N.} 24., 10., 1 et 2.}

L'huile de cyprès possédait, assure Pline,\footnote{\emph{Ibid.}} des qualités semblables ou analogues à celles qu'on se plaisait à reconnaître aux feuilles et aux racines. Elle est de plus recommandée, comme un puissant antidote contre les piqûres de tous les insectes venimeux, en général, et contre toute espèce de poison corrosif. On lui attribuait enfin les mêmes propriétés qu'à l'huile de myrte,\footnote{\emph{Ibid.} 23., 44., 1 ; 45., 1.} c'est-à-dire, à l'huile qu'on tirait d'un arbuste consacré à Vénus, comme le cyprès.

Ainsi l'arbre dont l'Occident, à l'exemple de l'Orient, faisait le symbole de la vie, l'attribut et même l'emblème des dieux créateurs ou générateurs, avait, chez les Grecs et les Romains, sinon, comme en Chine,\footnote{Ci-dessus, p. 161.} le privilège de donner l'immortalité, celui du moins de fournir des médicaments réputés efficaces pour combattre tout ce qui peut attaquer les sources de la vie, ou les organes de la reproduction. Il n'est pas inutile d'ajouter que l'expérience des médecins modernes est loin d'avoir confirmé toutes ces prétendues propriétés. Les cas, en très-petit nombre, où ils prescrivent encore des médicaments fournis par le cyprès, deviennent de plus en plus rares.

Je ne m'arrêterai point à rapporter ici les témoignages des auteurs anciens qui nous font connaître les divers usages domestiques auxquels jadis, en Occident, chez les Grecs et les Romains surtout, on employait le cyprès pyramidal. Ces témoignages sont cités, ou du moins indiqués, dans le recueil des œuvres du Dr. Lampe,\footnote{Tom. 1., p. 574-593.} dans la Géographie de M. Carl Ritter,\footnote{11., p. 567-582.} trésor inappréciable d'érudition, dans le recueil des Mémoires de l'Académie d'Herculanum,\footnote{Tom. 5., p. 393.} et, en particulier, dans le Mémoire de feu M. Avellino sur le mythe de Cyparisse.\footnote{Pages 7, 8, 31.} Quelques mots qui se trouvent dans un passage des \emph{Géoponiques}\footnote{2., 4.} semblent résumer les titres du cyprès à la faveur dont il jouissait chez les anciens : « Le cyprès, est-il dit à la fin de ce passage, a été donné par les dieux pour charmer les hommes.\footnote{« ... \textgreek{διὰ τέρψιν ἀνθρώποις} »} »

\subsubsection{Étrurie et Picentine}
\paragraph{}
Les institutions religieuses ou civiles des Étrusques, et les monuments figurés découverts sur le sol qu'habita ce peuple si intéressant à étudier, sont trop fortement empreints tout à la fois de coutumes asiatiques et d'usages propres aux Grecs, pour que le culte du cyprès pyramidal n'ait pas laissé en Étrurie, comme en Grèce, des traces faciles à retrouver. J'ai déjà dit\footnote{Ci-dessus, p. 68 ; p. 215.} que les Étrusques avaient reçu de l'Asie occidentale le culte d'une divinité féminine appelée \emph{Cypra} ou \emph{Cupra},\footnote{Au sujet de cette déesse et de son temple en Étrurie, on peut consulter : Tite-Live, 5., 21., 22. ; cf. 27., 37. Turnèbe, \emph{Advers.} 16., 19. Gori, \emph{Mus. etrusc.} t. 2., p. 84, 164. Passéri, \emph{Dissertat.} 1., \emph{ibid.} t. 3., p. 48. Inghirami, \emph{Monum. etrusc.} t. 2., p. 235, 242. Dr. Frédéric Creuzer, \emph{Symbolik}, t. 2., p. 490 ; t. 3., p. 256, 553, 666, 3\textsuperscript{e} édit. --- Remarquez qu'aucun de ces écrivains n'a aperçu l'identité du nom de \emph{cypra} ou \emph{cupra} avec les noms ou les mots que j'indique un peu plus loin.} nom identique avec celui de \emph{Cypris}, et avec un de ceux du cyprès et du cuivre, arbre et métal consacrés à Vénus. J'ai dit aussi\footnote{Ci-dessus, p. 217, note 2.} que la culture du cyprès, anciennement introduite dans la Grande Grèce, s'y était si bien nationalisée que Caton, traçant aux Romains des préceptes pour la propagation de cet arbre, le désigne sous la dénomination de \emph{cyprès de Tarente} ; mais je n'ai pas encore rapporté un passage de Strabon, qui doit trouver sa place ici. Ce géographe, décrivant le littoral de la Picentine (\emph{Picenum}), s'exprime en ces termes\footnote{\emph{Geogr.} 5., p. 241.} : « De là, on trouve le temple de Cypra (\textgreek{Κύπρα}), bâti par les Tyrrhéniens et dédié par eux à la déesse Héra, que ces peuples révèrent sous le nom de Cypra. » Silius Italicus\footnote{8., 432.} fait allusion à ce temple lorsqu'il dit :

\begin{quotation}
Et quis litoreæ fumant altaria Cupræ.
\end{quotation}

\paragraph{}
La déesse avait même imposé son nom à deux villes de la Picentine. Dans les auteurs anciens, ou dans les inscriptions lapidaires, elles sont appelées, la première, \textgreek{Κούπρα μαριτίμα},\footnote{Ptolémée, \emph{Geogr.} 3., p. 69 ; ed. Bertio.} \emph{Cupra maritima},\footnote{Gruter, \emph{Inscript. antiq.} p. 108. n° 7.} \emph{Cupra litorea},\footnote{Silius Italicus, \emph{loc. cit.}} ou simplement \emph{Cupra}\footnote{Pline, \emph{H. N.} 3., 18. Pomponius Mela, 2., 4., 10 ; \emph{ibique} Solin.} ; la seconde, \textgreek{Κούπρα μοντάνα}.\footnote{Ptolémée, \emph{loc. cit.} p. 72. Pline (\emph{loc. cit.}), parlant des habitants de cette ville, les désigne sous le nom de \emph{Cuprenses cognomine montani}.} Dans l'\emph{Agro Firmano}, diocèse de Ripatransona, le lieu nommé aujourd'hui \emph{Le grotte a mare}\footnote{Selon quelques-uns, \emph{Cupra maritima} aurait été située non dans ce lieu, mais dans le voisinage et au nord de Marano, château du diocèse de Ripatransona, par conséquent, dans l'\emph{Argo Massignano}, qui fait partie du territoire de S. Quirico. (Voy. la traduction française de Strabon, t. 2., p. 237, 238, note 3.)} paraît répondre à l'ancienne \emph{Cupra maritima}. C'est près de là que fut trouvée une plaque de marbre, où se lit une inscription latine, destinée à perpétuer le souvenir d'un acte de munificence de l'empereur Trajan,\footnote{\emph{Trib. potest.} 11., cos. 3.} qui avait restitué le temple de la déesse Cupra. Cette inscription a été publiée par Gruter,\footnote{Recueil cité, p. 1016., n° 2.} d'après Ursin.

Déjà ces divers faits nous autorisent à supposer qu'avec le culte asiatique de Cypra ou Cupra s'était introduit en Picentine et en Étrurie celui du cyprès pyramidal, évidemment aussi inséparable du culte de cette déesse qu'il l'était du culte de la Vénus assyrienne, soit dans l'Asie occidentale, soit en Égypte, soit chez les Grecs. Ma supposition se change en certitude à la vue de quelques monuments de l'art exhumés, depuis plusieurs années, des hypogées étrusques. Au nombre de ces antiquités se trouvent des sarcophages ou des urnes cinéraires dont je me réserve de parler en détail dans le mémoire suivant. Toutefois, par anticipation, je dirai brièvement ici que, sur sa face antérieure, un de ces monuments funéraires\footnote{Voyez le n° 4 de la planche 13 ci-jointe.} est décoré d'un bas-relief où l'on reconnaît sans peine l'image de Vénus placée entre deux cyprès, comme sur les bas-reliefs asiatiques. Ici la tête de la déesse est ornée de deux grandes cornes de vache, ce qui nous reporte aussi dans les contrées de l'Orient où Vénus était représentée tantôt avec une tête de vache au lieu d'une tête humaine, tantôt sous la forme d'une vache allaitant son veau. Les autres monuments que j'ai à citer ne sont ni des urnes cinéraires, ni des sarcophages ; et, bien qu'ils aient été découverts dans des tombeaux et qu'ils soient ornés de plusieurs symboles souvent employés avec un sens funéraire, on ne doit peut-être pas les considérer comme primitivement destinés à des usages funèbres. En première ligne, se place ici un vase peint, de la collection Candélori, publié par feu M. Micali\footnote{\emph{Storia degli antich. pop. ital.} (Firenz. 1832), t. 3., p. 163-165 ; pl. 95., A-E.} et déclaré par lui d'origine purement étrusque : \emph{a me pare tutto etrusco}, nous dit-il en propres termes. Ce vase, à deux anses et à figures noires, blanches et rouges sur un fond jaune, est couvert de peintures réparties dans cinq bandes ou divisions horizontales. On y voit Hercule combattant les Centaures,\footnote{\emph{Storia degli antich. pop. ital.} pl. 95., B, E.} et plusieurs autres scènes\footnote{\emph{Ibid.} B, D.} qui se rapportent, comme celle-ci, à l'initiation aux mystères. Dans un de ces cinq tableaux, deux griffons ailés sont placés chacun à l'extrémité opposée d'un groupe composé d'un lion et de deux panthères ou lionnes, dont la seconde dévore une vache. Entre la première et le lion s'élève un cyprès pyramidal, qui n'a point été remarqué par M. Micali, et qui, à mes yeux, est l'image symbolique de la déesse chargée de présider aux mystères. Je montrerai, dans mon ouvrage sur le culte de Mithra, comment les griffons et le groupe de quadrupèdes que nous voyons auprès de cet arbre se rattachent directement à la série des symboles employés dans les initiations aux mystères de la Vénus assyrienne et du dieu des Perses.

C'est avec plus de certitude encore que nous devons rapporter aux mystères de la première de ces deux divinités deux plats d'argent doré, qui proviennent d'un tombeau étrusque et qui nous offrent chacun un exemple de l'emploi religieux du cyprès pyramidal. Ce tombeau, de grandes dimensions, fut découvert, en 1836, non loin des ruines de \emph{Cæré}, par M. le général Galassi et M. l'abbé Rigolini. Tous les objets qu'il renfermait ont été déposés à Rome, au Musée grégorien. En 1841, M. le chevalier Louis Grifi\footnote{\emph{Monum. di Cere antica spiegati coll. osservanz. del culto di Mitra} (Rom. 1841, 1 vol. petit in-fol. fig.).} en a publié la description et les dessins. L'année suivante, ils furent également figurés dans la description du Musée étrusque grégorien dont nous sommes redevables au P. Marchi.\footnote{\emph{Mus. etrusc. gregorian.} parte 1. (Rom. 1842, gr. in-fol. fig.).} Je ne m'arrêterai pas à démontrer ici combien peu est fondée l'opinion du premier de ces deux archéologues, qui, tout en proclamant la haute antiquité des ouvrages d'art trouvés dans le grand tombeau de Cæré, prétend les rattacher aux mystères de Mithra. Je traiterai ailleurs cette question ; mais, en attendant, je dirai que l'origine phénicienne ou assyrienne des deux plats dont j'ai à parler ne saurait être mise en doute par aucune personne tant soit peu versée dans la connaissance des antiquités asiatiques. Le style, le travail, les sujets, les symboles, tout concourt à justifier cette assertion. Un médaillon central et en relief orne le fond de chacun de ces deux plats. L'un\footnote{M. Grifi, ouvrage cité, pl. 10., n° 1. \emph{Mus. etrusc. gregorian.} parte 1., tav. 65., n\textsuperscript{i} 1 e 2.} représente deux colombes, oiseau sacré de Vénus, voltigeant au-dessus d'une vache qui allaite son veau, image symbolique de la déesse, considérée comme mère de l'Amour.\footnote{Voy. ci-dessus, p. 59. M. le Dr. Chr. Walz, \emph{De Nemesi Græcor.} p. 16. --- Un fait très-curieux, qui me paraît avoir jusqu'à ce jour échappé à l'attention de tous les archéologues, ressort du médaillon central où, sur le plat d'argent doré dont je parle, on a représenté le groupe symbolique de la vache allaitant son veau. Ici (M. Grifi, \emph{ubi supra. Mus. etrusc. gregor.} parte 1., tav. 65., n° 2), devant la vache, et la tête élevée vers elle, nous voyons un second veau, dont l'attitude indique qu'à son tour, il voudrait être allaité comme son jumeau. Peut-on se défendre de supposer qu'une telle représentation était destinée à constater que, sur cette coupe, on avait voulu perpétuer symboliquement le souvenir de la naissance d'Éros et d'Antéros, représentée ailleurs par deux enfants ou deux Amours qui accompagnent Vénus, leur mère, ou qui sont placés tantôt sur ses bras, tantôt dans son giron ?} L'autre médaillon nous offre un groupe composé de deux lions, ou d'un lion et d'une lionne, dévorant un taureau,\footnote{M. Grifi, ouvrage cité, pl. 5. n° 1. \emph{Mus. etrusc. gregor.} parte 1. tav. 66., n\textsuperscript{i} 1 e 2.} groupe caractéristique dont l'invention et la signification appartiennent indubitablement à l'art hiératique des Assyriens, des Phéniciens et des Perses. Je crois avoir mis ce fait hors de doute\footnote{Voy. \emph{Mémoires de l'Acad. des inscript.} nouv. série, t. 15., 2\textsuperscript{e} partie, p. 63-127.} bien avant la découverte des bas-reliefs de Nimroud, qui nous ont permis de constater que, dès une haute antiquité, des compositions semblables ou analogues faisaient partie des scènes religieuses sculptées sur les murs des palais érigés par les rois d'Assyrie.\footnote{Voyez Layard's \emph{Monum. of Nineveh}, pl. 9. ; pl. 45., n° 2 ; pl. 46., n° 1 ; et pl. 48., n° 2. Cf. pl. 8.} De plus, les sujets que représentent les deux médaillons de nos plats reposent chacun sur un soubassement en échiquier ou en treillis et semi-circulaire, qui se rencontre exclusivement dans les monuments assyriens ou phéniciens. Ces deux médaillons sont entourés chacun de deux bandes circulaires, qui offrent, en relief, des scènes analogues à celles qu'on voit tout à la fois sur les deux coupes citées\footnote{Ci-dessus, p. 64, p. 95.} de l'île de Cypre, sur les murs des palais de Nimroud et de Persépolis, sur les coupes de bronze assyriennes provenant des dernières fouilles de M. A. H. Layard, et sur des cylindres et des cônes de travail babylonien, assyrien, phénicien ou persépolitain. La présence, sur un des deux plats de Cæré, du groupe symbolique de la vache allaitant son veau, l'identité de style, de forme, de matière et de travail que présentent entre eux ces deux monuments, tout nous autorise à les rapporter l'un et l'autre, non, comme l'a fait M. Grifi, aux mystères de Mithra, divinité mâle, mais aux mystères de la déesse que l'Asie occidentale, Corinthe et ses colonies adoraient sous la forme d'une vache allaitant son veau, ou d'une femme à cornes de vache, et sous les noms de Mylitta, Anaïs, Astarté, Aphrodite, etc. ; les Étrusques, sous les mêmes formes et sous les noms de Cypra ou Cupra, de Turan ou Touran, etc. Je m'abstiens de présenter ici l'interprétation des figures et des symboles qui entourent le médaillon central de chacun de nos deux plats. Je l'ajourne par la même raison qui m'a obligé d'ajourner l'interprétation des deux coupes citées de l'île de Cypre, c'est-à-dire, par l'impossibilité de me faire comprendre sans avoir préalablement exposé l'ensemble de mes idées sur les symboles et les cérémonies propres à l'initiation aux mystères. Je me borne à prier l'Académie de remarquer que, sur l'un des deux plats,\footnote{M. Grifi, ouvrage cité, pl. 10., n° 1. \emph{Mus. etrusc. gregorian.} parte 1., tav. 65., n\textsuperscript{i} 1 e 2.} dans les deux bandes circulaires, plusieurs colombes voltigent entre les cyprès pyramidaux plantés au bord de la voie sacrée que parcourt une pompe religieuse. Sur l'autre plat,\footnote{M. Grifi, \emph{ibid.} pl. 5., n° 1, \emph{Mus. etrusc. gregor.} parte 1., tav. 66., n\textsuperscript{i} 1 e 2.} la bande qui entoure immédiatement le médaillon central nous offre des combats mystiques, qui se livrent au pied de plusieurs cyprès entre lesquels voltigent aussi des colombes. De plus, on voit ici deux palmiers femelles\footnote{Nous retrouvons ces deux palmiers femelles, ou un seul palmier femelle, sur plusieurs cylindres ou cônes, de travail babylonien ou assyrien, et sur un bas-relief mithriaque, d'époque romaine. (Voy. \emph{Rech. sur le culte de Vénus}, pl. 1., n°\textsuperscript{s} 1 et 8 ; 4., n° 12 ; et \emph{Recherches sur Mithra}, pl. 16., n° 4 ; 25., n° 6 ; 27., n° 7 ; 51., n° 2 ; 52., n° 4 ; 62., n° 3 ; 74.) --- Nous avons vu plus haut (p. 91, 92 ; pl. 6., n°\textsuperscript{s} 1 et 2 ; et pl. 7., n°\textsuperscript{s} 4 et 5) que, sur les médailles impériales, frappées dans la Phénicie, le palmier et le cyprès figurent alternativement comme emblèmes ou symboles d'Astarté. Il en est de même sur les cylindres asiatiques consacrés aux mystères de la déesse. Alciat (\emph{Emblemat. libell.} emblem. 198), à qui ces monuments étaient inconnus, n'a pas ignoré cependant l'espèce de parenté que l'antiquité avait établie entre ces deux arbres symboliques.} et deux vautours,\footnote{Je montrerai ailleurs que le vautour remplit ici le même rôle qui lui est assigné sur plusieurs bas-reliefs et cylindres assyriens.} symboles consacrés à la Vénus assyrienne. La seconde bande représente, comme les deux bandes du premier plat, une pompe religieuse, qui suit une voie bordée de cyprès ; entre ces arbres volent des colombes et un vautour. Dans le mémoire suivant nous retrouverons, sur un sarcophage étrusque, des cyprès plantés au bord d'une voie que parcourent des personnages montés sur des chars.

On le voit donc, le cyprès, chez les Étrusques, comme chez les peuples de l'Asie occidentale, de l'Arabie, de l'Égypte et de la Grèce, remplissait, dans la composition des monuments figurés, un rôle essentiellement religieux. Ce n'est pas le seul exemple qu'on pourrait citer du culte des arbres en Étrurie, et je demande à l'Académie la permission de lui rappeler que j'en ai, il y a bien des années, signalé d'autres exemples à l'occasion des peintures trouvées dans les grottes Marzi et Querciola.\footnote{\emph{Annal. de l'Instit. arch.} t. 5., p. 90-113.}

\subsubsection{Rome et empire romain}
\paragraph{}
Diverses remarques ou citations placées dans ce mémoire, divers monuments de l'art, dont j'ai déjà invoqué le témoignage, nous ont préparés à reconnaître que le culte du cyprès pyramidal avait pénétré chez les Romains bien des siècles avant l'époque où il leur arriva, uni au culte persique de Mithra. À Rome, dès le temps même de la fondation de la ville, nous le trouvons lié au culte des divinités nationales. Pline\footnote{\emph{H. N.} 16., 76., 1.} fait une mention expresse d'un cyprès et d'un lotos qui, selon Masurius, étaient aussi anciens que la ville ; ils passaient pour avoir été plantés auprès du temple que Romulus victorieux venait d'ériger en l'honneur de Vulcain, avec la dîme du butin enlevé à l'ennemi. Le cyprès, ajoute Pline, tomba vers la fin du règne de Néron et ne fut pas remplacé. De son côté, Tite-Live\footnote{5., 21-23. ; 27., 37. Obsequens, \emph{De prodigiis}, cap. 106.} rapporte que les Romains avaient enlevé de la citadelle de Véies (\emph{Veii}), en Étrurie, et placé dans un temple bâti exprès sur le mont Aventin, deux antiques statues de bois de cyprès, représentant Junon, surnommée Reine, \emph{Regina}. Ce précieux renseignement nous ramène tout à la fois à la \emph{Junon Reine}, \textgreek{Ἥρα βασιλίς}, qu'on adorait dans l'île de Samos, primitivement appelée \emph{Cyparissia}, et à la \emph{Junon Cypra} ou \emph{Cupra} des Étrusques, dont le culte, sans aucun doute, était d'origine asiatique. On peut juger de la profonde vénération qu'inspiraient aux Romains les deux statues de bois de cyprès conquises à Véies, lorsqu'on lit, dans le même Tite-Live,\footnote{27., 37.} qu'elles furent solennellement portées en procession, accompagnées de vingt-sept vierges et de dix prêtres, pour expier les prodiges arrivés pendant la guerre d'Hannibal en Italie.\footnote{Sous le consulat de Cn. Cornélius Lentulus et de P. Licinius.} A côté de ce passage vient se placer celui de Pline\footnote{\emph{H. N.} 16., 79., 3.} où nous apprenons que, depuis l'an 661 de la fondation de Rome,\footnote{Il me paraît bien difficile de ne pas supposer qu'il y a ici, dans le texte, une erreur de copiste. La statue dont parle Pline remontait probablement à une époque beaucoup plus reculée que l'an de Rome 661.} il existait au Capitole\footnote{Des écrivains postérieurs disent que cette statue était placée dans un temple situé entre la citadelle et la roche tarpéienne.} une statue de Jupiter Véjove, de bois de cyprès, qui s'était parfaitement conservée jusqu'au temps où cet auteur écrivait. Il est curieux de rapprocher de ces témoignages les vers cités de Virgile,\footnote{Ci-dessus, p. 81.} qui nous ont montré que les Romains n'ignoraient point l'attribution faite du cyprès à Rhéa ou Cérès par les Troyens, et un long passage de Claudien,\footnote{\emph{De raptu Proserp.} 3., v. 370-403.} où nous voyons Cérès, à la recherche de sa fille Proserpine, arriver sur le mont Etna, y couper deux cyprès, les prendre en main, et les plonger, par leur cime, dans le cratère du volcan, pour les allumer et s'en faire deux immenses torches.

Les Romains, conquérants de l'Asie occidentale, purent donc regarder comme une nouvelle preuve de l'origine asiatique qu'ils aimaient à s'attribuer, l'usage répandu dans cette contrée de consacrer le cyprès pyramidal à des divinités qui s'identifiaient avec leur Jupiter, avec leur Junon, avec leur Vénus, avec la Mère des Dieux. Une même pensée dut se présenter à leur esprit, lorsque, chez les peuples de l'Orient, ils virent la consécration de cet arbre s'étendre à plusieurs autres dieux ou déesses, nommément au Soleil et à la Lune divinisés ? La métamorphose de Cyparisse en cyprès, chantée par Ovide, la belle fresque qui, à Pompeï, représentait cette ingénieuse métamorphose, quelques vers de Virgile, les savants commentaires de Servius, et enfin un passage d'Apulée et plusieurs monuments figurés dont il me reste à entretenir la Compagnie, ici ou dans le mémoire suivant, nous prouvent, en effet, que, chez les Romains, dès une époque antérieure à leurs premières expéditions en Asie, le culte du cyprès était intimement lié à celui d'Apollon, de Diane, de Vénus, de Mercure et de quelques autres divinités.

Je ne reviendrai pas sur les divers traits que, déjà, j'ai mis en relief, à l'occasion du récit d'Ovide et de la fresque de Pompeï,\footnote{Ci-dessus, p. 201-205.} deux tableaux dont la poésie et la peinture grecques avaient certainement fourni les modèles aux Romains. Je ne reviendrai pas non plus sur les passages de Philostrate, de Nonnus, de Virgile, de Servius et de Lactantius Placidus qui se rapportent au mythe de Cyparisse. Ce que j'en ai dit longuement me laisse à peine le soin de répéter ici que tout, dans ce mythe, décèle une origine asiatique et la transmission successive en Grèce et en Italie des traditions écrites ou des monuments figurés qui nous en ont conservé le souvenir.

L'imitation d'un type grec, imité lui-même ou plutôt transformé d'après un type asiatique, ne se décèle pas moins dans un disque d'argent, de travail romain, représentant Apollon avec l'attribut du cyprès, et dans le revers des trois médailles citées plus haut,\footnote{Pag. 105-107 ; pl. 13., n°\textsuperscript{s} 1, 2 et 3.} qui appartiennent à la numismatique d'Alexandria-Troas devenue colonie romaine. Ces trois pièces et le disque, rapprochés entre eux, fournissent un témoignage important et irrécusable. Le disque est figuré ici, avec les dimensions mêmes de l'original, sous le n° 5 de la planche 20. Il n'est pas inédit, mais il a été décrit et lithographié sans beaucoup de soin dans le \emph{Catalogue des collections de feu M. Toussaint Grille, d'Angers}.\footnote{Angers, 1851, in-8°, fig. (Voy. p. 43, n° 585 ; pl. 1., n° 25. ; pl 2., même numéro.)} Il provient d'un trésor qui, en 1836, et, par conséquent, peu d'années après l'heureuse découverte faite à Berthouville, près de Bernay, fut trouvé enfoui dans la terre, à Notre-Dame-d'Allençon, près de Brissac.\footnote{Département de Maine-et-Loire.} Celui-ci se composait de trente-quatre objets, dont le plus précieux, sous le rapport de la mythologie, du symbolisme et de l'art, est, sans contredit, notre disque d'argent. Ces trente-quatre objets, après avoir appartenu à M. Grille, ont récemment été acquis par le musée du Louvre, grâce à la sollicitude éclairée qui préside à la direction de ce bel établissement. Le disque a dû jadis orner le fond d'un plat ou de tout autre vase d'argent, soit votif, soit employé dans des cérémonies religieuses. Il représente, en relief et vu de face, Apollon debout, presque nu, le bras gauche appuyé sur un trépied à griffes de lion ; dans la main droite, le dieu solaire tient une branche de laurier. Une légère draperie, qui part de l'épaule gauche, couvre une portion des cuisses et toute la jambe gauche jusqu'au cou-de-pied ; les deux pieds sont nus. À droite et à gauche d'Apollon s'élèvent des rochers dont la disposition, pour le dire en passant, rappelle les rochers qui, sur un grand nombre de monuments romains, forment la grotte au milieu de laquelle est placé Mithra, le dieu solaire des Perses. Pour compléter l'analogie, un corbeau, fidèle compagnon d'Apollon, comme de Mithra, est posé sur les rochers placés à la droite du dieu. De là, à l'exemple du corbeau des bas-reliefs mithriaques, il semble contempler attentivement le sacrifice auquel il assiste ; mais ses regards s'arrêtent sur une grande roue qui forme une saillie très-prononcée. Cette roue ne se rencontre sur aucun des monuments mithriaques que nous ont légués l'antiquité persique et l'antiquité romaine. Je ne sais jusqu'à quel point il est permis de croire que, sur ces derniers, elle est, en quelque sorte, remplacée par les roues d'un char qui porte tantôt le soleil, tantôt Mithra lui-même. Symbole connu d'Apollon et de Némésis, et fréquemment placée sous la patte d'un griffon, animal solaire, la roue rappelle tout à la fois le mouvement rapide du soleil, et les arrêts du Sort ou du Destin. C'est à ce double titre qu'on la voit ici auprès d'Apollon, dieu solaire et créateur, qui préside à la vie et à la reproduction. Un grand cyprès pyramidal, planté entre le trépied et les rochers qui s'élèvent à l'extrémité droite de ce petit tableau, achève de caractériser, comme symbole de vie, les fonctions dévolues à ce dieu ; et cet arbre, nous allons le retrouver sur des monuments romains consacrés à Mithra. Ces divers rapprochements, la légende presque entière d'Apollon et le mythe de Cyparisse, en particulier, nous reportent encore une fois dans l'Asie occidentale. C'est là, en effet, que nous avons trouvé, à Alexandria-Troas, trois monnaies coloniales dont les types présentent la plus grande analogie avec le sujet figuré en relief sur le disque d'argent que je viens de décrire. Cette analogie est surtout frappante lorsque nous comparons ce précieux monument avec le revers de la pièce autonome coloniale d'Alexandria-Troas,\footnote{Pl. 13., n° 1.} qui représente Apollon debout et vu de face, comme ici et comme à Pompeï.\footnote{Pl. 12.} Sur d'autres médailles autonomes coloniales et sur quelques médailles impériales coloniales de cette ville,\footnote{Vaillant, \emph{Col.} 2., 63. Sestini, \emph{Descriz. delle med. ant. del mus. Hederv.} t. 2., p. 131, n° 24. C. M. H. n° 4765. Mionnet, \emph{Descript. de méd.} 2., 643, n° 103 ; 646, n° 120 ; 648, n° 132 ; \emph{Suppl.} 5., 512, n° 94 ; 522, n° 158 ; 522 et 523, n° 162 ; 523, n°\textsuperscript{s} 163 et 164 ; 529, n° 215 ; 539, n° 283.} de même que sur notre disque et sur la fresque de Pompeï, nous voyons une branche de laurier placée dans la main droite de ce dieu. D'autres médailles impériales d'Alexandria-Troas\footnote{Mionnet, \emph{Descript. de méd.} 2., 646, n° 116 ; \emph{Supplém.} 5., 519, n°\textsuperscript{s} 137 et 142.} nous montrent un corbeau posé aux pieds d'Apollon. Enfin, le style du disque d'argent du Louvre permet, ainsi que je l'ai annoncé plus haut,\footnote{Ci-dessus, p. 107.} de faire remonter ce monument au 1\textsuperscript{er} ou au 2\textsuperscript{e} siècle de l'ère chrétienne, c'est-à-dire à une époque peu éloignée du temps où Auguste avait établi une colonie romaine dans la ville même dont la numismatique vient de nous fournir des rapprochements qui ne sont pas, ce me semble, sans intérêt pour l'histoire de l'art et des origines mythologiques.

L'attribution du cyprès à Apollon, sur un disque romain trouvé dans les Gaules, nous autorise sans doute à présumer que ce symbole appartenait aussi à la Diane romaine, de même que nous l'avons reconnu parmi les attributs de l'Artémis grecque. Toutefois aucun des monuments figurés que nous ont légués les Romains ne nous en fournit la preuve. Mais un passage de Virgile\footnote{\emph{Æneid.} 3., v. 679-681.} supplée au silence de l'iconographie : on y trouve une mention expresse de forêts de chênes et de plantations de cyprès, les premières consacrées à Jupiter, les secondes à Diane. Le commentaire de Servius\footnote{\emph{Silva alta Jovis, lucusve Dianæ Jovis, propter quercum ; Dianæ, propter cupressum.} Servius, \emph{ad} Virgil. \emph{Æneid.} 3., v. 681.} confirme pleinement le témoignage du chantre de l'Énéide.

Un vers déjà cité du même poète\footnote{\emph{Georgic.} 1., v. 20.} et les observations de son commentateur nous ont appris qu'en Italie, dès une haute antiquité, le culte du cyprès se liait à celui d'un dieu appelé Silvain ou Sylvain, qui pour nous est resté très-énigmatique. J'ai rapporté plus haut\footnote{Pages 206, 207.} et le vers de Virgile et le commentaire de Servius. Je ne reviendrai pas sur les remarques qu'ils m'ont donné lieu de soumettre au jugement de l'Académie. Seulement je répéterai ici que la substitution de Silvain à Apollon dans le mythe de Cyparisse, les monuments de l'art cités par Heyne\footnote{\emph{Ad} Virgil. \emph{Georgic.} 1., v. 20.} et par Visconti,\footnote{\emph{Mus. Pio Clementino}, t. 8., p. 16-18 ; pl. 10. Cette planche 10 reproduit un bas-relief de marbre, qui représente plusieurs divinités rustiques, parmi lesquelles on distingue Silvain portant à la main un arbre dont toutefois l'espèce est difficile à reconnaître. On ne peut dire si c'est un pin, un cyprès, ou tout autre arbre.} et enfin l'inscription latine découverte dans l'Abruzze citérieure\footnote{Ci-dessus, p. 206.} concourent à établir que l'inimitable auteur des Géorgiques se conformait à une ancienne tradition lorsqu'il attribue à Silvain le symbole du cyprès.

D'autre part, l'ingénieux épisode des amours de Léda et de Jupiter métamorphosé en cygne a fourni au ciseau d'un sculpteur ancien l'occasion de montrer que l'attribution de ce même symbole à Vénus était tout aussi connue des Romains. Un monument trouvé, vers la fin du siècle dernier, dans le jardin supérieur de l'hospice de la charité à Vienne, en Dauphiné, représente, sous la forme d'un cyprès pyramidal, la déesse assistant, avec son fils Cupidon, au bonheur que goûtent les deux amants réunis sur le mont Taÿgète.\footnote{« Le Taÿgète, dit Strabon (\emph{Geograph.} 8., p. 362, 363), est une haute montagne à pic, peu éloignée de la mer, et qui, du côté du septentrion, touche aux pieds des montagnes d'Arcadie, avec lesquelles elle forme un vallon à l'endroit où la Messénie confine à la Laconie. »} Ce monument, publié une seule fois et de manière à être peu remarqué, est, jusqu'à ce jour, resté, pour ainsi dire, ignoré ou inaperçu. Dans leur recueil intitulé : \emph{Monuments romains et gothiques de Vienne, en France},\footnote{Paris, 1831, gr. in-fol. fig. lith. Voy. p. 11, et pl. 6.} MM. Rey et Vietty ont eu le tort de n'en donner que des dessins en miniature ; et, de plus, ces dessins, médiocrement lithographiés, sont disposés de façon à se confondre avec d'autres, sur une même planche. Au milieu de plusieurs monuments groupés ensemble, il n'est rien moins que facile de démêler les trois bas-reliefs qui ornent celui dont j'ai à parler, ni de comprendre quelle place ils y occupent. Enfin le texte explicatif ne fait aucune mention du cyprès planté à gauche de la scène érotique représentée sur le plus grand de ces trois bas-reliefs. Je mets sous les yeux de l'Académie d'autres dessins de ces sculptures, exécutés au musée de Vienne par M. Muret et gravés par M. Bigant avec toute l'exactitude et tout le soin que l'on connaît à ces deux habiles artistes.\footnote{Voy. les n°\textsuperscript{s} 3, 3a et 3b de la pl. 14. ci-jointe. --- Le monument original est carré ; il a 425 millimètres de haut sur autant de large.} Désormais, les archéologues qui n'auront pas eu l'occasion de voir sur les lieux le monument original, pourront prendre ici une idée juste de la forme, de la composition et du style des bas-reliefs dont ce bloc carré, de marbre blanc, est orné sur trois de ses quatre faces. MM. Rey et Vietty le désignent sous la dénomination d'autel de Jupiter. M. Delorme, qui, en 1841, l'a décrit dans son Catalogue du musée de Vienne,\footnote{Sous le n° 174.} suppose qu'il servait de piédestal à une des deux colonnes placées à l'entrée d'un \emph{sacellum}. Une lettre que j'ai sous les yeux, et qu'il a écrite, le 25 mars 1852, à M. le comte de Nieuwerkerke, directeur général des musées de France,\footnote{Je saisis avec empressement cette occasion de rendre publique l'expression de la reconnaissance que m'inspirent les procédés dont M. le comte de Nieuwerkerke veut bien user à mon égard.} m'apprend qu'il persiste dans son opinion ; il ajoute que ces deux colonnes étaient sans doute adossées au sacellum, ce qui, selon lui, expliquerait pourquoi la face postérieure du piédestal est restée brute, tandis que les trois autres sont richement sculptées. « Au-dessus de la corniche du piédestal, remarque-t-il, on distingue une plinthe, et, au-dessus, un tore couvert d'ornements. De plus, la face supérieure de ce tore n'est point unie et porte les traces du marteau, circonstance qui me paraît indiquer qu'elle était recouverte par une autre partie offrant les autres membres de la base de la colonne. » En 1841, j'ai vu le monument sur les lieux, et je n'ai pas hésité à le prendre aussi pour un piédestal. Mais je ne puis partager l'opinion de M. Delorme, quant à la destination qu'il lui suppose. Bien que ce marbre ne porte aucune inscription, aucune dédicace, et bien qu'il appartienne à une époque où déjà l'art romain était en décadence, il serait par trop insolite, ce me semble, de le considérer comme ayant servi de base à une colonne. Les bas-reliefs qui ornent trois de ses faces, et les sujets que représentent ces bas-reliefs, me porteraient plutôt à présumer qu'il était le piédestal d'une statue de Jupiter. Les sculptures de la face antérieure\footnote{Pl. 14., n° 3.} se rapportent directement à la longue liste des amours de ce dieu voluptueux. Les bas-reliefs des deux faces latérales\footnote{\emph{Ibid.} n°\textsuperscript{s} 3a et 3b.} se rattachent moins clairement à sa légende. On peut néanmoins supposer, avec quelque vraisemblance, qu'ils font allusion à la naissance ou à l'enfance de Jupiter ; car, sur la face latérale gauche, nous voyons deux Amours ou deux Génies ailés, dont l'un tient par le cou un bouc qui se dresse sur ses deux pieds de derrière. Entre les deux Amours, on remarque un autel et un vase qu'ombrage un chêne, arbre consacré à Jupiter. Sur le troisième bas-relief, c'est-à-dire sur la face latérale droite du piédestal, nous retrouvons un chêne. Au pied de cet arbre, est assis un personnage, qui semble être un pâtre, et non Mercure, comme on l'a avancé.\footnote{M. Vietty, ouvrage cité, p. 11.} Devant lui on voit une chèvre debout, peut-être la chèvre Amalthée, nourrice de Jupiter.\footnote{Peut-être est-il permis de considérer Amalthée, nourrice du maître de l'Olympe, comme une déesse dont l'image symbolique était une chèvre. Dès lors, le bouc figuré sur l'autre bas-relief latéral deviendrait pour nous l'image symbolique du dieu époux d'Amalthée ; et cette remarque nous expliquerait pourquoi le bouc était consacré à Vénus, divinité primitivement androgyne et qualifiée mère des Dieux. Que ces diverses conjectures soient fondées ou non, il convient d'ajouter que les Romains aimaient à varier la composition des sujets destinés à perpétuer le souvenir des rapports de Jupiter enfant avec la chèvre Amalthée. A l'appui de cette assertion, je me bornerai toutefois à citer un médaillon de bronze et un grand bronze d'Antonin le Pieux, qui, au revers de l'effigie de cet empereur, nous montrent Jupiter enfant assis sur une chèvre, devant un autel. (Voy. Mionnet, \emph{De la rareté et du prix des méd. rom.} t. 1., p. 213 et 218.)} En face du quadrupède, est placée une maison rustique, demeure probable du chevrier, transformée, bien à tort, en « un temple rustique, » sous la plume du premier éditeur.\footnote{M. Vietty, \emph{loc. cit.}}

Je n'insisterai pas sur la connexion qui peut exister entre les sujets de ces deux petits bas-reliefs latéraux et la scène érotique où, sur le grand bas-relief qui occupe la face antérieure de notre piédestal, nous voyons Léda recevant dans ses bras Jupiter métamorphosé en cygne. Je me hâte de soumettre à l'Académie les observations que m'ont suggérées les accessoires de cette scène. C'est pour la première fois, si je ne me trompe, que l'antiquité nous montre le groupe de Jupiter et Léda placé entre un cyprès et Cupidon, bien que ce groupe soit un de ceux que les artistes se plaisaient à reproduire sur des monuments de toute espèce. Le sculpteur, fidèle ici à la tradition hiératique, ou plutôt simple imitateur d'un modèle grec, a placé le lieu de la scène au sommet d'une montagne ; et, sans doute, il nous transporte sur le mont Taÿgète, situé entre la Messénie et la Laconie, car, dans l'hymne homérique composé en l'honneur des Dioscures, nous lisons ces paroles, remarquables sous plus d'un rapport\footnote{Je me sers ici de la traduction de Dugas Montbel.} : « Sur les sommets du Taÿgète, haute montagne, Léda, s'étant unie d'amour au formidable Jupiter,\footnote{Mot à mot : « Au formidable fils de Cronos. »} donna le jour à des fils pour être les sauveurs des faibles humains\footnote{\textgreek{Τοὺς ὑπὸ Τηϋγέτου κορυφῇ, ὄρεος μεγάλοιο,\\\hspace*{5mm}Μιχθεῖσ' ἐν φιλότητι κελαινεφέϊ Κρονίωνι,\\\hspace*{5mm}Σωτῆρας τέκε παῖδας ἐπιχθονίων ἀνθρώπων},\\\hspace*{5mm}... ...\\\hspace*{10mm}(Vers 4-6.)} ... » Mais, dans ce passage, il n'est fait, on le voit, aucune mention de la présence d'Aphrodite et d'Éros. C'est donc à une autre source qu'avaient été puisés les éléments de la composition sculptée qui est sous nos yeux. Remarquons à ce sujet que le mont Taÿgète dominait une partie de la Grèce où le culte d'Aphrodite et le culte du cyprès remontaient à une époque très-reculée. Nous avons trouvé la preuve de ce fait dans les traditions qui se rapportent, soit à la Vénus armée des Lacédémoniens, soit à la Vénus des Corinthiens et de leurs colonies, et dans le nom de \emph{Cyparissia} ou \emph{Cyparissies} que portèrent chacune une ville de Laconie, une ville de Messénie, et qui s'étendit à une rivière, à un golfe, à un promontoire et à un port de cette dernière province. D'autre part, le sculpteur romain, en plaçant sur un point culminant de la montagne où Jupiter s'unit à Léda un cyprès pyramidal\footnote{M. Delorme, qui avait pris cet arbre pour un peuplier, ne fait pas difficulté, dans la lettre citée, de reconnaître qu'il s'était trompé, et qu'au lieu d'un peuplier le bas-relief porte réellement un cyprès. Cet arbre ne peut servir ici à caractériser le mont Taÿgète, car ni Strabon, ni l'hymne homérique aux Dioscures, dans les passages cités ci-dessus, ni aucun autre écrivain ancien ou moderne, ne fait mention d'une plantation de cyprès sur cette montagne escarpée.} qui domine la scène, et le poète grec, en nous disant que de cette union fortunée naquirent les Dioscures, sauveurs des nochers, nous reportent, l'un et l'autre, aux traditions écrites et aux monuments figurés qui, dans un autre mémoire,\footnote{\emph{Mém. de l'Acad. des inscript.} nouvelle série, t. 15., 2\textsuperscript{e} partie, p. 261-305.} m'ont servi à démontrer qu'une antique théologie avait établi entre les Dioscures et Vénus, d'une part, et entre les Dioscures et Mithra, de l'autre, des rapports où les deux jumeaux célestes se montrent subordonnés à chacune de ces divinités. Ils remplissent auprès d'elles les fonctions d'assesseurs, et sont, par conséquent, associés à une déesse et à un dieu qui, l'un et l'autre, président aux mystères, et qui, à ce titre, avaient chacun mérité la qualification de \emph{sauveur}, comme l'Artémis qu'on adorait à Bœæ, en Laconie,\footnote{Pausanias, 3., 22., 9.} comme les Grandes Déesses des mystères d'Éleusis.

Si l'un des bas-reliefs de l'autel palmyrénien du Musée capitolin\footnote{Pl. 1., n° 2.} nous a donné lieu de reconnaître Vénus sous la forme d'un cyprès pyramidal qui enfante l'Amour, pouvons-nous, sur le monument de Vienne, ne pas reconnaître aussi comme une image symbolique et vivante de la même déesse le cyprès pyramidal placé auprès du groupe de Jupiter et Léda ? Cet arbre, n'en doutons pas, représente ici la divinité qui, chargée de présider à la vie, à la reproduction, aux amours, aux hyménées, comme aussi à la destinée des âmes, assiste à l'union divine de laquelle naîtront les Dioscures-Sauveurs. Le rôle que Cupidon joue dans cette scène ne semble pas permettre une autre interprétation. Le groupe des deux amants est placé entre ce jeune dieu et le cyprès, image de Vénus. Debout et armé d'un arc dont il dirige les traits enflammés\footnote{« Flammiferas pennas, » dit Silius Italicus (11., 412). « Nec Idalia lenta incaluisse sagitta, » avait-il dit précédemment (5., 19). On trouve dans Horace (\emph{Od.} 2., 8., 15) : « Ardentes sagittas. » Ces diverses expressions sont empruntées aux poètes grecs. Ceux-ci, parlant des flèches d'Éros, les qualifient \textgreek{πυρόεντα βέλη} (\emph{Antholog.} 4., 470), \textgreek{ὀϊστοὺς πυριπνεύστους} (Musée, vers. 88), etc.} vers le couple amoureux, Cupidon agit ici en présence de sa mère, et obéit à son ordre exprès en excitant ainsi le feu allumé dans le cœur des deux amants. Ne nous étonnons donc pas si les poètes de l'antiquité donnent à l'Amour un arc et des flèches de bois de cyprès, et le disent vainqueur de Jupiter lui-même.\footnote{Silius Italicus, 8., vs. 453 \emph{sqq.}} Et remarquons qu'une peinture de vase grec justifie pleinement le sculpteur romain d'avoir placé entre Vénus et Cupidon le groupe de Jupiter et Léda. Ce vase, acquis à Naples par M. Kestner, et récemment publié par M. Éd. Gerhard,\footnote{\emph{Denkmäler und Forsch.} nov. et déc. 1851, n°\textsuperscript{s} 35, 36, p. 387-392 ; pl. 36., \emph{Archäolog. Anzeig.} décemb. 1851, n° 36, p. 123.} représente Vénus assise sur un trône et portant sur la main gauche une colombe ou une \emph{jynx} ; la déesse, continuant le rôle que nous venons de lui voir remplir auprès de Jupiter et de Léda, assiste à l'entrevue amoureuse de deux personnages que mon savant ami prend, avec toute raison, ce me semble, pour Hélène et Pâris. N'oublions pas que cette princesse était née des amours mêmes de Léda avec Jupiter changé en cygne, et que Théocrite, dans un passage cité plus haut,\footnote{Ci-dessus, p. 238.} compare réellement la beauté d'Hélène à celle de Vénus elle-même, lorsqu'il compare cette princesse à un cyprès. Au revers du vase de M. Kestner, sont peints deux cygnes qui, sans doute, font allusion à la naissance d'Hélène, et qui, pour le dire en passant, nous donnent lieu de supposer qu'à une époque ancienne, le langage symbolique exigeait que l'on représentât Jupiter et Léda sous la forme de deux cygnes, l'un mâle, l'autre femelle.\footnote{On voit que cette supposition s'appuie tout à la fois sur un monument figuré et sur des considérations du genre de celles qui m'ont suggéré la conjecture exposée ci-dessus, dans la note 1 de la p. 267.}

N'oublions pas non plus que, dans la mythologie romaine, la métamorphose de Vénus en cyprès ne nous est point attestée par le seul témoignage du bas-relief de Vienne. Dans leurs compositions écrites, les mythographes latins, comme les sculpteurs, dans les œuvres de leur ciseau, faisaient parfois apparaître aussi la déesse sous la forme de cet arbre. N'étaient-ils que l'écho d'une tradition grecque qui ne nous est point parvenue, ou s'inspiraient-ils seulement de quelque monument figuré qu'ils avaient sous les yeux ? Les deux suppositions me paraissent admissibles ; et je ne saurais dire, par exemple, laquelle, de préférence, s'applique à un récit où Apulée,\footnote{\emph{Metamorph.} 5., 103, 104 ; edd. Oudendorp et Bosscha.} voulant peindre le fils de Vénus dans le giron de sa mère, nous le montre placé dans le feuillage d'un cyprès pyramidal. Dans ce récit, riche en sujets pour le poète, pour le sculpteur, pour le peintre, et composé, soit d'après une antique légende écrite, soit en présence d'une œuvre de l'art, Cupidon se réveille au moment où, sur son épaule, tombe brûlante une goutte d'huile de la lampe que tient à la main Psyché, attirée auprès de son amant par une irrésistible et coupable curiosité. Rendu à l'usage de sa raison, il reconnaît, mais trop tard, qu'il a désobéi aux ordres de Vénus. Il se sépare de Psyché, et s'envole pour retourner dans les bras de sa mère. Psyché s'élance, le saisit par une jambe, s'y suspend, et, dans cette humiliante position, traverse les airs avec lui. Bientôt la fatigue l'accable ; ses mains refusent de la tenir plus longtemps attachée à la jambe de son perfide séducteur ; elle retombe sur la terre.\footnote{« ... tandem fessa delabitur solo. »} A cet instant, Cupidon dirige son vol vers un cyprès voisin. Du sommet de cet arbre, comme du giron de sa mère, et non sans une profonde émotion, il adresse à son amante délaissée de graves paroles, qui commencent par cette phrase : « Trop crédule Psyché, j'avais oublié les ordres de ma mère\footnote{« Nec deus amator humi jacentem deserens, involavit proximam cupressum, deque ejus alto cacumine sic eam graviter commotus affatur : Ego quidem, simplicissima Psyche, parentis meæ Veneris præceptorum immemor ... »} ... » Ainsi, dans le tableau sculpté sur une des faces de l'autel palmyrénien, nous voyons l'Amour naître des flancs d'un cyprès pyramidal. Sur le monument de Vienne, en présence de cet arbre, image vivante et symbolique de la déesse qui l'a conçu, il exerce sur les cœurs l'empire qu'il a reçu de Vénus. Enfin, dans le tableau tracé par Apulée, Cupidon, coupable d'avoir séduit Psyché, c'est-à-dire l'âme, reconnaît sa faute, et vient chercher dans les bras de sa mère, représentée sous le même symbole du cyprès, le pardon qu'il attend de la divinité qui, je le répète, préside aux mystères, c'est-à-dire aux destinées de l'âme humaine. Si, jusqu'à ce jour, les commentateurs d'Apulée n'ont fait aucune remarque sur le trait le plus saillant de son récit, c'est tout simplement qu'ils n'ont pas compris que là le cyprès tient la place de Vénus elle-même.

Dans mon second mémoire, plusieurs monuments funéraires nous permettront de constater que les Étrusques et les Romains n'ignoraient point les rapports qu'une antique théologie avait établis entre Vénus, reine de la terre ou des vivants, reine des enfers ou des morts, et le cyprès pyramidal, symbole de vie et tout à la fois symbole funéraire.

Si les traditions écrites et les monuments de l'art concourent ainsi à prouver que, dès une époque reculée, le culte de cet arbre symbolique s'était introduit à Rome, comme dans l'Étrurie et la Picentine, avec le culte de Vénus, il est encore plus certain qu'à une époque beaucoup moins ancienne, le culte du même arbre, associé à celui de Mithra, fut apporté de l'Asie occidentale dans l'empire romain. Nous avons sur ce point le témoignage formel de plusieurs monuments figurés. Je citerai, en particulier, cinq bas-reliefs mithriaques que nous a légués l'antiquité romaine ; et, à ce sujet, il ne sera pas inutile de répéter que les Grecs de l'Asie Mineure reçurent des mains des Perses les types de tous les monuments consacrés à Mithra. Là, après avoir subi, dans des limites assez étroites, l'influence de l'art grec, ces types furent adoptés par les légions romaines qui, pendant et depuis la guerre de Pompée contre les redoutables pirates de Cilicie, séjournèrent au milieu de populations asiatiques, formées de Grecs et de barbares.

L'un des cinq bas-reliefs qui m'obligent à reproduire ici cette remarque est celui que l'on conserve à la villa Altiéri.\footnote{J'en ai publié un dessin exact dans mes \emph{Recherches sur Mithra}, atlas, pl. 74.} On y voit deux cyprès plantés, l'un à côté du buste du Soleil, l'autre à côté du buste de la Lune. Conformément aux prescriptions des livres sacrés des Perses, Mithra est représenté ici, comme ailleurs, ayant, à sa droite, le Soleil, à sa gauche, la Lune. De plus, il est debout sur le taureau, à la manière des divinités figurées sur les monuments asiatiques. Au-dessous du buste du Soleil et tout auprès d'un palmier mâle, un lion sort d'un rocher. Au côté opposé, un second palmier mâle est accompagné d'un coq ; plus bas un aigle, portant une foudre dans ses serres, surmonte un troisième palmier mâle. Ce curieux bas-relief nous offre donc, avec trois animaux, symboles du soleil ou du feu céleste, deux espèces différentes d'arbres symboliques, l'une monoïque,\footnote{C'est-à-dire réunissant, sur un même pied, des fleurs mâles et des fleurs femelles.} consacrée au soleil, à la lune et aux divinités primitivement androgynes, telles que Vénus et Mithra ; l'autre dioïque,\footnote{C'est-à-dire portant, sur deux pieds séparés, ses fleurs mâles et ses fleurs femelles.} consacrée à Mithra, lorsqu'elle porte des fleurs mâles, et à Vénus, lorsqu'elle est chargée de fleurs femelles et de fruits.

Le second monument mithriaque est un bas-relief à deux faces, qui fut découvert, en 1832, dans un mithræum, près de Heddernheim,\footnote{Il est déposé dans une des salles de la bibliothèque grand-ducale de Wiesbaden. La société instituée dans le grand-duché de Nassau pour les progrès de l'archéologie en a publié les dessins et la description dans le 1\textsuperscript{er} volume de ses \emph{Annales}. Ce sont ces dessins que reproduisent les pl. 90. et 91. de l'atlas de mes \emph{Rech. sur Mithra}.} et que, plusieurs fois déjà, j'ai eu l'occasion de recommander à l'attention des archéologues. Il a droit, comme le bas-relief cité de la villa Altiéri, à être compté au nombre des monuments romains qui fournissent les témoignages les plus précieux de l'origine asiatique du culte de Mithra. Sur la face antérieure, divisée en plusieurs tableaux ou compartiments, trois cyprès, de l'espèce pyramidale, sont sculptés dans chacune des deux divisions supérieures. Le tableau qui repose immédiatement sur un zodiaque semi-circulaire, représente, si je ne me trompe, les quatre phases de la vie humaine : à gauche, un enfant, coiffé du bonnet mithriaque, naît de l'intérieur d'un arbre, comme le fils de la Vénus assyrienne naît des flancs d'un cyprès, sur un des bas-reliefs de l'autel palmyrénien du Musée capitolin,\footnote{Pl. 1., ci-jointe, n° 1.} et comme naît de l'arbre \emph{Reivas}, selon les livres sacrés des Parses,\footnote{\emph{Zend-Avesta}, t. 2. (\emph{Boun-déhesch}, § 15), p. 376, 377. Voy. sur le \emph{Reivas}, Hamzah Isphâani ou d'Ispâhân, cité dans le \emph{Modjmel-el-tewarikh} (Anquetil, \emph{Zend-Avesta}, t. 2., p. 352, note 1 ; p. 353, note 2). Cf. Hyde, \emph{De relig. veter. Persar.} appendix, n° 9., p. 541 \emph{sqq.}} \emph{Meschia}, le premier homme. Dans ces livres, nous lisons, de plus, que le Reivas était né lui-même de la semence du premier être créé par Ormuzd, le taureau \emph{Aboudad}, symbole du principe de la vie. Les trois groupes qui font suite à la naissance de l'homme, je l'expliquerai plus amplement ailleurs, se rapportent aux trois autres phases de son existence terrestre. C'est pendant la durée de cette existence que le sectateur de Mithra doit s'efforcer d'acquérir les trois degrés de pureté : \emph{la pureté d'action, la pureté de parole} et \emph{la pureté de pensée}. Ces trois degrés constituent l'\emph{état lumineux} ; ils sont la condition indispensable de l'admission des âmes ou des féroüers dans le séjour céleste appelé le \emph{Gorotman} ou l'\emph{Albordj}. Dans un des trois groupes qui font allusion à cette purification successive de l'âme, nous voyons Mithra, le dieu psychopompe du \emph{Zend-Avesta}, poser la couronne mystique\footnote{Tertullien, \emph{De corona}, 15., p. 3, C.} sur la tête de l'initié, au pied de trois cyprès, comme le grand prêtre de Mylitta ou Anahid dut autrefois placer sur la tête de Sôs, au pied des cyprès sacrés d'Armavir,\footnote{Ci-dessus, p. 65-73.} la couronne royale d'Arménie. Plus loin, le myste, nu, comme la vérité, mais la tête entourée d'une auréole qui nous atteste qu'il est parvenu à l'état lumineux ; le myste se jette aux genoux de Mithra et implore sa médiation pour obtenir d'Ormuzd l'entrée du séjour éternel des bienheureux. Il est écouté ; il a mérité de l'être ; et, dans le tableau supérieur, qui représente la région céleste, nous le voyons prendre place, auprès du dieu \emph{médiateur},\footnote{C'est ainsi qu'est qualifié Mithra dans le \emph{Zend-Avesta}, t. 2. (\emph{Iescht} de Mithra, 10\textsuperscript{e}, 12\textsuperscript{e}, 20\textsuperscript{e} et 21\textsuperscript{e} cardé), p. 212, 213, 218, 219.} sur le char solaire qui s'élance vers un des deux sommets de l'Albordj, tandis que de l'autre sommet\footnote{Il n'est pas sans intérêt de remarquer que la haute montagne du Caucase qui conserve son ancien nom d'\emph{Albordj}, sous la forme moderne \emph{Elbrouz}, présente également un double sommet.} descend sur la terre un second char. La Lune personnifiée y est couchée, enveloppée dans le voile de la nuit ou des ténèbres. Un cyprès pyramidal est planté entre les deux points culminants de la montagne céleste, de cet Olympe des Perses, où Mithra, selon le \emph{Zend-Avesta},\footnote{Tom. 1., 2\textsuperscript{e} partie, p. 28, 418 ; t. 2., p. 13, 99, 209, 212, 213, 228.} réside habituellement, ayant, comme ici, à sa droite le soleil, à sa gauche la lune. Un second cyprès s'élève derrière le char du Dieu-Soleil, et un troisième à côté du char de la Lune. Ces trois arbres, on le voit, deviennent ici l'image symbolique d'une triade céleste, composée de Mithra, du Soleil et de la Lune.\footnote{Sur le grand bas-relief mithriaque du musée du Louvre (\emph{Rech. sur Mithra}, atlas, pl. 75.), cette triade est représentée par trois pins au lieu de trois cyprès.} Ils nous ramènent aux trois cyprès du trône votif de la Déesse de Syrie,\footnote{Ci-dessus, p. 51-58 ; et pl. 5., n° 1.} et aux cyprès du soleil et de la lune qui, sur les médailles impériales de Perga\footnote{Ci-dessus, p. 55, 56 ; et pl. 14, n°\textsuperscript{s} 1 et 2.} et sur les deux stèles consacrées à Hâthôr,\footnote{Ci-dessus, p. 167-183 ; et pl. 10., n° 2 ; et pl. 11. \emph{Recherch. sur Vénus}, pl. 14 F.} forment aussi une triade, soit avec un cône, symbole d'une divinité assimilée à Vénus, soit avec une fleur de lotus, emblème d'un dieu ou d'une déesse primitivement androgynes. En même temps, nous trouvons dans les trois cyprès de la région céleste, sur le bas-relief de Heddernheim, une allusion au triple caractère que reçoit Mithra dans le système religieux des Perses,\footnote{Roi du ciel mobile (\emph{Zend-Avesta}, t. 1., 2\textsuperscript{e} partie, p. 28, 82, note 10 ; t. 2., p. 13, 99, 206, 207, 209, 212, 213, 216, 218-221, 225, 228-230, 418) ; roi de la terre ou des vivants (\emph{ibid.} t. 2., p. 205, 206, 210, 214, 215, 222, 223), et roi des morts ou des enfers (\emph{ibid.} t. 2., p. 15, 211, 212, 223, 230).} comme Mylitta, dans la théologie assyrienne, et une allusion aussi à l'épithète de \textgreek{τριπλάσιος} donnée par les Grecs au même dieu.\footnote{Voy. \emph{Mém. de l'Acad. des inscriptions}, nouv. série, t. 14., 2\textsuperscript{e} partie, p. 174, 175.} Dès lors, ne suis-je pas autorisé à considérer comme des symboles de vie les trois cyprès qui, dans la région située au-dessous de la région céleste, font partie des trois groupes que je rapporte aux trois grandes phases de la vie humaine que précède la naissance ?

Un troisième et un quatrième bas-reliefs mithriaques, d'époque romaine, nous offrent de nouveaux exemples de l'emploi symbolique du cyprès ; et ces exemples méritent d'autant plus notre attention, qu'ici cet arbre est répété sept fois sur chacun de nos deux monuments, comme sur les médailles impériales citées d'Apollonia d'Illyrie,\footnote{Ci-dessus, p. 226-231 ; et pl. 8., n° 4.} comme sur une urne cinéraire étrusque\footnote{Pl. 13., n° 5.} dont il sera question plus loin. L'un de ces deux bas-reliefs provient d'une découverte heureuse, faite, en 1838, à Neuenheim, près de Heidelberg. C'est à mon savant confrère et ami M. le Dr. Frédéric Creuzer, que les archéologues doivent la connaissance de ce monument et la belle dissertation dont il a été le sujet.\footnote{\emph{Das Mithraeum von Neuenheim bei Heidelberg.} Heidelb. 1838, 94 pag. avec 2 pl. lithogr. La planche 92. de l'atlas de mes \emph{Recherches sur Mithra} reproduit le dessin que donne ici (pl. 1.) M. Creuzer du bas-relief de Neuenheim.} Toutefois j'exprimerai ici à l'illustre auteur de la \emph{Symbolique} le regret de n'avoir pas trouvé dans cette dissertation les observations que nous devait peut-être, sur le symbole du cyprès, son érudition si riche et si variée. Elles m'eussent été d'un grand secours pour mon travail ; et j'en suis privé probablement parce que M. Creuzer, s'arrêtant à la supposition que, chez les Perses, le \emph{hôm} et le cyprès étaient identiques,\footnote{Dissertation citée, p. 44 ; 85, note 76.} aura, par-là, été détourné de l'idée d'entreprendre des recherches de la nature de celles que je soumets au jugement de l'Académie. Je regrette aussi qu'il ne nous ait pas fait connaître les raisons qui le portent à identifier le cyprès avec le hôm.\footnote{M. de Hammer (\emph{Mithriaca}, p. 42, 46) pense que le cyprès répond au \emph{barsom} du \emph{Zend-Avesta} ; mais il oublie que le \emph{barsom} n'est ni un arbre, ni une branche d'arbre : c'est un faisceau de rameaux cueillis sur des arbrisseaux dont la liturgie persique ne détermine ni ne limite l'espèce ; quelquefois même ces rameaux sont remplacés par un faisceau de tiges de cuivre ou de laiton. (Voy. Anquetil, \emph{Zend-Avesta}, t. 1., 2\textsuperscript{e} partie, p. 90, note 3 ; t. 2., p. 532.)} On ne découvre, il faut bien le dire, rien qui, dans les textes ou sur les monuments figurés, puisse justifier son opinion. D'une part, les observations ou les renseignements recueillis sur les lieux, par Anquetil du Perron, nous montrent que, chez les Parses de l'Inde, le hôm est un arbrisseau et non un arbre, et qu'il est fort difficile d'en déterminer l'espèce. D'autre part, le hôm, tel qu'on le voit figuré, soit sur les bas-reliefs assyriens découverts à Khorsabad et à Nimroud, soit sur les cylindres et les cônes, de travail babylonien, assyrien, phénicien ou persépolitain, est également un arbrisseau, quelquefois même une simple plante fleurie, et toujours un végétal dont les formes conventionnelles varient beaucoup et ne sauraient être rapportées avec certitude à une espèce déterminée.

Ainsi que je l'ai annoncé, on compte sept cyprès sur le bas-relief mithriaque de Neuenheim. Ils sont plantés, à côté l'un de l'autre, sur le sommet de la montagne céleste, l'Albordj ou le Gorotman, qui s'élève ici derrière la tête de Mithra et entre le buste du Soleil et celui de la Lune. Cette disposition, par rapport aux sept cyprès, mérite d'être rapprochée de celle que l'on observe sur quelques cylindres consacrés aux mystères de Mylitta ou aux mystères de Mithra. Ces cylindres nous offrent l'emblème de la triade divine, ou le \emph{mihr}, symbole de Mylitta et de Mithra, placé auprès de l'astérisque du soleil, du croissant de la lune et de cinq ou de sept petits globes qui, de même que les sept cyprès, sur notre bas-relief et sur les autres monuments cités, représentent les planètes.\footnote{On en peut voir des exemples dans l'atlas de mes \emph{Rech. sur Mithra}, pl. 16., n° 1 ; pl. 29., n° 5 ; pl. 30., n° 7 ; pl. 54 A, n° 5. Cf. pl. 29., n° 6 ; pl. 35., n° 9 ; pl. 54 A, n° 7.} L'encadrement du bas-relief de Neuenheim, comme celui de la face antérieure du bas-relief de Heddernheim, est divisé en plusieurs compartiments où l'on retrouve un enfant qui naît de l'arbre Reivas, et trois groupes qui, si je ne me trompe, représentent les trois autres phases de la vie humaine.

Le quatrième bas-relief, comme le cinquième, fut découvert, il y a quelques années, dans la Transylvanie, cette belle province de l'empire autrichien qui déjà nous a fourni plusieurs monuments importants pour l'histoire du culte de Mithra chez les Romains. Il provient des ruines d'\emph{Apulum} ; il est de marbre blanc, et se conserve à Carlsbourg, dans une des salles de l'Institut battyanien. Je dois à l'obligeance de feu Mme. la baronne Jósika et de feu l'abbé Henne le dessin que j'en ai publié dans mes \emph{Recherches sur Mithra}.\footnote{Atlas, pl. 79., n° 1. --- Le dessin gravé sous ce numéro est, m'assure-t-on, plus exact que celui dont un célèbre orientaliste de Vienne a enrichi le recueil de planches qui accompagne ses \emph{Mithriaca} (pl. 11.).} Ce bas-relief présente plusieurs particularités dont il n'est pas nécessaire que j'entretienne ici l'Académie. J'ai seulement à faire remarquer qu'au-dessus de la grotte du dieu des Perses, c'est-à-dire au-dessus de la région terrestre, et, par conséquent, dans la région céleste, on a sculpté, entre le buste du Soleil et celui de la Lune, sept groupes composés chacun d'un poignard, d'un autel allumé, d'un bonnet phrygien et d'un cyprès. Si les sept arbres qu'avec l'abbé Henne,\footnote{Notes manuscrites.} M. de Köppen,\footnote{\emph{Wiener Jahrbücher der Literatur}, t. 24., Anzeige-Blatt, n° 24., p. 9 et 10. L'auteur décrit (\emph{ibid.} p. 5 et 6) un autre bas-relief mithriaque, fort endommagé, où il croit apercevoir des traces d'autels et de cyprès. Je ne connais pas ce monument.} M. le baron de Hammer\footnote{\emph{Mithriaca}, p. 46, 93, 182. --- J'aurai ailleurs l'occasion de montrer que le savant orientaliste de Vienne n'est pas fondé à reconnaître (\emph{ibid.} p. 42, et 126, § 5) le palmier et le cyprès dans le \emph{hôm} et le \emph{barsom} du \emph{Zend-Avesta}.} et M. le professeur N. Müller,\footnote{\emph{Annal. des Vereins für nassauisch. Alterthumsk. und Geschichtsf.} t. 2., 1\textsuperscript{re} partie, p. 19, 20.} je prends pour des cyprès, affectent une forme beaucoup moins élancée que ne l'est réellement la forme du cyprès, il faut remarquer que les figures qui ornent ce bas-relief sont elles-mêmes très-courtes et très-ramassées, comme il arrive maintes fois sur les monuments d'une époque de décadence. Ici, de même que sur plusieurs autres bas-reliefs mithriaques bien connus,\footnote{J'en puis citer quatre : un, de terre cuite, qui se conservait autrefois à Rome, dans le palais Zéno (Lafréry, \emph{Specul. roman. magnif. Mém. de l'Acad. des inscript.} nouv. série, t. 14., 2\textsuperscript{e} partie, pl. 5. \emph{Rech. sur Mithra}, atlas, pl. 89.) ; deux bas-reliefs de grès, découverts en Transylvanie (\emph{Mém. de l'Académie des inscript. ubi supra}, pl. 1., n°\textsuperscript{s} 1 et 2. \emph{Recherches sur Mithra}, atlas, pl. 96., n°\textsuperscript{s} 1 et 2) ; et enfin un bas-relief de marbre blanc, du musée Chiaramonti (\emph{Recherches sur Mithra}, atlas, pl. 79., n° 2). --- La présence de sept autels ou pyrées sur ces quatre bas-reliefs et sur celui de l'Institut battyanien (\emph{ibid.} n° 1) nous rappelle les sept pyrées que, selon Strabon (\emph{Geogr.} 15., p. 15), on voyait, en Cappadoce, dans le temple d'Anaïs et Omanès. Ces divers témoignages suffisent, sans doute, pour confirmer un passage du \emph{Ferhenghi-Djihanguiri}, dont on a contesté la véracité, et où il est dit qu'anciennement sept \emph{atesch-gâh} ou sept pyrées avaient été établis en Perse. M. le lieutenant-colonel Rawlinson, qui probablement n'avait pas consulté les cinq monuments mithriaques que je viens d'indiquer, a lui-même rejeté ce passage en parlant du pyrée de l'Aderbaïdjan, dans un mémoire sur la position d'Ecbatane, inséré, en 1840, dans le 10\textsuperscript{e} vol. du Journal de la société royale de géographie de Londres (1\textsuperscript{re} partie, p. 114 et suiv.).} les sept autels allumés représentent les sept feux du ciel mobile ou les sept planètes ; les sept cyprès, remplacés par sept lauriers sur un bas-relief du musée Chiaramonti,\footnote{\emph{Rech. sur Mithra}, atlas, pl. 79., n° 2.} sont les images symboliques des sept intelligences divines, appelées \emph{Amschaspands} ou \emph{Immortels},\footnote{Remarquons que les sœurs de Phaëthon, filles du Soleil, appelées tantôt \emph{Phaëthontides} ou \emph{Phaëthontiades}, tantôt \emph{Héliades}, et comptées au nombre de trois, de cinq ou même de sept, furent changées en arbres. Leurs larmes, disent les mythographes, sont du succin ou de l'ambre jaune. Par-là, nous sommes amenés à considérer comme altérées les traditions qui métamorphosent en aune ou en peuplier noir les sœurs de Phaëthon. Il est infiniment probable que, dans la légende primitive, les Phaëthontides étaient métamorphosées, comme les filles d'Étéocle (ci-dessus, p. 234, 235), en arbres verts et résineux, tels que des cyprès, des pins ou des lauriers.} qui président aux sept planètes et aux sept cieux des planètes. C'est par les sept cieux et les sept planètes que doivent faire leur ascension les âmes tombées du ciel sur la terre et rendues à la liberté ou à la vie spirituelle après l'accomplissement des sept sacrifices que rappellent les sept poignards, et après la mort du corps. Peut-être est-il permis d'inférer de là que le nombre vingt-huit, qui résulte des quatre objets répétés ici sept fois chacun, fait allusion aux vingt-huit stations lunaires par lesquelles doivent également passer les âmes, et aux vingt-huit Izeds ou Génies qui président à chacune de ces stations. Rappelons-nous que Mithra, dont la place est marquée, dans le ciel mobile, entre la porte du soleil et la porte de la lune, est un des trois dieux dont se compose la triade suprême qui régit les sept Amschaspands et les vingt-huit Izeds. N'oublions pas surtout qu'il préside aux destinées humaines et, par conséquent, à chacune des phases de la descente et de l'ascension des âmes.

Le cinquième bas-relief mithriaque dont il me reste à parler est inédit, m'assure-t-on, et a été découvert, il y a peu d'années, dans des ruines romaines situées aux environs de Carlsbourg. La planche 15 jointe ici le reproduit d'après un dessin dont je suis également redevable à feu Mme. la baronne Jósika. L'original est fruste en plusieurs endroits, et partout d'un travail passablement barbare. Mais il rachète ces défauts par quelques particularités qui ne se sont pas encore rencontrées dans les monuments de cette catégorie. Au-dessus de la tête de Mithra immolant le taureau est sculpté un mufle de lion, vu de face et inscrit dans un disque,\footnote{Peut-être, au lieu d'un disque solaire ou lunaire, faut-il voir ici une ouverture circulaire pratiquée dans un rocher dont sortirait le lion. Cette interprétation que la grossièreté du travail rend aussi douteuse que la première, nous ramènerait, d'un côté, à la célèbre tradition qui fait naître Mithra d'un rocher (\emph{ex petra natus}) ; et, de l'autre, au bas-relief de la villa Altiéri (\emph{Recherch. sur Mithra}, atl. pl. 74.) et à un camée du Musée impérial de Vienne (\emph{ibid.} pl. 101., n° 1), deux monuments où nous voyons le lion mithriaque, symbole igné, sortir d'un rocher, comme l'étincelle sort de la pierre.} emblème très-propre à caractériser le dieu des Perses, divinité solaire, assez souvent représentée avec une tête de lion sur les épaules, au lieu d'une tête humaine. Cet emblème nous ramène à la plaque de bronze du Musée royal de Berlin,\footnote{Pl. 7., n° 6.} qui nous a offert une scène d'initiation dominée par un mufle de lion placé entre deux serpents qui s'élancent du centre de deux cyprès plantés, l'un au-dessous du buste du Soleil, l'autre au-dessous du buste de la Lune. Ici deux cyprès correspondent de même à deux bustes qui représentent ces astres personnifiés ; mais de leur sein ne naissent pas deux serpents ; et, ce qui ne se voit point sur les quatre autres bas-reliefs mithriaques, ces deux arbres sont plantés chacun derrière un des assesseurs de Mithra. On sait que ces deux assesseurs représentent l'équinoxe du printemps et l'équinoxe d'automne, époques de l'année où, sur la terre, se manifestent successivement les deux phénomènes qui caractérisent la vie et la reproduction, la germination ou la feuillaison et la fructification. Le cyprès, comme symbole de vie, est donc d'autant plus convenablement placé ici que, selon Caton, je le répète, la graine de cet arbre se sème au printemps. De plus, nous constatons que si, en Égypte, comme nous l'avons vu,\footnote{Ci-dessus, p. 183-190 ; et pl. 10., n°\textsuperscript{s} 3, 4 et 5.} le même arbre, sur les monuments religieux, sert à caractériser le mois de \emph{tôbi}, dans lequel tombe l'équinoxe vernal, il faut en conclure que les Égyptiens et les Perses avaient puisé à une même source leurs emblèmes astronomiques. Or cette source ne peut être cherchée ailleurs que chez les Chaldéens d'Assyrie, inventeurs des mystères institués avec le culte public d'une divinité qui fut successivement appelée Mithra\footnote{J'espère en donner la preuve dans mon ouvrage sur le culte public et les mystères de cette divinité.} dans leurs sanctuaires, Mylitta et Anahid chez les Babyloniens et les Assyriens, Brathy ou Bérouth (cyprès), Astarté, Atergatis et Dercéto chez les Phéniciens, Athtor ou Othtor chez les Arabes, Hâthôr ou Hâthyr chez les Égyptiens, Mithra chez les Perses.

On le voit donc, un accord parfait, quant à l'emploi symbolique du cyprès, se manifeste sur les monuments religieux de ces diverses nations et sur ceux des Grecs, des Étrusques et des Romains. De cet accord et de l'origine asiatique de cet arbre sacré résulte la preuve irrécusable qu'à des époques très-reculées il a existé entre les peuples de l'Asie, de l'Afrique et de l'Europe, des communications dont le foyer ou le point de départ est l'Asie antérieure, patrie primitive du cyprès pyramidal. Ce sont précisément ces communications sur lesquelles, dès mes premiers mémoires d'archéologie, je n'ai cessé d'appeler l'attention de l'Académie et celle des savants de tous les pays. La vie d'un homme est si courte, je le sens chaque jour, qu'il ne faut pas moins que le concours de tous les archéologues pour explorer avec succès un aussi vaste champ d'études. Mais, à chaque instant, pour ainsi dire, je m'applaudis d'avoir ouvert cette nouvelle voie aux investigations des amis de la science.

Chez les Romains, la série nombreuse des bas-reliefs consacrés à Mithra finit à peu près avec le 4\textsuperscript{e} siècle de notre ère. Toutefois, dans le siècle suivant, le symbole du cyprès continue à se montrer sur d'autres monuments de l'art. Les médailles contorniates en fournissent la preuve. Plus tard, nous retrouverons ce symbole sur les monuments funéraires chrétiens, et nous le verrons se perpétuer, d'âge en âge, jusqu'à nous.

Les médailles contorniates, selon l'opinion de Morel et de Mahudel, généralement adoptée aujourd'hui, embrasseraient une période qui ne commence qu'avec le règne de Constantin et qui s'arrête à celui de Valentinien 3. Ce serait donc à cet intervalle de temps qu'il faudrait rapporter les pièces qui, dans la série des contorniates, nous offrent des exemples de l'emploi symbolique du cyprès. Je produis ici deux de ces pièces\footnote{Pl. 18., n°\textsuperscript{s} 3 et 4. --- Elles sont de bronze, comme toutes les médailles contorniates.} ; elles appartiennent l'une et l'autre au riche médaillier de la Bibliothèque impériale. La première est à l'effigie de Vespasien ; et, bien que je n'aie pas la prétention de lui assigner pour date le règne même de ce prince, je ne puis me défendre de penser que le style de cette médaille permet de la faire remonter à une époque antérieure au règne de Constantin. Elle a pour type, au revers, un héros ou guerrier armé de sa lance et combattant à pied, secondé par son chien, un énorme sanglier, qui semble sortir d'un antre. Derrière ce groupe, s'élève un cyprès pyramidal. Si le héros est Méléagre, le cyprès ferait allusion à la mort de ce personnage, puisque sa victoire sur le sanglier de Calydon fut suivie de scènes violentes, qui, selon les poètes ou les mythographes, causèrent la fin tragique du vainqueur, en excitant au plus haut degré un désir de vengeance dans le cœur d'Althée, sa propre mère. Cette hypothèse admise, le cyprès pourrait être considéré ici comme l'emblème de l'immortalité promise au héros grec pour prix de sa lutte victorieuse avec le sanglier, animal qui, là comme ailleurs, remplit sans doute un rôle symbolique. Dans le mémoire suivant, je reviendrai sur cette médaille contorniate à l'occasion de sujets analogues représentés en relief au fond de deux plats funéraires d'argent, que nous a pareillement légués l'antiquité romaine. Je ne dois pas omettre d'ajouter ici que le type gravé au revers de la tête de Vespasien se retrouve sur deux autres contorniates de la Bibliothèque impériale, l'une à l'effigie de Néron, l'autre offrant la tête d'Homère, entourée de la légende \textgreek{ΟΜΗΡΟϹ}. Mais quel rapport à chacune de ces trois têtes avec un pareil type ? J'avoue bien humblement mon insuffisance à résoudre le problème. L'opinion commune est que les médailles contorniates n'appartiennent point au temps où vécurent les personnages dont elles portent l'effigie, et qu'elles ont pour type, à leur revers, des sujets qui n'offrent aucun rapport avec ces personnages. Par là on est amené à les considérer comme des jetons frappés pour des particuliers, et non comme des médailles ou des monnaies frappées par ordre de l'autorité publique. J'adopte sans réserve cette manière de voir ; et j'en reconnais de nouveau la justesse lorsque j'examine attentivement la seconde pièce dont j'ai à parler.

Bien que celle-ci ne paraisse point appartenir à la même série ni au même âge que les trois autres contorniates citées plus haut, il est impossible de ne pas reconnaître que, inférieure à celles-là sous tous les rapports, elle doit être classée parmi les jetons qui ne sont susceptibles d'aucune interprétation satisfaisante. Elle représente,\footnote{Pl. 18., n° 4.} d'un côté, la tête d'un personnage inconnu, vu de profil, tourné à droite et accompagné de trois signes dont la valeur m'est également inconnue. Au revers s'élève un arbre grossièrement dessiné, mais qui me semble être de l'espèce du cyprès. À gauche de cet arbre, est gravée une légende composée de sept lettres dont il ne m'est pas donné de comprendre le sens.

Le cyprès pyramidal, chez les Romains, n'était pas seulement consacré par la religion. On en tirait des médicaments. Les passages de Pline qui sont cités plus haut\footnote{Ci-dessus, p. 248-250.} nous le prouvent. Ce que j'ai dit, en même temps, des divers usages auxquels les médecins grecs faisaient servir le cyprès, et des rapports qui existaient, à cet égard, entre la thérapeutique et les idées religieuses qu'on attachait au symbole du cyprès, s'applique à l'art médical des Romains tout aussi bien qu'à l'art médical des Grecs, puisque l'un dérivait de l'autre.

Les belles proportions qu'atteint le cyprès dans les pays chauds, sa forme pyramidale, et le précieux avantage qu'il a de ne jamais perdre ses feuilles, l'avaient aussi fait rechercher pour l'ornement des parcs et des jardins.\footnote{C'est peut-être parce que les Romains, à l'exemple des Grecs et des peuples de l'Asie occidentale, prisaient infiniment la beauté du cyprès, qu'Horace (\emph{Ars poetic.} 19-21) emprunte à cet arbre l'opposition qui se fait remarquer dans les vers où il dit :\\\hspace*{15mm}... ... et fortasse cupressum\\\hspace*{15mm}Scis simulare : quid hoc, si fractis enatat exspes\\\hspace*{15mm}Navibus ære dato qui pingitur ? ...\\\hspace*{5mm}Toutefois les commentateurs ne sont pas d'accord entre eux sur le sens qu'il convient d'attacher à ces vers.} Plusieurs passages de Caton, de Pline et de divers autres écrivains de l'antiquité\footnote{Tous ces passages, je l'ai dit plus haut (p. 250, 251), ont été recueillis ou cités par les savants auteurs des Mémoires de l'Académie d'Herculanum, et par MM. Avellino et Carl Ritter. Je me bornerai à rapporter ici quelques détails curieux qui se trouvent dans Pline (\emph{H. N.} 16., 60., 1). Cet auteur, après avoir dit qu'on fait des haies épaisses en coupant les cyprès et en les tondant, ajoute : « On fait aussi représenter au cyprès diverses figures, comme des chasses, des flottes, et d'autres tableaux auxquels se prêtent ses feuilles courtes, étroites et toujours vertes. » Un usage semblable a longtemps existé dans le midi de la France et ailleurs.} nous l'attestent, et nous apprennent, en même temps, combien était répandue et soignée, dans toute l'Italie, la culture de cet arbre exotique.\footnote{Cette culture était même un objet très-lucratif de spéculation. Jadis, en Italie, une forêt de cyprès assurait au possesseur un revenu considérable. Aussi, dit Pline (\emph{H. N.} 16., 60., 2), les anciens appelaient-ils ces sortes de plantations « la dot de leurs filles. » Les voyageurs modernes rapportent que, dans l'île de Crète, le cyprès reçoit une dénomination qui exprime la même idée.} A ces témoignages écrits, on peut joindre celui qui résulte d'une fresque découverte à Herculanum.\footnote{\emph{Le pitture antich. d'Ercolano}, t. 5., p. 393 ; pl. 39., gravée en vignette à la p. 173.} Elle nous montre sept cyprès pyramidaux, rangés autour d'un bâtiment somptueux, orné de statues. Ce n'est, à ce qu'il paraît, ni un édifice religieux, ni un monument funéraire : il répond mieux à l'idée qu'exprime, chez les modernes, la dénomination de \emph{villa}.\footnote{Les sept cyprès dont est ornée cette \emph{villa}, rappellent les cyprès qui, au même nombre de sept, décorent le temple de Vénus à Apollonia d'Illyrie, le bas-relief du mithræum de Neuenheim, et une urne cinéraire dont il sera question plus loin. Cette coïncidence m'avait d'abord donné lieu de penser que l'édifice représenté dans la fresque citée d'Herculanum avait peut-être une destination religieuse. Un examen plus approfondi et la lecture des observations judicieuses des éditeurs de cette fresque m'ont conduit à adopter l'opinion exposée dans leur texte explicatif.} Mais l'architecture en est d'un goût fort médiocre.

\subsubsection{Ibérie et Lusitanie}
\paragraph{}
L'importation du cyprès pyramidal en Afrique, en Grèce, en Italie, par les Phéniciens, et l'établissement de nombreuses colonies phéniciennes sur les côtes de la péninsule hispanique sont deux faits trop généralement connus pour qu'on puisse se défendre de quelque étonnement à la lecture des ouvrages qui traitent du culte de cet arbre symbolique : on n'y trouve aucune recherche sur les traces de ce culte dans l'Ibérie et la Lusitanie. Cet étonnement redouble peut-être, je le dis à regret, lorsque la même lacune se découvre dans le volume où M. le Dr. Movers\footnote{\emph{Die Phönizier}, 2\textsuperscript{e} partie, t. 2., ch. 12., p. 588-659.} a recueilli avec tant de soin et d'érudition les témoignages historiques, numismatiques et géographiques qui se rapportent aux colonies fondées par les Phéniciens dans ces deux contrées, aux divinités dont ils y introduisirent le culte, et nommément à leur Vénus-Astarté.

C'est en vain que j'avais moi-même entrepris des recherches pour découvrir dans la péninsule quelque œuvre de sculpture antique, qui pût fournir une preuve certaine du culte qu'on y rendait au cyprès pyramidal. Mes investigations, mes correspondances n'avaient eu qu'un résultat négatif, lorsque M. le duc de Luynes, qui poursuit avec tant de persévérance et de succès l'étude des médailles frappées dans les pays jadis occupés par les Phéniciens, appela mon attention sur quelques pièces ibériennes ou lusitaniennes publiées par Florez et par Sestini. J'ai tout lieu de m'applaudir de n'avoir pas négligé ce bienveillant avis, et de m'être par-là trouvé conduit à consulter la collection de médailles d'Ibérie et de Lusitanie que possède la Bibliothèque impériale. Les recueils de Florez et de Sestini\footnote{J'avais négligé de consulter ces deux recueils, parce que le grand ouvrage d'Eckhel et le recueil de Mionnet n'indiquent la présence du cyprès sur aucune médaille de la péninsule hispanique.} m'ont offert, dans la numismatique du royaume de Valence,\footnote{\emph{Hispania tarraconensis.}} la description et les dessins d'une monnaie autonome de la ville d'\emph{Ostur} ou \emph{Otour},\footnote{\emph{Medallas de las colonias, municipios y pueblos antiguos de España} (Madrid, 1758, in-4°), 3\textsuperscript{e} partie, p. 113 ; pl. 65., n° 6.} qui, de l'aveu de ces deux numismates, porte, au revers d'un sanglier, un grand cyprès pyramidal planté entre deux rameaux, dont ils ne déterminent pas l'espèce,\footnote{\emph{Un cipres entre dos ramos}, dit Florez (\emph{loc. cit.}). Il a soin d'ajouter qu'il publie cette médaille d'après Jacques de Bary, et que le texte de cet auteur indique ici un épi de blé et non un cyprès. Le dessin joint à la description faite par Bary, observe-t-il, représente un cyprès et non un épi.} et que je soupçonne être des branches de myrte. Dans ces mêmes recueils, mon attention a aussi été attirée par quelques autres pièces, frappées, les unes à \emph{Olunt} ou \emph{Olunk}, ville dont la position géographique n'est pas encore déterminée, les autres à \emph{Anticaria},\footnote{Aujourd'hui \emph{Antequera}. Sur les médailles, le nom ancien est écrit \emph{Antikaria}.} dans la Bétique, et à \emph{Balsa} en Lusitanie. Les premières\footnote{Florez, ouvrage cité, p. 510, pl. 35., n° 2. --- Mionnet n'indique aucune médaille de cette ville d'Olunt qui ait pour revers le type dont il s'agit.} ont pour type, au revers d'une tête virile imberbe et inconnue, un arbre de forme conique ou pyramidale, que je prends pour un cyprès, et qui est désigné par Florez sous la dénomination de pomme de pin. Bien que je n'aie pas eu sous les yeux les médailles originales, je n'hésite pas à les ranger dans une même catégorie avec les médailles d'Ostur que j'ai pu examiner à la Bibliothèque impériale. Sur celles-ci,\footnote{Pl. 20., n°\textsuperscript{s} 3 et 4.} le cyprès, gravé très en relief, est facilement reconnaissable ; on a peine à comprendre qu'Eckhel\footnote{\emph{D. N.} t. 1., p. 54.} et surtout Mionnet,\footnote{\emph{Descript. de méd.} t. 1., p. 47, 48, n°\textsuperscript{s} 346 et 347.} qui avait eu sous les yeux plusieurs de ces pièces, aient pu confondre cet arbre avec un gland de chêne. Parmi ces médailles, les unes sont conformes à la pièce décrite et figurée dans Florez, d'après Jacques de Bary\footnote{Voy. ci-dessus, p. 288, note 3.} ; les autres, d'un plus petit module, portent, au droit, un grand cyprès pyramidal, sans aucun accessoire, mais avec la légende OSTVR, et au revers, sans accessoire aussi et sans légende, deux rameaux semblables à ceux qui, sur les premières, flanquent un cyprès pyramidal.

Les médailles d'Anticaria et celles de Balsa ont, pour revers, le même arbre que, sur les monnaies d'Olunt, comme ici, Florez et Sestini\footnote{\emph{Descriz. delle medagl. ispane apparten. alla Lusitan. alla Betic. e alla Tarragon. che si conserv. nel mus. Hedervar} (Florence, 1818, in-4°, fig.), p. 23.} prennent pour une pomme de pin et qui me paraît être un cyprès pyramidal. Sa forme large et ramassée nous rappelle les cyprès sculptés sur un des bas-reliefs mithriaques romains du musée de Carlsbourg que j'ai cités plus haut.\footnote{Ci-dessus p. 278-281 ; \emph{Rech. sur Mithra}, atlas, pl. 79., n° 1.} Sur les pièces qui appartiennent à la numismatique d'Anticaria, on voit, au revers d'une tête virile, imberbe et inconnue, le croissant de la lune et un astre gravés dans le haut de la médaille, à droite de l'arbre sacré. Autour de cet arbre, une légende en lettres liées nous offre le nom abrégé de la ville et le nom également abrégé d'un magistrat, qui s'appelait \emph{Marcus Sempronius}.\footnote{Sestini, ouvrage cité, p. 23, n°\textsuperscript{s} 1 et 2 ; pl. 1., n° 17. --- Le numismate italien décrit sous ces numéros deux pièces à peu près semblables, sur lesquelles on avait cru lire \emph{municipium sempiternum}, au lieu de \emph{Marcus Sempronius}, qu'il a raison de donner pour la véritable leçon. Il ajoute qu'à cette même ville d'Anticaria, on peut rapporter une médaille décrite et figurée, parmi les incertaines, dans le recueil de Pellerin (3., pl. 118, n° 16), et rangée également parmi les incertaines par Mionnet (\emph{Descript. de méd.} 6., 641, n° 196). --- Cette pièce ne porte pas de légende. Elle a pour type, au revers, selon Pellerin, une pomme de pin, qui probablement est un cyprès. L'original ne se trouve pas à la Bibliothèque impériale.} Les médailles de Balsa\footnote{Sestini, ouvrage cité, p. 3, pl. 1., n° 2. --- Mionnet n'indique pas cette médaille ; mais, dans son \emph{Supplément} (t. 1., p. 3, n° 9), il en décrit une en ces termes, d'après un dessin qui lui avait été communiqué, et celle-ci me paraît très-analogue, sinon semblable à la pièce publiée par Sestini : « BALSA. Épi couché. ℞. Pomme de pin sur un croissant. Cabinet inconnu. BR. 4. »} nous offrent, au droit, un épi de blé avec la légende BALSA, et, au revers, un cyprès planté dans le centre d'un grand croissant. Les dessins mêmes que l'on trouve d'une monnaie d'Olunt dans Florez, d'une pièce d'Anticaria et d'une pièce de Balsa dans Sestini, me justifient de prendre pour un arbre de cette espèce ce qu'ils désignent, ainsi que Mionnet,\footnote{Voy. la note 5 de la page précédente.} sous le nom de pomme de pin ; car leurs prétendues pommes de pin auraient toutes un long pédoncule, tandis qu'en réalité ce fruit est presque sessile, comme le fruit du cyprès. De plus, on peut voir, au revers d'une jolie médaille autonome d'Irippo,\footnote{Florez, ouvrage cité, p. 474 ; pl. 30., n° 5. \emph{Catalog. mus. Hederv.} 1., 3, n° 50 ; pl. 1., fig. 4. Sestini, ouvrage cité, p. 61, n° 1 ; cf. \emph{ib.} n°\textsuperscript{s} 2-5. Mionnet, \emph{Descr. de méd.} 1., 50, n°\textsuperscript{s} 402, 403 ; \emph{Suppl.} 1., 113, n° 646.} ville de la Bétique, désignée par Pline\footnote{\emph{H. N.} 3., 3., 7. --- Florez n'a pas tenu compte de la judicieuse restitution faite par le P. Hardouin dans le passage de Pline que j'indique ici, restitution qui a permis d'identifier Irippo et Orippo. S'il se montre disposé (\emph{loc. cit.}) à comprendre Irippo au nombre des cités de la Bétique, c'est par la seule raison qu'on a trouvé en Andalousie plusieurs exemplaires des médailles autonomes de cette ville. Eckhel et Mionnet l'ont rangée parmi les villes incertaines.} sous le nom d'\emph{Orippo}, une divinité féminine assise, tenant sur la main une pomme de pin dont la forme, les détails et surtout le très-court pédoncule rendent manifeste l'erreur commise par Florez, par Sestini et par Mionnet.

Je donne ici,\footnote{Pl. 20., n° 4.} d'après un exemplaire que possède la Bibliothèque impériale, un dessin exact de la médaille autonome d'Ostur, publiée par Florez, qui s'est borné à copier la mauvaise figure qu'on en trouve dans l'ouvrage de Bary. L'exemplaire de Paris, bien que fourré de plomb et mutilé, m'a permis, à l'aide des empreintes ou des dessins de deux autres exemplaires, de reproduire la pièce entière et avec le caractère qui lui est propre. Je donne également un dessin fidèle\footnote{\emph{Ibid.} n° 3. C'est le n° 347 de Mionnet (\emph{Descript. de méd.} t. 1., p. 48.), inexactement décrit, ainsi que je l'ai dit plus haut.} de la médaille autonome d'Ostur, de plus petit module, dont j'ai parlé ci-dessus. Il s'en trouve, à la Bibliothèque impériale, trois autres exemplaires moins bien conservés, qui me semblent être d'un même coin. La variante que présentent le droit et le revers de cette pièce, comparée au grand module, n'a été connue ni de Florez, ni d'Eckhel, ni de Sestini.

J'ajoute ici,\footnote{Pl. 8., n° 11.} d'après le dessin publié par le dernier de ces trois numismates,\footnote{Ouvrage cité, pl. 1., n° 17.} la médaille autonome citée d'Anticaria, dont un exemplaire unique se conservait autrefois dans le musée Hédervar. Je ne le retrouve ni dans le médaillier de la Bibliothèque impériale,\footnote{Cette médaille n'a été ni connue d'Eckhel, ni décrite dans le recueil de Mionnet.} ni dans la collection de M. Rollin, et j'ignore quel en est le possesseur actuel. Je ne sais non plus ce qu'est devenue une médaille analogue, décrite aussi par Sestini.\footnote{\emph{Loc. cit.} p. 23, n° 2.}

Enfin je reproduis,\footnote{Pl. 20., n° 2.} également d'après le dessin qu'il en a donné, la médaille autonome de Balsa dont il a été question plus haut. Comme les deux précédentes, elle appartenait au musée Hédervar ; mais je n'ai pu découvrir en quelles mains elle est passée.

Les témoignages qui résultent des médailles d'Ostur, d'Olunt, d'Anticaria et de Balsa suffisent, sans doute, pour confirmer notre présomption à l'égard de l'importation du symbole du cyprès pyramidal en Ibérie et en Lusitanie. Dès lors, ne nous est-il pas permis de croire que cette importation fut l'œuvre des Phéniciens, et qu'elle se rattachait particulièrement à l'introduction du culte de Baal et d'Astarté ? Le croissant et l'astérisque que nous montrent les médailles autonomes d'Anticaria, le grand croissant placé au pied du cyprès, sur celles de Balsa, semblent nous autoriser à supposer même que les Ibériens et les Lusitaniens avaient appris des Phéniciens à représenter, sous la forme symbolique du cyprès, la déesse Astarté, considérée comme une divinité luni-solaire ou simplement lunaire.\footnote{Remarquons qu'au revers de la médaille autonome d'Irippo, citée plus haut, on voit le croissant de la lune posé au bord du trône où est assise une divinité qu'il est bien difficile de ne pas prendre pour Vénus. D'une main, elle tient une pomme de pin ; de l'autre, une corne d'abondance. Or nous savons, non-seulement que le pin était un arbre consacré à Vénus en Orient et en Occident, mais que la pomme de pin joue un grand rôle dans les cérémonies religieuses qui, chez les Assyriens, se célébraient en l'honneur de Mylitta (voy. les bas-reliefs de Nimroud, publiés par M. A. H. Layard). Nous savons aussi combien sont nombreuses les médailles asiatiques qui représentent Astarté portant une corne d'abondance.} D'autre part, les médailles autonomes d'Ostur nous ramènent aux médailles asiatiques et au trône votif de la Déesse de Syrie, qui nous ont montré Vénus formant une triade avec le soleil et la lune. Là, en effet, nous avons trouvé tantôt un cyprès placé entre ces deux astres personnifiés,\footnote{Pl. 3., n° 1.} ou entre un taureau et un lion,\footnote{Pl. 6., n°\textsuperscript{s} 1 et 2 ; cf. \emph{ibid.} n° 3.} emblèmes de la lune et du soleil ; tantôt un grand cyprès planté entre deux petits cyprès,\footnote{Pl. 5., n° 1.} autre triade symbolique, bien plus analogue à celle que nous offre, au revers d'une des médailles d'Ostur,\footnote{Pl. 20., n° 4.} le grand cyprès planté entre deux rameaux ou deux petits arbrisseaux.

De nos jours, la culture du cyprès pyramidal se perpétue dans la péninsule hispanique, de même qu'en France et en Italie ; et cette culture, on le voit, s'étend, comme jadis, des côtes orientales de la Chine aux côtes occidentales du Portugal. 
\clearpage
\setcounter{footnote}{0}
\section{Second Mémoire --- Du cyprès pyramidal considéré comme symbole funéraire et comme emblème ou attribut des divinités infernales en orient et en occident}
\subsection{Première Section --- Orient}
\subsubsection{Asie occidentale, Asie orientale, Arabie, Égypte et littoral de Carthage}
\paragraph{}
Si, en Orient et en Occident, le cyprès pyramidal fut l'image symbolique et vivante des divinités génératrices et, en particulier, de Vénus, du Soleil et de la Lune; s'il fut l'emblème du feu créateur et de cette flamme qui, des autels dressés sur la terre, s'élance vers les cieux, comme l'âme qui aspire à rentrer dans le sein de l'Éternel ; si une antique prescription voulait que les objets les plus propres à exprimer les idées de création, de génération, de reproduction, ou de vie terrestre, servissent d'hiéroglyphes idéographiques pour inscrire sur les monuments funéraires le dogme de l'immortalité de l'âme et l'annonce d'une nouvelle vie, il ne faut pas s'étonner de trouver, chez tous les peuples civilisés de l'antiquité, le cyprès pyramidal placé auprès des divinités infernales, ou employé comme emblème funèbre. Un tel choix n'était-il pas justifié par les conditions d'existence propres à cet arbre ? Sa longévité,\footnote{... perpetua nunquam moritura cupresso,\\\hspace*{5mm}dit Martial (\emph{Epigramm.} 6., 73., v. 7).} la nature incorruptible de son bois,\footnote{Ci-dessus, p. 9, note 1. --- A l'occasion de deux statues de bois (\emph{xoanon}) qu'il avait vues près de la colonne d'Œnomaüs, dans l'Altis, et qui représentaient, l'une, Praxidamas d'Égine, l'autre, Rhexibius d'Opus, les deux premiers athlètes qui furent vainqueurs aux jeux olympiques, Pausanias (6., 18., 5) remarque que la seconde, faite avec du bois de cyprès, avait beaucoup mieux résisté aux injures du temps que celle de Praxidamas, qui était de bois de figuier. --- Voyez les diverses autorités citées par Hirt, au sujet de la nature incorruptible du bois de cyprès, \emph{Kunstbau}, t. 1., p. 374, 375.} son feuillage toujours vert, et cette circonstance qu'une fois coupé, le cyprès ne repousse pas,\footnote{Cette observation, qui s'étend à tous les arbres verts, n'avait point échappé aux anciens. Voyez Pline (17., 39., 2), Servius (\emph{ad} Virgil. \emph{Georg.} 1., 20 ; \emph{Æneid.} 3., 64, 680, 681 ; 4., 507 ; 6., 216) et un mythographe latin anonyme (\emph{Scriptor. rer. mythic. latini tres}, t. 1., fab. 6, § 28, p. 191, ed. Bode). Toutefois Théophraste avait été induit en erreur lorsqu'il dit (\emph{Hist. plantar.} 4., 6.) que, dans l'île de Crète et nommément à Tarrha, le cyprès coupé repousse du tronc et même des racines.} tout ne contribuait-il pas à faire d'un tel arbre un emblème expressif de vie, de mort et d'immortalité ?

Un fragment d'Asclépiade, conservé par Piérius Valérianus,\footnote{\emph{Hieroglyphica}, lib. 2., cap. 4. --- Je transcris ici en entier le passage de Piérius : « Aiunt enim (Græci), Asclepiadis authoris gravissimi testimonio, Boream, Celtarum regem, cum filiam Cyparissam nomine amisisset, arborem hanc in ejus tumulo sevisse, a qua inde nomen inditum cyparisso : mox a satu illo luctuosam feralemque haberi cœptam. » Ce fragment d'Asclépiade a été reproduit par Werfer dans le tome 2 des \emph{Acta philolog. monacensium}. C'est par erreur que Needham, premier éditeur des \emph{Geoponica}, dit que le P. Rapin a fait entrer dans le deuxième chant de son poème sur les jardins la fable de Cyparisse, fille d'un roi des Celtes. Le P. Rapin n'est coupable que de la témérité d'avoir, après Ovide, essayé de mettre en vers latins la métamorphose de Cyparisse (voy. \emph{Hortorum lib. 4., et cultura hortensis}, 2., v. 395-419, ed. Brotier ; Paris, 1780). Il est fâcheux que le second éditeur des \emph{Geoponica}, Nicolas Niclas, ait reproduit la note de Needham sans relever la méprise que j'indique.} nous apprend, comme je l'ai dit dans mon premier mémoire, que, chez les Grecs, une tradition voulait que le cyprès eût reçu sa signification funéraire et son nom, \emph{cyparissus}, de \emph{Cyparissa}, fille d'un roi des Celtes, nommé \emph{Boreas}, qui, ayant eu le malheur de la perdre, avait planté sur son tombeau un arbre de cette espèce. Sans nous arrêter à un tel récit, remarquons que le témoignage le plus authentique qui nous soit parvenu sur l'emploi de cet arbre dans les anciens rites funéraires, résulte des renseignements recueillis par Virgile, lorsque, se disposant à écrire l'Énéide, il voulait s'y montrer aussi versé dans la connaissance des rites sacrés que le furent Homère, Hésiode, Pindare, Eschyle, Sophocle et tant d'autres poètes de la Grèce. Déjà nous avons vu le chantre latin du fils d'Anchise et de Vénus mettre dans la bouche de son pieux héros des vers qui attestent que, chez les Troyens, le culte du cyprès remontait à une haute antiquité et s'associait au culte de la Mère des Dieux.\footnote{Ci-dessus, p. 80, 81.} Plus loin,\footnote{\emph{Æneid.} 3., 61-67.} il nous montre Énée, occupé, sur les côtes de la Thrace, à rendre les honneurs funèbres aux manes de l'infortuné Polydore, après avoir offert un sacrifice à Vénus, sa mère, et au tout-puissant Jupiter. Par son ordre, des autels funéraires se dressent ; on les couvre de branches de cyprès ; et des femmes troyennes les entourent, les cheveux épars, selon une coutume nationale :

\begin{quotation}
... ... stant manibus aræ

Cæruleis mæstæ vittis atraque cupressu,

Et circum Iliades crinem de more solutæ.
\end{quotation}

\paragraph{}
Parvenu au rivage que rendait si célèbre l'antre de la sibylle de Cumes, le héros fugitif reçoit de la prêtresse d'Apollon l'injonction de remplir un devoir semblable envers un de ses compagnons d'armes, Misène, dont le corps, à son insu, gisait, sans sépulture, sur le sable de ce rivage. À sa voix, les Troyens élèvent un immense bûcher (\emph{pyra}) de chêne et de bois résineux ; ils y attachent des branches de cyprès, et, en avant, ils plantent des arbres de cette même espèce\footnote{\emph{Æneid.} 6., 211-217. Cf. Servius, \emph{ad} Virgil. \emph{Æneid.} 6., 215, 216.} :

\begin{quotation}
Ingentem struxere pyram, cui frondibus atris

Intexunt latera, et ferales ante cupressos

Constituunt ... ...
\end{quotation}

\paragraph{}
Mais de tels usages avaient, n'en doutons pas, la même origine que la coutume de consacrer le cyprès à la divinité féminine dont l'antique temple, situé hors des murs de Troie, fut le lieu de rendez-vous d'Énée et de ses compagnons d'infortune, au moment de quitter pour toujours le sol de la patrie. Cette origine, il faut l'attribuer au grand peuple de qui les Phrygiens, comme les Phéniciens, les Arméniens et les Arabes, avaient reçu le culte du cyprès avec celui de Mylitta, d'Astarté, d'Anahid ou d'Alilat. Si les révolutions politiques et physiques qui ruinèrent de fond en comble l'empire d'Assyrie, n'ont laissé subsister aucun monument littéraire, aucun monument figuré, où nous puissions découvrir un exemple de l'emploi du cyprès pyramidal dans les cérémonies funèbres ou sur les sépultures des Assyriens, nous trouvons néanmoins, dans un autre récit de Virgile, une troisième preuve de ces usages, très-indirecte sans doute, mais non moins concluante que les deux premières. Ici, le lieu de la scène est Carthage. A ce nom, nos souvenirs se pressent en foule pour nous rappeler que Carthage, colonie de Tyr, rendait, sous divers noms, un culte particulier à la Vénus assyrienne, naturalisée en Phénicie. Nous nous rappelons, en même temps, que, chez les Phéniciens, cette déesse, appelée par eux \emph{Bérouth} ou \emph{Brathy},\footnote{Ci-dessus, p. 88-91.} c'est-à-dire \emph{Cyprès} (\emph{Cypris}), était représentée, ainsi que ses deux assesseurs, le soleil et la lune, sous la forme même de cet arbre symbolique.\footnote{Ci-dessus, p. 19 et suiv.} Dès lors, si Virgile,\footnote{\emph{Æneid.} 4., 503-507.} décrivant l'immense bûcher de chênes et de pins, dressé, par l'ordre de Didon, au fond de son palais, nous dit que la reine couronna de rameaux funèbres ce bûcher,

\begin{quotation}
... ... et fronde coronat

Funerea ... ...
\end{quotation}

\paragraph{}
n'entend-il pas désigner ici, par cette expression,\footnote{Stace (\emph{Thebaid.} lib. 4., v. 460, 461), dans une imitation du passage de Virgile, est plus explicite, car il dit :\\\hspace*{10mm}... frondes, atque omne cupressus\\\hspace*{10mm}Intexit plorata latus ...} comme ailleurs,\footnote{Ci-dessus, p. 295.} par les mots \emph{frondibus atris}, les branches de l'arbre qu'il appelle ailleurs aussi\footnote{\emph{Ibid.} Sur cette épithète d'\emph{atra}, appliquée au cyprès, on peut consulter Servius (\emph{ad} Virgil. \emph{Æneid.} 3., 64 ; 6., 215). Ce commentateur l'explique en rappelant que les dieux Lares sont qualifiés \emph{atrati}, et que les feuillages noirs conviennent aux cérémonies funéraires.} \emph{atra cupressus} ? A l'appui de ma conjecture, je ne dois pas omettre de dire, d'après un renseignement qui m'est communiqué par un voyageur digne de foi, qu'on a récemment trouvé, dans un cimetière phénicien ou punique, situé près du vieil Arzeu, à une journée à l'est d'Oran, deux cents pierres tumulaires environ, parmi lesquelles plusieurs portent une figure d'homme ou de femme placée sous un portique flanqué de deux cyprès.

Les monuments de l'art qui se sont conservés en Égypte attestent que, dans cette partie si célèbre de l'Afrique, les Assyriens ou les Phéniciens avaient porté avec le culte de la Vénus de Babylone et de Ninive, appelée \emph{Hâthôr} ou \emph{Athyr} par les Égyptiens, l'usage d'employer le cyprès pyramidal comme un emblème funéraire. Cet arbre étranger au sol de l'Égypte et qu'à Dendérah, néanmoins, nous avons reconnu\footnote{Ci-dessus, p. 164-166.} au milieu d'une scène d'adoration, sur les murs du temple de Vénus-Hâthôr ; cet arbre que nous avons vu placé, avec la signification de symbole de vie, sur deux stèles consacrées à la même déesse\footnote{Ci-dessus, p. 167-183, et pl. 11. ; \emph{Rech. sur Vénus}, pl. 14.} et sur des bas-reliefs où il sert à caractériser \emph{Ammon générateur} et \emph{tôbi}, le premier mois de l'équinoxe du printemps\footnote{Ci-dessus, p. 183-190 ; pl. 10., n°\textsuperscript{s} 3-5.} ; cet arbre, nous le trouvons figuré, non-seulement sur des papyrus funéraires, mais sur un grand nombre de caisses de momies égyptiennes,\footnote{On en peut voir quelques-unes au Musée du Louvre et au cabinet des médailles et antiques de la Bibliothèque impériale.} où sont peintes des scènes psychologiques, auxquelles préside la déesse elle-même, tantôt sous un nom, tantôt sous un autre.

De plus, les beaux dessins rapportés de Karnac, par M. E. Prisse, fournissent la preuve, bien que ce voyageur n'en ait pas fait la remarque, que, dans l'ancienne Égypte, le cyprès pyramidal, comme symbole funèbre, n'était pas seulement placé sur les caisses de momies. Un arbre de cette espèce est planté, entre deux figures humaines, au milieu de la partie supérieure d'un bas-relief dessiné par M. E. Prisse dans l'édifice de Tahraka, situé à l'extérieur du mur d'enceinte du palais de Karnac, près de la salle hypostyle.\footnote{\emph{Monum. de l'Égypte}, p. 7, et pl. 23.} Or le voyageur français, après avoir examiné avec attention l'ensemble et les détails de l'édifice de Tahraka, n'hésite pas à déclarer que cet édifice était consacré à des rites funéraires.

Le rôle qu'en Égypte remplissait la déesse Hâthôr, comme reine à la fois des vivants et des morts, est conforme à celui que remplissaient Mylitta, chez les Assyriens ; Astarté, chez les Phéniciens ; Aphrodite, chez les Grecs ; Vénus, chez les Romains ; Mithra, chez les Perses. Ce sont autant de divinités médiatrices et psychopompes, qui règnent dans le ciel, sur la terre et dans le royaume des morts.\footnote{Voyez mes \emph{Recherches sur Vénus}, p. 97-100.}

Si les divers témoignages que j'ai produits nous ont permis de remonter au berceau primitif du culte du cyprès pyramidal, et si nous remarquons, avec tous les voyageurs, que les peuples musulmans de la Perse et de la Turquie d'Asie emploient cet arbre comme emblème funéraire, il devient impossible, ce me semble, de ne pas admettre que, lorsque les Arabes, d'un côté, les Turcs, de l'autre, envahirent ces contrées, ils durent y trouver cet usage établi de temps immémorial. Ils l'adoptèrent ; et personne n'ignore que, depuis bien des siècles, sans interruption et malgré tout le zèle, toute la persévérance que mirent les khalifes à abolir partout le culte du cyprès, image vivante de quelque antique divinité, les musulmans, sur toute l'étendue de la Turquie d'Asie, comme sur toute l'étendue de la Turquie d'Europe, plantent habituellement des cyprès pyramidaux dans les \emph{champs des morts}, c'est-à-dire dans leurs cimetières. Là, souvent un arbre de cette espèce est sculpté au revers d'une sorte de stèle\footnote{Sur leur face antérieure, les stèles musulmanes portent, gravés en relief, les noms du défunt ou une sentence religieuse.} dressée à l'une des extrémités de la tombe qui renferme le corps du défunt, ou de la pierre tumulaire qui le recouvre. Fréquemment aussi l'on voit des branches de cyprès, et même des cyprès entiers sculptés sur les pierres tumulaires qui marquent des sépultures de femmes musulmanes.

En Perse, cette dernière coutume ne semble pas avoir été assez attentivement observée par les voyageurs pour qu'il soit possible de décider si elle s'étend ou non aux sépultures des hommes,\footnote{Feu sir William Ouseley (\emph{Travels}, etc. vol. 2., p. 93) parle de quelques pierres tumulaires du cimetière de Fassa, qui sont ornées d'un cyprès sculpté ; mais il ne dit pas si elles appartiennent à des sépultures d'hommes ou à des sépultures de femmes. Il fait seulement remarquer que cet usage n'est point particulier au cimetière de Fassa, et qu'il l'a observé depuis le golfe Persique jusqu'à la mer Caspienne, et depuis le Mazendéran jusqu'à Constantinople.} et je n'ai moi-même conservé aucun souvenir qui puisse me permettre de trancher la question.

Thomas Herbert,\footnote{Ouvrage cité, p. 238, traduct. franç. p. 380.} décrivant sur les lieux, en 1627, les cérémonies funèbres des Persans, dit expressément que les personnes qui suivent le corps d'un parent décédé portent à la main des branches de cyprès ou de laurier.\footnote{Je ne sais si cet usage était déjà tombé en désuétude au temps où Chardin visitait la Perse, mais on n'en trouve aucune mention dans la relation de ce voyageur, et je n'en puis citer aucun exemple d'après mon propre témoignage.} D'autre part, les plantations de cyprès pyramidaux devant ou autour des sépultures attestent, dans toute l'étendue de ce pays,\footnote{Voy. M. Carl Ritter, \emph{Die Erdkunde}, 11., p. 574, 575.} qu'elles sont la continuation d'un ancien usage. Je citerai, en particulier, les cyprès qui, de nos jours, ornent encore deux tombeaux érigés à Hafiz et à Saadi, dans les environs de Schîrâz.\footnote{Kæmpfer, \emph{Amœnit. exotic.} p. 368, 369. M. J. Morier, \emph{A Journey through Persia, Armenia,} etc. \emph{in the years 1808 and 1809}, p. 104. Sir William Ouseley, \emph{Travels}, vol. 2., p. 3 et 4 ; pl. 24.} On prétend même que le premier de ces deux célèbres poètes persans avait planté de ses propres mains les cyprès qui se voient au pied de sa tombe,\footnote{M. J. Morier, \emph{loc. cit.}} dans une des cours de l'édifice appelé, de son nom, l'\emph{Hafizée}. Cette cour est ornée d'une autre plantation de beaux cyprès\footnote{Aga Beschir, chef des eunuques du palais de Schîrâz, a été la cause d'une émeute, en faisant imprudemment couper deux de ces arbres (voy. M. J. Morier, \emph{loc. cit.} Ouseley's \emph{Travels}, vol. 1., p. 400).} ; et deux lions de ronde bosse sont posés sur la plate-forme du mur, de chaque côté de la porte d'entrée, comme pour rappeler qu'anciennement, chez les Perses, le lion, aussi bien que le cyprès, avait parfois une acception funéraire.\footnote{Voy. \emph{Mémoires de l'Acad. des inscript.} nouv. série, t. 15., 2\textsuperscript{e} partie, p. 63-127.} Sur la route d'Ispahân à Tehérân, et dans plusieurs autres lieux de la Perse, on rencontre des pierres tumulaires qui portent, grossièrement sculptés, des lions et des cyprès.\footnote{Ouseley's \emph{Travels}, vol. 3., p. 83.} Non loin de Schîrâz, des arbres de cette espèce ornent la cour qui précède l'édifice funéraire nommé \emph{Heft-ten} (Les-sept-corps), en mémoire de sept personnages extraordinaires et pieux, qui avaient vécu, comme des derviches, sur l'emplacement où, par l'ordre de Kérim khân,\footnote{A Schîrâz, dans la ville même, s'élève un majestueux cyprès devant la première cour de la mosquée de Schah Mirza Hamza, qui renferme, dans une pièce séparée, le tombeau d'Abdul Rakim khân, dont la construction est également due à Kérim khân, père de ce dernier prince (voyez M. J. Morier, ouvrage cité, p. 103).} fut construit l'édifice. Ainsi, pour me servir d'un rapprochement piquant, que j'emprunte à M. Carl Ritter,\footnote{Ouvrage cité, t. 11., p. 574.} sur le territoire même où les sectateurs de Zoroastre ont laissé les ruines d'un grand nombre d'\emph{autels du feu} (\emph{ateschgahs}), les cyprès qui jadis entouraient ces pyrées ombragent maintenant les tombeaux où reposent des \emph{imâms}, c'est-à-dire des personnages réputés saints parmi les sectateurs de Mahomet.

A ces témoignages, vient s'ajouter celui qui se tire du nom même que le cyprès pyramidal conserve dans la langue persane. Ce nom, \emph{serv-azâd}, est un exemple mémorable de l'influence directe qu'exercèrent sur la formation du langage les castes sacerdotales, dépositaires tout à la fois des croyances religieuses et des connaissances scientifiques, et, par conséquent, dispensatrices du trésor des idées. En effet, \emph{serv-azâd} offre par sa composition, je dois le répéter ici, une double signification, qui, à défaut d'autres preuves, nous révélerait l'usage où furent anciennement les Perses d'attribuer au cyprès une acception funéraire. Il est formé de deux mots\footnote{$\arabicAAD$, qui signifient littéralement : « cyprès libre. » Les Persans disent aussi : $\arabicAAE$ \emph{serv-bulend}, c'est-à-dire « cyprès élevé » (\emph{alta cupressus}).} : le premier est le nom même de cet arbre\footnote{Le nom sous lequel les Persans désignent plus habituellement le cyprès est \emph{schemschâd}, $\arabicAAF$, dont on ne connaît pas le sens intime.} ; le second est un adjectif dérivé du participe du verbe persan \emph{zâden}, qui signifie \emph{vivre}, et qui, comme le verbe grec \textgreek{ζάω}, « vivre, » appartient au même radical que \emph{gaya, gava} ou \emph{gueïé}, dont la double signification, en zend, est \emph{taureau} et \emph{vie}. L'\emph{a} privatif placé devant \emph{zâd} donne donc à l'adjectif \emph{azâd} la propriété d'exprimer l'idée de \emph{non-vie} ou de \emph{liberté}\footnote{C'est avec le sens de \emph{libre} que nous trouvons cet adjectif employé dans la qualification d'\emph{azâd-mard}, qui signifie littéralement « homme libre, » et qui fut donnée à Lohrasp, père de Gustasp (\emph{Modjmel-al-téwarikh}, fol. 272 v°, manuscrit de la Bibliothèque impériale). Cette qualification répond, on le voit, au \emph{Freiherr} des nations germaniques.} que l'antiquité attachait à l'état de l'âme après la mort du corps. Cet état faisait dire aux philosophes et aux poètes grecs, comme aux philosophes de l'Orient, leurs devanciers : « La vie du corps est la mort de l'âme, et \emph{vice versa}, » où : « Notre vie est la mort des âmes, et leur vie\footnote{C'est-à-dire la liberté de l'âme.} est notre mort.\footnote{\textgreek{Ζῆν ἡμᾶς τὸν ἐκείνων (ψυχῶν) θάνατον · καὶ ζῆν ἐκείνας τὸν ἡμέτερον θάνατον} ... Héraclite, \emph{apud} Porphyr. \emph{De antr. nymphar.} 10., p. 12, ed. van Goens. Cf. Euripide, \emph{Polyidus} (fragm. v. 15, 16) ; \emph{Phrixus} (fragm. v. 34, 35).} »

Faute d'avoir rapproché de l'antique doctrine psychologique qui avait cours en Orient et en Occident, la double acception du cyprès, symbole de vie\footnote{Dans l'idiome de la basse Bretagne, le nom du cyprès est \emph{hivi} ou \emph{ivi}, altération peut-être des mots sémitiques ou araméens \emph{hayy, heuïa, héva, héyo}, qui signifient « la vie. » Le nom de l'arbre vert que nous appelons \emph{if} aurait-il la même origine ? En espagnol et en portugais, on dit \emph{iva} ; dans l'ancien allemand, on trouve \emph{îwa}, et dans l'allemand moderne, \emph{eibe}.} et symbole de mort ou de liberté\footnote{M. de Hammer (\emph{Mithriaca}, p. 157) cite trois passages de prosateurs ou de poètes persans qui font allusion à la \emph{liberté}, dont le cyprès est l'emblème : « Si tu le peux, dit Saadi, sois fertile comme le palmier ; si tu ne le peux, sois libre comme le cyprès. » --- « Qui rendra ton âme libre comme le lis et le cyprès ? Qui lui donnera la félicité au milieu de ses liens ? » s'écrie à son tour le grand poète mystique Djelal-eddîn Roumî, cité à l'article \emph{Azâd}, dans le \emph{Ferhenghi Chououri} (1., 68). --- Un autre écrivain persan dit : « Zerdéhuscht (Zoroastre) planta un cyprès libre devant la porte du pyrée. »} ; et, faute aussi d'avoir connu la composition du nom philosophique que cet arbre conserve chez les Persans, les philologues et les archéologues modernes, sans en excepter même le savant auteur du mémoire cité sur le mythe de Cyparisse, n'ont pas aperçu le sens philosophique ou psychologique que présente le cyprès pyramidal, consacré aux divinités infernales, ou placé auprès des sépultures. Füller\footnote{\emph{Miscellan. sacr.} 4., 5.} voulait que cet arbre eût reçu une acception funéraire, « parce qu'il avait servi à fabriquer l'arche de Noé, qui, pour ainsi dire, fut longtemps le tombeau ou la sépulture des vivants. » Le Dr. Lampe\footnote{Ouvrage cité, p. 26, 27.} a pris la peine de réfuter cette opinion : à ses yeux, le cyprès est un symbole d'éternité, parce qu'il vit indéfiniment. Aussi fait-il observer que jadis cet arbre figurait dans les funérailles des personnages qui avaient mérité les honneurs de l'apothéose, ou accompli des actions dignes de passer à la postérité. C'est pourquoi, dit-il, l'usage de planter un cyprès ou un \emph{cupressetum} auprès des tombeaux fut limité aux héros ou aux hommes illustres. A l'appui de cette assertion, il cite le vers où Lucain\footnote{3., 442.} a pu dire :

\begin{quotation}
Et non plebeios luctus testata cupressus.
\end{quotation}

\paragraph{}
Il cite aussi un passage de Spartien,\footnote{\emph{In Severo}, 22.} d'où l'on est en droit de conclure qu'une couronne de cyprès était le prélude de l'apothéose.\footnote{J'ai dit, dans le mémoire précédent (ci-dessus, p. 142, 143), que les couronnes de cyprès qui, selon Dinon, étaient en usage chez les Perses, avaient peut-être une acception funéraire. --- Voy. sur les couronnes considérées comme des emblèmes funèbres et comme des emblèmes de l'apothéose, les judicieuses observations de M. Raoul-Rochette (\emph{Mém. de l'Acad. des inscript.} nouv. sér. t. 13., p. 159, p. 209-217). Elles s'étendent et s'appliquent parfaitement aussi au symbole de la palme.} Il n'oublie pas d'invoquer le témoignage d'Hermippe que j'ai rappelé dans mon premier mémoire.\footnote{Ci-dessus, p. 216, note 3.} Enfin, il rapporte un passage où Jamblique,\footnote{\emph{Vita Pythagoræ}, 155.} répétant le précepte de Pythagore\footnote{Voyez les observations du Dr. Lampe (ouvrage cité, t. 1., p. 578 et suiv.) sur ce précepte et sur les commentaires dont il a été le sujet dans les écrits de Bochart et de Giraldi.} cité par Hermippe, dit aussi, de son côté, que les pythagoriciens s'abstenaient de se faire inhumer dans des cercueils de cyprès, parce que le sceptre de Zeus est fait avec le bois de ce même arbre ; mais, ajoute Jamblique, il se peut qu'un tel précepte repose sur quelque autre considération mystique.\footnote{... \textgreek{ἢ δι' ἄλλον τινὰ μυστικὸν λόγον}.} Feu M. Émeric David rejette d'une manière absolue le double sens attribué au symbole du cyprès : « Cet arbre prétendu de la mort, dit-il à l'occasion de la métamorphose de Cyparisse,\footnote{\emph{Recherches sur Jupiter}, t. 2., p. 437.} est réellement l'arbre de la résurrection, l'arbre de la métempsycose ; il s'élance du sein de la terre, comme le rayon solaire jaillit du milieu des ténèbres, au lever de l'astre du jour, et il tue le cerf, emblème des vapeurs humides ... » A l'exemple de plusieurs autres de ses devanciers, M. Avellino, s'occupant de l'attribution funéraire du cyprès chez les Grecs et les Romains seulement, cherche le motif de cette attribution dans un ordre d'idées qui exclut toute acception philosophique ou psychologique ; il croit le trouver dans une simple règle d'euphémisme : « Il nous semble impossible, dit-il, de mettre en doute que, dans le sens funèbre attribué au cyprès, on ne doive reconnaître un des nombreux exemples de cet \emph{euphémisme} par lequel l'antiquité grecque embellissait les idées les plus tristes, et, principalement dans le langage des arts, les revêtait des formes les plus riantes et les plus agréables.\footnote{« Noi non crediamo potersi dubitare che nel senso funebre dato al cipresso riconoscer debbasi uno de' tanti esempli di quell' \emph{eufemismo}, per cui l'antichità greca abbelliva le idee tutte di lutto, e specialmente nel linguaggio delle arti le vestiva di ridenti e piacevoli apparenze. » (Mémoire cité, p. 28, 29.)} » Partant de ce point de vue, le savant académicien de Naples attribue à l'expression de Virgile \emph{læta cupressus}, au lieu d'un sens philosophique,\footnote{Dans sa \emph{Somme des vices et vertus} (manuscrits français de la Bibliothèque impériale, n° 7283, fol. 33 v°), un frère prêcheur, Lorens (\emph{Laurentius Gallus}), exprime sur la mort la pensée religieuse du moyen âge en ces termes énergiques et naïfs, qui peuvent servir à expliquer le sens que Virgile attachait à l'expression \emph{læta cupressus} : « Mes li sainz homme qui Deu aiment ... mort sunt au pechié et mort au monde ... et desirent la morz corporel. Car cest damoisele porte-joie que la morz, qui touz les seinz corone et met en gloire ... »} un sens tiré de l'emploi du cyprès comme ornement des jardins : « Et l'usage qu'en faisaient les anciens pour l'ornement des sites les plus délicieux ne permet pas, observe-t-il, de douter que, loin d'attribuer à cet arbre la triste idée de la mort, ils ne lui reconnussent, au contraire, une signification agréable.\footnote{« E l'uso che gli antichi ne facevano per ornamento de' siti più deliziozi, non ci lascia dubitare che lungi dall' attribuire a quella pianta la trista significazione della morte, la riguardavano al contrario come piacevole e grata. » (Mémoire cité, p. 32.)} » Reiff\footnote{\emph{Ad} Artemidor. \emph{Oneirocr.} 2., 25.} me semble s'écarter bien davantage de la véritable signification lorsque, s'efforçant d'expliquer \emph{læta cupressus} par \emph{alta cupressus}, il rappelle l'expression \emph{lætas segetes} qu'emploie Virgile dans ses Géorgiques.\footnote{1., 1.} M. de Hammer, je le dis à regret, ne me paraît pas avoir été plus heureux dans ses explications. Les raisons qu'il donne\footnote{\emph{Gesch. der schön. Redekunst Persiens}, etc. (Vienne, 1818, in-4°), p. 26, 27 ; \emph{Mithriaca}, p. 42, 43.} pour faire comprendre comment le cyprès, chez les Persans, est devenu le symbole de la liberté, me portent à penser qu'il a un peu trop négligé de considérer la corrélation établie par une philosophie religieuse entre l'idée de la mort et l'idée de la nouvelle vie ou de la liberté de l'âme.

À l'est de la Perse, nous trouvons jusqu'à l'Indus un vaste pays, où nous savons\footnote{Ci-dessus, p. 151-153.} que croît spontanément le cyprès pyramidal. Les voyageurs modernes nous ont laissé ignorer si l'on y rencontre des arbres de celte espèce plantés dans les \emph{champs des morts}, ou auprès des sépultures particulières. Mais ce que j'ai eu à faire remarquer au sujet de la Perse et de la Turquie d'Asie\footnote{\emph{Ibid.} p. 298, 299.} s'applique de même à l'Afghanistân, et nous permet de supposer que là, comme dans toutes les contrées soumises à l'islamisme, le cyprès ombrage les tombeaux, ou figure dans les sculptures qui ornent les pierres tumulaires.

Je ne serai pas aussi affirmatif à l'égard des immenses régions situées à l'est de l'Indus ; et j'exprime ici le regret de ne trouver dans les relations modernes aucun renseignement qui nous fasse connaître si là on emploie le cyprès comme emblème funèbre, et si cet usage est commun aux populations musulmanes et aux diverses sectes qui n'ont pas embrassé l'islamisme, ou s'il contribue, avec les autres croyances religieuses, à établir une ligne de séparation entre les populations indigènes.

Toutefois, si, chez les Hindous, le cyprès pyramidal ne se montre pas davantage autour des sépultures et sur les pierres tumulaires qu'il ne figure auprès des divinités représentées dans les pagodes ou dans les grottes consacrées par la religion, il reçoit, dans la langue sanscrite, une dénomination, \emph{soka}, dont la signification propre est \emph{tristesse, douleur}.\footnote{En sanscrit, \emph{asoka}, c'est-à-dire \emph{soka} avec l'\emph{a} privatif, signifie : « sans douleur. »} Les Persans conservent dans leur idiome ce mot \emph{soka} sous la forme \emph{soug}, avec le sens de \emph{deuil, tristesse} ; mais ils ne l'appliquent pas au cyprès. De \emph{soug}, ils ont formé \emph{sougwar}, qui signifie \emph{triste, affligé}. Je ne sais si ces mots \emph{soka} et \emph{soug} ont quelque rapport avec \emph{sôs}, nom du cyprès chez les Arméniens,\footnote{Ci-dessus, p. 65-71. --- Il ne faut pas perdre de vue que la langue arménienne appartient à la même souche que les langues sanscrite, zende et persane.} et avec les usages funèbres auxquels nous voyons cet arbre servir chez tant de peuples anciens ou modernes. J'ignore, par conséquent, s'il serait permis de conclure d'un tel rapprochement qu'à des époques fort reculées, dont le souvenir s'est effacé dans la mémoire des hommes, les Indiens, comme la plupart des autres nations civilisées de l'Orient, avaient fait du cyprès un emblème funéraire. Devons-nous croire, par exemple, que lorsque, à Babylone, le philosophe indien Calanus faisait jeter du bois de cyprès dans le bûcher (\emph{pyra}) où volontairement il allait chercher la mort,\footnote{Ælien, \emph{Var. histor.} 5., 6.} il observait un usage propre à l'Inde, plutôt qu'il ne se conformait à une coutume babylonienne ? Ou bien faut-il admettre qu'un tel usage, quant au bois de cyprès, ne fut jamais pratiqué dans l'Inde, et que le nom de \emph{soka} et sa signification \emph{tristesse, douleur}, y étaient une importation étrangère ? En ce cas, devrait-on attribuer cette importation au grand mouvement qui, durant une période très-reculée, semble avoir fait rayonner d'un centre commun vers les diverses parties du monde la haute civilisation à laquelle étaient parvenus, sur les bords du Tigre et de l'Euphrate, les Chaldéens et leurs élèves, les Assyriens ? Je n'entreprendrai pas de traiter ici, encore moins de décider ces diverses questions. Attendons, pour le faire, la publication complète de tous les livres religieux des Hindous.

Une seule de ces questions se représente quant à la Chine. Dans ce vaste empire, nous avons trouvé, avec le secours des traditions écrites, quelques traces du culte que jadis on y rendait au cyprès, considéré comme symbole des dieux ou des esprits.\footnote{Ci-dessus, p. 161.} Déjà aussi ces mêmes traditions nous ont donné la certitude qu'anciennement les Chinois étaient dans l'usage de planter des cyprès autour des tombeaux\footnote{\emph{Ibid.} p. 160, 161.} et que cet usage était, en particulier, pratiqué par les princes feudataires de l'empire.\footnote{\emph{Ibid.} p. 161.} De nos jours encore, on conserve, à la Chine, la coutume de placer les sépultures à l'ombre de diverses sortes d'arbres verts, parmi lesquelles les voyageurs distinguent deux espèces de cyprès, le pyramidal,\footnote{\emph{Ibid.} p. 163. --- Sir William Ouseley (\emph{Travels}, etc. vol. 2., p. 93) affirme, mais sans citer aucune autorité, que le cyprès est consacré aux morts en Chine, comme en Turquie.} et celui que M. Fortune appelle \emph{funereal cypress}.\footnote{\emph{Cupressus funebris.} Voy. ci-dess. p. 163.} Les Chinois continuent aussi à tirer du cyprès divers médicaments, et qualifient ses feuilles de \emph{médecine de l'immortalité}.\footnote{Ci-dessus, p. 161.} Mais de quel peuple de l'Asie ont-ils primitivement reçu le cyprès pyramidal à la fois comme symbole des dieux ou des esprits, comme emblème funéraire et comme moyen thérapeutique ? J'ai déjà, en partie, dans le mémoire précédent,\footnote{Ci-dessus, p. 161, 162.} répondu à cette question ; j'ai fait remarquer quelle affectation mettent les écrivains chinois à nous dire que l'arbre \emph{pe}, qui est notre cyprès pyramidal, se tourne constamment vers le couchant. Cette remarque m'a conduit à supposer que par-là ces auteurs entendent, soit la région des morts, soit l'Asie occidentale, contrée où, en effet, toutes les observations réunies dans mon premier mémoire nous ont autorisés à placer le lieu natal de ce bel arbre et l'origine des usages religieux auxquels l'antiquité le faisait servir. Et si nous nous rappelons la longue durée de l'empire assyrien, la place immense qu'il occupe dans la topographie et dans l'histoire religieuse, civile et politique de l'Asie occidentale, nous serons portés à penser que l'importation du culte du cyprès chez les Chinois remonte à une époque où ils avaient des relations plus ou moins indirectes avec l'empire d'Assyrie. D'autres faits pourraient justifier cette conjecture ; mais ce n'est ici le lieu ni de les exposer, ni de les examiner.

Le Thibet et le Japon reçurent-ils anciennement, comme la Chine, le symbole du cyprès pyramidal ? Emploient-ils aujourd'hui cet arbre à ombrager leurs tombeaux et à fournir à leurs médecins des préparations thérapeutiques ? Je l'ignore, mais je ne saurais me défendre de le supposer.

Avant de quitter l'Asie et de retourner en Europe, il nous reste à rechercher si l'Arabie ancienne et l'Arabie musulmane nous offrent des indices certains de semblables coutumes. Bien que nous ayons trouvé dans les écrivains grecs, dans les écrivains musulmans et dans quelques monuments figurés ou lapidaires la preuve que le culte de la Vénus assyrienne était autrefois très-répandu en Arabie,\footnote{Ci-dessus, p. 117-126.} ainsi que le culte de son arbre symbolique, le cyprès, je ne puis, pour les temps anciens, découvrir la preuve que, chez les Arabes, cet arbre remplissait un rôle dans les rites funéraires. Le très-petit nombre de monuments antiques restés debout sur le sol de l'Arabie, et la destruction de ses trésors littéraires, antérieurs aux prédications de Mahomet et de ses successeurs, ne me laissent aucun espoir d'acquérir cette preuve. Mais je tiens de quelques voyageurs européens, qu'en plus d'un lieu les Arabes pratiquent maintenant l'usage de planter des cyprès autour des tombeaux, comme on le fait en Perse et dans la Turquie d'Asie. Nous sommes donc autorisés à supposer, par induction, que si une telle coutume existe aujourd'hui en Arabie, elle avait dû y être apportée de Babylone avec le culte de Mylitta et avec le symbole du cyprès. Les témoignages que j'ai recueillis dans mon premier mémoire\footnote{Ci-dessus, p. 117 et suiv.} ne laissent, ce me semble, aucun doute sur cette double importation.

\clearpage

\subsection{Deuxième Section --- Occident}
\subsubsection{Grèce et colonies grecques}
\paragraph{}
Il n'en est pas de la Grèce et de l'Italie comme de l'Arabie et de quelques autres parties du continent asiatique : ici, les témoignages les plus directs de l'emploi funéraire du cyprès abondent pour les temps anciens, aussi bien que pour le moyen âge et pour le temps présent. Déjà la métamorphose de Cyparisse, au moment où il meurt de douleur\footnote{On peut rapprocher de cette métamorphose opérée par Apollon les vers suivants des \emph{Dionysiaques} (2., 80-83), dans lesquels Nonnus fait intervenir Zéphyre et Phœbus à l'occasion du cyprès caractérisé comme emblème funéraire :\\\hspace*{10mm}\textgreek{Καὶ Ζέφυρος δεδόνητο, κυλινδομένων κυπαρίσσων\\\hspace*{10mm}Αὐχμηροῖς πετάλοισι · φιλοθρήνοισι δὲ μολπαῖς\\\hspace*{10mm}Αἴλινα Φοῖβος ἄειδε, δαϊζομένων ὑακίνθων,\\\hspace*{10mm}Πλέξας πένθιμον ὕμνον ... ...}} ; la métamorphose des filles d'Étéocle ; tombées dans un puits ; les cyprès plantés, à Psophis, autour de la sépulture d'Alcmæon, et enfin la tradition recueillie par Asclépiade,\footnote{Ci-dessus, p. 294.} nous ont préparés à reconnaître que, chez les Grecs, cet arbre était un symbole funèbre. Les cyprès du tombeau d'Alcmæon nous ont même donné lieu de constater que l'antiquité grecque, comme l'antiquité asiatique, se plaisait à représenter sous l'emblème du cyprès les personnages mythologiques qui assistent, dans ses fonctions, Vénus représentée elle-même sous la forme symbolique de cet arbre, et considérée comme reine à la fois du ciel, de la terre ou des vivants, et des enfers ou des morts. Développant ici toute ma pensée, je rappellerai qu'Aphrodite, chez les Grecs, était surnommée tantôt \emph{Épitymbie} (\textgreek{ἐπιτυμβία}),\footnote{Plutarque, \emph{Opp.} t. 7. (\emph{Quæstion. roman.}), p. 89, 90 ; ed. Reiske.} tantôt l'\emph{aînée des Parques},\footnote{\emph{Pausanias}, 1., 19., 2.} tantôt même \emph{Hadès}.\footnote{Dans un fragment d'un ancien tragique cité par Plutarque, \emph{Opp.} t. 9. (\emph{in Amator.}), p. 33.} Un hymne orphique\footnote{54., 5, ed. Hermann.} nous la montre exerçant un empire absolu sur les trois Parques, et Plutarque\footnote{\emph{Opp.} t. 7., \emph{loc. cit.}} a soin de nous apprendre qu'à Delphes on faisait, devant la statue d'Aphrodite Épitymbie, des libations pour les morts. De leur côté, Lucien\footnote{\emph{De Dea Syr.} §§ 31, 32 ; ed. Lehmann.} et Proclus\footnote{\emph{Hymn. 3., in Vener.} v. 1-3, dans les \emph{Analecta} de Brunck, t. 2., p. 444.} nous fournissent le moyen de remonter à l'origine asiatique de semblables surnoms et de semblables coutumes en nous disant, l'un, qu'à Hiérapolis la Déesse de Syrie était représentée avec les attributs des Parques, l'autre, que les grands de Lydie avaient souvent échappé aux coups de la mort par la toute-puissante protection d'Aphrodite. À son tour, Mithra, chez les Perses, qualifié roi du ciel, roi des vivants et roi des morts,\footnote{Voy. ci-dessus, p. 276, note 3.} remplit, comme Hermès ou Mercure, les fonctions de dieu psychopompe : assisté de ses deux hamkars ou assesseurs, il juge, sur le pont Tchinevâd, qui unit la terre au ciel, les bonnes et les mauvaises actions des âmes. Ces diverses traditions me conduisent à rappeler aussi que, dans la même ville où, selon Pausanias,\footnote{Ci-dessus, p. 235.} les cyprès du tombeau d'Alcmæon s'appelaient \emph{les vierges}, il avait existé un très-ancien temple d'Aphrodite Érycine, dont les ruines se voyaient encore au temps où cet écrivain visitait la Grèce.

Servius,\footnote{\emph{Ad} Virgil. \emph{Æneid.} 3., 64.} de son côté, affirme que les Athéniens avaient la coutume de couvrir de branches de cyprès les maisons mortuaires. Je serais disposé à croire que cet usage était passé de l'Asie occidentale dans l'Attique avec le culte de Minerve ou de Vénus, divinités primitivement identiques. D'autres témoignages nous apprennent que, chez les Grecs, le cyprès fut consacré à Hadès, comme il le fut à Pluton, chez les Romains. A l'appui de cette assertion, on peut citer une inscription grecque antique, gravée sur une plaque d'or, et commentée, en 1836, dans le Bulletin de l'Institut archéologique.\footnote{Ann. 1836, p. 149. Cette plaque d'or fut trouvée dans le royaume de Naples ; l'inscription est ainsi conçue :\\\hspace*{10mm}\textgreek{ΕΥΡΗΣΕΙΣ Δ ΑΙΔΑΟ ΔΟΜΩΝ ΕΠ ΑΡΙΣΤΕΡΑ ΛιμνΗΝ\\\hspace*{10mm}ΠΑΡ Δ ΑΥΤΗΙ ΛΕΥΚΗΝ ΕΣΤΗΚΥΑΝ ΚΥΠΑΡΙΣΣΟΝ.}} Au milieu d'une phrase relative à la destinée des âmes, elle fait mention d'un \emph{cyprès blanc}, \textgreek{λευκὴν κυπάρισσον}, comme déjà je l'ai dit\footnote{Ci-dessus, p. 162.} ; et peut-être est-il permis de conclure de l'épithète \textgreek{λευκή}, que le cyprès était d'argent ou argenté. En ce cas, nous aurions ici le cyprès de la lune, et, conséquemment, un symbole très-propre à caractériser la partie du monde sublunaire où règne Hadès. Je dois donc répéter que le poète italien cité plus haut\footnote{P. 158, 159.} fait du cyprès de la lune un arbre « brillant et blanc, comme l'argent, » et que, par une coïncidence bien plus digne d'attention, le cyprès pyramidal porte, dans la langue chinoise,\footnote{Ci-dessus, p. 160.} un nom qui signifie littéralement l'\emph{arbre blanc} ; car il se compose du signe générique \emph{mou} \textchinese{木} (\emph{arbre}), et du mot \emph{pe} \textchinese{白} (\emph{blanc}). Or les auteurs chinois, nous l'avons vu, expliquent cette dernière qualification en disant que le cyprès montre constamment l'Occident ou le couchant, et que la couleur blanche est affectée à l'Occident. Il devient, par conséquent, assez probable qu'en Chine, de même que chez les Grecs et les Romains, cette partie du monde sublunaire était considérée comme l'empire des morts, et le cyprès qualifié \emph{blanc}, parce qu'à la fois symbole de la lune et emblème funéraire, il appartient à la région des mânes.

Quelques renseignements particuliers, qui ne sont pas sans intérêt pour l'histoire du cyprès pyramidal, considéré comme emblème funéraire par les Grecs, nous ont été conservés dans divers écrits plus anciens que l'inscription citée. C'est ainsi qu'au rapport de Thucydide,\footnote{2., 34. Cf. les commentateurs grecs de cet historien.} les guerriers morts pour la défense de la patrie avaient droit à être ensevelis dans des cercueils de cyprès. Pythagore et ses disciples, je l'ai dit plus haut,\footnote{Ci-dessus, p. 303.} recommandaient, au contraire, qu'après leur mort on n'enfermât point dans des cercueils de ce bois leur dépouille mortelle. L'espèce de contradiction qui semble résulter de ces deux faits a beaucoup embarrassé les savants. Elle disparaît peut-être lorsqu'on remarque que la recommandation des pythagoriciens avait pour motif l'usage hiératique de consacrer aux dieux le cyprès pyramidal. Une telle humilité sied bien à des philosophes animés de sentiments religieux. Mais les idées qui, depuis une époque fort reculée, avaient cours chez les Grecs, comme chez les peuples civilisés de l'Asie occidentale, assimilaient les héros aux demi-dieux. Dès lors on conçoit facilement que le scrupule qui se découvre au fond de la pensée de Pythagore et de ses disciples, ne se soit pas élevé dans l'esprit du législateur chargé de régler les honneurs funèbres que la patrie reconnaissante doit rendre aux guerriers généreux, morts en combattant pour sa gloire et sa défense. Ce scrupule n'arrêta pas non plus le sentiment d'orgueil qui, en Grèce, portait les grands à ambitionner la vaine satisfaction d'être ensevelis dans des cercueils de cyprès. L'exemple des pythagoriciens trouva peu d'imitateurs ; l'usage dont je parle ne fut pas interrompu parmi les personnages d'un rang élevé ; il se perpétua même durant le bas-empire et le moyen âge.

Le continuateur incertain de Constantin Porphyrogénète\footnote{\emph{Scriptor. post Theophan.} p. 217 A ; éd. du Louvre, 1675.} et le moine George Cédrène\footnote{\emph{Historiar. compend.} p. 593, C, D ; édit. du Louvre, 1647. Cf. Leo, p. 475. Zonare, t. 2., p. 176. Lebeau, \emph{Hist. du Bas-Emp.} 72., 4. (t. 13., p. 335 ; édit. de Saint-Martin).} prennent soin de constater chacun qu'en 889 le corps de Michel 3, fils de Théophile et empereur d'Orient, fut rapporté à Constantinople\footnote{Par l'ordre de Léon le Philosophe, fils de Basile le Macédonien.} et placé dans un cercueil de bois de cyprès.\footnote{\textgreek{Ἐν γλωσσοκόμῳ κυπαρισσίνῳ}, selon l'un ; \textgreek{ἐν κυπαρισσίνῳ λάρνακι}, selon l'autre.} Mais ils ne nous disent pas si à ce genre de sépulture se rattachait une pensée pieuse, puisée dans les textes de la Bible où l'on trouve une mention de cet arbre. Il y a tout lieu de le croire ; car je dois à feu M. Buchon la communication de plusieurs observations que renferme le journal inédit de son voyage en Grèce, et qui établissent, d'une manière incontestable, qu'au moyen âge, dans cette contrée, il existait une alliance intime entre le symbole du cyprès pyramidal et le signe révéré des chrétiens. A Chalcis, dans l'église de Saint-Démétrius, et près de la porte d'entrée, le voyageur français remarqua, sur un fragment d'une ancienne pierre tumulaire, la croix du Christ, sculptée, entre deux cyprès, au milieu d'un cartouche. Sur la route de Delphes à Livadia, dans le monastère de Saint-Luc, un semblable cartouche s'offrit à ses yeux sur une autre pierre tumulaire ; celle-ci est encastrée dans le mur, au-dessus de la porte d'une cellule. À Néopatras,\footnote{Ou Néopatra.} en Thessalie, une troisième pierre tumulaire, scellée dans les murs d'une église, présenta à M. Buchon un cartouche où la croix, également plantée entre deux cyprès, est de plus accompagnée de deux fleurs de lis sculptées dans les angles supérieurs. À Piada, dans la presqu'île de Methana, les deux fleurs de lis sont remplacées par deux rosaces ou deux étoiles, sur un monument du même genre, encastré dans le mur intérieur d'une petite église ; mais ici, au-dessus de la croix, le sculpteur a ajouté deux oiseaux\footnote{Vraisemblablement deux colombes.} becquetant une grappe de raisin. Sur une cinquième pierre tumulaire, on voit, sculptée entre deux cyprès, une grande épée de forme cruciale ; au-dessus de la poignée de cette épée, une rosace est placée entre deux paons ; et, au-dessous des deux cyprès, sont couchés deux lions, la tête tournée, l'un à droite, l'autre à gauche. En plusieurs autres endroits notre voyageur rencontra des monuments funéraires analogues. Les uns et les autres lui parurent appartenir, soit à l'époque byzantine, soit à l'époque champenoise et bourguignone, soit à des temps plus modernes. Dans tous ces monuments, il remarqua, sans pouvoir en deviner la raison, que le sommet de chaque cyprès est constamment courbé en dedans. On dirait, à la vue des croquis qui accompagnent ses descriptions, que les deux arbres s'inclinent comme pour rendre hommage à la croix. Ici se décèle un des emprunts que l'art chrétien, à sa naissance, fut obligé de faire à la série des types symboliques inventés par l'antiquité païenne. Sur une stèle consacrée à la Vénus égyptienne, plusieurs siècles avant la venue du Christ, n'avons-nous pas vu deux cyprès pyramidaux, placés auprès d'une fleur symbolique de lotus,\footnote{Pl. 10., n° 2.} incliner leur sommet vers la déesse en signe d'adoration ? Nous verrons tout à l'heure, sur des lampes romaines,\footnote{Pl. 19., n°\textsuperscript{s} 1 et 2.} deux arbres de cette espèce s'incliner aussi devant une urne cinéraire. De nos jours, sur un grand nombre de pierres tumulaires qui recouvrent la dépouille mortelle de femmes musulmanes, on retrouve sculptés deux cyprès à sommet incliné. J'en ai vu moi-même plus d'un exemple dans la Bulgarie, à Constantinople et dans l'Anatolie. Ces tombes ne présentent aucun autre emblème ; et j'ignore si les deux cyprès qui les ornent en relief doivent être considérés comme l'imitation directe d'un usage observé sur des monuments antiques, ou seulement comme l'imitation d'une coutume pratiquée par les Grecs modernes, à l'exemple de leurs ancêtres.

Ce qui n'est pas douteux, c'est qu'en Grèce les générations actuelles continuent d'attacher au cyprès des idées funéraires qui rappellent certaines traditions anciennes. Nous en avons la preuve dans une touchante complainte, où ces idées sont mêlées à des images érotiques et au récit d'un suicide amoureux qui, en opposition directe avec la doctrine chrétienne, ne peut être considéré aussi que comme un emprunt fait aux poètes de l'antiquité. Cette complainte, recueillie sur les lieux, publiée et traduite par M. le comte de Marcellus,\footnote{\emph{Chants du peuple en Grèce}, tom. 2., p. 72-75.} est intitulée : \emph{L'amour au tombeau}. Un jeune marié revient de la chasse au moment même où l'on ensevelit son Eugénule, morte en son absence ; il tire un poignard et se l'enfonce dans le cœur. « Là où on ensevelit le jeune homme, dit le poète moderne, naquit un cyprès. Là où on ensevelit la jeune femme, naquit un roseau. --- Quand le puissant Borée souffle, le cyprès se courbe ; quand souffle le Zéphyr, le roseau se penche. --- Le roseau se penche, et vient baiser le cyprès. »

De nos jours, chez une nation puissante, qui a reçu de la Grèce, avec ses croyances religieuses, plus d'une coutume qui rattache les temps modernes aux temps anciens, on voit se pratiquer, parmi quelques familles considérables, la coutume d'ensevelir la dépouille mortelle des parents dans des cercueils faits avec du bois de cyprès apporté de la terre sainte. Aucune classe de la société ne se sert de bois de cette espèce pour des meubles, des lambris d'appartement, des ustensiles, ou pour tout autre usage domestique, non religieux. De plus, en Russie, on peint souvent l'image du Christ, de la Vierge ou des saints, sur des tablettes et des panneaux de bois de cyprès, destinés, soit à être exposés à la vénération publique dans une église, soit à prendre place dans un oratoire privé et même dans toute autre partie des maisons d'habitation. Les Russes tirent de la terre sainte, non-seulement leurs cercueils, mais aussi ces tablettes et ces panneaux, aimant à se rappeler pieusement que l'arche de Noé fut construite avec du cyprès, et que Salomon fit lambrisser avec du bois de cette espèce, coupé dans le Liban, certaines parties intérieures du temple de Jérusalem. Comme nous, ils recherchent et vénèrent des objets d'art créés par la religion chrétienne et sculptés, à l'époque byzantine, dans le cyprès ou dans le cèdre. De ce nombre sont des reliquaires travaillés à jour avec une grande délicatesse, et surtout des croix, non moins habilement travaillées, qui offrent, dans leur intérieur, de très-petites figures représentant quelques scènes de la passion de Notre-Seigneur Jésus-Christ.\footnote{Voyez ci-dessus (p. 77-79) un passage de l'Histoire des Aghouans, par Moïse Gaghagantouatsi, qui nous apprend qu'au 7\textsuperscript{e} siècle, l'évêque Israël avait fait ériger une croix de bois de cyprès, et l'avait ornée de sculptures représentant la passion de Notre-Seigneur Jésus-Christ.} Ces diverses circonstances semblent ne pas contribuer peu à entretenir l'idée de sainteté qu'en Russie on attache au bois de cyprès provenant de la terre sainte, bien que, sous cette désignation topographique, il faille peut-être entendre, dans le langage religieux des Russes, non-seulement Jérusalem, ses environs, le Liban et toute la Syrie, mais aussi le mont Athos et une partie de l'Anatolie et de l'Arménie.

\subsubsection{Étrurie}
\paragraph{}
Nous avons constaté l'identité qui existe entre le nom de la Vénus des Étrusques, \emph{Cypra} ou \emph{Cupra}, le nom de la Vénus des Grecs et des Romains, \emph{Cypris}, et le nom même du \emph{cyprès}, l'arbre consacré à Vénus en Étrurie, aussi bien que dans toutes les contrées où fut établi le culte de la déesse ; sur le sol étrusque, les monuments figurés nous ont montré le cyprès planté, comme un emblème sacré, au milieu de pompes religieuses que je me crois en droit de rapporter aux mystères de la Vénus asiatique introduits en Étrurie par les Phéniciens, ou par les Tyrrhéniens ; telles sont les deux principales considérations qui, sans autre examen, conduisent à supposer que les Étrusques avaient également reçu, avec cette importation, la coutume d'employer le cyprès dans les rites funéraires. Bientôt cette supposition se change en certitude à la vue de quelques monuments funèbres découverts en Étrurie, la plupart déjà publiés, mais restés sans les interprétations nécessaires pour mettre en évidence et faire comprendre le rôle important qu'y remplit le symbole du cyprès.

En tête de ces monuments se placent deux urnes cinéraires, de terre cuite, qui, en 1850, furent apportées d'Italie et vendues publiquement à Paris, où M. Muret eut l'occasion de les dessiner. Les numéros 4 et 5 de la planche 13 ci-jointe reproduisent ses dessins. La première de ces deux urnes a perdu son couvercle\footnote{Sur le couvercle, selon le dessin publié par Gori et cité plus bas, était couchée une figure de femme vêtue.} ; elle provient du mont Politien et se conservait anciennement au musée Bucelli. On en trouve dans le \emph{Musée étrusque} de Gori\footnote{Tom. 1., p. 37. ; tom. 2., p. 295 ; pl. 157., n° 5.} un dessin très- réduit, fort médiocre et mal gravé ; la description qui se rapporte à ce dessin qualifie arbitrairement de \emph{Déesse de la tempête, Dea tempestatis}, la tête de femme qui, sur le devant de l'urne, est placée entre deux cyprès pyramidaux.\footnote{Peut-être faut-il aussi reconnaître deux arbres de cette espèce sur le devant d'une autre urne cinéraire de terre cuite, également publiée par Gori (\emph{Mus. etrusc.} t. 3., p. 158., n° 4).} Cette tête, vue de face et caractérisée par deux grandes cornes de vache, représente, non une divinité que l'antiquité ne connut point, mais la Vénus asiatique, nommée, chez les Étrusques, \emph{Cypra} ou \emph{Cupra}, et \emph{Touran}. Comme la Vénus-Hâthôr des Égyptiens,\footnote{\emph{Recherches sur Vénus}, atlas, pl. 14 D, n° 5.} et comme la Vénus de Babylone et de Ninive,\footnote{Layard's \emph{Monum. of Nineveh}, pl. 9. --- Voy. aussi mes \emph{Recherches sur Vénus}, atlas, pl. 19 B, n° 6.} elle porte les cornes\footnote{On devine sans peine que Gori n'a pas compris la véritable signification de ce symbole.} du quadrupède femelle qui, dans l'Asie occidentale et surtout en Phénicie, servait à représenter symboliquement Mylitta ou Astarté.\footnote{Ci-dessus, p. 59, 60.} Placée, comme celle-ci, entre deux cyprès, images symboliques et vivantes de ses deux assesseurs, c'est-à-dire du soleil et de la lune, qui sont les deux portes du ciel, elle forme avec eux une triade céleste ; et c'est ici que se révèle soudainement à nos yeux l'origine asiatique d'un tel type. Il nous reporte aux monuments syriens, palmyréniens ou phéniciens, qui nous ont montré tantôt Vénus placée, sous la forme d'un cyprès, entre deux autres cyprès,\footnote{Pl. 5., n° 1.} ou entre le soleil et la lune personnifiés,\footnote{Pl. 3., n° 1.} ou entre un lion et un taureau,\footnote{Pl. 6., n°\textsuperscript{s} 1 et 2 ; cf. n° 3.} hiéroglyphes idéographiques de ces deux astres ; tantôt Vénus, sous une forme purement humaine, assise dans un temple flanqué du cyprès du soleil et du cyprès de la lune\footnote{Pl. 8., n° 4.} ; tantôt enfin Artémis, la Vénus lunaire, figurée par le symbole d'un cône posé entre le cyprès du soleil et celui de la lune.\footnote{Pl. 14., n°\textsuperscript{s} 1 et 2.} Comme Mylitta ou Astarté et comme Aphrodite ou Vénus, la Cypra ou la Touran des Étrusques était donc à la fois reine du ciel, reine de la terre et reine des enfers. Sur la plinthe de l'urne cinéraire où je reconnais cette triple Vénus qu'adorait l'Étrurie, on remarque une inscription en caractères étrusques, dont Gori ne s'est nullement occupé. Jusqu'à ce jour les habiles philologues que j'ai consultés ne peuvent en déterminer le sens.

Une inscription différente, qui ne paraît offrir que le prénom et le nom de la défunte, $\svgAAF$, se lit sur la plinthe d'une petite urne cinéraire, de terre cuite, presque semblable à celle que je viens de décrire. Cette urne est figurée dans l'ouvrage que Dempster publia sur l'Étrurie,\footnote{\emph{De Etrur. regal.} libr. 7. (Florence 1724 ; 2 vol. in-fol. fig. éd. de Th. Coke), t. 2., pl. 83., n° 5.} en 1724, treize ans, par conséquent, avant la publication du Musée étrusque de Gori. Elle appartenait autrefois à la famille des Buonarota. J'ignore ce qu'elle est devenue. Le couvercle est orné d'une figure de femme couchée et tenant, de la main droite, un serpent qui se replie sur lui-même et qui peut être considéré ici comme le symbole de la vie éternelle. Si l'on s'en rapporte au dessin publié par Dempster, les deux cornes de vache implantées sur la tête de la Vénus étrusque seraient moins épaisses et mieux accusées sur cette urne qu'elles ne le sont sur celle de l'ancien musée Bucelli\footnote{N° 4 de la planche 13., ci-jointe.} ; les deux cyprès seraient mieux modelés, et la figure de la déesse aurait une expression gracieuse ou riante, tandis que la Vénus de l'autre urne paraît avoir l'air sévère ou courroucé, qui caractérise les divinités infernales.

L'origine asiatique du sujet modelé sur la face antérieure de la seconde urne cinéraire étrusque, figurée ici,\footnote{N° 5 de la pl. 13.} est tout aussi évidente que celle du bas-relief qui orne chacune des deux urnes dont je viens de parler. Ce sujet, composé dans un esprit purement symbolique, nous offre, au-dessous d'un couvercle qui supporte une figure de femme couchée, une porte cintrée et décorée de deux guirlandes dans sa partie supérieure. Sous cette porte, on voit, plantés sur un soubassement, un grand cyprès pyramidal et, à côté, un objet difficile à déterminer,\footnote{Cet objet représente-t-il symboliquement Adonis, l'amant chéri de Vénus ? Je n'ose décider la question ; mais si, par hasard, elle rencontrait une réponse affirmative, il conviendrait de remarquer qu'à Yazili Kaïa (\emph{Descript. de l'Asie Mineure}, par M. Ch. Texier, 1\textsuperscript{re} part. t. 1., planche 78.), la Vénus orientale est suivie de son époux, représenté sous les traits d'un jeune homme dont la petite taille contraste avec celle de la déesse. C'est un exemple frappant du procédé qu'employait l'art asiatique pour exprimer, comme ici, la prééminence de l'épouse sur l'époux ; ailleurs, la suprématie d'un dieu ou d'un roi sur un autre. (Voy. ci-dessus, p. 57.)} peut-être une pomme de pin. À droite et à gauche de la porte sont rangés trois autres cyprès, de moins fortes proportions. Nous avons donc ici, comme autour du temple d'Aphrodite à Apollonia d'Illyrie,\footnote{Pl. 8. n° 4.} comme sur deux des cinq bas-reliefs mithriaques décrits dans le mémoire précédent,\footnote{Ci-dessus, p. 276-281.} sept cyprès, emblèmes des sept planètes. Mais sur notre urne, la Vénus-Cypra des Étrusques, au lieu de se montrer sous la forme humaine qu'elle revêt dans les bas-reliefs des deux autres urnes citées et dans le temple d'Apollonia, est figurée elle-même, comme chacune des six autres planètes, sous le symbole du cyprès. La différence que je signale nous reporte donc, plus directement même que ces derniers monuments, dans cette Asie occidentale qui fut le berceau du culte de Vénus et du langage symbolique. Elle rattache le bas-relief de notre urne cinéraire, comme le trône votif de la Déesse de Syrie,\footnote{Pl. 5., n°\textsuperscript{s} 1 et 2.} à la série des antiquités figurées qui, uniquement composées avec des éléments ou des signes idéographiques, propres à la langue des symboles, n'empruntent encore rien à l'anthropomorphisme. Cette urne, dont la provenance précise ne m'est pas connue, me paraît être restée inédite. Je ne sais en quelles mains elle se trouve actuellement.

Un monument funéraire de marbre, publié par Dempster,\footnote{\emph{De Etrur. regal.} t. 3., pl. 85., n° 1. --- Sur le marbre que j'indique ici, on voit sculptée une amphore semblable à celles où se renfermaient les cendres des morts. Gori (\emph{Mus. etrusc.} t. 2., p. 261) n'a pas omis de faire remarquer cet exemple de l'emploi de l'amphore comme symbole funéraire chez les Étrusques ; mais le renvoi qu'il fait ici à Dempster est fautif : il faut lire pl. 85., au lieu de pl. 80.} nous offre de nouveau, chez les Étrusques, un exemple de l'acception funèbre attribuée au cyprès. Mais le sculpteur, tout en traçant des arbres de cette espèce, n'a fait intervenir ici, ni sous une forme symbolique, ni sous une forme humaine, la déesse qui, en Occident comme en Orient, présidait à la destinée des âmes. Faut-il en conclure que ce marbre recouvrait la dépouille mortelle d'un homme et non d'une femme ? On serait presque tenté de répondre affirmativement, si l'on pouvait ne prendre en considération que les trois urnes cinéraires où nous venons de reconnaître l'image de la Vénus étrusque : l'Académie a dû remarquer que ces trois urnes renfermaient jadis les restes mortels de trois dames étrusques, qui avaient ainsi placé leur âme sous la protection de la déesse.

Nous retrouvons des cyprès sur un fragment de bas-relief étrusque, que l'on conserve, à Chiusi, dans le musée Casuccini. Il est figuré parmi les monuments inédits de Micali.\footnote{\emph{Monum. ined. a illustraz. della stor. degli antichi popoli italiani} (Florence, 1844, 1 vol. in-8°, avec un atlas in-f°), pages 145, 146, pl. 24., n° 2.} Ce savant n'en indique ni la provenance, ni la destination ; il ne désigne même pas sous le nom spécifique qui leur appartient les arbres qu'on y voit sculptés : \emph{Una fila di alberelli}, se borne-t-il à dire, \emph{indica chiaramente stadio campestre}. Or ces \emph{alberelli} sont de véritables cyprès pyramidaux, plantés au bord d'une voie que suit une course de chars attelés chacun de trois chevaux lancés à fond de train. Des représentations analogues, mais où l'on ne voit aucun cyprès, se font remarquer sur plusieurs autres sarcophages étrusques ou romains. C'est une image sensible de la rapidité du mouvement qui nous emporte dans le tourbillon de la vie. Les cyprès achèvent de montrer qu'ici-bas la mort du corps est le terme de notre existence terrestre ; mais que la vie éternelle est promise à l'âme dans la région céleste. Je n'hésite donc pas à penser, malgré le silence de Micali, que ce bas-relief ornait jadis un monument funéraire. Il mérite d'être rapproché des deux plats d'argent, ciselés et dorés, dont j'ai déjà parlé\footnote{Ci-dessus, p. 254-257.} et qui, en 1836, furent trouvés à Cæré, dans un grand tombeau souterrain. Si je les cite ici, c'est pour rappeler à l'Académie qu'ils nous ont offert des pompes religieuses de personnages qui suivent, les uns dans des chars, les autres à pied ou à cheval, une voie sacrée, bordée de cyprès et de myrtes, arbres consacrés à Vénus, et symboles à la fois de vie et de mort. Ces pompes religieuses, je les rapporte aux mystères qui, en Étrurie, comme chez les Assyriens et les Phéniciens, se célébraient sous les auspices de la déesse ; et, sans qu'il soit nécessaire d'entrer ici dans de plus amples détails à ce sujet,\footnote{On trouvera ces détails, comme appendice, dans mes \emph{Recherches sur le culte public et les mystères de Mithra}.} chacun comprendra facilement que, si l'initiation à ces mystères conduit à une nouvelle vie, à la vie spirituelle ou éternelle, les cyprès plantés au bord de la voie que suivent triomphalement les initiés sortis vainqueurs des épreuves et des combats mystiques dont se composait le rituel des sanctuaires, ces cyprès, dis-je, remplissent ici le même rôle que les cyprès figurés sur le bas-relief du musée Casuccini. Cette observation s'applique également, ce me semble, aux arbres de même espèce qui, sur les deux coupes d'argent doré, trouvées dans l'île de Chypre,\footnote{Ci-dessus, p. 64, p. 95.} et sur les coupes de bronze découvertes par M. Layard dans les ruines d'un palais assyrien,\footnote{Ci-dessus, p. 256.} sont plantés au milieu de scènes où l'on voit des initiés livrer ces mêmes combats mystiques dont j'entends parler. Dès lors chacun aussi comprendra, sans plus de peine, que les plats de Cæré étaient convenablement placés dans un tombeau où l'on avait réuni plusieurs autres ustensiles ornés de scènes qui, par des symboles ou par des luttes mystiques, marquent les différents grades obtenus par l'initiée dont ce tombeau renfermait la dépouille mortelle.\footnote{Je me réserve de montrer ailleurs que ce tombeau est réellement celui d'une femme qui, dans les mystères de la Vénus étrusque, remplissait les fonctions de grande prêtresse. Je montrerai, en même temps, que les personnages qui figurent dans les pompes religieuses, représentées sur les deux plats déposés auprès de la dépouille mortelle de cette grande prêtresse, appartiennent également au sexe féminin. Ces diverses remarques paraissent avoir échappé à l'attention des archéologues qui ont écrit sur les objets d'antiquité trouvés dans la riche sépulture dont je parle.}

Le rapprochement que je viens de faire reçoit un nouveau degré d'intérêt d'une remarque qui n'avait pas échappé à l'attention de Micali. « C'est pour la première fois, dit-il,\footnote{\emph{Loc. cit.}} que l'on rencontre sur un monument étrusque le costume dont est revêtu chacun des personnages qui, sur le fragment de bas-relief du musée Casuccini, conduisent des chars ; ce costume n'est ni grec ni étrusque. » Je ferai observer, à mon tour, que les personnages figurés sur les deux plats de Cæré portent aussi des vêtements qui n'appartiennent ni aux usages de la Grèce, ni à ceux de l'Étrurie. Comme le costume déclaré étranger à ces deux pays par Micali, ces vêtements me paraissent avoir une origine asiatique.

Je ne mets pas en doute qu'un examen attentif des monuments recueillis sur le sol de l'antique Étrurie ne fournît de l'acception funèbre attribuée au cyprès des exemples plus nombreux que je n'ai pu le faire d'après les recueils ou les ouvrages publiés jusqu'à ce jour. Il ne m'a pas été donné de visiter les riches collections d'antiquités étrusques que possèdent la Toscane et les États Romains. Mais mon assertion repose à la fois sur des inductions que chacun peut tirer de l'influence exercée en Étrurie par les colonies venues de l'Asie occidentale, et sur le témoignage de Passéri,\footnote{\emph{Dissert. 3., apud} Gor. \emph{Mus. etrusc.} vol. 3., p. 177, 178.} qui, parlant du seul musée Bucelli, nous apprend qu'on y voyait beaucoup d'urnes cinéraires étrusques, ornées de cyprès.

\subsubsection{Rome et empire romain}
\paragraph{}
Chez les Romains, les traditions écrites\footnote{Ajoutez aux écrivains qui déjà ont été cités et à ceux qui le seront plus loin : Horace, 2., \emph{Od.} 14., vers 21-24. Stace, 5., \emph{Silvar.} 1., 135, 136. Silius Italicus, 10., 535, 536. Claudien, \emph{De raptu Pros.} 2., 108. Cf. Gunther, \emph{De jure Man.} 1., 19. Kirchmann, \emph{De funer. roman.} 1., 14. Nicolas, \emph{De luctu Græc.} 5. et 13. Aringhi, \emph{Rom. subterran.} 6., 42. Sagittarius, \emph{De januis}, 31. Wouweren, \emph{ad} Petron. \emph{Satyr.} c. 120.} et les monuments figurés concourent à établir qu'aux funérailles des personnages d'un rang élevé, ou des guerriers morts pour la patrie,\footnote{Bellori, \emph{Colonna trajana}, pl. 42.} on employait le cyprès, soit dans les rites, soit dans la décoration des sépultures.\footnote{Il paraît, d'après deux vers de Lucain (3., \emph{De bell. civ.} v. 442, 443), que cet usage ne s'étendait pas aux classes inférieures de la société romaine :\\\hspace*{15mm}Et non plebeios luctus testata cupressus,\\\hspace*{15mm}Tunc primum posuere comas.\\\hspace*{5mm}On lit aussi dans Alciat (\emph{Emblem. libell.} emblem. 198) ces paroles non moins explicites : « Videmus enim in magnatum et nobilium sepulchralibus ac pollincturis spargi cupressi frondes, sed iste ritus apud omnes non æque celebratur. » Il ajoute (\emph{ib.}) que l'ache, comme la sauge, l'hysope, la menthe et le buis, tenait lieu de cyprès dans les cérémonies funéraires des plébéiens :\\\hspace*{15mm}Funesta est arbor, procerum monumenta, cupressus,\\\hspace*{15mm}Quale apium plebis promere fronde solet.} Cette double coutume remontait à une époque fort ancienne, nous ne pouvons guère en douter. Sans se prévaloir des traditions qui, pour flatter l'orgueil du peuple et de ses souverains, assignaient aux fondateurs de Rome une origine troyenne, on doit remarquer, d'une part, que, dans le Latium, le culte du cyprès se liait intimement à celui du dieu national Silvain, comme je l'ai déjà dit,\footnote{Ci-dessus, p. 206.} et au culte de Pluton\footnote{Pline, \emph{H. N.} 16., 33., 60.} et de Proserpine,\footnote{Servius, \emph{ad} Virgil. \emph{Æneid.} 3., 64, 681.} divinités toutes trois rangées parmi celles qui président aux enfers. J'ai dit aussi que, du temps de Pline le naturaliste, on conservait au Capitole\footnote{Aulu-Gelle (\emph{Noct. attic.} 5., 12.) dit que le temple de \emph{Véjove} était situé entre la citadelle et le Capitole.} une ancienne statue\footnote{Pline (16., 79., 3 ) parle de cette statue dans un chapitre où il a réuni des exemples fort remarquables de la longévité de certains arbres et de l'incorruptibilité de leur bois. Il cite des plantations, des statues, des portes de temple et des lambris de cyprès, dont la date remontait à bien des siècles. Cette circonstance me confirme dans la pensée, déjà exprimée plus haut, qu'une erreur de copiste a introduit dans le texte de Pline, un chiffre (661 de Rome) qui rapproche beaucoup trop, de l'époque à laquelle écrivait cet auteur, la date de la statue de bois de cyprès consacrée à Jupiter Véjove.} de Jupiter Véjove, faite avec du bois de cyprès. Or il résulte des témoignages réunis et comparés des auteurs latins,\footnote{Voy. Aulu-Gelle (\emph{loc. cit.}) et les notes de Gronovius et de Conrad sur le chapitre 12. du livre 5. de cet auteur (édit. de Leipzig, 1762).} que \emph{Vejovis} était assimilé à Jupiter Stygius, à Pluton, et, par conséquent, considéré comme une divinité infernale. D'autre part, Ovide fait une mention expresse de la coutume qu'avaient les Romains d'entourer d'arbres de cette espèce leurs autels funéraires.\footnote{Funeris ara mibi ferali cincta cupresso\\\hspace*{5mm}Convenit, et structis flamma parata rogis.\\\hspace*{10mm}Ovide, \emph{Trist.} 3., \emph{Eleg.} 13., vers 21 et 22.} Varron,\footnote{\emph{Apud} Serv. \emph{ad.} Virgil. \emph{Æneid.} lib. 6., v. 216.} commentant l'usage de brûler des branches de cyprès autour des bûchers des grands, en parle comme d'un usage ancien, et non comme d'une coutume récente.\footnote{Patin (\emph{Thesaur. Polen.} 2., col. 1079) ne met pas en doute que les Romains n'eussent reçu des Grecs l'usage d'employer le cyprès comme emblème funéraire. Il ne tient aucun compte des faits qui tendent à prouver que cet usage avait pu être apporté directement en Italie, par des colonies asiatiques, bien avant la fondation de Rome. Il est juste d'ajouter que plusieurs de ces faits résultent du témoignage de monuments figurés qui ont été découverts très-longtemps après Patin.} Pline,\footnote{\emph{H. N.} 16., 30., 60.} Servius\footnote{\emph{Ad} Virgil. \emph{Æneid.} 3., 64 ; 4., 506, 507 ; 6., 216.} et Festus\footnote{\emph{De signific. verbor.} sub voc. \textbf{Cupressi} ; ed. K. O. Müller. --- Ici, le savant éditeur loue Scaliger d'avoir considéré comme récents l'usage dont il s'agit et même l'importation du cyprès chez les Romains. Ainsi que Patin, il néglige les passages de Pline et de Tite-Live que j'ai rapportés plus haut, et qui semblent assigner une date ancienne à cette importation, ou du moins à la coutume de consacrer aux dieux nationaux des statues de bois de cyprès. Le témoignage cité de Servius et des deux mythographes anonymes du Vatican doit aussi, quant à l'usage de représenter, un cyprès à la main, Silvain, dieu infernal, s'appliquer à une époque romaine fort reculée ; car ce renseignement est la seule trace qui nous reste d'une semblable coutume. Ces remarques acquièrent peut-être quelque importance à la vue des monuments étrusques où le cyprès figure comme un arbre sacré, soit dans les mystères, soit dans les rites funèbres (voy. ci-dessus, p. 253-257, 317-324). N'oublions pas non plus les témoignages recueillis par Pline (16., 87., 1) pour montrer que le culte des arbres, dans le Latium, était antérieur de plusieurs siècles à la fondation de Rome.} rapportent que, chez les Romains aussi, on plaçait des branches de même espèce devant la porte ou dans le vestibule de la maison où se trouvait le corps d'une personne décédée. Le premier de ces écrivains explique très-brièvement cette coutume en se bornant à dire que le cyprès est consacré à Pluton : \emph{Diti sacra cupressus}. Le second nous apprend de plus que cet usage avait été institué, soit à l'imitation des Athéniens, qui, je l'ai déjà dit, couvraient de branches de cyprès les maisons mortuaires ; soit pour empêcher les prêtres de se souiller en entrant imprudemment dans une maison où reposait un mort ; soit enfin parce que le cyprès, une fois coupé, ne repousse plus. Festus s'en tient à cette dernière explication, ajoutant seulement ces paroles : « De même il n'y a plus rien à espérer d'un mort ; et c'est la raison pour laquelle on estime que le cyprès a été placé sous la protection de Pluton. » Enfin Claudien\footnote{\emph{De raptu Proserpinæ}, 2., 108.} atteste la coutume d'élever les tombeaux à l'ombre des cyprès.

S'il n'est parvenu jusqu'à nous aucune antiquité figurée qui puisse, chez les Romains, fournir, une exemple de l'acception funéraire du cyprès antérieurement au temps des empereurs, en revanche, la colonne trajane,\footnote{Bellori, \emph{loc. cit.}} un grand nombre de sarcophages, d'urnes cinéraires, de lampes de terre cuite, comme aussi quelques plats ou vases d'argent, nous attestent qu'à Rome, et dans tout l'empire romain, l'usage de placer cet arbre sur les monuments et les ustensiles funéraires était très-répandu sous le règne des Antonins. Il se perpétua jusqu'aux dernières années du paganisme. Mon intention n'est pas d'entreprendre ici l'énumération de tous les objets d'antiquité figurée qui confirment ces deux assertions. Ils sont, en général, parfaitement connus des archéologues. Mais j'appellerai l'attention de l'Académie sur un sarcophage, sur un joli petit vase d'argent, inédit, sur deux plats de même métal, que je crois également inédits, et sur deux lampes de terre cuite, qui me paraissent mériter une mention particulière, bien que déjà elles aient été figurées dans plusieurs recueils.

Le sarcophage provient de la villa Borghèse et se conserve au musée du Louvre. Publié plusieurs fois dans le cours de trois siècles, il a été commenté, en dernier lieu, par M. Panofka, ainsi que je l'ai dit plus haut.\footnote{Ci-dessus, p. 211, 212.} Je ne reparlerais pas ici de ce monument si mes savants devanciers ne m'avaient laissé le soin de faire remarquer que le cyprès y figure tout à la fois comme symbole de vie, comme emblème funéraire et comme indication d'un lieu consacré à une déesse qui préside à la vie et à la mort. En effet, sur un des deux bas-reliefs qui ornent le devant de ce sarcophage, on voit\footnote{Bouillon, \emph{Musée des antiq.} t. 3., bas-reliefs, pl. 17. Clarac, \emph{Mus. de sculpt. antiq. et mod.} n° 315, pl. 113. et 114.} Diane occupée à se baigner, avec quelques nymphes, dans une grotte qui renferme une source.\footnote{Voy. Pausanias, 9., 2. ; 3.} Quatre cyprès, deux à droite et deux à gauche, sont plantés au bas des rochers qui bordent l'ouverture de cette grotte ; un grand pin s'élève au-dessus du cintre. Le bas-relief sculpté auprès de celui-là nous montre Actéon et ses serviteurs se préparant à entrer en chasse avec sept chiens. Sur une des faces latérales du sarcophage, un troisième bas-relief,\footnote{Bouillon, \emph{loc. cit.} Clarac, ouvrage cité, n° 315, pl. 113.} que précédemment j'ai eu l'occasion de citer,\footnote{Ci-dessus, p. 211, 212.} représente Actéon attaqué par ses propres chiens au moment où commence sa métamorphose en cerf. Cette scène se passe dans une grotte ornée, à gauche, de deux cyprès, et, à droite, d'un grand pin. Enfin, sur la face latérale opposée,\footnote{Bouillon, \emph{loc. cit.} Clarac, ouvrage cité, n° 315, pl. 113. et 114.} Actéon, étendu mort sur le sol, mais non métamorphosé en cerf, est entouré de femmes qui semblent se disposer à lui rendre les devoirs funèbres. Ici nous retrouvons de nouveau la grotte, les deux cyprès et le pin du bas-relief précédent. Cette description, toute succincte qu'elle est,\footnote{Je m'abstiens de placer ici les nombreuses remarques que me fourniraient la composition de ce beau sarcophage, considéré sous des rapports psychologiques, et le choix des symboles ou accessoires employés dans les sculptures dont il est orné. La grotte, la source, les griffons, le nombre des chiens, peuvent surtout donner lieu à des observations d'un grand intérêt.} suffit sans doute pour faire comprendre que, placé auprès de Diane, personnification de la lune, porte du ciel par laquelle les âmes descendent sur la terre et remontent dans les régions célestes, le cyprès est ici le symbole de la vie, quant au corps, et le symbole de la mort, quant à l'âme.\footnote{Voy. \emph{Mém. de l'Acad. des inscr.} nouv. série, t. 15., 2\textsuperscript{e} partie, p. 63, 64 ; et ci-dessus, p. 301, 302.} Placé auprès d'Actéon, qui vient de perdre la vie pour n'avoir pas su résister à une coupable curiosité, ou, en d'autres termes, pour n'avoir pas su dompter ses passions, le cyprès devient évidemment un emblème funéraire.

Le petit vase d'argent dont je me suis proposé d'entretenir la Compagnie, après le sarcophage du Louvre, a la forme d'un \emph{crater} ou d'un \emph{poterium} ; ses dimensions n'excèdent pas six centimètres et deux millimètres de hauteur, sur dix centimètres de diamètre. On le conserve à la Bibliothèque impériale. Selon les uns, il aurait été retiré du Rhône, au même lieu et vers la même époque où fut découvert le disque d'argent que possède aussi cet établissement, et qui longtemps a été désigné sous la dénomination erronée de \emph{bouclier de Scipion}. Selon d'autres, notre petit vase aurait été trouvé à Aire ou Ayries,\footnote{Département des Landes.} sur les bords de l'Adour.\footnote{On ajoute qu'au moment de la découverte, ce petit vase contenait un certain nombre de médailles d'argent et une fibule de même métal, attachée à une longue chaînette.} Par son travail et par son style, il peut remonter au 2\textsuperscript{e} siècle de l'ère chrétienne. Il a été coulé ; le sujet qu'il représente a, en général, fort peu de relief ; mais quelques têtes d'animaux, qui sont rapportées, forment une saillie considérable ; elles ont été ajustées au moyen de soudures, comme on l'a pareillement pratiqué pour les parties les plus saillantes des objets qui ornent plusieurs des vases d'argent trouvés à Berthouville près de Bernay. Je joins ici, sous les numéros 6 et 6 \emph{a}, de la planche 6, deux dessins qui reproduisent avec exactitude la forme du vase et le développement du sujet curieux qui en décore toute la circonférence extérieure, sur une hauteur de quatre centimètres. On y remarque un autel allumé, placé, sur le premier plan, entre deux cyprès pyramidaux. En arrière de celui qui est planté à droite de l'autel, s'élève une colonne surmontée d'une urne cinéraire ; à défaut des deux cyprès, elle suffirait pour marquer que notre petit vase avait dû servir dans les rites funéraires. Tout près de cette colonne, on voit une lionne dévorant un sanglier, et, plus loin, un taureau dévoré par un lion.\footnote{Voyez, quant à l'acception funéraire d'un pareil groupe, mon \emph{Mémoire sur une urne cinéraire du musée de la ville de Rouen} (\emph{Mémoires de l'Académie des inscriptions}, nouv. série, t. 15., 2\textsuperscript{e} partie, p. 63-127 ; pl. 1., 2., 3.).} Un laurier, dont le tronc a atteint de fortes dimensions, est placé, sur le premier plan, entre ces deux groupes d'animaux. Le second groupe est mutilé : à l'exception de la tête et du cou, le corps du lion manque, par suite d'un déchirement fâcheux qu'à éprouvé le vase sur une longueur et une hauteur moyennes de près de six centimètres. Sur les deux bords de cette fracture, subsistent encore, à droite, la portion droite d'une tête de taureau ou de génisse, vue de face ; à gauche, une jambe allongée de quadrupède, la croupe et la queue d'un animal herbivore, un petit bucrane, et un troisième cyprès pyramidal, dont le tronc a disparu. À droite de cet arbre ainsi mutilé, un pin, resté intact, étend ses branches chargées de feuilles et de cônes. Entre ce pin et le cyprès planté à gauche de l'autel, on remarque un troisième groupe d'animaux : un énorme lion dévorant un quadrupède herbivore, à pied fendu, qui fléchit sur ses jambes de devant, et qui probablement est un taureau. La tête et le cou de ce quadrupède manquent ; ils se sont dessoudés et perdus dans le choc violent que le vase doit avoir éprouvé.

Quoique endommagé et incomplet, ce petit monument réunit à mes yeux plus d'un genre d'intérêt. Laissant de côté les considérations d'art et de rareté, pour m'occuper seulement du sujet dont il est orné, je remarque que nous avons ici, sur un vase de travail romain, un exemple, unique jusqu'à ce jour, d'une composition où se trouvent tout à la fois trois cyprès, un pin, un laurier, un autel allumé, une colonne surmontée d'une urne cinéraire et trois groupes d'animaux herbivores ou frugivores, dévorés par des animaux carnassiers. Une telle composition prête, ce me semble, un puissant appui à quelques observations, à quelques conjectures que renferment mes deux mémoires sur le culte du cyprès et une dissertation précédente, dont le sujet principal était une urne cinéraire romaine, ornée du groupe d'un lion dévorant un taureau.\footnote{\emph{Mém. de l'Acad. des inscript.} loc. cit.} Choisis parmi les arbres verts qui, en Occident comme en Orient, furent consacrés aux divinités génératrices, le cyprès, le pin et le laurier ont chacun ici une acception funéraire, précisément parce qu'ils sont ailleurs des symboles ou des emblèmes de vie.\footnote{M. Avellino (mém. cité, p. 28), je le remarque à regret, ne tient pas compte de cette double acception, et oublie même les monuments qui attestent que le pin et le laurier furent, aussi bien que le cyprès, employés comme emblèmes funéraires.} Ils accompagnent des groupes d'animaux qui, à leur tour, reproduisent hiéroglyphiquement l'idée de la mort et l'idée d'une nouvelle vie.\footnote{\emph{Mém. de l'Acad. des inscript.} loc. cit.} D'autres groupes semblables ou analogues s'étaient déjà montrés à nos yeux, soit isolés, soit réunis, sur l'urne citée, sur un tombeau grec découvert à Myra, sur des scarabées étrusques, sur un miroir mystique trouvé près de Viterbe et sur plusieurs sarcophages romains. Nous avions été conduits à reconnaître\footnote{\emph{Ibid.}} que leur composition et la double acception de mort et de vie, ou de génération et de vie spirituelle, qui leur fut attribuée chez les anciens peuples civilisés d'Europe, étaient des emprunts faits à l'art et aux croyances religieuses de l'Asie occidentale. L'usage primitif d'employer le cyprès, le pin et le laurier avec les deux mêmes acceptions est aussi revendiqué par cette dernière contrée ; et, sur notre petit vase funéraire romain, une particularité plus importante qu'elle ne semble l'être au premier aperçu, ajoute un fait du même genre à la liste de ceux que, pour ma part, j'ai plusieurs fois déjà recommandés à l'attention de l'Académie. Je veux parler de la forme particulière et insolite de l'autel placé ici entre deux cyprès. Cette forme, dont je ne connais pas d'autre exemple dans la série nombreuse des antiquités figurées romaines, est exactement celle qu'affectent d'autres autels qui se voient au milieu de scènes psychologiques, sur des cylindres assyriens ou phéniciens, dont j'ai publié les dessins.\footnote{\emph{Rech. sur Mithra}, atlas, pl. 28., n° 12 ; pl. 42., n° 13 ; pl. 49., n° 2.} Ces autels, comme celui-ci, sont composés de deux cônes, dont l'un, renversé, est superposé à l'autre, de telle manière qu'ils s'unissent tous deux par leur sommet.\footnote{D'autres fois l'autel est formé d'un seul cône, ou affecte une forme conique plus ou moins modifiée. (Voy. \emph{Rech. sur Mithra}, atlas, pl. 17., n° 3 ; pl. 32., n° 11 ; pl. 36., n° 11 ; pl. 50., n° 3 ; pl. 54 \emph{A}, n° 9 ; pl. 54 \emph{B}, n° 2.)} Le cône, symbole de deux divinités génératrices, Vénus et Artémis ou Diane, symbole du feu créateur, symbole phallique, et, en conséquence, symbole de vie et de génération, devenait, par ces raisons mêmes, susceptible d'une acception funéraire. C'est pourquoi il entre dans la composition des autels placés, comme ici, au milieu d'une scène mystique et funèbre, et, comme sur les cylindres, au milieu de scènes qui de même ont un double sens mystique et funèbre, puisqu'elles représentent les âmes passant de la vie corporelle ou matérielle, qui est leur mort et leur tombeau, à la vie spirituelle, qui les rend à leur destinée primitive. C'est aussi pourquoi la forme conique est employée dans un grand nombre de sépultures antiques, soit en Orient, soit en Occident. C'est enfin pourquoi encore la pomme de pin, dont la forme est celle d'un cône, a souvent une acception et une destination funéraires, en même temps qu'elle est un emblème du principe igné et un symbole de vie.\footnote{Sur les monuments assyriens, les deux principes nécessaires à la vie, l'eau et le feu ou ce que nous appelons le calorique, sont représentés tantôt par des figures qui portent, d'une main, un vase à contenir un liquide, et, de l'autre, une pomme de pin ; tantôt par des ornements où l'on remarque des méandres et des pommes de pin.}

Les deux plats d'argent inédits appartiennent à la Bibliothèque impériale, comme le petit vase dont je viens de parler. Le premier fut trouvé sur les bords du Rhin, il y a quelques années ; il est de forme ronde, et plus profond que le second ; ses bords sont renversés ; son diamètre est de trente-cinq centimètres et quatre millimètres. On peut le citer comme un exemple des ouvrages de plaqué que nous a légués l'antiquité romaine ; car il a été coulé en bronze et recouvert ensuite d'une feuille d'argent, tant extérieurement qu'intérieurement. Les dessins que j'en donne ici\footnote{Pl. 16., n°\textsuperscript{s} 1-6.} reproduisent le profil de l'original,\footnote{\emph{Ibid.} n° 6.} le développement du sujet représenté en relief sur ses bords,\footnote{\emph{Ibid.} n°\textsuperscript{s} 1-4.} et le médaillon, de plus haut-relief, qui en occupe le centre.\footnote{\emph{Ibid.} n° 5.} Rapprochée du second plat,\footnote{Pl. 17., n°\textsuperscript{s} 1-3 ; pl. 18., n° 1.} qui porte gravée une formule de consécration à Mercure, dieu psychopompe, la composition de ce premier plat annonce, à défaut d'une dédicace, qu'il fut aussi consacré sur les autels d'une divinité chargée de présider à la vie, aux mystères ou aux initiations, et à la mort. Ses bords sont ornés de divers objets, qui tous ont un sens à la fois mystique et funèbre : ce sont des masques scéniques,\footnote{Personne n'ignore que des masques scéniques se plaçaient, non-seulement sur les tombeaux, mais aussi sur les lampes funéraires. Ce qui est beaucoup moins connu, c'est que des masques d'un caractère particulier, sur lequel je me réserve de m'expliquer ailleurs, se voient sur des cylindres asiatiques, où ils reçoivent une double acception, mystique et funèbre, à raison du sens psychologique que présentent les scènes gravées sur ces petits monuments, si intéressants à étudier pour l'histoire des mystères. L'atlas de mes \emph{Recherches sur Mithra} fournit plusieurs exemples des masques dont j'entends parler ; voy. pl. 18., n° 8 ; pl. 27., n° 8 ; pl. 53., n° 1. Cf. le n° 8 de la pl. 50. et le n° 1 de la pl. 51.} des animaux carnassiers, tels que des lions, des lionnes et des ours, qui attaquent ou poursuivent des animaux herbivores ou frugivores, tels que des antilopes ou des bouquetins, des biches, des sangliers et des truies ; ce sont des édicules funèbres et des autels ; ce sont enfin des cyprès, des arbres morts ou dépouillés de leurs feuilles, et des plantes aquatiques. Le médaillon central représente un Romain combattant à cheval un énorme sanglier. C'est sans doute le personnage même qui avait fait la consécration du plat. La déesse de la Victoire s'avance dans les airs pour poser une couronne sur la tête de ce cavalier ; elle nous montre, par cette insigne récompense, que, grâce au dieu psychopompe et sauveur dont il est un des adorateurs fervents, il a pu sortir vainqueur, non d'un simple combat avec une bête féroce, mais d'une lutte qui, si je ne me trompe, fait une allusion mystique au combat perpétuel qu'ici-bas nous devons livrer à nos passions.

Une allusion semblable se découvre dans le médaillon en relief qui orne le centre du second plat\footnote{Pl. 17., n°\textsuperscript{s} 1-3 ; pl. 18., n° 1.} que j'ai à décrire. Ici le cavalier combat, au lieu d'un sanglier, une lionne et une louve, qui l'attaquent simultanément ; et, bien que la déesse de la Victoire n'intervienne pas dans cette scène, le double sens, mystique et funéraire, de la lutte n'est pas plus douteux ici que sur le premier plat ; car l'attaque simultanée d'une lionne et d'une louve n'est assurément pas la représentation d'un fait réel. De plus, des masques scéniques, des édicules et des autels funèbres, des cyprès, des quadrupèdes carnassiers poursuivant des quadrupèdes herbivores sont figurés, en relief, sur les bords intérieurs de ce second plat, tout comme sur le premier,\footnote{Pl. 16., n°\textsuperscript{s} 1-4.} et lui donnent, de même qu'à celui-ci, un caractère particulier, que justifient pleinement les attributions du dieu psychopompe à qui nous allons voir qu'on l'avait consacré. L'un et l'autre de ces deux plats avaient donc été composés dans un même esprit, dans un même but et avec les mêmes éléments idéographiques. J'ajoute que, par leur forme, leurs dimensions, leur travail et leur style, non moins que par leurs sujets, ils appartiennent tous deux à une même époque de l'art, à une période comprise entre le 2\textsuperscript{e} et le 3\textsuperscript{e} siècle de l'ère chrétienne. Cet intervalle de temps, on le sait, fut marqué par un grand nombre de monuments où les artistes romains se plurent à reproduire des symboles, des figures, et des types empruntés à l'Orient. Comme le petit vase que j'ai décrit ci-dessus, le plat qui est sous nos yeux\footnote{Pl. 17., n°\textsuperscript{s} 1-3 ; pl. 18., n° 1.} a droit, plus encore que le premier, à être rangé dans cette catégorie : on y voit des lions poursuivant des taureaux ; sur le petit vase, ce sont des lions occupés à dévorer des quadrupèdes de cette même espèce. Je demande à l'Académie la permission de lui répéter que, dans une de mes dissertations précédentes,\footnote{\emph{Mémoires de l'Académie des inscriptions}, loc. cit.} je n'ai pas eu de peine à établir que le groupe d'un lion attaquant ou dévorant un taureau fut emprunté à l'Asie occidentale par les Grecs et les Romains. Ces peuples, n'ignorant point l'acception mystique et funéraire dont il est susceptible, le placèrent sur leurs tombeaux, où il exprime, par rapport à l'âme et au ciel, l'idée d'une nouvelle vie, en même temps que, sculpté à Persépolis, comme emblème de l'équinoxe vernal, il exprime là aussi, par rapport à la terre, l'idée d'une nouvelle vie. Les beaux bas-reliefs découverts par M. Layard, dans les ruines des palais de Nimroud, ont montré, depuis le mémoire auquel je me réfère, que je ne m'étais pas trompé en assignant à la conception idéographique de ce groupe une origine assyrienne ou plutôt chaldéenne.

Le plat qui me suggère ces diverses remarques provient du trésor trouvé, en 1830, à Berthouville, près de Bernay. Peu de temps après, il fut décrit par M. Charles Lenormant\footnote{\emph{Bulletino dell. Instit. arch.} (Lettre à M. Th. Panofka), n° 5., mai 1830, p. 109. --- La même année, 1830, M. Raoul Rochette publia dans le \emph{Journal des savants} (Juillet, p. 417-430 ; août, p. 459-473) les deux premières parties d'une intéressante notice sur les vases de Bernay. Il est à regretter que la troisième n'ait pas paru. C'est là, sans doute, que le savant archéologue aurait parlé du plat qui nous occupe.} et par M. Auguste le Prévost\footnote{\emph{Mém. sur la collection de vases antiques trouvés, en mars 1830, à Berthouville (arrondissement de Bernay)}, p. 10, 11 (Caen, 1832 ; gr. in-8°, fig.). --- Cette belle dissertation est tirée du 6\textsuperscript{e} volume des Mémoires de la Société des antiquaires de Normandie.} ; mais ces deux savants n'en donnèrent pas les dessins et ne paraissent point avoir observé qu'au nombre des objets représentés sur les bords de ce plat, on voit des cyprès. De tels arbres ne contribuent pas peu à assigner à ce monument le double caractère que je lui assigne et qu'ils ne lui ont pas reconnu.\footnote{M. Lenormant (\emph{loc. cit.}) se borne à dire : « Un rebord orné de chasses et de génies, dans le goût de Pompeï. Au centre, un médaillon représenté par un homme à cheval, attaqué par un loup et par un lion. » --- M. le Prévost s'exprime ainsi (Mém. cité, p. 17, 18) : « Le bord est relevé en bosse et orné de lions, de masques et autres attributs scéniques d'une fort belle exécution ; au centre, on voit un cavalier attaqué par un lion et un loup ... Malgré la nature des ornements de la bordure, nous sommes porté à voir dans les figures du centre le souvenir d'un péril auquel aurait échappé le donateur, plutôt que le combat d'un gladiateur contre les deux animaux ... » Toutefois la pénétration dont est doué notre savant confrère ne lui a pas fait défaut, car il ajoute : « Dans l'une et l'autre hypothèse, l'attaque simultanée de ces deux animaux est une circonstance peu vraisemblable. Peut-être y devrait-on chercher un sens allégorique ... » Ces deux dernières phrases nous donnent à comprendre que si, avant de les écrire, M. le Prévost avait eu sous les yeux le plat d'argent, fourré de bronze, que j'ai décrit plus haut, il se serait arrêté davantage à l'idée d'attribuer au médaillon du plat de Berthouville un sens allégorique, ou plutôt un sens mystique.} Il a trente-cinq centimètres et demi de diamètre, ce qui, à très-peu de chose près, est aussi la dimension du premier plat. Coulé en argent,\footnote{Il pèse un peu plus d'un kilogramme.} il n'est pas fourré de bronze comme celui-ci. Son médaillon central présente un plus fort relief que les objets figurés sur ses bords. Le sujet de ce médaillon est entouré circulairement d'un ornement particulier et de l'inscription suivante, tracée en caractères d'une forte proportion et d'une assez belle forme : DEO • MERCVRIO • KANETONNESSI • (\emph{sic}) C • PROPERT • SECVNDVS • V • S • L • M • Par cette dédicace, nous apprenons que le plat d'argent où elle est gravée était un don offert par \emph{Caïus Propertius Secundus} à Mercure \emph{Kanetonnessis},\footnote{Ce surnom se retrouve, sous la forme \emph{Canetus}, dans les formules de consécration gravées sur plusieurs autres objets d'antiquité découverts à Berthouville en même temps que notre plat d'argent ; on y lit : MERCVRIO. CANETO. Voy. le mémoire cité de M. le Prévost.} pour l'accomplissement d'un vœu (\emph{votum solvit lubens merito}). Nous apprenons aussi que ce dieu avait un temple en un lieu appelé \emph{Kaneto},\footnote{\emph{Caneto}, ou \emph{Canet}.} dont la position topographique nous reste inconnue. Mercure, divinité psychopompe, présidait nécessairement à la destinée des vivants et des morts, aux mystères et aux rites funèbres. Dès lors, ne trouvons-nous pas dans l'offrande de Caïus Propertius Secundus, personnage que l'histoire ne nous fait pas connaître, la preuve que le donateur s'était mis, durant sa vie, sous la protection particulière de Mercure ? et, cette protection ayant été aussi efficace qu'il l'avait désiré, n'avait-il pas voulu, avant de mourir, accomplir un vœu dicté par le soin du salut de son âme ?

Un manche d'argent que je crois inédit, et qui autrefois, sans doute, était adapté à une patère ou à quelque autre vase employé pour les libations, dans les rites funèbres, confirme, ce me semble, l'interprétation que je donne du médaillon placé au centre de chacun de nos deux plats funéraires. On voit, en effet, sur la face supérieure de ce manche,\footnote{Planche 18., n° 2.} un édicule funèbre, parfaitement caractérisé par deux cyprès pyramidaux, plantés, l'un à sa droite, l'autre à sa gauche. Cet édicule, qui renferme une dépouille mortelle, nous rappelle les petits édifices funèbres et les cyprès figurés sur un des bas-reliefs de la colonne trajane\footnote{Bellori, \emph{loc. cit.}} et sur les bords des deux plats d'argent de la Bibliothèque impériale. Il est placé ici entre un capricorne et l'image de Mercure. Le capricorne est debout devant une porte cintrée, qui, sans aucun doute, est la porte par laquelle nous savons que les âmes, selon les doctrines propres aux mystères, rentrent dans la région céleste, après avoir passé par la planète Saturne, dont le domicile est dans le signe du Capricorne.\footnote{Voy. \emph{Mémoires de l'Acad. des inscript.} nouv. série, t. 14., 2\textsuperscript{e} partie, p. 99-134.} Mercure, armé du caducée, qui lui sert à conduire les âmes, est placé dans la partie supérieure de ce petit tableau ; il lève la main droite vers le ciel, et, de l'index, il montre ainsi aux âmes leur demeure primitive, le séjour où les attend un bonheur éternel, qu'elles ne sauraient trouver sur la terre. Ce geste si expressif semble nous révéler le rôle que remplissait Hermès Criophore dans les mystères des Grandes-Déesses,\footnote{Ci-dessus, p. 26.} et nous rappelle le célèbre passage où l'empereur Julien\footnote{\emph{Cæsares}, Opp., p. 336 ; ed. Spanhem.} place ces paroles dans la bouche de Mercure : « Je t'ai fait connaître notre père Mithra\footnote{\textgreek{Σοὶ δὲ, πρὸς ἐμὲ λέγων ὁ Ἑρμῆς, δέδωκα τὸν πατέρα Μίθραν ἐπιγνῶναι} ...} ; observe fidèlement ses commandements, et conduis-toi de manière à trouver en lui, pendant ta vie, un port et un refuge, et à emporter, lorsqu'il te faudra quitter cette terre, la douce espérance d'avoir dans ce dieu un guide propice. » Ces paroles servent de commentaire à un bas-relief mithriaque romain,\footnote{\emph{Rech. sur Mithra}, atlas, pl. 95., n° 2.} où Mercure précède le char sur lequel Mithra conduit au ciel l'âme d'un initié parvenu aux trois degrés de pureté\footnote{Ces trois degrés sont : \emph{la pureté de pensée, la pureté de parole} et \emph{la pureté d'action}. (Voy. \emph{Zend-Avesta}, t. 1., 2\textsuperscript{e} partie, \emph{Vendidad Sadé}, p. 104, 164 et ailleurs ; t. 2., p. 34.)} qui lui ont mérité cette insigne récompense.\footnote{Le nombre des monuments romains où Mercure remplit les fonctions de conducteur des âmes (psychopompe) est considérable. Il faut y ajouter plusieurs lampes funéraires sur lesquelles on voit tantôt le buste de ce dieu (Passéri, \emph{Lucern. fictil.}, t. 3., tab. 97., fig. 1 et 4), tantôt seulement son symbole, un bélier (\emph{id. ibid.} et 4 ; fig. 2 et 3).} Si les médailles de Mostène de Lydie\footnote{Ci-dessus, p. 108-111 ; pl. 3., n°\textsuperscript{s} 3 et 4 ; pl. 4., n°\textsuperscript{s} 2 et 3.} nous ont montré Mercure servant d'introducteur à des amazones dans le sanctuaire où se pratiquait l'initiation aux mystères de Vénus ; si là ce sanctuaire est caractérisé par un autel allumé et par un cyprès pyramidal, image vivante et symbolique de la déesse, ici, sur notre manche de patère, nous trouvons deux colombes et deux têtes de cygne destinées à rappeler l'intime alliance qui existe entre Vénus, déesse des mystères, et Mercure, dieu psychopompe. De plus, l'idée de la vie future, déjà exprimée par la porte du Capricorne, l'est une seconde fois par deux corbeilles remplies de fruits et placées chacune tout auprès d'une colombe, symbole de Vénus, comme de Mithra. Chaque fruit ne porte-t-il pas en soi le germe d'un nouvel être ou d'une nouvelle vie ?

J'avais donc quelque raison de dire que le sujet figuré sur ce petit manche de patère confirme le sens mystique ou psychologique que je prête au combat représenté sur le médaillon central des deux plats funéraires décrits plus haut, et dont l'un est nominativement consacré à Mercure. La même interprétation s'applique tout naturellement à un grand nombre d'autres vases funéraires et de sarcophages où l'on voit, en relief, des combats de guerriers, de héros, ou des luttes que des gladiateurs soutiennent corps à corps, soit avec d'autres gladiateurs, soit avec des animaux féroces. À cette série de monuments appartiennent deux fragments d'un vase de terre cuite, rouge et de travail romain, qui proviennent des fouilles entreprises, il y a quelques années, à Rheinzabern, dans la Bavière rhénane. Ces deux fragments sont figurés dans un ouvrage posthume de M. Schweighæuser fils, publié par les soins de M. Matter, mais peu répandu.\footnote{Il est intitulé : \emph{Antiquités de Rheinzabern} ; un volume in-4° (avec fig.), sans indication de lieu ni date d'impression.} Le premier fragment\footnote{Ouvrage cité, pl. 8., n° 5.} représente un cyprès pyramidal planté entre deux médaillons qui renferment chacun une figure de guerrier combattant. Le second\footnote{Ouvrage cité, pl. 8., n° 6.} nous montre, avec un autre guerrier combattant, un lièvre, un cheval et un lion, tous symboles funéraires. A l'occasion de ces deux fragments, M. Schweighæuser s'exprime ainsi\footnote{Ouvrage cité, p. 3.} : « Parmi les figures du bas, un cyprès, arbre aujourd'hui inconnu à nos contrées. Mes fragments représentent plusieurs plantes et plusieurs animaux sur lesquels on peut faire la même observation. »

Un autre fragment, trouvé dans la même localité, offre un second exemple de la présence du cyprès pyramidal sur une poterie romaine.\footnote{\emph{Ibid.} p. 3., pl. 9.} Mais ici nous ne voyons aucun combat mystique de guerriers, ni d'animaux. L'opposition des idées de vie, de mort et d'éternité est exprimée par un cyprès planté au milieu de plusieurs oiseaux aquatiques, symboles du principe humide, et, par conséquent, symboles de vie, puisque les premiers êtres vivants sont nés dans les eaux, et que l'eau est, avec l'air et le feu ou la chaleur, un des trois principes indispensables à la manifestation et à la conservation de la vie. C'est à ce même titre que, sur des urnes cinéraires et des sarcophages d'origine romaine, sont sculptés, non-seulement des oiseaux aquatiques, mais des plantes aquatiques et des arbres verts de diverses espèces.

Parmi les monuments funéraires romains où nous retrouvons le cyprès pyramidal, je n'oublierai pas de citer une belle urne cinéraire de marbre, publiée par Boissard\footnote{\emph{Roman. urb. topogr. et antiquit.} pars 6., tab. 62 (Francof. 1597-1602).} et reproduite par Montfaucon.\footnote{\emph{L'Antiquité expliquée}, t. 5., 1\textsuperscript{re} partie, p. 72, pl. 43.} Elle est ornée d'un bas-relief qui représente, entre deux arbres de cette espèce, une dame romaine occupée à faire des libations sur un autel allumé. L'inscription latine gravée au bas de ce sujet nous apprend que cette dame s'appelait \emph{Cypris}, et qu'elle avait consacré le monument à son fils et à ses petits-fils, victimes, ainsi que sa belle-fille, des ravages d'une maladie épidémique.\footnote{Cette inscription est ainsi conçue :\\\hspace*{10mm}PATREM • MATREM •\\\hspace*{15mm}LIBEROS •\\\hspace*{10mm}VNA • LVES • SVSTVLIT •\\\hspace*{15mm}LACRVMIS • (\emph{sic})\\\hspace*{10mm}CONFECTA • CYPRIS •\\\hspace*{10mm}FILIO • ET • NEPOTIBVS •} Il est infiniment probable, comme l'a pensé Montfaucon, que les deux cyprès, en même temps qu'ils sont ici un emblème funéraire très-expressif, font une allusion directe au nom de la malheureuse mère. C'est ainsi que sur la pierre tumulaire d'une autre dame romaine, nommée \emph{Labéria Daphné}, on avait sculpté l'image de la nymphe Daphné changée en laurier.\footnote{Fabretti, \emph{Inscript.} C. 3., n° 37., p. 186.} A cet exemple, j'en pourrais ajouter beaucoup d'autres analogues, que je juge inutile de rapporter. Mais je ne puis me dispenser de faire remarquer comment de la composition du sujet qui décore l'urne consacrée par Cypris ressort la preuve incontestable que ce nom s'était perpétué, chez les Romains, comme nom de femme, et que son identité avec le mot cyprès n'était pas mise en doute.

Un autre monument funéraire, quoiqu'il ait été rapporté de Smyrne et qu'on y lise une inscription grecque, doit trouver place ici parmi ceux qui appartiennent à l'époque romaine. C'est un sarcophage dont Patin\footnote{\emph{Thesaur. Polenian.} t. 2. (\emph{Comment. in tres inscript. græc.}), col. 1071-1082 ; pl. pour la pag. 1071-1072.} a publié la description et un dessin. Il renfermait la dépouille mortelle d'un jeune personnage nommé dans l'épitaphe\footnote{Voici comment cette épitaphe a été copiée :\\\hspace*{10mm}\textgreek{ΛΟ · ΜΟΥΡΔΙΟϹ\\\hspace*{10mm}ΗΡΑΚΛΑϹ ΕΤΩΝ \textoverline{Κ}\\\hspace*{10mm}ΗΡΩϹ ·}} \emph{Lucius Murdius Héraclas}, et resté inconnu dans l'histoire. Un bas-relief, qui décore la face antérieure du sarcophage, nous montre Héraclas monté sur un cheval. Suivi d'un serviteur et d'un chien, il s'avance pour faire une libation sur un autel allumé ; à côté de cet autel, un serpent s'enroule autour du tronc d'un cyprès pyramidal.\footnote{Sur la face antérieure du grand bas-relief trouvé dans un mithræum, à Heddernheim (\emph{Recherches sur Mithra}, atlas, pl. 90.), l'idée de vie et de printemps est exprimée par un serpent qui s'enroule autour d'un arbre couvert de fleurs, et planté auprès du groupe de Mithra immolant un taureau.} Pouvait-on exprimer par un emblème plus significatif la double idée de vie et d'immortalité ? et cette double idée ne semble-t-elle pas en harmonie avec la qualification de \emph{héros} qui, dans l'épitaphe d'Héraclas, comme dans toutes les inscriptions funéraires, se traduit par \emph{défunt}, mais qui implique l'idée de l'apothéose, ou, tout au moins, de l'immortalité ?

Une stèle funéraire, publiée par Passéri,\footnote{\emph{Lucern. fictil.} t. 3., tab. 60.} exprime probablement la même pensée, à l'aide d'un procédé différent. On y voit, placé entre deux cyprès pyramidaux, un aigle, symbole du feu céleste, gardien de la foudre et signe caractéristique de l'apothéose, dont les honneurs se décernaient aux héros, comme aux initiés parvenus à l'un des grades supérieurs des mystères de Mylitta et des mystères de Mithra.\footnote{Dans mes \emph{Recherches sur Mithra}, je m'expliquerai à cet égard avec tous les détails convenables.}

Non moins variés dans leur composition que les bas-reliefs qui ornent les sarcophages, les stèles et les vases funéraires, les sujets modelés sur les lampes de terre cuite, trouvées dans des tombeaux romains, nous offrent, à leur tour, de nombreux exemples de l'acception funèbre du cyprès. Je n'en citerai que quelques-uns. Les deux premiers me sont fournis par deux lampes que Bellori,\footnote{\emph{Le antich. lucern. figur.} (Rom. 1729, in-f°.), parte 1., p. 6-8 ; tav. 11., tav. 13.} au commencement du siècle dernier, avait publiées d'après des dessins inexacts de Sante Bartoli. Elles ont été apportées d'Italie et mises en vente à Paris, il y a peu d'années. M. Muret, qui les a eues entre les mains,\footnote{J'ignore qui les a acquises.} n'a pas laissé échapper cette occasion de les dessiner ; et j'ai pensé que les archéologues me sauraient gré de placer sous leurs yeux une planche qui reproduit ces deux lampes\footnote{Pl. 19., n°\textsuperscript{s} 1 et 2.} d'après un aussi habile dessinateur. La première\footnote{Elle appartenait autrefois à Bellori.} représente un autel ou une urne cinéraire qu'ombragent, en inclinant leur sommet pyramidal, deux cyprès plantés, l'un, à droite, l'autre, à gauche. Derrière chacun de ces deux arbres, on voit, placé sur un piédestal carré, un jeune homme imberbe, la tête, les bras, la moitié inférieure des cuisses, les jambes et les pieds nus, le reste du corps vêtu d'une double tunique courte et d'une chlamyde flottante. D'une main, il porte un vase ansé\footnote{Les deux vases ansés ou les deux seaux sont plus nettement modelés sur une autre lampe du recueil de Bellori (3\textsuperscript{e} part. pl. 2.), qui représente un sujet analogue.} ; de l'autre, une corne de taureau,\footnote{Sur la face postérieure du grand bas-relief mithriaque de Heddernheim, déjà cité, on voit un prêtre et une prêtresse qui offrent un sacrifice à Mithra. Le prêtre porte, de la main gauche, une corne de taureau, destinée à recevoir le jus du raisin que la prêtresse tient dans ses mains. (Voy. mes \emph{Recherches sur Mithra}, atlas, pl. 91.)} qu'il incline au-dessus de sa tête comme pour faire une de ces aspersions d'eau, de lait et de sang, prescrites dans les sacrifices offerts en faveur des morts aux divinités infernales.\footnote{Bellori (ouvrage cité, p. 7) rapporte fort à propos le passage où Virgile (\emph{Æneid.} 5., v. 77, 78), parlant des rites accomplis par Énée au tombeau d'Anchise, dit :\\\hspace*{15mm}Hic duo rite mero libans carchesia Baccho\\\hspace*{15mm}Fundit humi, duo lacte novo, duo sanguine sacro.\\\hspace*{5mm}On retrouve, sur d'autres lampes funéraires, dans le même recueil de Bellori (1\textsuperscript{re} partie, pl. 14. ; 3\textsuperscript{e} partie, pl. 2.), mais sans les deux cyprès, des exemples de l'accomplissement des rites dont il s'agit. Une de ces lampes (1\textsuperscript{re} part. pl. 14.) fournit même un témoignage plus décisif que la lampe ici figurée ; car on y voit distinctement le liquide contenu dans chaque corne de taureau tomber sur les flammes d'un autel funéraire.} La face antérieure de l'autel ou de l'urne que caractérisent les deux cyprès, est ornée d'un bucrane et de deux guirlandes de fruits ; tous objets dont nous avons, dans une autre occasion,\footnote{\emph{Mém. de l'Acad. des inscrip.} nouv. sér. t. 15., 2\textsuperscript{e} partie, p. 124-126.} reconnu l'acception funéraire.

La composition du sujet qui décore la seconde lampe ici figurée\footnote{Pl. 19., n° 2. L'original se trouvait autrefois dans la collection de Sante Bartoli.} est beaucoup plus simple. C'est une urne cinéraire, d'une forme particulière, posée entre deux cyprès pyramidaux, qui inclinent aussi leur sommet vers ce petit monument, comme s'ils voulaient rendre hommage aux cendres qu'il renferme, ou montrer qu'elles sont placées sous leur protection. Au milieu de l'urne, on remarque une espèce de cartouche carré, qui probablement indique la place où devait se lire une inscription en l'honneur du défunt.\footnote{« Nel mezzo dell' urna, dit Bellori (\emph{loc. cit.} p. 7), si vede il contrasegno del titolo sepolcrale solito à gli Dei Mani, ed alla memoria del defonto. »}

Une troisième lampe de terre cuite, analogue à la précédente, se trouve à Paris dans le cabinet de M. le général baron Brunet-Denon. Elle représente un monument funéraire érigé entre deux cyprès pyramidaux. La forme de ce monument est celle d'un sacellum distyle, d'ordre ionique, surmonté d'un fronton triangulaire. Une couronne, d'où pendent deux bandelettes, occupe le milieu du tympan.\footnote{Voyez, sur l'acception funéraire de la couronne, ci-dessus, p. 303, et note 3.} Les deux battants de la porte sont entr'ouverts ; à chaque battant est attaché un gros anneau.

Sur quelques autres lampes de cette même catégorie, les deux cyprès sont plantés précisément à l'entrée d'un monument funèbre,\footnote{Passéri, \emph{Lucern. fictil.} t. 3., tab. 44.} ou aux deux côtés d'un autel funéraire, au-dessus duquel est placé un \emph{simpulum}.\footnote{\emph{Id. ibid.} tab. 48.}

S'il était permis de croire que les médailles contorniates ont eu quelquefois une destination funéraire, la série des monuments romains qui nous offrent le cyprès comme un emblème funèbre devrait se clore par celles de ces médailles où cet arbre remplit le même rôle. Dans cette hypothèse, j'aurais à rappeler ici les trois pièces qui, au revers de la tête d'Homère, de Néron ou de Vespasien,\footnote{Ci-dessus, p. 283-285 ; pl. 18., n° 3.} nous ont montré, devant un cyprès pyramidal, un personnage à pied, armé d'une lance et attaquant, ainsi que son chien, un énorme sanglier réfugié dans un antre. J'aurais aussi à rapprocher ce sujet du médaillon qui, placé au centre de l'un des deux plats funéraires dont j'ai eu l'honneur d'entretenir l'Académie,\footnote{Ci-dessus, p. 333, 334 ; pl. 16., n°\textsuperscript{s} 1-6.} représente un cavalier attaquant de même un sanglier formidable. Je devrais enfin répéter que la couronne présentée à ce cavalier par la déesse de la Victoire\footnote{Pl. 16., n° 5.} ne nous a pas permis, non plus que le sujet funèbre figuré sur le rebord du plat, de considérer le médaillon central comme la représentation d'un simple épisode de chasse, ou d'un danger couru par un voyageur. Dès lors la présence du cyprès au revers de nos trois contorniates ne semblerait pas nous permettre davantage d'attacher au type de ce revers un sens pris dans un ordre d'idées aussi vulgaires. Soit donc qu'un tel type reproduise un fait mythologique, Méléagre, par exemple, combattant le terrible sanglier de Calydon ; sort que le combat ne se rattache nullement à cette légende, et doive être assimilé aux luttes mystiques que nous trouvons sur nos deux plats funéraires et sur une multitude d'urnes cinéraires et de sarcophages, il devient presque certain que le cyprès présente une signification funèbre au revers des trois médailles contorniates de la Bibliothèque impériale.

Ne nous hâtons pas toutefois de conclure de cette remarque et des divers monuments cités ou décrits plus haut, que, de tout temps, les Romains connurent les raisons qui, chez les peuples de l'Asie occidentale, avaient fait ranger le cyprès au nombre des emblèmes employés dans les rites funèbres. Sous le règne des empereurs, beaucoup de traditions religieuses s'étaient perdues à Rome, comme ailleurs, bien que les coutumes propres à une très-ancienne liturgie se fussent perpétuées, d'âge en âge, sans interruption. Quelquefois les traditions étaient encore vivantes au fond des sanctuaires, ou déposées dans des livres sacrés, dont les prêtres avaient seuls la possession. Si Virgile s'exprime en termes qui nous donnent lieu de croire qu'il avait emprunté à ces deux sources ce qu'il nous dit du cyprès, comme symbole de l'antique divinité génératrice des Phrygiens et comme emblème funéraire, Servius,\footnote{\emph{Ad} Virgil. \emph{Georgic.} 1., 20.} son commentateur, et les deux mythographes latins\footnote{\emph{Scriptor. rer. mythic. latini tres}, t. 1., fab. 6, p. 2 ; fab. 178, p. 135.} cités ci-dessus ne font pas preuve d'un grand savoir quand ils rapportent qu'on attribue le cyprès à Silvain, dieu infernal, parce que cet arbre, à cause de la faiblesse de ses racines, convient aux rites funèbres. De son côté, Varron, réputé le plus docte des Romains, ne justifie pas à nos yeux cette réputation lorsque, cherchant à expliquer pourquoi on entoure de cyprès les bûchers où l'on brûle le corps des personnages d'une haute distinction, il met en avant, au lieu d'une raison tirée des croyances et des usages religieux, une supposition d'un ordre physique, qui assurément ne peut satisfaire aucun bon esprit : « On entoure ainsi de cyprès les bûchers, dit-il, afin que l'odeur forte qu'exhalent les corps en combustion n'incommode pas les assistants.\footnote{« Pyras ideo cupresso circumdari, propter gravem ustrinæ odorem, ne offendatur populi circumstantis corona. » Varron, \emph{apud} Servium \emph{ad} Virgil. \emph{Æneid.} 6., 216. Cf. Isidor. Hispalens. \emph{Origines}, 17., 7., n° 34.} » Servius et Festus, nous l'avons vu,\footnote{Ci-dessus, p. 326, 327.} semblent se contenter de cette explication. Saint Isidore\footnote{\emph{Loc. cit.}} la reproduit sans nommer Varron. Faut-il croire que la religion défendait à ce dernier écrivain et aux deux autres de divulguer les motifs qui, dans les sanctuaires, avaient fait attribuer au cyprès une acception funèbre ? Je laisse à de plus habiles que moi le soin de résoudre cette question.

\subsubsection{Europe chrétienne}
\paragraph{}
L'ignorance du vulgaire sur le sens intime du symbole, soit qu'on figurât des cyprès sur les sarcophages, sur les pierres tumulaires et sur les lampes sépulcrales, soit qu'on en plantât autour des tombeaux, n'empêcha pas ces coutumes de se répandre dans tout l'empire romain. Les chrétiens, succédant aux païens en Asie, en Europe, en Afrique, et trouvant partout ces coutumes établies, ne purent se défendre d'adopter un emblème funéraire consacré par tant de peuples, par tant de siècles.\footnote{N'oublions pas qu'en Orient et en Occident le culte des arbres, bien que d'institution païenne, bien que prohibé par nos livres saints (\emph{Deuteron.} 12., 2, 3 ; 16., 21), se conserva longtemps parmi quelques chrétiens, comme il s'était conservé dans quelques tribus israélites. Il se perpétua même fort tard en Europe ; et, entre autres preuves qu'il serait facile d'en donner, je citerai celle que fournissent les statuts d'un synode tenu à Trèves, sous la présidence de l'archevêque et postérieurement à l'année 1274 : on y trouve la défense expresse d'adorer les fontaines, les arbres, etc. (Voy. Martène, \emph{Amplissim. Collect.} t. 7., col. 107 ; col. 113, § 53.)} Peut-être même, se rappelant que, par l'ordre du Seigneur, Noé avait construit l'arche avec du bois de cyprès, et que les cyprès du Liban avaient servi à réédifier le temple de Jérusalem, les premiers chrétiens furent-ils tout naturellement portés à considérer cet arbre comme un des symboles légitimés par le texte de l'Ancien Testament et légués aux disciples de la nouvelle loi. Déjà, dans la Grèce du moyen âge, j'ai eu à signaler des pierres tumulaires qui attestent une alliance intime entre la croix de Jésus-Christ et l'emblème du cyprès pyramidal\footnote{Ci-dessus, p. 313-315.} ; alliance qu'au 7\textsuperscript{e} siècle nous avions trouvée établie chez un peuple du Caucase converti au christianisme.\footnote{Ci-dessus, p. 77-79.} En Italie, les chrétiens des premiers siècles ont laissé, parmi une multitude de monuments funéraires, plusieurs pierres,\footnote{Voy. Aringhi, \emph{Roma subterran.} 6., 42. Mamachi, \emph{Origin. et antiquit. christianar.} lib. 20. (Rom. 1749-1755), t. 1., p. 408, cul-de-lampe.} plusieurs lampes ornées de sujets où cet arbre figure comme emblème funèbre. Je n'appellerai l'attention de la Compagnie que sur deux lampes de terre cuite. La première peut remonter au 3\textsuperscript{e} siècle de notre ère, ou au commencement du 4\textsuperscript{e} ; elle est inédite, et fut apportée de Rome à Paris, il y a peu d'années, pour y être vendue.\footnote{J'ignore en quelles mains elle se trouve, actuellement.} Je n'ai pas vu l'original ; mais M. Muret, qui l'a eu entre les mains et qui possède si bien le sentiment de l'art antique, m'en garantit l'authenticité.

Au centre de cette lampe,\footnote{Voy. le n° 1 de la pl. 20. ci-jointe.} Jésus-Christ debout, vu de face et figuré avec les attributs du Bon Pasteur, est entouré de sept fidèles, représentés sous l'emblème de sept brebis, qui, par leur attitude, témoignent leur joie de son retour ; il a retrouvé et il rapporte sur ses épaules la brebis égarée, image du pécheur qui a quitté la bonne voie. Au-dessus du Bon Pasteur, dans la région céleste, la Vierge,\footnote{Et non le Soleil personnifié.} reine du ciel, s'avance sur un nuage, la tête ceinte d'un diadème, et montre du doigt, sur la tête de son divin fils, une couronne formée de sept étoiles,\footnote{« Et habebat in dextera sua stellas septem. » (\emph{Apocalyps.} 1., 16. Cf. \emph{ibid.} 1., 20 ; 2., 1 ; 3., 1.)} qui répondent aux sept branches du chandelier du Temple,\footnote{\emph{Exod.} 25., 37 ; 37., 23. \emph{Num.} 8., 2. Il n'est pas rare de trouver ce chandelier représenté sur des lampes chrétiennes (voy. Bellori, ouvrage cité, 3\textsuperscript{e} partie, pl. 32.). Quelquefois il est remplacé par sept orifices ronds, pratiqués dans la partie supérieure de la lampe et destinés à recevoir sept mèches (\emph{id. ibid.} pl. 33.). On peut rapporter à la même source l'usage qui, de nos jours, se conserve en quelques endroits, de tenir, pendant sept jours et sept nuits, au chef du lit d'un mort, une lampe ou une lanterne allumée.} aux sept dons du Saint-Esprit, aux sept sacrements, aux sept principales vertus,\footnote{Trois vertus théologales, et quatre vertus cardinales.} et enfin aux sept sceaux, aux sept esprits, aux sept anges et aux sept églises de l'Apocalypse.\footnote{5., 1, 5, 6 ; 3., 1 ; 8., 2 ,6 ; 15., 7 ; 1., 4, 11, 12, 13, 16, 20.} Dans le champ de la lampe, à la hauteur des mains de Jésus-Christ, on remarque, à sa droite, et posée sur l'arche de Noé, la blanche colombe, symbole de lumière. Lâchée par Noé, pour la seconde fois, après un intervalle de sept jours,\footnote{\emph{Genes.} 8., 10. Euseb. \emph{Chronic.} 2., p. 237, edd. Maïo et Zohrabo.} elle est revenue dans l'arche. Messagère fidèle, et tenant dans son bec un rameau d'olivier, elle rapporte ainsi le gage de la réconciliation entre Dieu et son peuple, et la preuve manifeste du retour de la verdure, ou d'une nouvelle vie sur la terre, après le retrait des eaux du déluge, ce grand baptême qui régénéra le genre humain tombé dans la corruption et le péché, bien des siècles avant la venue du divin Rédempteur. À gauche, nous voyons l'oiseau au noir plumage, le corbeau, symbole des ténèbres.\footnote{Remarquons bien qu'en hébreu la racine \texthebrew{ערב} a le double sens de \emph{corbeau} et de \emph{nuit} ou \emph{ténèbres}. Il en est de même dans la langue arabe.} Il ne rentra pas dans l'arche, bien qu'il en fût sorti le premier\footnote{\emph{Genes.} 8., 6.} ; et nous le trouvons ici dans le giron de la nuit, représentée par le buste d'une femme vue de face, la tête surmontée du croissant de la lune et d'un voile que, de ses deux mains, elle tient au-dessus du croissant, pour achever d'indiquer les ténèbres.\footnote{On reconnaît facilement dans ce buste l'imitation d'un type souvent employé pour représenter Diane sur les monuments de l'époque des Antonins.} Au-dessous de ce groupe si caractéristique, on en distingue un autre non moins expressif et parfaitement en rapport avec l'idée de la nuit, c'est le corps de Jonas ou d'un juste, qui s'est endormi du sommeil de la mort au pied d'un cyprès pyramidal, emblème tout à la fois de vie, de mort et d'immortalité ou d'éternité. Un cep de vigne, chargé de pampres et de trois grappes pendantes de raisin,\footnote{On voit fréquemment, sur les lampes chrétiennes trouvées en Italie, dans des cimetières ou des tombeaux, l'image ou le monogramme du Christ entouré d'un cep de vigne chargé de feuilles et de grappes de raisin. J'en puis même citer deux qui représentent Jésus-Christ sous les traits du Bon Pasteur, comme la lampe ici figurée, mais sans aucun autre accessoire que huit grappes de raisin placées dans une bande circulaire. (Voy. Bosio, \emph{Rom. soterran.} (Rom. 1632), pl. à la pag. 203. Mamachi, ouvrage cité, t. 3., pl. à la pag. 79.) --- La lampe publiée par Bosio est reproduite dans la nouvelle \emph{Roma soterranea}, t. 3., pl. 208. ; Rom. 1737, 3 vol. in-fol. --- Toutes ces représentations, aussi bien que notre lampe, se rapportent au passage du Nouveau Testament où Jésus-Christ se compare lui-même à la vigne : « Ego sum vitis vera » (Évangile de saint Jean, 15., 1, 5). Sur la seconde lampe dont je donne le dessin, une tige de courge remplace le cep de vigne.} s'élève au-dessus de ce corps gisant, le protège, et fait une allusion directe au vin, qui, dans la célébration de nos saints mystères, représente le sang de Jésus-Christ, répandu sur la croix pour notre rédemption. En regard de ce groupe, c'est-à-dire à la droite du Bon Pasteur, Jonas sort, plein de vie, du ventre d'une baleine, après y avoir passé trois jours et trois nuits,\footnote{\emph{Jonas}, 2., 2.} de même que Jésus-Christ, le troisième jour, est sorti vivant du tombeau. Cet emblème, fréquent sur les lampes funéraires chrétiennes, exprime donc l'idée de la résurrection.\footnote{« Libera me ... de profundis aquis, » dit le Psalmiste (Ps. 68., v. 15) ; « non me demergat tempestas aquæ, » ajoute-t-il (v. 16). Au psaume 17. (v. 17), on lit : « ... assumpsit me de aquis multis. »} Ainsi, en même temps qu'au centre d'une lampe, symbole de lumière, le divin Sauveur se montre sous la figure du Bon Pasteur, à sa gauche sont exprimées les idées de nuit, de ténèbres, de mort et de rédemption ; à sa droite, les idées de lumière, de baptême et de résurrection. Que les personnes qui s'obstineraient à croire l'art chrétien dépourvu d'invention et de poésie, daignent prendre la peine d'étudier avec nous la composition de la grande scène représentée sur cette lampe, et que, la main sur la conscience, elles nous disent si un tableau aussi habilement tracé ne satisfait pas tout à la fois nos croyances et notre imagination !

La seconde lampe chrétienne dont j'ai à parler se conserve au musée royal de Berlin.\footnote{N° 289 (1018) du nouveau Catalogue des terres cuites de ce musée.} Elle faisait anciennement partie du cabinet de Bellori, d'où elle passa dans la collection de l'Électeur de Brandebourg.\footnote{Bellori, \emph{Le antiche lucerne fig.} parte 3., p. 10.} Elle offre le même sujet et presque les mêmes accessoires que la première. Comme celle-ci, elle fut donc composée dans un esprit conforme tout à la fois aux enseignements de l'Ancien Testament et à ceux du Nouveau Testament. C'est ce que ne semblent pas avoir bien compris les écrivains qui l'ont décrite et commentée. Au premier abord, on pourrait croire que les deux lampes sont une copie l'une de l'autre. Un examen attentif ne permet pas de conserver longtemps cette opinion. Ces deux terres cuites offrent entre elles des différences notables ; elles ne sortent ni d'un même moule, ni de la même main ; elles n'appartiennent même pas à une même époque de l'art, bien qu'on puisse les considérer chacune comme l'imitation plus ou moins fidèle d'un seul type original. Le style et le travail de celle que possède le musée royal de Berlin ne sauraient remonter plus haut que la dernière moitié du 4\textsuperscript{e} siècle de notre ère. L'Académie en jugera par le dessin que je joins ici\footnote{Pl. 21., n° 3.} et dont je suis redevable à l'inépuisable complaisance de M. Édouard Gerhard. Parmi les différences à signaler entre les deux monuments, il faut surtout citer celle que présentent la chevelure et le costume du Bon Pasteur et la forme des sept étoiles et du cyprès. L'arbre, quant à son espèce, est même peu reconnaissable sur la lampe de Berlin, dans l'état où elle est actuellement. Personne encore ne s'est cependant refusé à le prendre pour un cyprès ; et je remarque même qu'il est beaucoup mieux caractérisé comme tel dans les dessins qui en ont été publiés jusqu'à ce jour que dans le dessin tout récemment exécuté sous les yeux de M. Gerhard. De plus, ici, la colombe ne porte pas à son bec le rameau d'olivier ; et le cep de vigne qui, sur la lampe apportée à Paris et dessinée par M. Muret,\footnote{Pl. 20., n° 1.} ombrage la figure gisant sur le sol, au pied du cyprès, est remplacé par la tige de courge que d'autres monuments nous montrent s'élevant au-dessus de Jonas, également étendu sur le sol. Enfin, sous la lampe de Berlin, on lit en assez beaux caractères, FLORENI ; c'est le nom du potier.

Depuis longtemps elle était connue de tous les archéologues : Bellori\footnote{\emph{Le antiche lucerne figur.} parte 3., p. 9, 10 ; tav. 29. --- Ce recueil a été imprimé en 1694, 1699 et 1729.} en 1691, Bottari\footnote{\emph{Scult. e pitture sagre}, t. 3., p. 79.} et Mamachi\footnote{Ouvrage cité, t. 3., p. 78.} vers le milieu du 18\textsuperscript{e} siècle, et successivement, depuis le commencement du 19\textsuperscript{e}, M. le Dr. Schöne,\footnote{\emph{Geschichtsforschungen über die kirchlich. Gebr. und Einrichtung. der Christen}, etc. t. 1. (Berlin, 1819, in-8°), p. 310, 311 ; pl. 1., n° 4.} M. le Dr. Frédéric Creuzer,\footnote{\emph{Symbolik und Mytholog.} t. 4., p. 418 ; pl. 7., n° 2.} feu Mgr. Münter, évêque de Seeland,\footnote{\emph{Sinnbilder und Kuntsvorstellung. der alten Christen} (Altona, 1825, in-4°), 1\textsuperscript{re} partie, p. 64, 65 ; pl. 3., n° 39.} et M. Guigniaut\footnote{\emph{Relig. de l'antiquité}, t. 4., 1\textsuperscript{re} partie ; Explication des planches, p. 419, 420 ; pl. 259., n° 914.} l'ont chacun publiée\footnote{Plusieurs autres archéologues, tels que Buonarroti, Passéri, Kopp, Bellermann et M. Raoul-Rochette (\emph{Mém. de l'Acad. des inscript.} nouvelle série, t. 13., 1\textsuperscript{re} partie, p. 98, 99) ont écrit sur cette lampe, mais succinctement et sans en donner un dessin.} d'après un dessin peu fidèle de Sante Bartoli, dessin qui chaque fois, pour ainsi dire, qu'il a été copié par un nouveau graveur, s'est écarté de plus en plus de l'original. L'interprétation qu'ils donnent du monument n'est pas satisfaisante. M. Raoul-Rochette en a fait avant moi l'observation,\footnote{\emph{Loc. cit.} p. 98 et 99, note 2.} quant aux six premiers de ces savants ; mais il ajoute avec raison que feu Mgr. Münter a du moins le mérite d'avoir judicieusement réfuté l'opinion de quelques érudits, tels que M. Schöne et même M. Creuzer, qui voyaient dans la lampe dont il s'agit un monument gnostique. Enfin M. Ferdinand Piper, professeur de théologie à l'université de Berlin, a rendu publique, en 1846, une dissertation\footnote{Elle est insérée dans le recueil périodique intitulé : \emph{Zeitschrift für die historische Theologie}, et il en a été fait un tirage à part que j'ai sous les yeux.} où il décrit cette même lampe\footnote{Pag. 13-15 du tirage à part.} et la reproduit, d'après l'original, par la lithographie.\footnote{N° 5 de la planche lithographiée jointe au tirage à part.} Je regrette d'avoir à dire que ce dessin et le commentaire dont il est accompagné laissent encore beaucoup à désirer. Comme le Dr. Münter, il constate qu'aucun élément gnostique n'est entré dans la composition du sujet qui décore la lampe chrétienne du musée de Berlin. Mais, à l'exemple de tous ses devanciers, il ne tient aucun compte du cyprès\footnote{Le Dr. Fr. Münter (\emph{loc. cit.}), à l'exemple de Bellori et de Mamachi, ne fait aucune allusion à cet emblème significatif ; il se borne à dire qu'on voit ici, « couché à l'ombre d'un arbre, Jonas, type de Jésus-Christ. »} qu'on y remarque, il ne voit dans le groupe qui, à mon avis, représente la nuit ou les ténèbres qu'une image de la lune considérée comme astre, et il commet la double méprise de prendre pour une seconde colombe\footnote{J'ai peine à comprendre que mes savants devanciers, au lieu de reconnaître ici le corbeau sorti de l'arche, aient tous indiqué une seconde colombe, sans tenir compte du texte de l'Écriture sainte. C'est Bellori, le premier, qui a commis cette méprise ; il s'explique en ces termes (ouvrage cité, 3\textsuperscript{e} partie, p. 10) : « L'altra colomba, che posa sotto la luna, può interpretarsi per l'Innocenza, Purità, Simplicità, Concordia, e per l'altre virtù christiane del defonto, di cui è simbolo questo animale. »} le corbeau, qui est une partie essentielle de ce groupe, et pour l'image du soleil\footnote{Je vois à regret que M. Raoul-Rochette (\emph{loc. cit.}), comme M. Piper, a partagé, sur ce second point, l'erreur de ses devanciers.} la figure de la Vierge, bien que Bellori et Mamachi eussent avoué eux-mêmes que les vêtements de cette figure sont ceux d'une femme. De là il est résulté que le savant professeur de Berlin n'a pas mieux compris que ses devanciers l'esprit dans lequel fut composé le sujet qui orne cette lampe. On ne peut se défendre de quelque étonnement en voyant tant d'habiles interprètes s'arrêter à l'idée que le Bon Pasteur y est placé entre le soleil, la lune et les étoiles, et que, par conséquent, ce petit monument doit être classé parmi ceux qui ont pour simple thème la création des corps célestes.\footnote{Les études théologiques de feu le Dr. Frédéric Münter ne l'ont pas empêché de partager cette opinion erronée : « Es kommt mir jedoch leichter vor, dit-il (ouvrage cité, p. 64), diese Bilder, die in einem Kreise von biblischen Geschichten stehen, demgemäss von der Schöpfung der Himmelscörper zu erklären. »} Ils restent ainsi insensibles à l'admirable accord qui, sur la lampe de Berlin, comme sur la lampe inédite que je publie,\footnote{Planche 20., n° 1.} se révèle entre les divers accessoires disposés autour du Bon Pasteur et les dogmes fondamentaux de l'Ancien et du Nouveau Testament. Cet accord, malgré les différences que présentent entre elles les deux lampes, dans les détails de leur composition, est, à mes yeux, tout aussi manifeste sur l'un que sur l'autre de ces deux précieux monuments.
\clearpage

\begin{center}
\textbf{Résumé.}
\end{center}

\paragraph{}
Si donc les premiers chrétiens plaçaient le cyprès pyramidal sur leurs pierres tumulaires, dans leurs cimetières et sur leurs lampes\footnote{Voy. Aringhi, \emph{Roma subterran.} 6., 42., p. 339-341.} ; si l'Arménie et la Grèce du moyen âge alliaient à ce symbole la croix de rédemption ; si, de nos jours, en Europe, on entoure de cyprès les sépultures royales et les sépultures privées ; si, sous nos yeux, les peuples musulmans, successeurs des antiques nations qui, en Asie, en Afrique, en Europe, faisaient de cet arbre un emblème funèbre, plantent des cyprès autour de leurs tombeaux isolés et dans leurs \emph{champs des morts} ; si les habitants d'un vaste empire, qui s'étend de l'Europe en Asie, attachent encore aujourd'hui des idées religieuses au bois de cyprès ; si, dans le midi de la France, s'est conservée, d'âge en âge, jusqu'à nous la coutume de faire bénir, le dimanche des Rameaux, des branches de cyprès, dont le feuillage et les fruits sont argentés et dorés ; si, en Asie, le culte de cet arbre symbolique remonte à plus de trois mille ans et se lie intimement à la légende de la Vénus assyrienne, comme aux origines de l'empire d'Assyrie, du royaume d'Arménie et des États fondés par les Phrygiens ou les Troyens, comme à l'histoire de l'importation de la civilisation chaldéenne ou assyrienne chez les Phéniciens, les Arabes, les Égyptiens, les Perses, les Grecs et les peuples italiques ; si enfin le cyprès pyramidal est connu, dans toutes les langues de l'Europe ancienne et moderne, sous un seul nom, précisément le même que portait la divinité androgyne dont cet arbre était, en Asie, l'emblème vivant, reconnaissons que le culte du cyprès, institué aux lieux mêmes où ce bel arbre croît spontanément, aux lieux mêmes habités par les premiers adorateurs de cette divinité, dut être importé partout où pénétra le culte de la déesse qui avait donné son nom à l'île de Cypre, comme au cuivre, et qui fut adorée, chez les Grecs et les Romains, sous celui de Cypris, et, chez les Étrusques, sous le nom de Cypra ou Cupra. Tant furent profondes les racines qu'avaient jetées en Europe les croyances religieuses apportées de l'Asie occidentale avec tout le cortège d'une haute civilisation ! Tant fut puissante l'influence exercée par les castes sacerdotales sur les idées, le langage, les coutumes et les institutions des peuples ! Tant il est vrai de dire que les traits les moins effacés et les moins effaçables de la civilisation d'un peuple, les traits surtout qui caractérisent sa vie intellectuelle, sont ceux que l'on peut rapporter aux dogmes et aux rites enseignés dans ses livres sacrés, ou inscrits en caractères idéographiques sur ses édifices, sur ses monuments figurés ! Tant, il est juste enfin de reconnaître que le culte du cyprès, emblème tout à la fois de vie, de mort, de vie future et d'éternité, concourt, avec bien d'autres faits, à prouver qu'aucun peuple civilisé de l'antiquité ne méconnut le dogme de l'immortalité de l'âme, inséparable, comme la connaissance d'un Dieu éternel, comme le dogme des récompenses et des peines dans une vie future, des conditions morales d'existence, de progrès et de durée de la société humaine !

\clearpage

\section{Explication des planches}
\subsection{Planche 1\textsuperscript{re}}
\vspace*{\fill}
\begin{enumerate}
    \item[N° 1.] Face antérieure d'un autel ou cippe votif, palmyrénien, à quatre faces, et de marbre. Hauteur, 3 palmes romains et 10/12 ; largeur, 2 palmes romains et 5/12.\footnote{Le palme de Rome équivaut à environ 212 [252] millimètres (8 pouces 3 lignes et 1/2).} \emph{Musée capitolin.}

    \item[N° 2.] Face postérieure du même monument.
\end{enumerate}
\vspace*{\fill}
\clearpage
\begin{landscape}
\vspace*{\fill}
\begin{figure}[H]
\centering
\includegraphics[height=0.9\textheight,keepaspectratio]{figures/plate-001.jpg}
\end{figure}
\vspace*{\fill}
\end{landscape}
\clearpage
\subsection{Planche 2}
\vspace*{\fill}
\begin{enumerate}
    \item[N° 1.] Face latérale gauche du même monument.

    \item[N° 2.] Face latérale droite du même monument.
\end{enumerate}
\vspace*{\fill}
\clearpage
\begin{landscape}
\vspace*{\fill}
\begin{figure}[H]
\centering
\includegraphics[height=0.9\textheight,keepaspectratio]{figures/plate-002.jpg}
\end{figure}
\vspace*{\fill}
\end{landscape}
\clearpage
\subsection{Planche 3}
\vspace*{\fill}
\begin{enumerate}
    \item[N° 1.] Bas-relief votif, palmyrénien, de marbre. Hauteur, 4 palmes rom. et 4/12 ; largeur, 3 palmes romains. \emph{Musée capitolin.}

    \item[N° 2.] Fragment d'un très-petit bas-relief de bronze, trouvé en Transylvanie, et reproduit ici avec les dimensions de l'original. Possesseur non connu.

    \item[N° 3.] Médaille autonome de Mostène (Lydie). BR. \emph{Bibliothèque impériale.}

    \item[N° 4.] Grand bronze d'Hadrien, frappé dans la même ville. \emph{Même collection.}
\end{enumerate}
\vspace*{\fill}
\clearpage
\begin{landscape}
\vspace*{\fill}
\begin{figure}[H]
\centering
\includegraphics[height=0.9\textheight,keepaspectratio]{figures/plate-003.jpg}
\end{figure}
\vspace*{\fill}
\end{landscape}
\clearpage
\subsection{Planche 4}
\vspace*{\fill}
\begin{enumerate}
    \item[N° 1.] Statuette inédite de bronze, provenant d'un \emph{sacellum} pratiqué dans les flancs d'une montagne située à 20 kilomètres environ de Seyde (Syrie). Hauteur de la statuette, 50 centimètres. \emph{Cabinet de M. le duc de Luynes.}

    \item[N° 2.] Moyen bronze de Commode, frappé à Mostène. \emph{Bibliothèque impériale.}

    \item[N° 3.] Moyen bronze de Gallien, frappé dans la même ville. \emph{Même collection.}

    \item[N° 4.] Médaillon de bronze, frappé à Éphèse (Ionie) en l'honneur d'Antonin le Pieux. \emph{Même collection.}

    \item[N° 5.] Autre médaillon de bronze, frappé à Éphèse en l'honneur du même empereur. \emph{Même collection.}
\end{enumerate}
\vspace*{\fill}
\clearpage
\begin{landscape}
\vspace*{\fill}
\begin{figure}[H]
\centering
\includegraphics[height=0.9\textheight,keepaspectratio]{figures/plate-004.jpg}
\end{figure}
\vspace*{\fill}
\end{landscape}
\clearpage
\subsection{Planche 5}
\vspace*{\fill}
\begin{enumerate}
    \item[N° 1.] Petit trône votif, de bronze, inédit, et vu de face. Hauteur, 28 à 29 centimètres ; largeur, 12 à 13 centimètres. \emph{Bibliothèque impériale.}

    \item[N° 2.] Le même monument, vu de profil.
\end{enumerate}
\vspace*{\fill}
\clearpage
\begin{landscape}
\vspace*{\fill}
\begin{figure}[H]
\centering
\includegraphics[height=0.9\textheight,keepaspectratio]{figures/plate-005.jpg}
\end{figure}
\vspace*{\fill}
\end{landscape}
\clearpage
\subsection{Planche 6}
\vspace*{\fill}
\begin{enumerate}
    \item[N° 1.] Moyen bronze de Caracalla, frappé dans l'île d'Aradus (Phénicie). \emph{Bibliothèque impériale.}

    \item[N° 2.] Moyen bronze d'Élagabale, frappé dans la même île. \emph{Même collection.}

    \item[N° 3.] Monnaie coloniale de Damascus (Cœlésyrie), à l'effigie de Trébonien Galle. BR. \emph{Même collection.}

    \item[N° 4.] Monnaie coloniale, inédite, de Carrhes de Mésopotamie (\emph{Aurelia Carrhenorum}), à l'effigie de Septime Sévère. BR. \emph{Biblioth. impériale.}

    \item[N° 5.] Monnaie coloniale d'Héliopolis (Cœlésyrie), à l'effigie de Philippe père. BR. \emph{Même collection.}

    \item[N° 6.] Développement de la périphérie d'un petit vase funéraire, d'argent, inédit, et trouvé, selon les uns, dans le Rhône, entre Vienne et Lyon ; selon les autres, à Aire (Landes), sur les bords de l'Adour. Le sujet qu'il représente est reproduit ici avec les dimensions de l'original. \emph{Même collection.}

    \item[N° 6. \emph{a.}] Réduction de la forme de ce vase.
\end{enumerate}
\vspace*{\fill}
\clearpage
\begin{landscape}
\vspace*{\fill}
\begin{figure}[H]
\centering
\includegraphics[height=0.9\textheight,keepaspectratio]{figures/plate-006.jpg}
\end{figure}
\vspace*{\fill}
\end{landscape}
\clearpage
\subsection{Planche 7}
\vspace*{\fill}
\begin{enumerate}
    \item[N° 1.] Monnaie coloniale de Damascus, à l'effigie de Philippe fils. BR. \emph{Bibliothèque impériale.}

    \item[N° 2.] Monnaie coloniale de Tyr (Phénicie), à l'effigie de Caracalla. BR. \emph{Même collection.}

    \item[N° 3.] Moyen bronze de Plautilla, frappé à Sicyone (Achaïe). \emph{Même collect.}

    \item[N° 4.] Monnaie coloniale de Tyr, à l'effigie de Volusien. BR. \emph{Même collection.}

    \item[N° 5.] Autre monnaie coloniale de Tyr, à l'effigie de Caracalla. BR. \emph{Même collection.}

    \item[N° 6.] Petite plaque de bronze, gravée en creux, inédite, acquise à Rome par M. le professeur Édouard Gerhard, et déposée au \emph{Musée royal de Berlin}.
\end{enumerate}
\vspace*{\fill}
\clearpage
\begin{landscape}
\vspace*{\fill}
\begin{figure}[H]
\centering
\includegraphics[height=0.9\textheight,keepaspectratio]{figures/plate-007.jpg}
\end{figure}
\vspace*{\fill}
\end{landscape}
\clearpage
\subsection{Planche 8}
\vspace*{\fill}
\begin{enumerate}
    \item[N° 1.] Moyen bronze d'Élagabale, frappé à Tripolis de Phénicie. \emph{Bibliothèque impériale.}

    \item[N° 2.] Moyen bronze de Caracalla, frappé dans la même ville. \emph{Même collection.}

    \item[N° 3.] Moyen bronze du même empereur, frappé dans la même ville. \emph{Même collection.}

    \item[N° 4.] Grand bronze de Julia Domna, frappé à Apollonia d'Illyrie. \emph{Même collection.}

    \item[N° 5.] Cône du \emph{Musée britannique.}

    \item[N° 6.] Cône de calcédoine saphirine. \emph{Bibliothèque impériale.}

    \item[N° 7.] Moyen bronze de Julia Domna, frappé à \emph{Germanicia\footnote{Par erreur, on a imprimé à la page 79, ci-dessus, \emph{Germanica Cæsarea}.} Cæsarea} (Commagène). \emph{Même collection.}

    \item[N° 8.] Monnaie autonome de Nagidus (Cilicie). AR. \emph{Même collection.}

    \item[N° 9.] Autre monnaie autonome de la même ville. AR. \emph{Même collection.}

    \item[N° 10.] Monnaie coloniale de Bostra (Arabie), à l'effigie de Trajan Dèce. BR. \emph{Même collection.}

    \item[N° 11.] Monnaie autonome d'Anticaria (Bétique). BR. Autrefois dans le musée Hédervar. Possesseur actuel non connu.
\end{enumerate}
\vspace*{\fill}
\clearpage
\vspace*{\fill}
\begin{figure}[H]
\centering
\includegraphics[height=0.85\textheight,keepaspectratio]{figures/plate-008.jpg}
\end{figure}
\vspace*{\fill}
\clearpage
\subsection{Planche 9}
\vspace*{\fill}
\begin{enumerate}
    \item[N° 1.] Cylindre de stéatite, vert foncé, à taches rougeâtres. Autrefois dans la collection de M. J. Robert Steuart. Possesseur actuel non connu.

    \item[N° 2.] Cylindre de jaspe noir. \emph{Musée britannique.}

    \item[N° 3.] Cylindre de chlorite terreuse, vert-noir. \emph{Bibliothèque impériale.}

    \item[N° 4.] Cylindre de stéatite verte. Légende en caractères inconnus, offrant peut-être un exemple de l'altération des caractères cunéiformes du système assyrien. \emph{Même collection.}

    \item[N° 5.] Cylindre de talc ollaire, vert-noir.\footnote{Hématite, selon M. Hawkins.} Collection de M. Cl. Scott Steuart, acquise par le \emph{Musée britannique.}
\end{enumerate}
\vspace*{\fill}
\clearpage
\vspace*{\fill}
\begin{figure}[H]
\centering
\includegraphics[height=0.85\textheight,keepaspectratio]{figures/plate-009.jpg}
\end{figure}
\vspace*{\fill}
\clearpage
\subsection{Planche 10}
\vspace*{\fill}
\begin{enumerate}
    \item[N° 1.] Table ou autel chargé d'offrandes et tiré d'une stèle égyptienne. \emph{Musée britannique.}

    \item[N° 2.] Emblème tiré de la même stèle.

    \item[N° 3.] Emblème tiré du plafond d'une des salles du palais de Rhamessès 3, à Thèbes.

    \item[N° 4.] Figure représentant le mois de \emph{tôbi}, et tirée d'un des bas-reliefs qui ornent le temple d'Edfou.

    \item[N° 5.] Même figure, tirée d'un autre bas-relief du même temple.
\end{enumerate}
\vspace*{\fill}
\clearpage
\begin{landscape}
\vspace*{\fill}
\begin{figure}[H]
\centering
\includegraphics[height=0.9\textheight,keepaspectratio]{figures/plate-010.jpg}
\end{figure}
\vspace*{\fill}
\end{landscape}
\clearpage
\subsection{Planche 11}
\vspace*{\fill}
Stèle égyptienne, inédite, de pierre calcaire. \emph{Musée du Louvre.}
\vspace*{\fill}
\clearpage
\vspace*{\fill}
\begin{figure}[H]
\centering
\includegraphics[height=0.85\textheight,keepaspectratio]{figures/plate-011.jpg}
\end{figure}
\vspace*{\fill}
\clearpage
\subsection{Planche 12}
\vspace*{\fill}
Dessin au trait d'une peinture à fresque, découverte à Pompeï en 1833 ou 1834.
\vspace*{\fill}
\clearpage
\vspace*{\fill}
\begin{figure}[H]
\centering
\includegraphics[width=0.9\textwidth,keepaspectratio]{figures/plate-012.jpg}
\end{figure}
\vspace*{\fill}
\clearpage
\subsection{Planche 13}
\vspace*{\fill}
\begin{enumerate}
    \item[N° 1.] Monnaie autonome coloniale, inédite, d'Alexandria-Troas. \emph{Bibliothèque impériale.}

    \item[N° 2.] Monnaie coloniale, inédite, de la même ville, à l'effigie de Commode. \emph{Même collection.}

    \item[N° 3.] Monnaie coloniale, inédite, de la même ville, à l'effigie de Maxime. \emph{Même collection.}

    \item[N° 4.] Urne cinéraire étrusque, de terre cuite, sans couvercle, vendue aux enchères publiques, à Paris, en 1850. Elle se conservait autrefois à Florence, dans le musée Bucelli, et avait été trouvée au mont Politien. Possesseur actuel non connu.

    \item[N° 5.] Urne cinéraire étrusque, de terre cuite, inédite, ayant son couvercle, et vendue aux mêmes enchères. Possesseur actuel non connu.
\end{enumerate}
\vspace*{\fill}
\clearpage
\begin{landscape}
\vspace*{\fill}
\begin{figure}[H]
\centering
\includegraphics[height=0.75\textheight,keepaspectratio]{figures/plate-013.jpg}
\end{figure}
\vspace*{\fill}
\end{landscape}
\clearpage
\subsection{Planche 14}
\vspace*{\fill}
\begin{enumerate}
    \item[N° 1.] Moyen bronze d'Aurélien, frappé à Perga (Pamphylie). \emph{Biblioth. imp.}

    \item[N° 2.] Autre moyen bronze d'Aurélien, frappé dans la même ville. \emph{Même collection.}

    \item[N° 3.] Face antérieure d'un piédestal carré, de marbre blanc, sculpté sur trois de ses faces. Hauteur totale, 425 millimètres ; plus grande largeur, même dimension. \emph{Musée da la ville de Vienne en Dauphiné.}

    \item[N° 4.] Face latérale gauche du même monument.

    \item[N° 5.] Face latérale droite du même monument.
\end{enumerate}
\vspace*{\fill}
\clearpage
\begin{landscape}
\vspace*{\fill}
\begin{figure}[H]
\centering
\includegraphics[height=0.7\textheight,keepaspectratio]{figures/plate-014.jpg}
\end{figure}
\vspace*{\fill}
\end{landscape}
\clearpage
\subsection{Planche 15}
\vspace*{\fill}
Bas-relief mithriaque romain, de marbre, trouvé, il y a peu d'années, aux environs de Carlsbourg en Transylvanie. Possesseur non connu.
\vspace*{\fill}
\clearpage
\vspace*{\fill}
\begin{figure}[H]
\centering
\includegraphics[height=0.85\textheight,keepaspectratio]{figures/plate-015.jpg}
\end{figure}
\vspace*{\fill}
\clearpage
\subsection{Planche 16}
\vspace*{\fill}
\begin{enumerate}
    \item[N°\textsuperscript{s} 1, 2, 3 et 4.] Développement du sujet représenté en relief sur le bord intérieur d'un plat d'argent, rond, fourré de bronze, inédit, et trouve sur les bords du Rhin. \emph{Biblioth. imp.}

    \item[N° 5.] Médaillon en relief, qui orne le fond du même plat.

    \item[N° 6.] Profil ou forme extérieure de ce plat. Diamètre, 354 millimètres.
\end{enumerate}
\vspace*{\fill}
\clearpage
\begin{landscape}
\vspace*{\fill}
\begin{figure}[H]
\centering
\includegraphics[height=0.85\textheight,keepaspectratio]{figures/plate-016.jpg}
\end{figure}
\vspace*{\fill}
\end{landscape}
\clearpage
\subsection{Planche 17}
\vspace*{\fill}
\begin{enumerate}
    \item[N°\textsuperscript{s} 1, 2 et 3.] Développement du sujet représenté en relief sur le bord intérieur d'un plat d'argent, rond, très-peu profond, et inédit, qui fait partie du trésor découvert, en 1830, à Berthouville, près de Bernay (Eure). \emph{Bibliothèque impériale.}

    \item[N° 4.] Profil ou forme extérieure de ce plat. Diamètre, 355 millimètres.
\end{enumerate}
\vspace*{\fill}
\clearpage
\begin{landscape}
\vspace*{\fill}
\begin{figure}[H]
\centering
\includegraphics[height=0.85\textheight,keepaspectratio]{figures/plate-017.jpg}
\end{figure}
\vspace*{\fill}
\end{landscape}
\clearpage
\subsection{Planche 18}
\vspace*{\fill}
\begin{enumerate}
    \item[N° 1.] Médaillon en relief, dont est orné le fond du plat d'argent représenté sur la planche précédente.

    \item[N° 2.] Manche d'argent, qui a dû appartenir à une patère ou à tout autre vase employé à des libations dans les cérémonies funèbres. En vente à Paris, il y a très-peu d'années. Possesseur actuel non connu.

    \item[N° 3.] Médaille contorniate, à l'effigie de Vespasien. \emph{Bibliothèque impériale.}

    \item[N° 4.] Autre médaille contorniate, à l'effigie d'un personnage inconnu. \emph{Même collection.}
\end{enumerate}
\vspace*{\fill}
\clearpage
\vspace*{\fill}
\begin{figure}[H]
\centering
\includegraphics[height=0.85\textheight,keepaspectratio]{figures/plate-018.jpg}
\end{figure}
\vspace*{\fill}
\clearpage
\subsection{Planche 19}
\vspace*{\fill}
\begin{enumerate}
    \item[N° 1.] Lampe sépulcrale, de terre cuite rougeâtre. Autrefois dans la collection de Bellori. Possesseur actuel non connu.

    \item[N° 2.] Autre lampe sépulcrale, de même matière. En vente à Paris, ainsi que la précédente, il y a peu d'années. Autrefois dans la collection de Sante Bartoli. Possesseur actuel non connu.
\end{enumerate}
\vspace*{\fill}
\clearpage
\begin{landscape}
\vspace*{\fill}
\begin{figure}[H]
\centering
\includegraphics[height=0.9\textheight,keepaspectratio]{figures/plate-019.jpg}
\end{figure}
\vspace*{\fill}
\end{landscape}
\clearpage
\subsection{Planche 20}
\vspace*{\fill}
\begin{enumerate}
    \item[N° 1.] Lampe sépulcrale, chrétienne, de terre cuite rougeâtre, inédite. En vente à Paris, chez un marchand d'antiquités, il y a peu d'années. Possesseur actuel non connu.

    \item[N° 2.] Médaille autonome de Balsa (Lusitanie). BR. Autrefois dans le médaillier du musée Hédervar. Possesseur actuel non connu.

    \item[N° 3.] Médaille autonome d'Ostur (royaume de Valence). BR. Petit module. \emph{Bibliothèque impériale.}

    \item[N° 4.] Autre médaille autonome de la même ville. BR. Grand module. \emph{Même collection.}

    \item[N° 5.] Disque d'argent, provenant d'un trésor trouvé dans la terre, en 1836, à Notre-Dame d'Alençon, près de Brissac (Maine-et-Loire). \emph{Musée du Louvre.}
\end{enumerate}
\vspace*{\fill}
\clearpage
\begin{landscape}
\vspace*{\fill}
\begin{figure}[H]
\centering
\includegraphics[height=0.85\textheight,keepaspectratio]{figures/plate-020.jpg}
\end{figure}
\vspace*{\fill}
\end{landscape}
\clearpage

\subsection{Planche 21}
\vspace*{\fill}
\begin{enumerate}
    \item[N° 1.] Petit bronze d'Antonin le Pieux, frappé à Épidaure (Argolide). Autrefois dans le musée Hédervar ; maintenant dans le médaillier de la \emph{Bibliothèque impériale.}

    \item[N° 2.] Autre petit bronze d'Antonin le Pieux, frappé dans la même ville. Même provenance. \emph{Même collection.}

    \item[N° 3.] Lampe sépulcrale, chrétienne, de terre cuite rougeâtre. Autrefois dans la collection de Bellori, maintenant au \emph{Musée royal de Berlin}.
\end{enumerate}
\vspace*{\fill}
\clearpage
\vspace*{\fill}
\begin{figure}[H]
\centering
\includegraphics[height=0.85\textheight,keepaspectratio]{figures/plate-021.jpg}
\end{figure}
\vspace*{\fill}
\clearpage
\end{document}
